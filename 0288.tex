\subsection{除夕的来袭}

周打工的咖啡厅在年末年初会关店,让员工能在过年期间好好地放松休息。

话是这么说,实际上能不能放松休息又是另一回事了。\\

「真昼~这个切法对吗?」\\

今天是一年的最后一天,除夕。\\

大扫除在前一天就完成了,所以最后一天就优雅地度过,不用在意时间……但怎么可能做得到。

真昼和去年一样在做年菜,周则站在厨房里帮忙。\\

和几乎不会做菜的去年不同,周现在也有了大众水平的厨艺。需要打工的日子,几乎都是由真昼负责做饭,事情全丢给她的话就有点太不是人了,所以周才像这样主动提出帮忙。\\

本来应该由周先做才对,但他几乎不知道年菜的做法,内容也只记得个大概,所以不可能由他来主导。

因此,他才会在真昼旁边帮忙。\\

真昼一边考虑着正在烹调或冷却的年菜的颜色、大小和分量,一边计算着色彩搭配对周下达指示。周看着她感慨地想:真亏她能同时处理这么多事情。

周按照真昼的指示,将年菜仔细地装进重箱里,让红白两色互相映衬。去年这些都是让真昼一个人做的,周心里感到非常过意不去。\\

「……去年你一定很辛苦吧」

「嘻嘻,你能理解做这些很费工夫的话,那就太好了。虽然每一道菜都不难做,但是菜色一多,就会很花时间」

「让你准备这些,真是抬不起头了」

「请不要真低头,现在还在烹调中」

「好——」\\

在炉子和烤箱全开的状态下,要是随便乱动会有危险,所以周安分地待着,同时在能力范围内帮忙烹调。\\

周能帮忙的范围有限。由于他有刚才差点把甜沙丁鱼干烧焦的前科,因此他决定完全遵照真昼的指示来烹调。\\

「……话说回来,刚才很自然地忽略掉了做年菜这件事,可是技能这一块上,稍微想想就会觉得很奇怪吧?年菜是能随随便便做出来的吗?」

「这是多亏了小雪阿姨的教导。为了以防万一,她教了我很多东西」

「那个人到底多厉害……」

「她自称是普通的主妇」

%
「普通……」\\

随便一想也能明白,小雪这个人的能力超乎寻常。\\

她是在孩子能够独自生活后才成为真昼家的管家,主妇经历应该相当长。不过,一个普通的家庭主妇却能教会真昼完美的家事技术,而真昼的人格又受到她很大的影响,恐怕她连情操方面都受过完整教育,是个不得了的人物。

她对并非亲生、只是工作上需要照顾的孩子,以毫无虚假的真心和爱情相待。

周很想吐槽她哪里普通了,但他们也不是随随便便就好联系的关系,只好把话吞回去。\\

「看着小雪阿姨,就会搞不清楚什么是普通呢」

「估计看着你,也会搞不清楚什么是普通的」

「咦……?」

「请你理解自己有多么厉害」\\

小雪是很厉害没错,但受到她教育的真昼也很厉害,问题是她本人似乎没有意识到。

真昼是那种会理所当然地努力、反复练习也不会拉落下的人,对此已经习以为常了,但那本应是值得称赞、可以自豪的事情。\\

「嗯,要说努力的话,我确实努力了……」

「嗯。很厉害。我一直都很佩服,也很尊敬你。因为我做不到,那是你花了很多年才学会的」

「……谢谢。夸我也没好处哦」

「夸你的话,你没准会害羞一下」\\

这种时候不夸奖真昼的话,她本人可能会意识不到自己的努力,所以心里想的事情应该要老实说出来。\\

这并不是客套话,是真昼真的很努力,周也很尊敬她所拥有的自己没有的部分,而真昼则是不太好意思正面接受夸奖,「真是的」她可爱地叹了口气。\\

真昼似乎并不觉得不高兴,不如说还很高兴的样子,看来她应该是听进去了。\\

「既然你这么夸,那我就允许你尝尝味道」\\

真昼从烤箱里取出鱼肉末鸡蛋卷的面团,似乎是正好这会儿烤好了,周在工作中和做蛋糕的时候已经充分体会到了烫铁板的危险,所以他先退到安全的地方拉开距离,然后才为这刚烤好的蛋卷发出欢呼。\\

去年的蛋卷非常好吃,周吃了不少。现在卷起来之前的鱼肉末鸡蛋卷面团正冒着热气,散发出诱人的香味,周的嘴里自然而然地流露出期待。\\

「……现在只能尝尝边角。而且得趁热卷起来,还要稍等一会儿」\\

真昼看穿了周现在就像只被吊着的狗一样,于是开口告诫他。周沮丧地垂下眉梢,真昼大概是觉得他这样很好笑,肩膀发着颤,不张嘴地嘻嘻笑着,一边摊开做蛋卷的卷帘。\\

周尽了自己的一份绵薄之力帮忙,结果从早上开始准备的年节菜,还没等到天黑就摆满了重箱。\\

虽说因为只有两个人吃,所以有斟酌过分量和菜色,但真昼竟然能这么熟练地做出这么多菜……周感慨地这么想着的时候,真昼若无其事地告诉他「有些菜是昨天就先准备好的」这让周再次为真昼的高超手艺感到佩服。\\

顺带一提,刚烤好的鱼肉末鸡蛋卷松软湿润,还带着微微的甜味和淡淡的高汤味,非常好吃。因为是过年时吃的菜,所以真昼只给了周尝一口的分量,不准他再多吃,但这也让周对明天的期待增加了,心情非常愉快。\\

接着只要等菜冷却下来,就可以开始收拾了。这时客厅传来了客人来访的门铃声,两人同时抬起头。\\

「我去看看是谁」

「好」\\

周完全想不到会是谁,还以为是推销员上门,但又觉得不太可能有人会在过年时上门推销。他疑惑地想着,可又不能让手里拿着沾满泡沫的海绵的真昼过去迎接。\\

刚好空出手来的周用毛巾擦干手上的水汽,然后确认了显示访客的灯在闪烁,再看了看对讲机屏幕,整个人愣住了。\\

「谁会在过年的时候……啊?」

「怎么了?」

「……呃,不是,为什么?」

「嗯?」

「我爸和树来了」

「……嗯?」\\

真昼也歪着头,一副不明所以的样子。\\

如果只有树一个人来,周倒还能理解,可是住在远方的父亲为什么会在这里?而且为什么还带着树来拜访?周完全无法想象,也无法理解。\\

「莫名其妙。我是说真的。我爸和树一起来的理由,我完全搞不懂。总之,我可以先让他们进来吗?」

「我是无所谓……」\\

两人恐怕来的时候都知道真昼会在这里,所以周先征求真昼的同意。真昼带着些疑惑,同意了让他们进来。

既然如此,让两人在外面等也不好,周便决定先进门再问话,于是打开公寓的门锁。不到几分钟,门铃就响了。\\

打开门一看,树和修斗真的来了,而且不知道为什么。这个组合实在太不可思议,让站在周旁边的真昼也难掩困惑。\\

「不好意思,这么突然来打扰。你吓到了吧?」

「呃,嗯,吓到了,不过先不说这个……」\\

周往旁边瞥了一眼,只见树穿着有些休闲的服装,明明是寒冬时节,却像是没有考虑过外面的寒冷似的。他的脸色看起来不太好。\\

「……椎名,抱歉,感觉这次真的非常打扰了你们」

「没关系,反正事情也告一段落了,不用在意。我去泡杯热饮」\\

真昼判断再这样下去树可能会感冒,于是先看了周一眼,然后对两人轻轻点头致意,便跑向厨房。

周也催促两人先进来,好让他们先歇一歇。修斗露出温和的笑容,树则是尴尬地垂下视线,犹豫着走进门内。\\

放在桌上的马克杯冒出袅袅热气,像是在鼓励脸色不太好的树。\\

真昼端出蜂蜜姜汤给身体受凉的树暖暖身子,再为修斗端来他喜欢的红茶。\\

她配合两人的喜好和状态端出饮料,然后把毛毯递给衣着单薄的树,接着往坐在地板上的周旁边放了坐垫,跪坐在上边。\\

「你们应该有很多事情想问,不过还是先由我开始说吧。好久不见了,周和椎名」\\

周打算等树能自己说话的时候再管他那边,先看向修斗,打算问问他是为什么过来。只见修斗脸上带着一如往常的微笑,仿佛不用他说也知道。\\

「也不至于好久吧,文化节不是才来过吗?」

「嗯,不过两个月没见,说好久不见也没错吧」

「是啊,毕竟两个月没见了。修斗叔叔,上次文化节之后就没见过了。也再次谢谢叔叔前些日子送的圣诞礼物,我在珍惜地使用着」\\

真昼之前已经传信息和视频通话道过谢,不过她似乎还是想当面道谢,于是恭敬地低头致意。\\

正如她本人所说,圣诞节之后没过多久,她就在写作业和自习的时候使用了礼物,觉得很好用,对此非常高兴。\\

%
「太好了,我还在担心会不会给你添麻烦」


「怎么会呢!我很喜欢,也很高兴收到礼物!」\\

看到真昼拼命挥手否认,修斗笑咪咪的,恐怕是看穿了她会珍惜使用礼物吧。\\

「那么,我可以问爸爸正事了吗?」

「嗯。要过来这件事情,其实我早上说了,不过看样子你没看到」

「咦?哇,真的哎。我们两个一起在做年菜,所以没看手机……抱歉,是我不好」\\

经这么一说,周确认了一下手机,发现修斗传了两则信息,内容确实告知了要来拜访。\\

周已经告诉父母除夕会待在家里,所以修斗应该是确定周在家,即使信息没有已读也还是过来了吧。这是周自己确认不足,并非修斗的不是。\\

「我也是,要是能提前说一声就好了,抱歉。吓到你了吧」

「啊,我是有吓到,但不是因为那个」\\

无论是树还是修斗要来,周都没有大吃一惊的地步。他惊讶的是,几乎没交集的两人竟然一起来到自己家里。\\

「总之,先从我为什么过来开始说起吧。我这边没什么重要的事,单纯是有无论如何都推不掉的事情要办,所以就顺便——唔,要说是主要目的的话倒也没错,总之是来观察情况的」

「那倒是能理解,跟预料中也大差不差。要是妈妈这么说的话倒是没法相信」\\

修斗的信息里有提到他有事要办,回来的时候打算顺道来一趟,这样倒也合情合理,可以理解。他也不至于会是单纯的心血来潮,只是为了看看周就特地搭新干线或开车过来——修斗的话。\\

如果是志保子,倒是不无可能。\\

「真过分」

「因为妈妈的话,很明显是主要来见真昼的。她为了这个目的,没事也能找点事过来」

「嗯,是这样没错。志保子还闹着别扭呢,说她也想来」

「我想也是。所以,为什么爸爸会跟树一起来?你们交换了联系方式吗?」

「不,不是那样……我过来的时候看到他一个人站在公园里,毕竟还有印象,就过去搭话了。幸好没认错人」\\

看来是记性很好的修斗找到了树,把他带了回来。\\

「……看你这样子,树在外面待了挺久的吧。发生了什么事吗?」\\

多亏了蜂蜜姜汤、毛毯和空调,树的脸色已经好转许多,逐渐恢复平时的样子,但表情依旧闷闷不乐。\\

周完全不认为比较怕冷的树会在这个时期穿得这么单薄出门,而且他一个人待在公园里,怎么想都不正常。\\

周猜想应该是发生了什么事,他离家出走了。\\

「啊……不,该怎么说呢」

「你跟你爸吵架了吧?」\\

树会跑出来大多都是因为这个原因。除夕当天应该有很多时间要和家人一起度过,原本就叛逆的树,大概又因为某些事情起了争执。

不出所料,树用一种「你怎么知道」的眼神看向周,紧咬着嘴唇。\\

「这次是因为什么?」

「……这次不是只有老爸的错,他也不是直接的原因」

「怎么说?」

「因为是年底,所以哥哥们他们不得已回来了。然后老爸又跟哥哥吵了起来,我也被牵连,所以就出来透透气,顺便冷静一下」

「……吵架的原因是你哥哥那边?」

「嗯」\\

整理树说的话,就是他哥哥因为之前那场争执,现在和女朋友一起离开老家生活。结果年底回来之后,他跟大辉起了争执,树受到牵连,最后受不了而离家出走。\\

然后碰巧经过的修斗发现了他,也带着保护的目的将他带了回来,大概就是这样。\\

「顺便问一下,你有跟大辉叔叔说你要出来吗?」

「我是没什么预告就出走的,所以他说不定已经知道了,也说不定因为忙着处理跟哥哥的争执,没注意到……毕竟继承人是哥哥,优先级比较高」\\

树放弃似的补充道,他的样子除了疲惫,给人悲伤之感,可以推测家里之前应该吵得很厉害。

树基本上很开朗,不会让人看穿他的内心,既然他都这么明显地表现出软弱的样子,想必是相当难受。\\

「树想怎么做?」

「也没什么好想的……我想问的是我自己。我想怎么做?接下来该怎么做才好?」

「……你们吵架,是你哥哥说他不想继承吗?」\\

周也隐约知道大辉的为人,他认为大辉虽然严格,但不会随便乱发脾气,能吵成那样,表示一定发生了什么严重的问题。\\

那么,跟树的哥哥和树有关系的事情里面,最有可能发生冲突的就是家业。周这么猜想,而他的猜测似乎大差不差。\\

树的身体抖了一下,眉毛皱成八字形,看起来很为难。\\

「与其说是不想继承吧,不如说是冷静下来后还是多少有些叛逆。他觉得爸爸过了这么久还是一成不变的。我眼看要闹大,就去劝架,结果他们两个都对我说『小孩子给我闭嘴』『你没要继承,立场轻松,不会懂的』,我也没辙了」

「树……」

「他们到底想把我怎么样啊?真是的。也替我这个被耍得团团转的人想想吧」\\

树苦涩地低语,这应该是他的真心话。\\

比起愤怒,树更感到有气没处使,他握着毛毯深深地叹了口气。\\

「……那个,我想问修斗叔叔」

「只要是我能回答的,你尽管问」\\

树一定是想问一个和自己无关的大人吧。

修斗不会因为树是周的朋友,就扭曲自己的想法。身为儿子的周可以断言,他不会在这方面有所顾虑。\\

「修斗叔叔,你会希望孩子继承自己的事业吗?」

「在回答问题之前,我希望你先留意,我之所以能这么说,是因为我处在不会被责任束缚的立场」\\

修斗并不清楚树的状况。

周也不会把朋友的烦恼透露给父母。他应该是根据刚才听到的内容和前几天见到大辉时的样子来判断的吧。\\

「我家的家业没有到需要传承的地步,所以和你想要的答案意思不太一样,不过,我认为孩子愿意走到自己这条道路的更远处,是一件令人高兴的事。因为这就代表他们超越了父母的背影」\\

藤宫家不像赤泽家那样是名门世家,虽然家族之间多少会继承一些东西,但也是个普通家庭。

所以,修斗无法真正理解树的烦恼,但他似乎连这一点都理解了,继续说道:\\

「不过,我也觉得这不是我们能强制的事情。我这边的立场比较轻松,所以能够说,既然要断绝家业,那也没办法。我觉得无所谓」

「……即使会断绝一路走来的道路?」

「是啊。当然,我认为孩子愿意开拓我们一路走来的道路是一件令人高兴的事,所以打算交给孩子的意志决定」

「爸爸……」

「嗯,我不认为传承家业这件事本身不好。我认为将代代相传的东西托付给下一代也很重要。有些东西就是这样建立起来的,所以无法一概否定,也不认为你父亲有错」

「这样啊」

「不过,即便逼人继承,我觉得也不会顺利」\\

「因为人类越是被逼迫,越会想要反抗」修斗微微露出苦笑,眼神有一瞬间飘向远方。\\

「到头来,父母和孩子是不同的人格,是不同的人,做不到完全如意的。我以前也叛逆过父母」

「咦?」

「因为我不是那么乖的孩子,让父母吃了不少苦」\\

从周的角度来看,修斗和祖父母的关系良好,完全没有争执,看起来是十分理想的亲子关系,但修斗年轻时应该发生过什么事。

周认为现在的修斗是沉着冷静的大人,从儿子的角度来看也是个优秀的人,不过他本人却说「我以前还算调皮」。\\

「你是因为家庭因素,正在烦恼自己的出路,是这样没错吧?」

「……是的」\\

大辉和千岁之间的关系、家世、继承问题——被这些弄得团团转的树脸色阴沉地点了点头,修斗则平静地继续说道:\\

「虽然我说什么都不可能负责,但在我看来,你的父亲是愿意听你说话的人」

「怎么可——」

「嗯,你可能不这么认为,从你的角度来看,他确实是个不讲理的人。但我认为他不是真的顽固到听不进别人的意见」\\

周也这么认为。大辉并不是完全不听儿子说话的顽固之人,也不是贯彻自己想法的死脑筋。如果他是那种人,周跟他说话也不会被听进去。\\

大辉确实很顽固,而且是意志坚定、毫不动摇的人,但也是个有温情的人。周认为他只是因为在树和千岁的事情之后,由担心而产生了顽固。\\

「只是,你现在还没能让令尊上桌听你说话。我认为他也还没有做好心理准备」

「上桌……?」

「我的意思是,凡事都需要事前准备。在暴风雨中是没办法进行建设性对话的」\\

修斗喝了口热气散去的红茶,平静地看向树。

他的眼神柔和又慈爱,同时又笔直地注视着树的内心深处。\\

「……我该怎么办才好?」

「树,你的愿望是什么?」

「我想和小千,和千岁在一起。」

「你父亲的愿望呢?」

「想尽快让我们分开吧」

「不对」

「周?」

「我觉得,不是这样的」\\

周无意袒护大辉,情感上也站在树这边,但他想反驳树认为大辉想强行拆散他和千岁的想法。\\

大辉确实没有接纳千岁,也没有认可她,但也没有采取行动要排除她。周甚至觉得他本人是想接纳千岁的。\\

只不过,他心中似乎有无论如何都无法接受的部分,所以才无法接纳千岁,为此感到纠结。\\

「你这么说是有根据的吧?」

「……虽然猜测别人父亲的想法不太好,但我觉得大辉叔叔并没有想硬让你和千岁分手。至少我从来没听他说过要你们分手」

「那不是因为顾虑到孩子的朋友吗?」

「我也想过这个可能性,但我觉得不是」

「……既然你这么认为,那在你看来就是这么回事吧。在我看来,结论就是叫我和小千分手」\\

周眼中的大辉和大辉亲生儿子眼中的大辉当然不同,从树的角度来看,周就像是在袒护朋友愤怒的矛头。

周看见树阴沉的脸上瞬间闪过一阵红晕。\\

「他和你好好照顾孩子的父母不一样!根本没考虑过我的心情!」\\

树说完后表情马上又蒙上阴霾,因为他客观地理解到自己正在激动。\\

他马上露出歉疚的表情,垂下愤怒的肩膀,一脸严肃地向周道歉,反而让周感到过意不去。\\

「别在意。我很能理解被别人说三道四时的不爽心情,明明对方什么都不知道。听起来像是我在袒护大辉叔叔,你也不会有什么好心情吧。不是你的错。是我不好,真的很抱歉」

「……为什么是你道歉啊,笨蛋」

「是我不好吧」

「才不是,不管怎么想都是我自己在乱发脾气,迁怒而已。你没有错,我只是在这里耍脾气而已」

「因为蜂蜜姜汤?」

「啰嗦」\\

考虑到树的性格,继续让气氛严肃也不好,于是周刻意开朗地反问,树似乎也理解了这一点,故意配合着说。\\

尽管愤怒本身并没有消失,但树也没有强烈地向周吐露,而是把情绪吞了回去,露出平常的轻松笑容。\\

修斗没有插嘴,只是在一旁看着。他确认周和树之间的气氛软化后,接着说道:\\

「我和大辉作为父亲的立场不同,所以不能深入说什么,但我觉得你们最好冷静地谈一次。大辉不是会不分青红皂白地否定一切的人,但我也能理解你说他听不进去的主张……你应该要准备一些筹码,让他愿意听你说」

「筹码?」

「对方的弱点、好处、坏处……什么都行。如果他不肯坐到这个位置上,就根本无从谈起。没有筹码就想单方面地让他理解或说服他,基本上是不可能的」

「……没有筹码的话,就没有听的价值,是这个意思吧」

「也不能说没有,但至少不会听吧。其实,不耍手段就能正面沟通才是最理想的,但就是因为做不到,你才会这么辛苦吧?」

「……是的」

「你的父亲大概不承认你是对等的存在,他应该认为你还是个需要庇护的孩子」\\

树也隐约感觉到了这一点,只见他绷紧了脸。\\

「你不希望关系决裂吧?那就该做好准备,把他拉到谈判桌上。要是你随便闹事或顶撞,对方也会变得顽固」\\

明明只交谈了很短的时间,修斗却似乎已经掌握了大辉的人品。

周因为没有深入接触,所以大辉在周眼中的形象和修斗所感觉到的人品是一致的,但至少没有偏离到树会否定的程度。\\

「人不是谁都能明事理,价值观也不一定相同。也会有人不希望你所期望的事情发生。至于正确与否,解释方式要多少有多少」

「……我觉得爸爸说的不对,但爸爸认为那是正确的,是这样吗?」

「反过来也一样。你所期望的事情,对令尊来说未必是正确的,所以他才会这么不肯退让」

「我……」

「所以为了沟通,最好还是采取行动让他坐上谈判桌。如果想避免最坏的手段,先准备好谈判筹码比较保险」\\

最坏的手段,想必是指断绝关系离家出走。树应该也预想过这个可能性。\\

不过,付诸实行的风险很大,而且千岁恐怕也不希望他这么做。

要是知道树为了自己要和父母断绝关系,她想必会拒绝。\\

「先准备好筹码,再来思考你想怎么做,有什么样的愿望,以及现在面临的问题,也就是如何实现愿望。有没有头绪,可以妥协到什么程度,最好先整理好这些再开始谈。如果只是含糊地说出愿望,没有明确的愿景,就不要觉得对方会接受。令尊看起来在这方面很严格」\\

顽固耿直、坚定不移的大辉,不会听进去半吊子的觉悟。\\

树似乎也明白这一点,他抿着嘴唇皱起眉头,修斗用温柔的眼神看着他。\\

「如果无论如何都无法互相理解,我也打算成为你的筹码。有其他大人站在你这边,也可以是你的筹码」

「……为什么要这么帮我?这对你没有任何好处,我也不记得自己做过什么让你中意的事」\\

树大概不明白这一点。

在树看来,修斗是周的父亲,两人并没有直接的交情。他们前几天才打过招呼,几乎可以说是初次见面。\\

像修斗这样终究是外人的父亲,不仅仔细地听他说话,还给他建议,甚至说会直接帮助他,也难怪树会怀疑。假如周受到大辉的亲切帮助,也会怀疑是不是有什么企图。\\

被怀疑的修斗眨了眨眼,然后忽然露出释然的微笑。\\

「因为你有恩于我」

「恩……?」

「我认为拯救了我们家周的人是你。因为有你向他搭话,对他伸出援手,周才没有落入黑暗之中。现在才能像这样安稳地生活……单凭这点,还不足以成为理由吗?」\\

对修斗而言,树的存在让修斗感到非常高兴,这甚至远远超过了周的料想。

周没有详细跟父母说过树的事情,但他们知道两人非常亲近,要说能敞开心扉的朋友,排第一的非他莫属。

周陷入消沉的时候,最担心他的人就是父母。\\

担心周离开不愉快的喧嚣,独自前往陌生之地的是父母。说到底,把他送过去的,也是父母。

树以朋友的身份接纳了选择独自生活的周,修斗对这一点十分感谢,这是周现在才得知的。\\

(……虽然身为当事人听他们这么说,超难为情的)\\

在一旁听着父母向朋友表达自己很高兴对方愿意和儿子交朋友,周难免会感到害羞。不过,对树和修斗都很感谢的周无论说什么,都会被当作他在掩饰害羞,所以周还是保持沉默。\\

「如果这样还不够的话……我想想。倒不是责怪他,不过我觉得你的父亲对你的态度不太好,而且我也想以个人身份守护你的选择。我喜欢专情的人,这是好事」\\

修斗这次则是用开玩笑的语气,说出他中意树的理由。树愣愣地看向依然散发着柔和氛围的修斗,然后像是在忍耐什么似的垂下眉梢。\\

「……太狡猾了」\\

周隐约察觉到树是在针对什么而说他狡猾,但他没有说出口,只是默默地看着树。

修斗也没有再多说什么,只是静静地守望着树垂头丧气的样子,直到他下定决心为止。
