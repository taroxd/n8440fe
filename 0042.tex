% 42 天使様の憧憬


% 「どうぞ」\\


%  実の両親とはいえ客人なのでもてなすのは当然なのだが、真昼がお茶を出すと言って聞かなかったので周は彼女に任せていた。


%  真昼が自分で飲む用に持ってきたティーセットと紅茶がまさかこんな所で役に立つとは思うまい。\\


%  普段は周と真昼二人で腰かけるソファに座った両親は、穏やかな笑みをたたえていた。\\


% 「あらありがとう真昼ちゃん。すっかり慣れてるわねえ」


% 「は、はい」


% 「本来は周がしなきゃいけないのよ?」\\


%  おそらく、周が淹れると紅茶の渋みだけ取り出しそうなので真昼直々にしているのだが、志保子はほんのりと呆れた顔を浮かべている。\\


% 「いえ、私がしたかったので……」


% 「まあ周がするとお湯の温度適当にするから仕方ないわね」\\


%  ごもっともなのだが、それを指摘されるのは些か腹が立つ。


%  しかしながら反論は出来ないので大人しく黙っていれば、志保子のにっこりした笑みを向けられる。\\


% 「そういえば周、ちゃんと真昼ちゃんの事名前呼びするようになったのね」\\


%  唐突な指摘に、周も真昼も体を強張らせる。\\


%  ナチュラルに呼んでしまっていたので気付かなかったが、以前母親に会った時、周は真昼を名前呼びしていなかったし、真昼は周をぎこちなく呼んでいた。


%  それが今ではどもる事もなく自然と呼びあっているので、志保子の事だから当然勘ぐるだろう。\\


% 「……別にいいだろ」


% 「いいと思うわよ。仲睦まじいのはいい事だし」\\


%  敢えて余計に追及せずに、ただにこにこと実に明るい笑顔でこちらを見守る志保子に、周は頬がひくりと震えるのを感じた。


%  まだからかわれた方がよかったかもしれない。こういう時の志保子は、確実に頭の中であらあらまあまあと仲を捏造して楽しんでいるのだ。\\


% 「志保子さん、あまり周をからかわない」\\


%  ただ、そこで止めにかかるのが修斗だ。\\


% 「志保子さんの悪い癖だよ。あまりつつかないでやりなさい」


% 「はぁい、残念だけど仕方ないわね」\\


%  志保子は修斗のいう事ならすんなり聞くので、振り回される息子としてはありがたい限りだ。\\


% 「でも、やっぱいいものよねえ、息子が可愛い女の子と仲良くしてるのを見るのは」


% 「志保子さんの悪い癖が暴走しないか、私はひやひやしてるんだけどね」


% 「あら、修斗さんがとめてくれるでしょう?」


% 「自覚があるなら直した方がいいと思うけど、志保子さんのそういう所も好きになったから仕方ないね」


% 「まあ……修斗さんってば」\\


%  止めたのはいいものの、今度は両親が微妙に二人の世界を作り出すので、周はため息を隠さない。\\


%  基本的には修斗は常識人ではあるのだが、妻を無意識に可愛がってしまう事があるので、時々他人を寄せ付けない空気を生み出してしまうのだ。


%  幸い、それは家族の前でしか見せないものだし外ではそういった露骨な雰囲気は出ないのだが、ここが周の家だから気が緩んだのかもしれない。\\


%  何歳になっても仲睦まじいというのは息子からすれば夫婦円満でよい事なのだが、それを見せつけられるこちらの身にもなって欲しいものである。\\


%  ああなると周としては割り込みたくないので、諦めてダイニングから持ってきた椅子に座って再度深くため息をついた。


%  真昼も、その隣に用意してあった椅子に腰かけてそっと周を窺う。\\


% 「……ご両親、仲がよいのですね」


% 「そうだな。まあ外ではあんな風ではないけど、家だとあんな感じだ」


% 「そうですか」\\


%  苦笑と共に答えれば、真昼は目を細めて志保子と修斗を見る。\\


%  その表情は不快そうなものではなく、むしろ、まばゆいものを見た時のもの。


%  憧憬と羨望の滲んだ、尊いものを見るような、そんな眼差しだった。\\


%  儚いと言い切れるほどに淡い笑みで見守る真昼の姿に、思わず手を伸ばしかけて――。\\


% 「あら周、どうかしたの?」\\


%  現実世界に戻ってきたらしい志保子の声に、即座に手を引っ込めた。\\


% 「どうしたの、じゃねえよ。母さん達が二人の世界に入ってるから俺達が居たたまれないんだよ」


% 「あら羨ましいの?」


% 「全く、これっぽっちも羨ましくない。そういうのは自宅でしてくれって思ってるんだよ」\\


%  どうやら真昼の手を握りかけていた事には気付いていなかったらしい。真昼も、同じように気付かなかったのか、周の言葉に苦笑している。\\


%  どうして、手を伸ばしたのかは、分からない。


%  ただ、なんとなく……あの真昼を一人にしたくない、そう思った。\\


%  もう普段の真昼に戻っているので、周は微かに安堵しつつ、悟られないようにいつもの仏頂面に戻す。\\


% 「で、母さん達は息子の顔見て満足したか」


% 「周より真昼ちゃん見て満足したのだけど……」


% 「おい」


% 「半分冗談よ。まだ目的果たしてないしねえ」


% 「目的?」\\


%  てっきり、新年の挨拶と真昼への挨拶目的だと思っていたのだが、志保子にはまだ他に目的があったらしい。\\


% 「周達まだ初詣行ってないのよね?」


% 「人が落ち着いてから行くつもりだったし」


% 「でしょ? 真昼ちゃんも行ってないわよね。メッセージで聞いたもの」


% 「はい」


% 「だろうと思って着物持ってきたのよー」\\


%  どうやら、志保子は真昼と初詣に行きたかったようだ。


%  満面の笑みを浮かべていて、随分と大きな荷物を持ってきた理由が今更ながらに分かって周は今日何度目か分からないため息をついた。\\


%  志保子は可愛い物好きであるし、人を着飾るという行為そのものが好きなので、こういった機会は逃したくないのだろう。


%  周が覚えている限りでも家に着物が幾つかあったので、それを持ってきたらしい。\\


% 「私娘に着物着せて初詣に行くの夢だったし……真昼ちゃんならきっと似合うと思って」


% 「母さんが単に着せ替え人形したいだけだろ」


% 「そんな事ないわよ? でも、真昼ちゃんに着せたいっていうのは大きいわねえ」\\


%  だってすごく似合いそうだもの、と自信満々に言う志保子の意見は正しい。


%  というか、あまり真昼に似合わない服装もなさそうだ。 \\


%  周が見た限り、ボーイッシュな服装もお嬢様のような品のある格好も、フリルやレースをふんだんに使ったいかにも女の子らしい服装も何度かしているが、どれもこれもよく似合っていた。美少女というのは着るものを選ばないらしい。


%  和装も恐らくではあるが非常に似合うだろう。\\


%  藤宮家は一人息子なので、娘を着飾りたかったらしい志保子としてはこのチャンスを見逃す事が出来ないらしい。\\


% 「……まあ真昼がいいって言うなら着せて行ってくればどうだ」


% 「なんで周はこない前提なの」


% 「いや真昼と出かけて学校のやつらにバレても困るし」\\


%  両親と真昼だけなら、別に初詣に行こうが家族に見られるだろうし問題はない。


%  そこに周が加わった場合が問題なのだ。\\


%  見るからにパッとしない周が真昼と並んで参拝しているのを同じ学年の人間にでも見られた場合、冬休み明けが阿鼻叫喚の地獄絵図になる事が予想出来た。


%  流石に、そのリスクを背負ってまで初詣に行きたいとは思わない。\\


% 「ばれなければいいの?」


% 「まあそうなんだろうが普通にバレ……いや母さん、まさかとは思うが」


% 「ふふ、こういう時のために色々と持ってきてるのよ?」


% 「どんな時だよ!?」\\


%  着物やら襦袢やら小道具やら、着物関連だけにしてはやけに荷物の量が多いと思えば、周をいじる用に更に荷物を持ってきていたらしい。\\


% 「修斗さんも結構乗り気よ」


% 「父さん……」


% 「折角の機会だし、いいんじゃないかな。私としては、恒例行事だし出来れば一緒に行きたいんだけどね」\\


%  そう言われると、断りにくいものがある。


%  家族の仲を大切にする修斗の意向もあって志保子が申し出ているのだ、それを突っぱねるのも悪い気がする。\\


% 「でもさあ」


% 「大丈夫、お母さんを信じなさい。必ず元の周とは似つかないようなかっこいい男にしてあげるから!」


% 「それ元の俺がカッコ悪いって言ってるよな」


% 「もちろん修斗さんに似てるからカッコ悪い訳じゃないけど髪型とか雰囲気が野暮ったいわねえ。陰気って言うのかしら」


% 「うるさい」\\


%  自分でも野暮ったいのは自覚しているが、好きでこんな格好をしているのだから一々指摘されたくないものである。\\


% 「整えたらそこそこ見れるのに、周ったらめんどくさがるから……」


% 「余計なお世話だ」


% 「勿体ない。……ねえ真昼ちゃん、真昼ちゃんも周がきっちり整えた姿見てみたいわよね?」


% 「え?」\\


%  突然話を振られて、真昼は目に見えておろおろとしている。


%  あまり真昼にぐいぐい押さないで欲しいものであるが、志保子は遠慮なしに迫っていた。\\


% 「周が着飾ったら真昼ちゃんも見直すと思うのよ。こう見えて、周は割と顔はいいのよ? 性格も素直ではないけど修斗さんに似て紳士的だし、ちゃんとすればほんとに良物件だと思うの」


% 「え、あの……そ、そうですね……?」


% 「一緒に初詣、行きたくない?」


% 「そ、それはその、行きたいです、けど」


% 「おい裏切るなよ」\\


%  出来れば万が一を考えて却下してほしかったのだが、真昼は突っ込んだ周をちらりと見る。\\


% 「……周くんが嫌なら、いいです」\\


%  しゅん、と少し気落ちしたような声で眉尻を下げられて、周はぐっと息を詰まらせた。\\


%  本人は隠しているようだが、明らかに残念そうにしている。これ見よがし、という訳ではなく自然と漏れたものらしい。


%  そっと長い睫毛を揺らして瞳を伏せる姿に、非常に罪悪感が湧いた。\\


%  志保子の方からは「真昼ちゃん悲しませた」といった責めるような視線が、修斗からは「諦めた方が早いよ」といった視線が送られて、周はうぐぐと小さく呻く。


%  これでは、自分が真昼をいじめているようではないか。\\


% 「……分かったよ」\\


%  あんな顔をされては、折れるしかなかった。

