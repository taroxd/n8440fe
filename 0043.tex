% 43 天使様と初詣


% 「はい、もういいわよ」\\


%  散々志保子にああでもないこうでもないと髪やら顔やらをいじられ服装のコーディネートをされ、ようやく解放された時には地味に疲労していた。\\


%  あまり服装には興味ない周としては苦痛の時間だったが、鏡で確かめてみれば苦労の甲斐あってか普段の周とは比べ物にならないくらいに整った男が居た。\\


%  志保子が選んだのは、ダークグレーのチェスターコートに白のタートルネック、黒のスラックスといったシンプルでありながらカジュアルさを抑えたコーディネートだ。


%  新年のめでたい行事なので軽装にならないように気を付けたらしく、フォーマルな雰囲気をほんのりと匂わせている。\\


%  周もあまりカラフルな服装は好きではないので、このモノトーンの落ち着いた格好は周の好みとも合致していた。\\


%  髪型も確認してみたが、やや長めの前髪はアイロンやワックスと志保子の腕によって上手い事横に流して、普段は前髪に隠れがちな瞳が出ていた。


%  目元をしっかり露出させた事で印象が大分明るくなるが、それだけではなく上手く髪全体にボリュームを持たせてセットした事により、洗練された雰囲気を醸し出している。\\


%  陰気臭いと母や樹に揶揄される周はそこにはおらず、どこの誰だといった爽やかさを感じる男が、鏡の前に居た。\\


% 「ちょっと弄るだけで爽やか好青年になるのにどうしてしないのかしらねえ」


% 「趣味じゃない」


% 「周そういう所あるわよね。まあ顔が仏頂面だから笑わないと爽やかにならないんだけど」\\


%  仏頂面は余計なお世話なのだが、事実なので否定は出来ない。\\


% 「じゃあ、私真昼ちゃんの調整に行ってくるからリビングで待ってるのよ」\\


%  周は自室であれこれやっていたので、一度自宅でお着替えしているらしい真昼の様子は知らない。


%  自分で着付けが出来るという事なので真昼は一度自宅に帰って着てくるらしいが、着付け出来るという時点で真昼のスペックの高さが伺えた。\\


%  部屋から先に出て行った志保子を見送り、もう一度鏡で自分を見る。\\


%  久しくこうした格好をする事はなかったので、自分が自分でないように思える。\\


% 「……まあ、悪くないのかな」\\


%  真昼の隣に並ぶにはみすぼらしい気がしなくもないが、普段の周より何倍もましだろう。


%  視界にちらつく事のなくなった前髪を少し弄りながら、たまにはこういうのも悪くないのかもしれないな、と小さく呟いた。\\


\vspace{2\baselineskip}

%  リビングで修斗と待つ事数十分、玄関のドアが開く音がした。


%  女子の支度には多大なる労力と時間がかかる、と聞いているので待つこと自体には不満はなかったのだが、真昼が志保子にセクハラされていないかという点が心配である。\\


%  やっとか、と座っていたソファから腰を上げて玄関の方を見たくらいで、真昼が静かにリビングにたどり着いていた。\\


%  真昼の姿を一目見た瞬間に、思わず呆けてしまう。\\


%  普段、真昼は和装なんてしないし、見る機会もない。似合うだろうな、とは思っていたが――まさかこんなにも似合うとは、思ってもみなかった。\\


%  流石に振り袖は人混みでは動きにくいという事で小紋にしたらしいが、淡いピンクを基調とした梅柄の小紋は、真昼が持ち主なのではないかと思うくらいに着こなしていた。


%  普段はピンク色をあまり着ない真昼だが、上品さの中にもフェミニンさを香らせている。


%  色素の薄い長い髪は横髪が一房だけ残されて、あとは上でかんざしによってまとめられている。真っ白なうなじやしゃらりと揺れる飾りが女性らしさを際立たせていて、なんとも色っぽい。\\


%  元の美しさを引き立てるようにほんのりと施された化粧もあいまって、これ以上になく清楚美人といった雰囲気を醸し出していた。\\


% 「どう? 中々可愛く出来たと思うんだけど。真昼ちゃんは素材がいいからほんと飾りつけ甲斐があったわ」


% 「うん、とても似合っているよ」\\


%  さらりと笑顔で褒めている修斗に、真昼もやや恥じらうように瞳を伏せる。その仕草すら色っぽいのだから、美人というのは本当に恐ろしい。\\


% 「ほら周、ちゃんと感想を言わなきゃ駄目よ」


% 「似合ってると思う」\\


%  流石に親達が居る前で絶賛など到底出来る筈もなく、無難な称賛を送ったのだが、志保子は非常に不服そうだった。\\


% 「……そういう所がダメなのよ?」


% 「うるさい」\\


%  志保子から駄目出しされてしまったものの、周は両親の前ではこれ以上褒めるつもりはないのでそっぽを向く。


%  そんな周に志保子は呆れたようだったが、周の性格をよく知っているのかため息一つで見逃してくれるようだった。\\


% 「全くもう。……ちなみに真昼ちゃん、どう? 周、こうしたら全然雰囲気違うでしょ?」


% 「は、はい。普段とは全く……」


% 「普段からこの格好してたらモテるんでしょうに、しないのよねえ。ほんと、損してるわぁ」\\


%  周としては余計なお世話なのだが、志保子は本気で残念がっているようでため息をついている。\\


% 「折角修斗さんに似ているのに、それを活かそうとしない周にはがっかりなのよ。勿体ないわー」


% 「まあまあ志保子さん。周も色々とお年頃なんだろう」


% 「お年頃ならモテたいんじゃないの?」


% 「周はどちらかと言えばただ一人だけでいいタイプだと思うからね。他は煩わしいんじゃないかな」


% 「まあ」\\


%  フォローの筈が、志保子の妄想に火をつけている。


%  確かに周は不特定多数に好かれるよりただ一人が居ればいい……というか修斗にそう教えられているし実際その方がいいと思っているが、今の状況だとその相手が真昼だと言っているように聞こえるではないか。\\


%  志保子の輝かんばかりの笑顔に頬をひきつらせ、顔を背ける。\\


%  どうしてこうも邪推されなければならないのか、とは思うのだが、実際問題他人からそう見えてしまうのも自覚している。


%  少なくとも、周にとって真昼は特別なのだ、と言い切れる程度には。\\


%  それは、事実ではあるが――。\\


%  ちらりと真昼に気付かれないように盗み見て、そっとため息。\\


% (そりゃ、好きと言えば好きだけど)\\


%  好ましい、とは思う。


%  ただ、恋愛感情と言い切るには、まだ違うと思うのだ。\\


% 「母さんが邪推してるような事は一切ないからな。んなくだらない事言ってないで、車の準備とかしようか」


% 「つれない子ねぇ……ほんと。まあいいでしょう、修斗さん、車の準備しましょうか」


% 「そうだね」\\


%  どうやら話を逸らす事に成功したようで、二人とも出掛ける準備の方に移り出す。


%  どの神社に行くのかは両親に任せておき、先に駐車場に向かい家を出ていった両親の背中を見送った。\\


% 「……俺は鞄に必要なもん入ってるしそんな準備要らないけど、真昼は?」


% 「え、このバッグに入ってますので」


% 「そっか」\\


%  急に二人きりになったのでほんのりとした居たたまれなさを覚えつつ、周は窓の戸締まりやら余分な電化製品のコンセントを抜いていった。\\


%  リビングの照明を消したところで、改めて真昼を見る。


%  やはり、よく見なくとも美人だと思う。両親の手前あまり手放しで褒められなかったが、誰が見ても和装美人な真昼は非常に目の保養になる。\\


% 「どうかしましたか、周くん」


% 「ん、いや似合ってるな、と。いかにも清楚な和装美人って感じだ。可愛いと思うぞ」\\


%  本来は修斗から女性がお洒落をしていたら褒めてあげるべき、と学んでいるので見た瞬間に褒めるべきだったが、流石に両親の目の前で称賛するのは恥ずかしかった。\\


%  素直な感想を口にすれば、真昼が大きく数回目をしばたかせて、それからほんのりと頬を染めてきゅっと唇を結んでいる。


%  前もそういう反応をされた事を思い出して、周は小さく苦笑した。\\


% 「ああ、褒められるの嫌だったっけ? すまん」


% 「そ、そんな事はないです、けど。……周くんは、割と」


% 「割と?」


% 「……何でもないです」\\


%  ぷい、とそっぽを向いた真昼に何なんだと思いつつも、口を割る気配はなかったので大人しく諦めて、真昼を伴って玄関に向かう。


%  はいていく靴は歩く事を考えて下駄ではなくブーツで和洋折衷のスタイルらしいが、それはそれで可愛らしい姿が見られそうだった。\\


%  しゃらしゃらとかんざしの飾りを揺らしながら何とかブーツをはいた真昼は、先に外に出て扉を支えていた周にそっと近寄る。


%  思ったよりも、距離が近い。珍しく真昼の方から接近して、そっと背伸びをしている。\\


%  耳を貸せ、という事なのだろうか、と玄関の扉を閉めて鍵をかけてから腰を屈めると、そっと真昼が掌で口元に輪を作りながら耳元に近付ける。\\


% 「周くん」


% 「ん?」


% 「その、……かっこいい、ですよ?」\\


%  それだけ小さく囁いて、横をすり抜けて足早にエレベーターホールに向かった真昼に、周はそのまま扉にごん、と額を押し付けた。\\


% 「……それはずるい」\\


%  仕返しと言わんばかりの囁きに、周の心臓が早鐘を打つように鼓動を刻んでいた。\\


%  真昼のせいで一気に火照りだした頬を冷やすのに時間がかかって、先に駐車場で待っていた両親に不審な目で見られる事になる周だった。
