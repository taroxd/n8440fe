\subsection{情人节前的一幕}

进入二月,社会上针对情人节的商战加速了。购物中心为了情人节商战准备了特别区域,就连平常的超市也出于情人节的关系,摆出了各种各样的巧克力点心。\\

当然,这种气氛传播到了各个地方,周的学校里也明显能感受到。随着情人节的接近,女生那边变得其乐融融,男生这边则是期待得坐立不安的学生越来越多。

周对这方面的事情其实没什么兴趣,不过今年无疑能收到真昼送的巧克力,所以也不是完全漠不关心。\\

话虽如此,周也不认为还会发生什么别的事情,所以大家兴致高昂的时候,他难免会在远一些的地方事不关己地观望。\\

「今年门胁在情人节好像会很不妙」\\

晨练结束后,门胁正坐在座位上翻开课本,直到第一节课前都在预习,认真得不得了。看到他那副模样,连别班的女生都跑来远远地投以热情的视线,周不由得这么嘀咕了一句。\\

特地跑到别班来瞻仰意中人的行动力实在令人佩服,但可怕的是,来的人不只一个,而是好几个,更可怕的是每次看过去,发现来的人都不一样,甚至还越来越多。\\

在一旁静静地翻着单词卡的树听到周的声音,抬起头来,然后眼神变得有些怜悯。他大概是想起了去年的事情吧。\\

「应该很不妙吧。他是田径社的社长兼王牌,长得帅,性格和头脑都很好,那女生当然是他推走多少,就有多少想把他拿下」

「别推了别推了」

「只是打个比方啦。不过,优太会不会为此感到高兴就另当别论了」

「看上去他应该是困惑更多一点」\\

优太对于女生的好感本身并没有厌恶,但似乎对于女生的尖叫加油声变成日常景象这件事有些意见,常常会困扰地垂下眉梢。\\

身为朋友,周担心优太这么受欢迎会不会引起其他男生的反感,好在由于门胁的人品以及他难掩辛劳的模样,并没有多少男生真的嫉妒他。

去年情人节时,目睹他几乎没有自由时间的样子,学生们全都表情僵硬,异口同声地说「到了那种程度,已经不是羡慕而是恐怖了」。\\

「希望他今年不要预估错了数量」

「光是需要预估就已经很不妙了。习惯真是可怕」

「就是说啊……看着门胁一年到头都很受欢迎,感觉都麻痹了」

「一般来说,光是被一个人表白,对当事人来说就是一件大事了」\\

一个人向另一个人表达强烈的爱意,这是一件很重大的事情,至少对周来说,这需要一生一次的决心和勇气。

而门胁却同时被许多人喜欢,比起佩服,周更觉得担心。\\

「优太本人不擅长应付女生的追求,而且他想专注在田径上,所以一律拒绝了。他还说在不了解对方的情况下交往也很失礼。即便任挑任选也都拒绝,很有优太的风格」

「毕竟门胁是个非常有常识又正经的人,简直像是诚实两个字成精了一样」\\

周非常喜欢门胁这种对他人认真和诚实的态度。

常常听说有人并没有喜欢上对方,却先试着交往,而且还跟好几个人一起进展的行为。周对此是完全无法理解的,而他看到门胁即便有条件随意挑选,也要真挚地观察对方然后拒绝的模样,很难不对他不产生好感,\\

说是理所当然或许也没错,但能把这种理所当然的事情视为理所当然,在此原则下行动,这是周认为非常可贵的品质,也是做人所必需的。\\

「你对优太的评价很高啊」

「那当然,看一眼都看得出。而且他确实是个很好的人」

「唔,你都不夸我」

「那你稍微改改那爱拿我开玩笑的毛病」

「那不夸就不夸了吧」

「喂」

「反正我知道你只是不会说出口,心里其实很认同我」

「烦死了烦死了」

「哈哈哈」\\

这家伙是怎样?周用眼神瞪了过去,树却毫不在意,反而愉快地咧嘴一笑。周又瞪了他一眼,然后叹了口气。\\

被本人看穿的那一刻,周就差不多算是输了,但被当面说出来还是让他觉得有些难为情,并且有点火大。\\

继续作出反应也只是徒增树调侃的材料,于是他不再吐槽,把树的身影从视野中移除。结果树又在视野之外欢快地笑了起来。\\

「哎,我们过情人节倒是挺轻松的」\\

和门胁相比,周他们既轻松又不着急。\\

周基本上不受欢迎,而且真昼也在,所以只要能收到真昼的巧克力就行了。树则有千岁。

如果和去年一样,周应该能收到千岁送的友情巧克力,但也仅此而已。\\

不用烦恼回礼的问题,也没有其他想要的东西,所以情人节应该能过得非常和平。\\

「啊,对了,敬请期待小千今年的奇迹手艺」

「这位男朋友请你制止她的暴行」

「你觉得我制止得了吗?」

「……应该不行」\\

千岁最近多少老实了一点,也展现出认真面对各种事情的态度,但遇到这种节日还是会往不该有的方向全力发挥。

去年从千岁那里收到的巧克力里面,正常的几个很好吃,周希望她能往那个方向发展,但他也隐约察觉到千岁不可能安分守己。\\

周不由得警惕起她今年会搞出什么花样,而树则是不知为何得意地竖起手指,脸上浮现无畏的笑容。\\

「今年好像和去年别有风味哦」

「听着不像是打比方」

「听说是精心构思、下足功夫的一样东西,你一定会喜极而泣」

「说的绝对是字面意思吧」

「爽吧,今年也确定会有刺激物了」

「感觉眼泪都要流出来了」\\

出于本人的信念和自尊心,以及负责踩刹车的真昼的劝阻,千岁绝对不会做出不能吃的东西,不过在能吃的范围内做出令人大跌眼镜的食物,就是千岁这个人一贯的作风。\\

她本人最喜欢刺激物,这也是周无法阻止她的最大原因。考虑到千岁那高于常人的耐受力,耐受力只是普通水平的周吃下去没准就会成为剧毒,她是时候明白这一点了。\\

树好像完全事不关己似的捧腹大笑,周一脸正色,开始认真考虑起来,是不是让他如字面意思一样不得不捧腹比较好。\\

「多好啊,现在就提前掉眼泪,说不定就能培养耐受力了」

「对啊,你也先流点眼泪培养耐受力吧。我去叫千岁让你这个负责品尝的试吃个够」

「你出卖我!?」

「没事没事,既然是你可爱又可爱的女朋友做的,那肯定能吃能吃」

「那也太强人所难了吧!」

「可不就是你给好朋友推荐的嘛」

「别把别人做的东西当成毒药好吗?」\\

趁着上课时间还没开始,周和树吵得不可开交。而话题的始作俑者,也是制作刺激物的专家千岁则气鼓鼓地插嘴进来。\\

她看起来没有生气,只是对两人的态度感到不满,于是对着树连连拍打。\\

「千岁,别再送刺激物给我了好吗?」

「才不要~」

「那就拿树做实验,做出能让人接受的东西吧。要让他吃多少都行」

「真拿你没办法」

「你刚刚明确出卖我了好吗!?小千!?」

「没事没事」\\

没事个鬼。\\

周虽然很想吐槽,但总之先成功制造出牺牲者了。他无视树拼命做出的可怜表情,移开了视线。\\

「我会做成能吃的东西啦。糟蹋食物不好」

「混入刺激物难道不算糟蹋食物吗?」

「我只是为了新的境界,开拓未知的味道罢了。我会控制在能吃的东西里面。我会一直做到昼儿为我担保的」

「……别把真昼的舌头弄坏了,我说真的」

「讨厌啦~我不会做那种会弄坏肚子或舌头的事啦~还有,昼儿的舌头比你的要耐辣许多,她没准还会吃得津津有味哦。我也会控制在好吃的范围内」

「然后树去年就惨叫了来着」

「阿树不喜欢吃辣的嘛。放心,阿树的份我会单独做。试吃的是另一回事!」

「哇——好令人开心的宣言——」

「别一字一顿地念啦,真是的!」\\

因为有前科,所以树在这方面的信任度为零。千岁对着树扬起眉梢,周则在心里为他合掌祈祷。照这个样子看来,树大概会被逼着吃更加刺激一点的东西吧。\\

「总之,敬请期待我和昼儿的巧克力。到当天为止都保密」

「随你高兴……我只要能拿到就很感激了」

「哼哼,好好感谢我吧」\\

千岁双手叉腰,一脸得意地说道。周很高兴能拿到巧克力,于是坦率地回了一句「谢谢你平时的照顾」,气势被削弱的千岁则嘀咕着「你就是这点让人受不了」。\\

「我说,周」

「怎么了?」

「你如果收到其他女生送的巧克力,会怎么做?」\\

千岁压低声音,以免被别人听见。周歪着头,疑惑地回答:\\

「啊,你说友情巧克力?如果有人送的话,那就收下并回礼呗。不过我完全不期待会收到」

「为什么是以友情巧克力为前提?」

「毕竟我有真昼了,很难不是友情巧克力吧。不可能不可能」\\

虽然不至于全校学生人尽皆知,但绝大部分的学生应该都知道周和真昼在交往,周完全不认为会收到本命巧克力,也没有想要收到的念头。\\

「哇,真自卑。」

「这哪里自卑了?在现在的情况下,还觉得会有人送自己本命巧克力才其他怪吧。我可没有那么自信过剩或轻浮」

「话是这么说没错啦」\\

千岁似乎很在意什么,脸色和声音都比平常低沉。\\

「可是啊,世上又不是那么单纯。喜欢上已经有明确对象的人也是有可能的。毕竟人的心意是不受控制的」

「之前也说过类似的话……结论就是我没办法回应。话说,为什么是以有那种人为前提啊?」

「就不能设想一下吗?」

「我说啊」

「从你的角度来看或许不可能吧。可是,我觉得也不是没有可能。而且,喜欢上一个人也不一定就会希望对方回应。你没有过这种经验吗?」

「……倒是有」\\

喜欢一个人,想要珍惜对方,但并不是只有独占欲,希望对方一定要和自己交往。周也曾经想过,只要真昼幸福就好。\\

「我觉得你最好还是多注意一点比较好,因为昼儿很担惊受怕的」

「呃,这个嘛,我知道的,而且我也不想让真昼难过,所以有在注意……不过,总想着自己可能被别的人喜欢,会不会太自我感觉良好了?」

「你这么说倒也挺有意思的」

「喂」\\

周瞪了过去,示意「这话是你先说的吧?」千岁却只是回以傻气的笑容,让周也跟着放松下来。\\

「总之,我只想说,要是惹昼儿哭的话,我可会生气哦」\\

听到「惹哭」这个词,周回想起去年的事情,顿时绷紧身子。千岁则是瞪大眼睛看着他,仿佛在说「不是吧?」树也一样看着这边,让周感到非常不自在。\\

「……真惹哭了?」

「最近没有惹哭她!」

「最近」

「……生日的时候有弄哭她就是」\\

周不可能主动去做出伤害真昼的事情,也一直尽可能地努力让她保持笑容。\\

只不过,生日的时候该算是另一码事了吧。\\

并没有让她伤心,也没有让她受伤,他相信那是喜悦的泪水,真昼也说过那是喜极而泣。\\

如果主动弄哭她这件事情本身不行,那周恐怕要遭到制裁了,但那次如果也作数,今后就会非常麻烦,所以周希望千岁能放过他。\\

「啊,那个算是过关吧」

「你有什么立场这么说……」

「昼儿挚友的立场!」

「行吧」\\

千岁得意地挺起胸膛,周则扶额表示疲惫。正和木户聊天的真昼担心地凑过来,周于是挥挥手表示没什么。\\

\vspace{2\baselineskip}

那天是需要打工的日子,周像平常一样出勤工作,这时收拾好空桌的宫本回到了吧台内,嘀咕道「我每年都很期待情人节哎」。\\

「是准备收很多巧克力吗?」

「不是不是。我们咖啡厅每个季节都会换菜单,而且会请我们试吃,味道都很不错」

「啊。确实挺好吃的」\\

配合季节更换菜单是大多数餐饮店的做法,而这间咖啡厅的菜单则是由丝卷决定。她决定菜单似乎都挺随性,但味道从来没有出过问题。\\

要到情人节了,菜单上推出使用巧克力的甜点和轻食,颇受欢迎。\\

甜点类的餐点都是由丝卷在幕后准备,她对味道很讲究,由她准备的东西不可能不好吃。

由于材料量的关系,难免会有剩下的情况,剩下的这些就成为员工吃的甜点,不着痕迹地分给大家,所以周也经常能吃到甜点。\\

考虑到发胖的可能性,以及真昼的晚餐,周都会克制着吃,但味道本身都非常好,所以他也十分满足。\\

「店长人很好,真庆幸能有这份工作」

「我记得宫本喜欢吃甜食来着?」

「还挺喜欢的。而且能省下伙食费,这点也很不错」

「你是一个人住?」

「对啊。我在大学附近租了房子。不过老家也没那么远」

「这家伙住的地段挺好的」\\

由于快要打烊,没了客人,工作也就有了空闲,同样来上班的大桥捧着洗好的虹吸壶回来了。\\

「毕竟我是考上推荐名额的,就先租了房子」

「炫耀自己头脑好?」

「成绩和生活态度都比你好啦」\\

最近周开始能分辨出他们不是在吵架,而是在互相轻轻拌嘴,也没必要特地劝架,周便放任他们继续了。\\

「顺便问一下,藤宫你情人节要怎么办?」\\

看来是回到了原本的话题,宫本一边随意地应付着大桥,一边漫不经心地向周问道。他们看排班表也都知道周请了假,没什么好隐瞒的,周便老实回答:\\

「正常和女朋友一起过」

「那是不是超亲热的」

「不陪女朋友过情人节,身为男朋友反而不合适吧」

「你说得对。茅野这次也排了休假来着」\\

今天不在的总司圣诞节时有来上班,不过这次排班表上写着休假。\\

「……宫本?」

「可惜,我要上班。不过,店长每年都会送甜点当慰劳品,所以还好啦」

「为了甜点来上班,你不对劲」

「你不也一样」

「因为能吃到限定商品,免费的」

「你不也一样」

「啰嗦,笨蛋笨蛋」\\

大桥说着有点孩子气的话,甚至构不成反驳。他看起来有点心神不定的样子,似乎并不是完全没有回报,周感觉到他们之间细微的进展,松了口气。

不过,周似乎不小心把想法表现在了脸上,被两人责备「你在笑什么?」于是他连忙绷紧脸,露出平时的客套笑容,匆匆忙忙地跟从餐桌上回来的等待清洗的餐具见面去了。
