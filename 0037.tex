% 37 天使様と年越し


%  日が暮れる頃にはすべての品を作り終えて重箱に詰めた真昼は、今度は晩ご飯の支度を始めていた。\\


%  といっても、年越し蕎麦なので蕎麦は茹でる手前まで作ってあるものを購入しただけだし、麺を茹でて具材を用意するだけなのだが。\\


%  かまぼこはおせちのものが余っているので、丁度いいだろう。ほうれん草はゆがくだけだしネギは刻むだけ。


%  一番手間がかかるのはえびの天ぷらなのだが、真昼は面倒であろう揚げ物を嫌な顔一つせずに揚げている。\\


% 「あと、かぼちゃが余ってたので、ついでに天ぷらにしておきますね」


% 「おー……豪華な年越し蕎麦だ」


% 「たまにはこういうのもいいでしょう」\\


%  そう言った真昼によって完成した年越し蕎麦は、実家で食べるものよりやはり贅沢なものとなっている。\\


%  大きな海老のてんぷらは一人二匹分用意されているし、おまけのかぼちゃの天ぷらもサックリとした仕上がり。ほうれん草とねぎはたっぷり、かまぼこは扇形に飾り切りされていた。\\


%  ちなみに真昼は天ぷらは後載せサクサクスタイルらしく、周の分も直接蕎麦には載せずに皿に分けられていて、ささやかな気遣いがありがたかった。\\


% 「おー」


% 「どうぞ召し上がってくださいな」\\


%  周はそれだけでは足りないだろうという事で、おせちの余りも小皿に盛られて出される。


%  真昼が席についたのを見てから互いに手を合わせていただきますと食物に感謝してから、蕎麦に手をつけた。\\


%  市販品ですよ、とは言っていたものの、お高めの蕎麦を買ってきたのか噛むと蕎麦の香りが広がる。


%  つゆも濃すぎず薄すぎず、 ほっと一息つけるような塩梅に仕上がっている。お腹の奥から温まる、寒い日にはぴったりな味だ。


%  


% 「はー……これぞ年末って感じだ……」\\


%  つゆを飲んでほぅ……と息を吐き、しみじみと呟く。\\


%  テレビを見ながらゆったりと蕎麦を食べて新年を待つ、というのはやはりよいものだった。


%  実家でも毎年年越し蕎麦を食べ年末特番を見て年一の歌番組を見て年越しするのが恒例だったので、今年も同じ過ごし方が出来たのがありがたい。側に居るのは、家族ではなく甲斐甲斐しい他人の少女であるが。\\


% 「年越し蕎麦を食べると一気に年が終わるって実感が湧きますよね」


% 「ほんとな。……今年は色々あったなあ」\\


%  といっても、色々のほとんどを占めているのが真昼との交流である。


%  一人暮らしを始めた時はこんな美少女がご飯を作ってくれるなんて、 一ミリも思っていなかった。\\


% 「周くん一人暮らし始めた年ですからねえ、そりゃあ大変だったでしょう」


% 「お前はめちゃくちゃ慣れてるよな」


% 「まあ、一通りなんでもこなせますからね。何も出来ないのに一人暮らししようとする周くんがダメダメなのですよ?」


% 「ぐっ。……そうだけどさあ」


% 「ほんと、仕方のない人ですよね、まったく」\\


%  呆れたというよりは微笑ましそうに窘めた真昼の表情は、柔らかい。


%  周の世話を焼く事を苦に思っていないらしく、あくまで穏やかな表情だ。\\


% 「……今年は本当に世話になった」


% 「まったくです」\\


%  小さく笑いながらの全肯定はほんのりと胸に刺さるものがあったが、真昼は嫌そうでないのが救いだろう。\\


% 「……来年もよろしく頼む」


% 「分かってますよ。周くんは私が居ないと不摂生自堕落生活にまっしぐらですので」


% 「否定出来ない」


% 「……分かってるなら気を付けるのですよ?」


% 「来年の抱負にするわ」\\


%  おそらく心がけても真昼にせっせと世話を焼かれて決意も溶かされてしまいそうな気がするが、本人には言わず心の中に押し留めておく。\\


%  もちろん身の回りの整理整頓やらなんやらはするが――彼女のご飯に頼るのは、間違いないだろう。


%  すっかり虜にされている、と自覚しているものの、もうどうしようもなかった。\\


%  改善すると真昼に宣言してみても笑われたので、ムッと表情を固くしたのだが、真昼は楽しそうに小さな笑みを浮かべるだけだった。\\


\vspace{2\baselineskip}

% 「そろそろ年明けますね」


% 「そうだな」\\


%  年越し蕎麦を食べ終えてソファで歌番組を眺めていれば、あっという間に時は過ぎて日付変更直前まで来ていた。


%  テレビを必要以上に見ないのか、あまり今時の歌に詳しくないらしい真昼が静かに、そして楽しそうに歌番組を見ているのを眺めていたら、思ったよりも早く時間が過ぎていたのだ。\\


%  中継で除夜の鐘をついている風景に画面が変わっていて、改めて年が変わるのを実感する。\\


%  隣に腰かけた真昼は、瞳を伏せながら静かに除夜の鐘の音を聞いている。\\


%  そうこうしている内に百七回目の鐘の音が聞こえて――。\\


% 「明けましておめでとうございます」\\


%  日付が変わった瞬間、こちらを見てきっちりと背筋を伸ばしてから腰を折った真昼に、つられて周も姿勢をただして同じように新年の挨拶をする。\\


% 「明けましておめでとう。……なんか変な気分だな、二人で年越しって」


% 「ふふ、そうですね。……今年もよろしくお願いしますね」


% 「こちらこそ……というかむしろこっちがお願いする立場というか」


% 「それは否定できませんね」\\


%  くすりと笑った真昼に周は苦笑した所で、膝の上で震えるスマホに気付く。\\


%  どうやら樹や千歳達から新年の挨拶が来ているらしく、アプリのアイコンに幾つか数字がついていた。


%  それは真昼も同様で……というか恐らく真昼の方が多いだろうが、真昼のスマホも震えている。\\


%  最近はメッセージを送るだけで新年の挨拶が出来るのだから、楽になったものである。\\


% 「少し返信しますね」


% 「俺もしとくわ」\\


%  おそらく真昼はたくさん挨拶が来ているだろう。何となくだが、男子には連絡先を教えていない気もするが。\\


%  慣れた手つきでフリック操作で返事を打ち込んでいく真昼の手際に「こういうところは女子高生だよなあ」と感心しながら自分も樹や千歳に返信を送っておく。


%  メッセージには普通に『明けましておめでとう』の他『椎名さんと仲良く年越ししたか?』と要らない詮索が入っていたりするので、図星ではあるものの否定のメッセージを送った。\\


%  すぐに樹から『またまたぁ』とからかうような返信がきたので、しばらく茶化されたり否定したりを繰り返して会話を楽しんでいたのだが。\\


%  ぽす、と二の腕に、重みがかかる。それから、甘い匂いがふんわりと香る。\\


%  恐る恐る横を見てみれば、瞳を閉じた真昼がこちらに寄りかかっているではないか。\\


% (――待て待て待て)\\


%  声には出さなかったが、周は相当うろたえていた。\\


%  うたたねは以前にもあったのだが、まさか、隣で、それも寄りかかって寝るなんて、誰が想像するだろうか。\\


%  なぜ真昼が寝てしまったのか、というのは考えずとも分かる。\\


%  現在の時刻は深夜零時半過ぎ。\\


%  規則正しい生活を送っているらしい真昼が夜更かしなどあまりする筈もないし、そもそも今日一日おせち作りに奔走して、表には出していなかったが疲弊していたのだろう。


%  睡魔に抗うほどの体力がなかったに違いない。\\


%  理由は、分かる。\\


%  分かるが、よりによってこのタイミングで寝落ちするとは。\\


%  周に寄りかかって寝ている真昼は、周の混乱や狼狽など知らないと言わんばかりの実に安らかな寝顔を見せていた。長い睫毛や整った鼻梁も桜色の唇も、無防備にさらされている。


%  寝顔は初めて見る訳ではないのだが、こんなにも至近距離で見る事なんてなく、体を強張らせた。\\


% 「真昼、起きろ」\\


%  遠慮がちに声をかけても、反応はない。


%  余程疲れていたのか睡魔に飲まれて深い眠りの海に落ちているらしく、声をかけても肩を動かして少し揺らしても覚醒の気配はなかった。\\


%  軽く腿を叩いても触れた体を揺らしても、起きてくれない。\\


%  そんな事をしていたらもたれている部分がずれて前のめりになり始めたので、周はあわてて真昼を受け止めて引き寄せた……のはよいものの、図らずも抱き寄せたような体勢になってしまって、更に慌てる事になった。\\


% (……すげえいい匂いする)\\


%  食事後一度帰宅して入浴やら何やらを済ませてきたというのもあるが、洗髪料のフローラルな香りに加えて本人の匂いなのかほんのりと甘い匂いがして、とてつもなく居心地が悪い。


%  おまけに、何か柔らかいものが当たっている気がしなくもないので、気が気ではなかった。\\


%  起こそうにも、あまりに熟睡しているので、起こすのは忍びないし、そもそも叩き起こすレベルでないと起きない気すらしている。\\


% (どうしたらいいんだ)\\


%  新年早々にこんなハプニングが訪れて、周は頭を抱えた。\\


