\subsection{天使大人的生日}

这一天是周迄今为止的人生中忙碌程度能够排进前三的日子。\\

周从早上开始就站在厨房里,发挥出近日特训出来的技术,在下午他喊来了支援,和他们一起商量房间的装饰和最大的惊喜,为了不让真昼感到寂寞,周推进准备工作的时候还见缝插针地与真昼联系。\\

周一边推进各种事情一边做着准备,差点就要崩溃。但这些与平时真昼同时在做的事相比起来,根本算不得什么。无论什么样的多任务真昼都能处理,周边在心里赞叹着,边努力做好准备。\\

那般匆匆忙忙的一段时间过后,太阳已经完全西斜。\\

当周终于对自己的成果感到满意时,看了眼钟表,已经是平常晚餐前的时间了,窗外也已从夕阳红过去,呈现出一片蓝紫到青褐的晕染。\\

本还想着来不及完成准备该怎么办,但总算勉强在最后一刻完成了,周从心底松了一口气,为了邀请在隔壁等着被叫来的真昼按响了她家的门铃。\\

真昼马上就打开了门,可谓是以她的方式同样做足了准备。\\

「让、让你久等了」\\

有些急切地从门后面出现的真昼结结巴巴地说道。周忍不住笑了出来,真昼似乎也明白他在笑什么,脸颊微微泛红,眼神有些尴尬地四处游移着。\\

「……请、请当作没看见吧」

「为什么?」

「不、不是,因为,那个,期待得兴高采烈什么的,不是很丢脸吗?」

「啊,这说明一直在期待吧?那我就太开心了」\\

如果这是场一厢情愿的庆祝,周对此也是会感到失落和难堪的,而真昼在高兴和期待中等待着他,光是这一点就让周非常满足了。\\

知道真昼等得如此兴奋,周也不知道能不能满足她的期望,不过周相信自己的准备没有疏漏。\\

接下来,只需要将迄今为止的累累硕果向真昼展示出来。\\

看起来真昼也做好了准备,一身可爱的装束打扮精心得不像是家居服。周牵着真昼的手问道「你那边都准备好了吗?」而后真昼羞涩地眯起眼睛,轻轻应了一声「嗯」。\\

周牵着只带了一个小挎包的真昼的手回到了自己家里。真昼发现屋内只有门口放鞋的地方开着灯,她使劲眨了眨眼睛。\\

「……咦,好黑啊」

「如果从门口看到就没意思了吧?」\\

虽说有一面墙把走廊和客厅分隔开来,但门上嵌着块玻璃,能看到里面的情况。\\

毕竟好不容易瞒着真昼筹划并实行了这一切,那么在最后的收尾阶段懈怠是绝对不行的。为了能给真昼提供惊喜,他必须要有收有放。\\

「那么,可以的话,我能遮住你的眼睛吗?黑暗可能会有点可怕,但有我在,能不能放心地把自己交给我呢?」

「呵呵,既然如此,那我就完全相信周君吧」\\

出于对周的信任,真昼爽快地点了点头。\\

周还没来得及捂住真昼的眼睛,她就没有丝毫犹豫地合上眼皮,那副危险的样子让周在心里嘀咕,其实她再怀疑点自己也不要紧的。与此同时,他把手绕到真昼的背后和膝盖下面,将她抱了起来。\\

她还是一如既往地轻,弄得周当真担心起来:她是不是再多吃一点比较好?周再用略自由的另一只手打开通往客厅的门,灯也给开了。\\

真昼依旧闭着双眼,直到周发话之前,她似乎都不打算睁开眼睛。真昼的听话让周松了一口气,他走向了准备好的生日席——沙发,小心翼翼地把真昼放下,确保她不会受伤。\\

真昼从走过的距离和坐下的感觉就知道了自己身在何处,她习惯性地挺直了背。\\

「啊,能不能答应我,先别睁开眼睛也不要动,再多等一下?」

「嘻嘻,别把我当小孩子啊,我理解周君想要把自己所准备的事情最大限度地展示出来,只是再多一会儿的话,我完全会满怀期待地等着的哦?」

「抱歉抱歉,还好有你这个善解人意的女朋友」\\

其实周觉得真昼过于通情达理,已经到了不正常的地步,不过这种聪明伶俐也确实符合真昼的性情。他苦笑着,迅速确认为真昼准备的东西有无疏漏。\\

「嗯,眼睛可以睁开了」\\

「这样就都就绪了」周温柔地告诉真昼,而翘首以盼的真昼则是旋即——却又缓缓地抬起了眼帘,尚且有些没适应光亮而眯缝起来的眼眸慢慢显露出来。\\

对于今天第一次进入这个房间所见的景象,真昼是怎么想的呢?\\

「……这些」\\

真昼细小的声音里有点颤抖。\\

这些指的是哪些,不用问也能通过真昼的视线与眼神中的光彩知道。\\

「我拜托树、千岁、门胁和木户帮我装饰了一下。那个,真昼不是说过即使让他们知道你的生日也无所谓吗?光靠我一个人的话是不可能弄出这么漂亮的装饰的,我去求他们帮忙,他们二话不说就答应了。在装扮这方面他们的品味比我好太多了,我很感谢他们。怎么样,很可爱吧?」

「可爱……太好看了」

「我拜托他们做一个盛大的生日装饰,他们就做成了这样」\\

在为数不多真昼所信任的朋友们的帮助下,周将这个客厅在今天一天内变成了生日祝贺的样式。\\

主题是『儿时憧憬的快乐生日』。\\

几束扎在一起的气球、相应颜色的纸花热热闹闹地挂在墙上,再加上用LED灯摆出的一串大字『HAPPY BIRTHDAY』也贴在墙壁上,更增添了一派华丽的气氛。\\

天花板上则悬挂着水晶玻璃装饰,随着供暖的风摇曳,又在灯光的照明下,不时闪烁出柔和而耀眼的光芒。\\

在真昼坐的沙发上,真昼喜爱的布偶们用丝带可爱地打扮过了的模样,严阵以待今日的主角。\\

这么多装饰难免会让空间显得过于花哨而杂乱无章,而为了不让色彩过于花里胡哨,装饰都统一采用了淡淡的暖色,或许是其布局和色彩所致,沉稳之中又很有前卫的风格。\\

即使是中途查看过装饰、自觉热情已经基本褪去的周,看到房间装饰完成后的样子也发出了感叹声。在真昼的视角下,这一切想必是一份很大的惊喜吧。\\

看她瞪大的双眼中映照着房间,熠熠生辉,正是周料想中的反应。他为自己的成功而欣喜,不由得翘起了嘴角。\\

「我们围绕真昼喜欢的颜色之类的主题,大家一起做了装饰。拜托千岁的调查也是没白忙活」

「啊……我记得是说过那些话题,是你拜托的千岁同学呢」

「看来是直到刚才为止都还没发现啊,到底是千岁的口才。真的,太感谢了」\\

从周的角度看到的真昼的喜好可能会与实际情况不符,所以他拜托了千岁调查,如今看到真昼眼睛里闪闪发光的样子,周再次感到拜托千岁是正确的选择。

两人视线交汇,真昼露出了软塌塌的笑容。周对此感到得意的同时也觉得害羞,为了掩饰自己逐渐微微发热的脸颊,轻咳了一声。\\

「……真昼能开心就好。那个,我已经准备好了晚饭,这就去端上来」\\

周以生日庆祝还要继续为借口站起身来,真昼则是抱着打扮得可爱的小熊布偶,腼腆地注视着他。\\

\vspace{2\baselineskip}

虽然只有几米,但远离了真昼的周总算抑制住了内心的激动,把事先做好的菜肴装盘摆上餐桌。\\

真昼恐怕是凭借气味已经隐约猜出都是些什么菜了,这里应该不会让她感到太多新鲜感和惊喜,但这毕竟是用来庆祝和慰劳她的,还是用她所熟悉的味道比较好。\\

周选择了以日本料理为中心,更偏关西的清淡口味,理由是真昼不太喜欢浓重的味道或是新奇的食物。\\

「我过生日时是真昼花了很多心思做的料理,这回轮到我了」\\

这时候真昼来到了餐桌,周朝她微笑着,自己也坐到了平时的座位上。\\

虽然厨艺远不及真昼,但周为了尽量让真昼高兴,摆上餐桌的都是他认为真昼喜欢的菜品。菜单谈不上生日宴会级别的豪华,不过都是由周觉得真昼喜欢的东西构成的。\\

真昼什么都会吃,没有特别讨厌的食物,但她当然也有偏好。她整体上更偏爱清淡素雅的味道,更确切地说是喜欢咸度较淡,充分展现原料味道与汤底风味的菜肴。\\

发挥食材本身的风味,比调味重的料理更难。\\

味浓的菜肴比较容易调整,并且保底也能用调味料蒙混过关,而清淡的料理却不能这样。并且发挥食材的味道和享用食材本身的味道又是两回事,各有各的烹饪和调味方法。

因而,掌握真昼喜好的调味对周来说是一个漫长的过程。\\

(不过将来还是希望能完美掌握啊)\\

周觉得做不出自己喜欢的人所喜欢的东西,作为伴侣会显得很丢脸,何况真昼能完美地制作周所喜欢的菜肴,更是让他如此觉得。

这次做得固然算是美味,但仍有很大的进步空间。周为自己的不足而暗自羞愧,而真昼则静静地观察着周的表情。\\

「……好吃」\\

真昼优雅地把嘴唇贴在碗边,静静地将高汤喝了下去,脸上荡漾出轻飘飘的笑容,呼出了一口气。\\

即使只是一份高汤,周也是严格按照真昼所教导的方法从头开始认真熬制的,因此最终的味道应该会符合真昼的喜好。

现实没有背叛周的预期,真昼保持着柔和的表情继续用餐。\\

「太好了,看来合你的口味。说实话,我刚才紧张得要命」

「我虽然评判过周君的料理,但应该从来没有抱怨过吧?」

「没,这点我还是知道的。但,这个和担心没法让你高兴还是两码事吧?」\\

周边用筷子掰开萝卜酿虾球边嘀咕着。而后不知为何他听到不满的声音说「话是这么说啦」。\\

「……周君,不知不觉间料理水平已经进步得很多了呢」

「我的也就是负分终于到了正50分左右的水平,跟真昼的100分200分这种还很有差距,怎么也追不上的啦」

「要是轻松被追上了,我也会很困惑的」

「我就算花一辈子也追不上你的水平了,不如说对我而言,你做的就是最好的,所以不论我的厨艺如何都无关紧要。话虽如此,我也会努力做出真昼心里的第一的」

「……周君又在说这种话」

「也不是一天两天了嘛」

「真是的」\\

这不是什么责备的话,而是无奈的意思,长时间的相处使周很明白这一点。\\

更进一步说,他还知道真昼说这句话表示她并没有觉得不好,反而很高兴。\\

而真昼则是对面露笑容的周微微瞠目,嘴唇抽动着说着「周君可真是的」,仿佛感到耀眼一般躲开了视线。\\

\vspace{2\baselineskip}

「多谢款待」

「招待不周」\\

原本分量就有限的晚餐转眼之间就被两人吃完了。\\

真昼的饭量并不至于太小,不过考虑到等在后面的东西,这会如果提供了太多食物,届时就会看到她吃不完了。于是他控制了菜量,并且不至于让真昼察觉。\\

周之所以选择日式料理,也有为了符合真昼的口味,并通过小碗及色彩来调整菜量和视觉印象的因素,而真昼似乎没有注意到这点。\\

「……没想到你会为我做这么多」\\

周告诉真昼说后续的收拾让他来做,并将作为主宾的她引导到沙发上坐下,而真昼在周洗完东西回来时感慨地嘟囔着,可能是因为想帮忙而微微有些不满。\\

「为了喜欢的人会不吝惜努力这点,真昼也是一样的吧」

「唔,话、话是这么说」

「要说的话,我做这些是因为我想为你做,或许并不是真正为了你好吧」\\

到头来,自己的行为终究只是靠着自我满足来驱动的,周不确定用「为了真昼好」这么漂亮的话来形容是否合适。\\

「因此,这只是我擅自做的事情罢了」

「……周君总是会这样呢,真是的」\\

啪,真昼责备似地拍了拍周的上臂。她也知道周是不会让步的,嘴角勾勒出一个仿佛混杂着困惑、无奈与高兴等复杂情绪的微笑。\\

「……但是,今天我真的好开心,这么……」

「啊,可以再稍等一下吗?」

「嗯?」\\

被打断话的真昼瞪大了双眼,但这里周绝不会让步。\\

「虽然有种要结束的气氛,但我还没打算就此结束呢。真昼的生日还没过完吧」\\

「哎?」周听到了困惑的一声,但不如说,生日的重头戏从现在才开始。

装饰房间与亲手准备晚餐不至于会让周准备近一个月。这段时间里,为了让真昼高兴,周在周围一众好人的帮助下到处奔走操劳,那份成果还没有让真昼看到。\\

真昼表现得非常满足又矜持,而周今天打算带给她更多的幸福,以至于足以冲散她的那份矜持。\\

「可以再闭上一次眼睛么?」\\

睁着眼的话,惊喜感就会少掉一半。因此周今天第二次请求真昼,真昼旋即紧紧闭上眼睛,将头仰起。\\

那模样,与其说是乖巧地听从周的话等待着,不如说像是期待着即将到来的什么,还夹杂着些许紧张。\\

明显是往那方面想的真昼实在是太过可爱,周忍不住就要开始傻笑,只好拿手捂住了嘴巴。

本来真昼也看不见,周并不需要掩饰什么,只不过恋人那等待中的模样,让他禁不住地感到怜爱。\\

「……抱歉,这回不是打算接吻」\\

毕竟过后再打破别人的期待不是什么好事,周便轻声在真昼耳边低语道。真昼雪白的眼皮中一下子露出焦糖色的眼眸,对焦到了周身上。\\

在那之后真昼很好懂地涨红了脸,既可爱又闹着别扭似的发出「笨蛋笨蛋」的声音,还顺着声音的节奏敲着身旁周的胸口,好像在打鼓一样。\\

那副模样可爱得让周差点又要忍不住傻笑了,他只能咬住脸颊的内侧来强行忍住,不然若是表现到了脸上,真昼小小的拳头恐怕就要演奏得更加激烈了。\\

「疼疼疼,抱歉啦。……是还有东西想要给你看,所以才请你闭上眼睛的」

「……早说啊」

「抱歉啦」\\

比起刚刚更闹别扭的真昼转过头去,闭上了眼睛。周将嘴唇轻轻凑上了真昼递到眼前如桃子般可口的粉嫩脸庞。\\

由于有过很多次了,真昼似乎已经能凭借体温和触感来作出判断,她睁开了眼睛,一下子愣住了。「好好地闭上眼睛哦」周笑着说道,真昼纤细的喉咙一颤,传来了低吟声。\\

这也是惊喜,周心满意足地追加了一个计划外的惊喜,便向厨房走去。\\

周之所以会把晚饭全部制作好,饭后收拾的工作也全部交由自己来,都是为了不让真昼靠近冰箱。\\

他连同盘子一起从盒中取出一大早就在准备的、这几周的集大成之作,用双手稳稳地端着。\\

周小心翼翼地将盘子缓缓搬到矮桌上。真昼从声音和气息中发觉到周回来了,便将脸转向了他,他见状暗暗笑了起来,十分期待真昼睁开双眼时的反应。\\

「眼睛还不能睁开哦」\\

「准备工作还没有结束呢」周一边低声说道,一边拿出藏起来的比普通蜡烛更纤细多彩的蜡烛,小心地插在白色的奶油大地上。\\

一根、两根,周默数着将蜡烛稳稳插上。蛋糕被蜡烛本体的亮彩所支配,以至于周都感觉,17根或许还是多了点。\\

最终蛋糕比周所预想的要更五彩缤纷,周稍稍反省自己的预判不足,不过都到这一步了,他还是用打火机点燃了蜡烛。\\

这么多根蜡烛,周也得稍微花些时间。顺利地将所有蜡烛都点亮后,他便用遥控器关掉了房间的灯。\\

周围一下子变暗了,但并非完全黑暗,真昼每有一岁,便有一束柔和的光晕,宛如罩上一层薄纱般,将装饰好的房间照亮。\\

「真昼,可以睁开眼了」\\

周对着到最后一直遵守着自己嘱咐的真昼温柔地耳语道。真昼怯生生地慢慢抬起了眼帘——\\

「……啊」\\

她情不自禁地发出了颤抖着的小小声音,既不是叹息,也不是惊愕。\\

在暗淡的光线中浮现出的真昼的脸庞上,是一副愣神的表情,从中已然见不到理性的抑制。

她的眼里泛起更大的波纹,映出摇曳的烛火。\\

这时,周干咳了一声,缓缓地张开紧闭的双唇。\\

坦白说周还是有点害羞,但更胜于此的,是他想把这份心意告诉真昼、传达给真昼的冲动。\\

尽管自己不太擅长,但周还是认真地、充满感情地,为真昼送上了小时候父母曾为他唱的那首,一提及生日就能联想到的小曲。\\

「十七岁生日快乐,真昼」\\

周送上了这句今天见面时故意没说,又一直想说得不得了的,为心爱之人的生日而祝福的话语。再往真昼那一看,她仅仅是僵在原地,一动不动。\\

她之所以会这样,想必是因为遭受到了预料之外的冲击,现在她肯定是在内心里拼命整理着信息和刚刚发生的事情吧。周很理解她愣住的缘由,便朝她微微一笑。\\

「虽然我觉得,或许会有点孩子气吧」\\

把房间装饰得很豪华,准备好生日席,制作好一整个蛋糕,插上很多蜡烛,唱生日歌——这种庆祝的方式不像是高中生,而是跟儿时一模一样。不过周觉得就得这样才好,所以才为了此时此刻做足了准备。\\

「但是,我们依旧还是孩子,我觉得像这样也挺不错的。小时候,我被这么庆祝过生日,有过非常开心的回忆,小时候的这些回忆,也一直留在我的记忆里」\\

虽说小时候的记忆有些模糊,但周至今还记在心里。\\

父母会将房间按照周的喜好进行装饰,让周和喜爱的布偶、玩具坐在一起,在周喜欢的蛋糕上插上蜡烛,并将吹灭蜡烛的特权交予周。\\

他们将数不尽的祝福与爱,都慷慨地给予了周。\\

那些儿时回忆至今仍存在于周的内心深处,给了他被爱着的自信,正因如此,周遇到了伤心事之后才能够克服。\\

「或许有点强人所难,但我想与你分享我的开心的事,我觉得你在小时候,肯定梦想过这样一场生日」\\

生日庆祝是许多孩子或多或少经历过、期待着的事。\\

尽管过于以自我为中心地考虑事情并不好,但对于小时候的周来说,没有什么日子能比一年一次的生日更令人兴奋了。\\

「我觉得,你也说不定会憧憬这样的生日庆祝」\\

周也在反省自己擅自想象了那些情景,但从真昼的反应来看,他确信自己没想错。\\

「所以,我希望真昼也能体验一下。虽然或许是我的私心吧」\\

6寸的蛋糕插上17根蜡烛或许是有些过火,但考虑到真昼到现在为止有多少没有过、没能过的生日,就一点都不觉得多了。\\

烛火被供暖的风吹得变形,与此同时真昼的眼睛有一道光芒无声地产生、滴落。\\

僵硬的状态刚刚消失,下一刻真昼的面庞便突然扭曲,滴落下许多清澈的泪滴,倒是弄得周慌乱起来。\\

「不、不喜欢吗?」

「怎么会,不喜欢,那个,我很开心,心里,满满的,就是说,我,像这样,真的好,吗」\\

夹杂着呜咽,一口气经历了迄今为止没能体验过的事物的真昼,费尽全力用语言传达出自己的心情。

真昼以不加掩饰的真实姿态啜泣着,同时拼命组织着语言。周在她身侧蹲下,温柔地包裹住真昼颤抖着的手掌,也有点想哭。\\

「如果你开心的话,就太好了。我拼命想了很久,想着怎样才能让你开心。要在你的生日做些什么,我考虑了很多,也找了许多人商量,看来一切都是值得的」\\

这回不同于第一次祝贺真昼的生日。\\

树和千岁这回知道对方是真昼而帮助了周,周的父母也为了真昼而陪他商量,还有以优太和彩香为首的朋友们、打工的前辈和店长都给予了援手。\\

「这不仅仅是我一个人的努力,还有很多人的帮助。你看,有那么多人来帮助我,有那么多人为真昼的生日感到高兴」

「……嗯」

「来,趁着蜡烛还没有融化,把它们吹灭吧。这可是寿星的特权哦?」\\

周用手帕擦拭掉真昼湿漉漉的泪水后,特意对她顽皮地一笑道,真昼微微哭湿了的面颊也被逗得放松了下来。\\

或许是在心里决意不能一直哭,真昼尽管眼中依旧湿润,但恢复了耀眼的光彩,她开心地眯起了眼睛,腼腆地喜笑颜开。

接着真昼就从沙发上下来,跪坐在地上,朝着蜡烛在昏暗中温和而有力地不停燃烧的光辉,轻轻地吹了一口气。\\

顽强的烛火抵抗着真昼温和的吐息,仅仅是晃了晃,真昼在挑战过几次后向周投去伤脑筋的视线。\\

那种,初次尝试时的困惑模样,比什么都要更惹人喜爱。\\

不熟练的事,真昼正思索着该怎么办才好,周温柔地告诉她「好歹是有17根,得更用力吹才行」,并抚摸她的后背为她加油,但他始终保持观望的姿态,吹还是要真昼自己来吹的。\\

因为,吹灭生日蛋糕上蜡烛的体验,也是由生日的主角来执行的仪式。\\

周在真昼背后的支持让她下定决心,真昼深吸了一口气,像要将种种忧愁统统吹散似的,往蜡烛上一吹。

随着蜡烛一根根熄灭,房间内的亮度也逐渐减弱,在不顾这些吹灭着烛火的真昼吹灭最后一根的瞬间,周打开了客厅的灯。

即便周选了较低的亮度,由暗转明也让眼球很受刺激,不过蛋糕的全貌却是一下子变得鲜明。\\

蛋糕选的是简单的草莓奶油蛋糕,以草莓和鲜奶油为主。周还是一边心里回想着自己第一次送给真昼的那块草莓奶油蛋糕,一边做的装饰。\\

但要说是否准确地复现了出来,答案是否定的。\\

除开装饰的美感不同之外,蛋糕中央摆放着一块巧克力板,装饰着周笨拙地手写的「生日快乐」几个字,还有一根根蜡烛填补着巧克力板和草莓间的空隙,可以说是不同的作品了。\\

尽管如此,周很想让那时的回忆能成为今日回忆的一部分,因而选择了这个蛋糕。\\

「一亮起来,这个蛋糕就更显得有些挫了啊……啊不是,味道是经过店长监制的,出不了岔子。我练过了哦?」

「哎,什么时候?在哪……?」

「打工的地方,帮店长试做蛋糕时,学了一些」\\

之前周帮店长试吃蛋糕的试作品时,顺势就拜托了店长,而店长比周预料的还要善意,很爽快地接受了周的请求,甚至让提出请求的周都感到了困惑。\\

「因此我能在回家时间不晚且不被你发现的情况下做出来,真的非常感谢店长」\\

尽管很忙,丝卷还是会抽出时间指导周,她嘴上说「会做的人多了,工作也会更轻松吧」,但这显然是为了不让周放在心上,所以他心里还是很过意不去。\\

在如此亲切的丝卷的指导下,周学会了号称「绝对不会失败的海绵蛋糕制作方法」的食谱。\\

理所当然地,制作蛋糕必须要能够复现,并且用家里的器具要能做出和刚才一模一样的结果,因此丝卷很细致地讲解了步骤等等的注意事项。\\

多亏了这样,周才能将蛋糕的海绵部分烤制得恰到好处。\\

虽然周还是不擅长抹奶油,但这不是一时半会儿能掌握的事情,因而他把技术练到拿得出手的水准,便投入到实战了。

至少做到了让真昼高兴。周松了口气,并看着为真昼制作的蛋糕。\\

「倒也有代价就是了」

「代、代价……为了我,没必要吧」

「说是要我告诉她有没有讨到你的欢心……怎么样,有没有?」

这是打从一开始就没打算要求什么代价。为了实现文华的请求,也为了得到自己在意的评价,周悄悄探头看向真昼的脸,发现她又是泫然欲泣地点了点头。\\

「我高兴得,无以言表。真的,非常感谢」\\

快要哭出来的真昼说道,脸上却浮现出感到开心的微笑。周松了口气同时在内心感谢文华,之后开始准备碟子和蛋糕刀。真昼微微蹙眉,抬头看着周。\\

「……我,这么幸福,真的好嘛?」

「不——行」\\

周出于本能地停顿后真昼呆住了,但周很快就意识到自己的话没说清楚,慌忙继续说了下去。\\

「啊,为了避免误会,我先解释下,因为你以为的『这么幸福』还远远不够。所以,不行。我要你让你更加更加幸福,别只是这样就满足了」

「……嗯」\\

误会完全解开,真昼微红着脸点了点头,对此周也放下心来,他做完吃蛋糕的准备后,重新坐到了真昼旁边的地板上。\\

「那就开始切吧。切得太多没把握好平衡的话,还请多多包涵」\\

虽然周下刀的时候考虑了布局的美观性,但由于巧克力和草莓巧妙地阻挡,切的时候也并非完全顺心如意。\\

周做出了努力,但不行的事情就是不行,而且还有蜡烛密集的区域,从制作者的角度来说,改良空间果然还是很大的。他将这个问题记在脑海中,为下次的机会做打算。\\

但这并不是现在要解决的问题,所以周暂时把这个问题放进心底,皱着眉头想着怎么切蛋糕。\\

「还是把这些先拿掉吧」稍稍烦恼过后,周得出结论并拔出了蜡烛。看真昼有些落寞的样子,周小心地将蜡烛放到另一个盘子里,成功地获得了真昼的心安。\\

「这块给你」\\

6寸的蛋糕,四等分就差不多了,周将蛋糕按易于食用又易于切割的切法切开,把那块带着巧克力板的、给最重要的主角食用的那碟蛋糕递给了真昼。\\

真昼郑重地接过盘子,她就像看到了什么宝物一样,嘴角翘起,眼中闪烁着光芒。周觉得真昼这样有点夸张,但这也佐证了她有多么开心,他有些心痒痒地把叉子递给了真昼。\\

「请用」

「我、我开动了」\\

真昼有些犹豫,可能是因为,如果不把以绝妙的平衡放置在上面的巧克力弄下来,是没法吃的。\\

一时间手足无措不知如何是好的真昼感到对不起地将巧克力从白色的舞台上拿了下来。将那底座切成一口的大小花了相当长的时间,而吃下去只是一瞬间的事。\\

一张一合间,红白两色对比艳丽的蛋糕块,被小小的嘴唇吞没了。\\

那双焦糖色的眼睛睁得大大的,随后又慢吞吞地轻轻眯起。

看到真昼那客气的表情中出现了柔和的惬意之色,周确信这半个月左右的努力有了回报。\\

「怎么样?」

「好吃」\\

真昼仔细地咀嚼蛋糕并咽了下去,然后腼腆地点了点头。周终于可以不用紧张了,他把没有吐干净的沉重空气从肺里赶出去,再以轻松的心情重新吸入了新鲜空气。\\

「太好了,真昼在我生日时为我做了蛋糕,所以我也想还给你一个」\\

将心比心,周想让真昼体会到与自己相同的喜悦。

为此他不得不全力以赴,但毕竟还是外行,由于时间有限,比起漫无目的地自学,他选择了找内行求教的方式。\\

结果很成功,他切实感受到,遇到困难是应该依靠别人的。\\

「但以我的手艺,无论如何也是达不到真昼的水准的。还是得靠别人啊」

「……是店长教你的呢」

「是啊,是我拜托的她。我说想为女朋友制作生日蛋糕,然后她直接笑开了花,怎么说呢,是她会做出的表情。不过我非常感谢她」\\

「越是这种简单的东西,食材的味道和技能越会直接关系到整体的味道」丝卷说着,把从基础开始的各种技术都一一教给了会做点菜但在制作点心方面完全是个门外汉的周。\\

丝卷给周提供了各方面的支持:基于不同的打蛋泡方式海绵纹理会完全不同,于是她让周品尝和比较;奶油的脂肪含量不同,打泡时间和方法会随之变化,于是她让周去触摸每一种材料;她还告诉了周烘焙用品店的位置,推荐去那里买要用的原料。\\

既雇佣了周作为临时工,又帮了周这么多忙,对于周而言,丝卷已是他再怎么感谢都不为过的对象。\\

「周君的生日我也受到了她的照顾,所以想找个机会亲自去道谢……记得周君不愿意来着?」

「不、不是说不愿意。是说最好等我待客能有点门道之后再来……要是太不像样的话,还是很难为情的」\\

过了一个多月,周自信已经熟练了工作本身,但要说能不能达到拿出来给真昼看的熟练度,周也有自信会全力给出否定的回答。\\

被恋人或朋友看到自己工作的地方,大部分人都会感到尴尬,而展示出自己接待客人的样子,就更是火上浇油。\\

周为了自己的虚荣,想着如果要向真昼展示的话,希望让她看到自己完美的表现,所以他很抱歉要让真昼等待……这点终究是不能让步。\\

想要尽可能地在女朋友眼里展示帅气的自己,周在这方面有他的执着,哪怕软弱和丢脸的地方已经被看了个遍,也一样如此。\\

「我觉得周君慌慌张张的也很可爱啊」

「这样可不行……我想帅气点」

「嗯。那么,我就等着咯」\\

真昼似乎同意了周让她再等等的做法,不过她明明被要求等着,却笑得很开心。周在心里对真昼的宽容感到拜服,自己也把蛋糕送入嘴里。\\

因为得到了丝卷的全面指导,蛋糕本身做得很好吃。\\

涂上糖浆的海绵部分不太重,口感温润细腻,和夹着的草莓一起在口中柔软地化开。归功于调整得没有过甜的奶油,草莓的酸甜也可以品味得到。\\

真昼虽然喜欢精致的和漂亮的东西,但最终总会回归到基本的形式,这次应该是再现出了真昼喜欢的口味吧。\\

周偷瞄了一眼真昼,她正珍惜地品味着,放松的脸颊、低垂的眉梢,看上去要比平时吃甜点的时候要惬意得多。\\

「真好吃」

「……太好了」\\

真昼喜欢,对于周来说就是巨大的成功。

这样的话,时不时为真昼做做蛋糕也不错。周边在心里想着这些事,边将真昼的微笑当作蛋糕的配菜,慢慢品尝着甜度比较低的甜蛋糕。\\

瞧见真昼吃完的时候,周去了一趟卧室。\\

包装好的礼物是准备在另一间屋子里的。若是直接放在亮堂堂的房间里,那实在是明显得不能再明显了。而真昼则是视线一直追随着周的身影,像是由于周不在而感到寂寞了,等看到周回来时手上的东西,她连连使劲地眨眼。\\

今天看到了好多次真昼那样的表情,周不禁感到高兴。他跪到沙发上困惑着的真昼脚边,把真昼的手轻轻扶到她并排着的大腿上,再把东西放了上去。\\

「生日礼物。来之不易的体验固然很重要,但我也想给你一份能作为物品留存下来的东西」\\

装饰和晚餐是开胃菜,而蛋糕、这份礼物,以及最后那样才是重头戏。\\

虽说是礼物,但周并不能送十分贵重的东西。高中生能送的,且还是真昼会喜欢的东西很有限,这是从那为数不多的选项中挑选出来的。\\

尽管如此,周觉得心意和选择的理由是很重要的,他以他自己的方式苦恼了一番,最后决定把这个送给真昼。\\

「坦白说,如果是需要的东西,你就会毫不犹豫地买下,贵重的东西会觉得过意不去而不开心吧。所以我十分苦恼」\\

之前也和千岁说过,真昼不想要什么东西并且没有什么物欲,而且对于有需要的东西,既然有需要就会毫不犹豫地买下,以送礼对象而言,属于非常困难的那一类。\\

显然,无论周送真昼什么她都会感到高兴,正因如此周才不知道送什么好,最后,周想象着自己至今以来眼中的真昼,以此选择了礼物。\\

「打开看看吧」周温柔地轻声说道,已经从僵硬中解放出来的真昼还有些犹疑地偷瞄着他,不禁让他笑了起来。\\

真昼对周的态度有点不悦,但还是一副诚惶诚恐的模样,小心地解开盒子上的丝带。至于她的手指有点颤抖这件事,还是不要说出来的好。\\

提到礼物,很多人不经意地就会想到亮着光泽的红缎带。真昼将它从束缚盒子的任务中解放出来之后,又非常小心地撕下包装纸,存放礼物的盒子本体这才终于亮相。\\

真昼再次向周确认,于是周笑着告诉她可以打开。真昼屏住呼吸轻轻地掀起了盒盖。\\

从里面拿出来的是缓冲材料,以及一个真昼拿双手正好能捧住的另一个盒子。\\

不,说是盒子并不贴切。礼物自不可能是像套娃一样送一个平平无奇的盒子。\\

现在真昼手里拿着的,是一个木制的古风收纳盒,其设计典雅,配色细腻,上面还雕刻着真昼喜欢的花朵,与其说是可爱,不如说是一个美丽又有格调的物件。\\

「呃,应该会是个你喜欢的收纳盒」\\

按理说,恋人的生日送首饰、化妆品之类的东西或许更为稳妥,但在听了各种各样人的意见后,周决定这么做。\\

这也有着真昼性格上的原因,真昼非常爱惜物品,从别人手里拿到的东西更是往往会珍重地保管。

听说真昼会把周送的每一样东西都保养好并收藏起来,防止其老化磨损,周甚至都对她的一丝不苟感到敬佩了。\\

正因如此,这回周想给那些为真昼所珍惜的,成为了真昼回忆的东西一个能好好地休息的地方。\\

「真昼,送给你的东西,你全部都很珍惜吧。要是有个能收起来的地方,感觉应该还挺不错的。啊,我想你自己本来也有地方存放的,不是说一定要用这个的啦!」\\

虽然送过去了,但并不是非用不可,这一点很重要,周在好好地说清楚后,轻咳了一声。\\

「我是觉得,今后,如果有能把我送的东西收起来的地方,应该会很不错吧」\\

直接面对面地说出来有些难为情,很难把话说出口,但他还是慢慢地,将心中的愿望说了出来。\\

「我希望能有那么一天,只靠我送的东西就能装满那个盒子……抱歉,都是我一厢情愿的想法」

「……不是一厢情愿」\\

周给出的理由里包含着自己的愿望,脸上露出了对自己的任性感到自嘲的笑容,而真昼却俯首摇头。

她的声音在微微颤抖着,啪嗒,大颗宝石从她眼中滑落,砸在她放在盒子的手背上,弹开。\\

无需言语,也看得出那泪水不是由悲伤而生。\\

「……周君想让我哭多少次啊?」

「如果是喜悦的泪水,多少次也不嫌多」

「……真是的」\\

微微闹着别扭、撒娇的声音,正是真昼对周信赖的证明。

抬起头的真昼向周展示着她爽朗而甜美的笑颜,一副心满意足的样子,连那因过度使用泪腺而微微红肿的眼角,也不再让人心疼了。\\

「回去之后,得好好收拾呢。手镯啊,发夹啊之类的。我的,宝物。嘻嘻,每次打开时候我都会变得很幸福」\\

真昼像处理易碎品一样,轻轻地打开了盒盖,盒子内没有隔板,简单地准备了三层宽阔的空间。

盒子虽然不是很大,但首饰或者厚度较薄的东西还是可以放下不少的。真昼声音很雀跃,似乎是难掩自己的激动。\\

「希望今后真昼的宝物能变得满满的啊」

「除了可以放首饰的空间外似乎还有很大的空间,其他重要的东西也得放进去呢」

「例如?」

「嘻嘻,充满了和周君一起度过的回忆的物品,都是些微不足道的东西,所以保密」\\

作为礼物送出去的东西,周都还记得很清楚,但听真昼的口吻,不是礼物的东西她也很珍惜着。

周对此没什么头绪,心里微微感到自责,真昼察觉到了周的心思,但还是不在意地对周笑了笑。\\

「这也要保密么?」

「嗯……大概真的是周君无意中送给我的东西,或者是在一起的时候得到的东西哦。所以不记得也是理所当然的,就留给你当作个期待吧。等好好装满了的时候,我再给你看,让你怀念一下,原来还发生过那样的事」\\

今后周也一定会和真昼一起创造出很多回忆吧。不,是一定要去创造。\\

倘若有一天,能用满满的幸福去装满这个箱子就好了啊。周用眼神向真昼诉说,而怀有相同心情的真昼平静地微笑着,双方彼此心怀对未来的期待,翘起了嘴角。
