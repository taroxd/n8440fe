% \subsection{164 天使様とかき氷}
\subsection{天使大人与刨冰}

% 「次はかき氷食べよー!」\\
「接下来去吃刨冰吧~! 」\\

% ぶらつくのを再開した四人だが、千歳の発言に再度足を止める事になった。かき氷の屋台は通りすぎた。恐らく進んだ先にもまだあるのだろうが、どこにあるのかは分からないので少し戻った方が早い。\\
又开始闲逛起来的四人,听到千岁的话后再次停下了脚步。刨冰摊子早就走过了。虽然前面应该也还有摊子,但毕竟不知道摊位在哪,还是稍微往回走一点来得快。\\

% 「どんな胃袋してるんだよほんと……」
「你这胃啥样的啊这么能装……」

% 「こんな胃袋だよー」\\
「就这样咯~」\\

% ぽんとお腹を叩いているが、真昼に負けず劣らずの細さが分かるだけだ。このお腹に焼そばと唐揚げとイカ焼きが格納されているのだから驚きである。
千岁嘭嘭地拍了拍自己的肚子,但能看见的只有那与真昼难分伯仲的柳腰。这个尺寸的肚子居然能装下炒面、炸鸡块再加上烤鱿鱼,实在是惊人。

% どこに仕舞われてるんだ……と真顔になってお腹を見ていたら、真昼も同じ事を思ったのか苦笑を浮かべていた。\\
周想着到底是装哪去了,一脸严肃的观察起了千岁的肚子,而真昼则是想到了同样的事情一般,露出了苦笑。\\

% 「千歳さん太りませんよね。すごくスレンダーで羨ましいです」
「千岁是不会胖的类型呢。这么苗条真是令人羡慕啊」

% 「健康的な細さだよな。引き締まってるし」
「是健康的苗条呢。感觉也挺有肌肉的」

% 「えへー、もっと褒めたまえ」
「嘿嘿~,来,再多夸夸本小姐」

% 「ほんとちぃは細いんだよなあ……抱っこした時とかすごく細いし」\\
「小千是真的很苗条呢……抱着的时候感觉是真的细啊」\\

% よくくっついている樹だから、千歳の細さもよく分かっているのだろう。樹は特段太いという訳ではなく中肉中背なのに、くっついていると千歳の細さが目立つのだからかなり細い。
经常跟千岁搂搂抱抱的树,想必是十分清楚千岁这苗条身材的。虽然树身材中等,实际上并不胖,但是两个人凑在一起的时候千岁的苗条感还是十分显眼,由此可见千岁的身材。

% それでいて筋肉がうっすらと浮かびつつごつくない絶妙な体つきをしているのだから、千歳の努力が窺える。\\
不仅如此,千岁身上还有淡淡的肌肉纹路,而且分寸十分完美,不会产生粗鲁感,足以窥见千岁的努力。\\

% 「よく食べるのに太らないんだよなあ」
「吃了这么多居然还不会胖呢」

% 「代謝いいもん」
「代谢好嘛」

% 「まあそれにちぃは体質的にも太りにくいからなあ。その分他のところにつかないんだけど」
「而且小千也是不容易胖的体质啊。虽说让某些地方也大不起来了呢」

% 「……いっくん、こっちにオイデ」\\
「……阿树,你给我过~来~下~」\\

% 口を滑らせたな、と一瞬で悟ったのは、千歳がにこやかな笑顔で抑揚のない声をあげたからだろう。
听见千岁那满脸笑容发出的毫无抑扬的声音,树才发觉自己说错话了。

% 千歳が地味に気にしている部位の事に触れたのだから、当然怒る。むしろ彼氏だからこそ余計に怒っている気がする。\\
毕竟是说了千岁超在意的部位的事情,想当然是要被发火的。而且感觉因为是男朋友,火气更大了。\\

% 「ごめん失言だったから脛蹴るのやめてください」
「抱歉我说错话了求你不要踢我的小腿啊」

% 「毎度言ってるけどいっくんは一言余計だなー? 向こうで仲良く話そ?」\\
「每次都跟你这么说,结果你每次都大嘴巴啊~?聊得挺嗨嘛你?」\\

% にこにこと笑いながら樹の腕にくっついてひっぱる千歳に、御愁傷様と口には出さず樹に送っておく。\\
看着千岁一边一脸笑容一边拧着树的手臂,周只得默默地对树表示节哀。\\

% 「雉も鳴かずば撃たれまい……」
「不作死就不会死呢……」

% 「何か言った?」
「你说了啥?」

% 「いーやなんでも」\\
「嗯,没什么」\\

% こちらに飛び火するのは勘弁なのでさらっと否定して、隣で困っている真昼に樹の救援要請をスルーすべくわざとらしく微笑みかける。\\
周可不想引火上身,随口糊弄了过去,然后故意一般向一旁正对树的救援请求不知所措的真昼微微一笑,让她不要插手。\\

% 「真昼はかき氷何食べる?」
「真昼想吃什么口味的刨冰?」

% 「え……い、いちご……?」
「呃……草,草莓……?」

% 「ん。じゃあ買いに行こうか。千歳ー、先にかき氷買ってくるからそこで仲良くしてろー」
「嗯。那我就跑一趟去买来吧。千岁~,我先去买趟刨冰,你俩就在那好好交流吧~」

% 「はーい」\\
「好~嘞」\\

% 樹を威圧しつつも笑顔で振り返って返事する千歳に小さく笑って、周は真昼の手を引いて一度道を戻る事にした。\\
千岁一边用威压镇住树,一边一脸笑容扭头回答。周对千岁微微一笑,拉起真昼的手回到了街上。\\

% \\


% 二人がかき氷を買って戻ってきても、千歳のお説教は終わっていなかった。
等到两人买好了刨冰回来了,千岁的说教还是没有结束。

% 道から少し外れたところで仲良く話し合いをしている二人を遠目に見て肩を竦めた周は、周の腕にくっつきながらなんとも言えなそうに苦笑いを浮かべている真昼を見る。\\
远远望着在离大街稍稍有点距离的地方进行着亲切交谈的两人,周耸了耸肩,然后转头看向正贴着自己的手臂,露出着无语笑容的真昼。\\

% 「……まだやってるんだよなあ」
「……看来还没有结束呢」

% 「仲いいですよねえ」
「关系挺不错的呢」

% 「まああいつらなりのいちゃつき方だよなあ。若干千歳が怒ってるけど」
「嘛大概是他俩特别的亲热方式吧。虽说千岁有点生气」

% 「あ、あはは……」\\
「啊,啊哈哈……」\\

% 本気で怒っている訳ではないのも分かっているので止めたりはせず、手にしていたかき氷のカップを真昼に手渡す。\\
周也知道,千岁这并不是真的气到心里去了,因此也没有插手两人。周把手上的刨冰杯子递给了真昼。\\

% 「ほら真昼」
「给,真昼」

% 「ありがとうございます。周くんは……なんか渋いですね」
「谢谢。周君的……稍稍有点涩呢」

% 「ほんとは宇治金時がいいんだけど流石に屋台にはなかった」\\
「其实我是想要宇治金时的,但摊子上再怎么也没那种东西啊」\\
% 宇治金时(うじきんとき)是一种日本的传统刨冰,以日式抹茶加砂糖及水煮成糖浆,淋在刨冰上,旁边加上以砂糖熬煮的红豆,制成色彩分明的甜品。“宇治”原指宇治抹茶,即以京都府宇治市周边地区所生产的日本抺茶;“金时”原指名为“金时豆”的红豆,后来演变为指以砂糖煮调过的红豆。

% ちなみに周は抹茶を選んだ。
顺带一提周买的是抹茶味的。

% あったならば宇治金時にしたのだが、流石に屋台にあんこと白玉を求めるのはきついものがあるので致し方ない妥協である。\\
本来有的话周是想买宇治金时的,但要小摊准备豆沙跟糯米团子就实在有些强人所难,没办法周只好买抹茶味的凑合了。\\

% 「周くんそういう甘いのは食べるのですね。あんまり食べようとしないですけど」
「周君居然会吃那么甜的东西呢。我记得你是不怎么吃甜味的东西来着」

% 「別に甘いもの嫌いじゃないぞ、好んで食べないだけで。あんこは好き。特につぶあん」\\
「我倒也不是不喜欢吃甜的东西啦,只不过是不好这口罢了。不过喜欢吃豆沙。特别是红豆沙」\\

% 甘いものは自分から食べないだけで出されたら食べる。自ら食べようとするのはカスタード系のものくらいだろう。それもあまり食べないので、好きというイメージはまずつかない。
甜口的东西周虽然不会自己弄来,但有现成的话还是会吃的。要周自己选的话,周更加青睐奶蛋类的食物。不过这些周也不怎么吃,所以倒也没有特别喜欢的印象。

% あんこが好きなのは抹茶や緑茶に合うからである。苦いものに甘いものは互いを引き立てあってとても合うので、実は好きだったりする。\\
周喜欢的是豆沙和抹茶或者绿茶混合的感觉。苦味的东西与甜味的东西互相衬托,简直是天作之合,十分符合周的口味。\\

% 「そうなんですか。……餡を炊くのは大変ですから何か作るのも苦労しますね」
「这样啊。……不过做豆沙做起来很麻烦的,想要做点什么也很累人呢」

% 「あんこを炊くところから考え始める真昼がすげえよ。市販のやつでいいだろ……」\\
「真昼你会去考虑自己做豆沙这也太强了吧。一般来说去买点不就好了么……」\\

% 普通小豆を炊いていくところから始めようという発想はないだろう。市販でもあんこは袋づめされて売られているのだから、手間隙と時間を考えたらそちらを選ぶ人間の方が圧倒的に多い。\\
一般人都不会产生从红豆开始做的想法吧。反正也有那种一袋一袋的豆沙卖,考虑到麻烦程度和时间,绝大多数人都会选择去买吧。

% ただ、真昼は手作りの方が先に来るようだ。\\
不过真昼似乎是先从自己做开始考虑的。\\

% 「好きな人には美味しいものを食べさせたい心なのです。市販のだと中々甘さは調整出来ませんし、粒の感触が残らないのが多いので」\\
「因为我想让喜欢的人吃到好吃的东西嘛。毕竟外面卖的那些不好控制甜味,而且很多也没有保留住颗粒的口感」\\

% 周くんには美味しそうに食べてほしいなんて健気な事を言って微笑む真昼に、周も申し訳なさやら愛されてる実感に幸福を感じるやらで、頬が緩めばいいのか引き締まればいいのか分からなかった。\\
听见真昼一边说着想让自己吃上美味的东西这般令人感动的话语,一边露出微笑,周也因那亏欠感与被爱着而产生的幸福感,而不知是该放下脸来还是绷紧脸来了。\\

% 「……じゃあ抹茶プリンにあんこ添えたやつたべたい。あとどらやき」
「……那我就预订个加了豆沙的抹茶布丁吧。再加上铜锣烧」

% 「ふふ、はぁい。お任せあれですよ」\\
「呼呼,知道啦~。交给我吧」\\

% 周くんのためなら何でも作りますよ、と真昼が言えば過言でもなさそうな言葉を口にしてかき氷を食べた真昼に、周は何とも言えない照れ臭さを感じて誤魔化すように自分のかき氷を口に運んだ。
真昼一边说着「如果是给周吃的话,什么我都会努力做的」这般对她来说毫不夸张的话,一边吃着刨冰;而听见这话的周,则是沉浸在一股难以言表的害臊感中,似是要掩饰这心情一般,也开始吃起了自己的刨冰。
