\subsection{不想见到的过去的象征}

刚才眼前的他说了什么呢。

如果没听错的话,这个少年是看向真昼,然后——\\

大脑慢了一拍理解了话中的意思后,周把视线转向真昼,只见她睁大眼睛盯着眼前的少年,然后又眯起了眼睛。\\

她的眼神中隐约流露出胆怯与厌恶,显得十分冰冷。\\

即使是身为局外人的周,也能看出她并不是直接对少年产生这种情绪,而是透过他看向别人。\\

周不知道少年是否理解了这一点,但他似乎明白真昼脸上没有任何善意,因而不安地绷紧了脸。\\

「……请问您是哪位?」\\

除了眼神以外,真昼脸上失去了所有感情,用缺乏抑扬顿挫的声音问道。

平时待人非常亲切,对任何人都会回以温和微笑的真昼,如今却露出了明确的拒绝表情。那冷淡的态度让人明显感受到「别和我扯上关系」的意思,拒人于千里之外。\\

「我想我们应该是初次见面」\\

真昼淡淡地说道。

她的声音听起来很生硬,冰冷得仿佛能刺痛人,甚至让人感到一丝疼痛。她慢慢地这么说道。\\

任谁听了这番话,都会觉得态度中毫无善意。少年虽然有些畏缩,但视线依然直勾勾地盯着真昼,然后又鼓起勇气似地在身体旁边握紧了拳头。\\

「你是椎名真昼,对吧?」

「……是的」

「我是,那个……你母亲的,相关人士」\\

少年烦恼许久才挤出声音,那嘶哑的嗓音让真昼的肩膀颤抖了一下。

站在旁边的周感受到她的颤抖甚至传到了指尖,但现在的插嘴似乎会逾越一个男朋友的本分,因此他只能轻轻握住她那无助的手。\\

「我今天来拜访,是想和你谈谈」

「……有什么事呢?」

「是关于你母亲的事。在这里就只是谈这个」\\

少年瞥了这边一眼,大概是因为周这个外人在场的缘故。

他当然有注意到周的存在,只是故意无视了而已。周本来还在担心要是对方说「你不要插嘴家里的事」的话该怎么办,不过少年似乎没有赶走他的意思。

只是,他有些为难地看向这边,或许是因为他也犹豫着不知道该怎么对待周。\\

以立场来说,周应该要回避才对,可若是让真昼和少年单独交谈,也不知道两人能否冷静。真昼现在表面上看起来冷淡平静,内心却可能已经涌起巨浪而非涟漪了。\\

「很不巧,我对母亲的事情没什么兴趣,也不打算和她扯上关系。请你回去」

「拜托了,请听我说……在你听我说之前,我不会回去,也不能回去」

「不能回去?」

「……那个,因为我是从家里跑出来,一路到这里的」\\

少年的目光非常尴尬地游移着,对真昼投以求助的眼神。

真昼用前所未有的冰冷视线回望那依赖的目光。\\

「回你家去不就好了?我可以帮你出出租车钱」

「在你听我说之前,我不会回去!绝对不会!」

「我不想听。请回」\\

真昼断然拒绝,冷冷地盯着还不肯放弃的少年。\\

「你知道我母亲是怎么对待我的吗?」

「……我知道,或者应该说,我能想象」\\

听到他的肯定回答,真昼惊讶地眨了眨眼。

然后,她像是再也忍不住似的,有些不甘心地抿紧嘴唇,微微皱起眉头。

她的眼神仿佛在控诉着「为什么?」,让周也咬紧了嘴唇。\\

「那你应该知道我不喜欢别人提起母亲的事情吧?」

「……我知道。可是,我也有我的苦衷」

「我们根本是在对牛弹琴。再说,我没有义务听你说话,只要穿过这个入口,我们就没有任何关系了。你不能进来这里,这样会构成非法入侵。到时候我会报警,而且你这个年纪还在外面游荡,也会被警察带去辅导吧。我觉得你还是乖乖回去比较好」

「就算这样——」\\

从认识真昼的人的角度来看,她那冷酷无情的态度令人难以置信,甚至可以说是固执,完全没有让步的意思。然而,与强硬的言辞相反,她的表情变得阴沉而软弱。

这是只有经常观察真昼的周才能察觉的变化,而真昼也被逼到了绝境。\\

从少年的话来看,他应该是真昼母亲的儿子吧。只不过,他没有自称是儿子,而是说相关人士,这一点很可疑。\\

无论他是不是儿子,对真昼来说,那都是不愿想起、不愿思考的事情。

证据就是,被周牵着的手正在发冷颤抖。如果周没有碰触到她,她可能已经离开了这个地方。真昼对现在的状况表现出强烈的抗拒。\\

「真昼,可以打扰一下吗?」\\

这样下去对真昼的心理健康不好,而且双方都不肯让步的话,事情就无法解决。现在是回家时间,也有可能被别人看见。真昼很怕别人知道她家之间的敏感问题,就算这栋公寓里没有同校的学生,最好还是避免引人注目。

正因为周是局外人,所以才应该介入其中,打破僵局。\\

「我觉得就算你继续拒绝,问答也不会结束。可以让我代替来谈一下吗?虽然你可能会觉得我干嘛多管闲事」\\

虽然不至于妨碍到谈话,但周确实被少年用「你谁啊」的眼神盯着。毕竟他的确跟少年没有关系,这也是没办法的事。

为了尽量不刺激到对方的戒心,周用柔和的语气询问。他或许是愿意听他说话了,于是用还留有些许稚气的脸庞看向这边。\\

「可以先告诉我你的名字吗?」

「……慧」\\

小声报上名字的少年——慧怯生生地抬头看向周。

这是周第一次正面与他对视,但他和真昼并不相似。要说的话,大概只有气质比较内敛这一点相似,长相本身并不像。\\

麻烦事又增加了。周在心里感到为难,同时凝视着慧的眼睛。\\

「你叫慧啊。你说是有真昼母亲的事情想说?」

「嗯」

「那是无论如何都要今天告诉她的事吗?」

「咦?」

「除了你内心的焦虑之外,那是非得在今天说不可的事情吗?」\\

慧对周的问题感到困惑,但周只是代替真昼来让事情和平收场的,所以并不打算对他有多余的顾虑。\\

「我能理解你很着急。可是,你认为真昼突然听到你提起自己母亲的事情,她会是什么心情?从你的样子来看,你应该已经察觉到真昼以前的处境了。所以,她现在非常不安,也很动摇。你换位思考一下,如果你是她,能够在那种状态下坦率、冷静地听你说吗?」

「……不会」\\

看到慧缓缓点头,周松了口气,但没有表现出来。

如果他坚持主张自己,什么都不管的话,周也打算毫不留情地拒绝他,不过看来不会发生那种情况。\\

「我认为你最优先的事情是让她听你说话。有非得现在说不可的条件吗?如果没有的话,我觉得改天再谈会更有机会让她听你说。至少会比现在更有余力接受你的存在」\\

看到真昼现在的固执态度,慧应该也明白她不是处于愿意听自己说话的状态。

真昼紧抿着嘴唇,握着周的手。周知道那只手正在颤抖,于是温柔地回握那只小巧的手,试图安抚并让她安心。\\

「我打算把真昼放在比你更优先的位置,所以如果真昼不愿意,我无论如何也不会让你接近她。不管对我还是对真昼来说,你都是外人。我不能只听你一个人的愿望,听了也没有益处」\\

从某种意义上来说,这个与真昼没有交集的人象征着她过去的痛苦。

虽然他本身没有任何过错,但如果真昼表现出拒绝的态度,周想尽可能优先考虑她的心情。\\

「不过,从你的样子来看,我知道你非常迫切。所以,你能不能先回去,改天约好时间再来拜访?能不能约好时间要看真昼的意思,所以我不能保证绝对能实现你的愿望」\\

周能感受到他无论如何都想说出口的意志,但考虑到真昼突然承受了这么大的负担,周实在不想就这样直接让他们交谈。\\

另一方面,周也知道如果把人赶回去,真昼心里就会一直有块疙瘩,让她烦恼不已。而且慧也不见得会放弃,要是每天都得害怕他什么时候会再来找上门,对真昼的身心也不好。\\

既然如此,不如先约好时间,做好心理准备后再面对,这样对精神来说更好。\\

不过,如果这样也会让真昼感到痛苦的话,周打算彻底隔绝他。\\

周观察着慧的反应。只见他垂下视线,尴尬地抓着自己的手腕缩起身体。\\

「……我今天……我妈不在家,所以我说要去住别人家一天,蒙混了过去。跟朋友也对好了口供……我只能趁今天这个机会……」\\

看来很难改变他的想法了。\\

既然如此,周当然会站在真昼这边,但他想知道真昼本人是怎么想的,于是将视线转向身旁。\\

「怎么办?虽然对这个孩子不好意思,但我认为也可以当作没这回事直接回家。以你的心情为优先比较好,我是站在你这边的」\\

周温柔地表示真昼的心情最重要,真昼随即皱起眉头,默默垂下视线。\\

「……我不想让他,进我家」\\

真昼烦恼了十几秒后,从唇间吐露出的话语中带着些许让步。\\

「那让他进我家的话,就可以了吧。怎么样?」\\

周也很清楚,她不希望自己的领域被一直逃避的过去的碎片所翻搅。

既然如此,提供自己的家作为谈话的地方怎么样?\\

其实去咖啡厅或家庭餐厅之类的地方谈话是最合适的,可是去家庭餐厅可能会被熟人听见,再说,晚上带着一个年龄大致是小学高年级到初中生的少年在外面走动,很有可能会被警察带去辅导。

在不知道会聊多久的情况下,最好还是避免在外面游荡。\\

听了周的提议,真昼依然一脸阴沉,尴尬地抬头看向他。\\

「可是,这样会,连累你」

「没关系,你连累我也没差。啊,如果你不想被听见,我就先离席」

「……我不希望你离开」

「嗯,我会陪在你身边」\\

如果真昼一个人承受不了,周就会陪在她身边。他们已经决定要互相扶持,真昼难受的时候不支持她,算什么伴侣?周应该要支撑她那不安定的背影。\\

真昼没有拒绝周的提议,反而露出安心和喜悦的神色,嘴角微微扬起。

第一次在慧面前露出的柔和表情让他有些动摇,不过他也因为真昼愿意听他说话而松了口气。\\

「……如果可以来我家的话,她愿意听你说。前提是你要接受这个条件。如果你不想被我听见的话,这件事就当作没发生过吧」

「那样,就可以了。不好意思,提出这么任性的要求」\\

慧彬彬有礼地弯腰鞠躬,真昼则是以厌恶感大幅减弱、困惑感增强的眼神盯着他的头顶,然后用力握紧了和周牵着的手。
