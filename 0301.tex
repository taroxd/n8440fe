\subsection{与树的真心交谈}

由于是考试前,打工排班比平时少,所以周上完一天的课后就老实地回家学习了。\\

和平常不同的地方在于:一是今天少了一节课;二是真昼一对一地教着千岁,到吃饭时间之前都不在这边;三就是树住进了周的家里。\\

周本以为树也会在自己家里学,结果他却说待在家里会想到父亲,这样学不进去,于是就来周的家里认真学习了。\\

只要树不吵闹,认真读书的话,周并不会感到厌烦。两人之间还可以互相出题,亦不失为一种学习方式,所以周没有特别拒绝,而是接受了树的请求。也许是心理作用,总觉得树的表情比平时更加无精打采。

周观察着坐在矮桌对面的树,想知道他为什么露出那样的表情。树察觉到他的视线,脸上露出淡淡的苦笑。\\

「没什么,只是在想今天早上老爸对我说的话而已」\\

树没有说父亲对他讲了什么,不过这种情况通常都会演变成争吵。

从树最近的态度来看,可以知道他非常认真地在准备考试,但大辉或许没有看到他努力的样子。说起来,这可能也是因为树不会主动出现在父亲面前。\\

「你和大辉叔叔还在冷战?」

「与其说是冷战,应该说几乎不说话?」

「那不就是冷战吗?」

「老实说,在他开始提继承人之前,我们就没怎么说话了」\\

如果只听树的说法,会觉得他们的家庭关系非常冷淡,但身为外人的周也不好说什么。\\

周在初中时期也没有像一般叛逆期那样反抗父母,和父母建立了比较良好的关系。不过,他这样似乎属于少数,很少有人能像他这样做到恰到好处的距离感。\\

树曾经甚至还说他希望能换掉自己的父母,可见他对父母,尤其是大辉,怀有着不满的情绪。\\

「该说是连候补都没考虑过我?反正我就是多出来的,随便我怎么做。考虑到这一点,现在这样反而算是有在交流了」

「……我觉得这也不太好」

「我也觉得」\\

树一边用遮字膜来记忆参考书的内容,一边重重地叹了好几口气。他看起来并没有明显地消沉,只是有些疲惫地垂下眼帘。

他和周对视了一会儿,然后像是整理好了心情,缓缓抬起头来,眼神比刚才更有力量。\\

「可是也没办法,老爸很顽固。如果要改变的话,我也必须改变才行。比起改变别人,还是改变自己更快」\\

树斩钉截铁地断言,他的眼神让周眯起眼睛,仿佛看到了什么耀眼的东西。\\

「你变了呢」

「所以我说改变自己更快,对吧?」

「……是啊」\\

正因为周自己也努力改变过,而且自觉到有所改变,所以才能切身体会到树所说的意思。\\

周开始和真昼交往后,也曾被别人说过配不上她,但自从他开始以毅然的态度应对之后,那些声音就很久没再听到了。

为了成为配得上真昼的人,周不断努力,不知不觉间周围的人也接受了这样的他。或许更正确的说法是,他们不再去在意了。\\

虽然情况和树有些不同,但至少都证明了只要自己不改变,情况就不会好转。与其期待别人改变,不如自己主动争取想要的状况会更快。\\

「哎,到头来要做的事还是一样。我只是想努力证明自己是个能好好履行学生本分的人。反正我本来就不怎么调皮,光是这样也能恢复一定程度的信赖了吧」

「大辉叔叔也真是难搞」\\

从旁人的角度来看,树的父亲在人际关系上相当笨拙。

当树不在的时候,周和大辉交谈过,那会儿的大辉不像是什么有问题的人,所以恐怕只有在面对树的时候才会搞砸吧。\\

「他从没不难搞过,我也不知道他在想什么。不过,总之我认真起来之后他就不再抱怨了,这应该算是他的一种让步吧。今天那个我就不懂了」

「我投不知道怎么相处,所以先放着不管一票」

「可能也有这个原因,可是……」

「嗯?」

「怎么说,为什么……为什么父亲会放弃用语言沟通?要别人看着背影领悟?白痴吗?平常没来往的人能指望对方这么细心吗?没有学到人与人之间要面对面沟通吗?」

「不、不是所有人都这样的……」

「是没错啦。但我爸就不行。为什么他明明想被理解,自己却不肯把话说清楚呢?要别人自己去察觉?那种事自己知道就好?就是因为不知道,我们才会生气啊?」

「冷静点冷静点,我知道你很郁闷」\\

树似乎积压了不少不满,声音变得很危险。周安抚着他,一边站起来打开冰箱,去找点东西给他多少转换下心情。

这段期间,客厅那边仍传来低沉的嘟哝。周心想着他真辛苦,暗自同情着树,尽管树恐怕并不想要这份同情。周慢慢地往杯子里倒入冰凉的汽水,试图让他气愤的脑袋冷静下来。\\

「啊真的好烦。这种时候,我真的很羡慕你爸妈。他们愿意听你说话,对你很好,还会默默守护着你」

「听你这么说,我是很高兴啦」\\

周顺便把薯片放在托盘上端回客厅,树便投来了羡慕的眼神。\\

「我先说清楚,我爸妈也不是哪里都那么宽容的哦?该骂的时候还是会骂」\\

周把餐桌上的湿巾放到矮桌上,耸了耸肩。

身为儿子的周不禁觉得,树是不是把修斗和志保子美化过头了。

的确,即使在周看来,他们也很尊重自己,比起当作儿子,更加把他当作独立的人来看待。以一个人的品性来说,他们都是很好的人,周身为儿子也很尊敬他们。

只不过,周依然觉得树是跟大辉一对比,把自己的父母看得太好了。\\

「话是这么说,但你也没怎么被骂过吧?」

「要说的话是我没不太捅娄子,但还是有被骂过……不过他们会听我把话说完再骂就是了」\\

周的父母不会不分青红皂白地斥责他。\\

他们认为孩子做的每一件事,通常都有其原因,不听原因就无法做出正确的判断。因此除了会造成物理性危险的事情以外,他们都会先听孩子解释。至于父母能不能接受孩子的说法先另当别论。\\

周一边回想往事,一边把薯片倒进盘子里,但他仍旧不记得父母曾经用严厉的表情斥责过自己。在这方面,父母算是很宽容吧。\\

「把那份宽容和冷静分一成给我老爸吧。有没有啥办法让我爸学学你们家?」

「没有」

「啧」\\

树一脸遗憾,但他也知道不可能,所以很快就放弃了这个念头,伸手去拿薯片。周见状,内心松了口气,觉得他的心情稍微平复了一些,于是重新坐回沙发上。\\

「好了,别再强人所难了,继续学习吧。你想让大辉叔叔刮目相看吧?」

「我知道,我会好好念书的。呜哇,这我完全没记住」

「快想起来,这是四个月前学过的」

「果然不重复复习就会忘掉啊……」\\

树似乎把世界史的年表忘得一干二净,并由此产生了悲痛的感悟。周把放在旁边的书包里厚厚一本参考书拿到树面前,要他再好好记一下,树随即幽怨地嘀咕道「你也真够斯巴达的」,不过语气中却带着一丝喜悦。\\

\vspace{2\baselineskip}

两人安静地做了一会儿参考书上的题目,互相出题考对方,但注意力能集中的时间是有限的。在书桌前坐了一小时左右后,两人决定休息一下,于是把自动铅笔放在桌上。\\

树的注意力似乎也到极限了,他马上赞同周的提议,伸了个懒腰。

可能是刚才太专注了,身体也变得僵硬起来,树转动肩膀舒缓筋骨,脸上也露出些许疲惫。\\

「你刚才很专注嘛」

「当然啊,这可是为了将来」\\

树从中途开始就以学习为优先,把抓薯片的事搁在一旁。他现在正啪嚓啪嚓地咬着剩下的薯片,缩起肩膀。

树口中说的将来,指的自然是和千岁的未来。\\

周好奇的是,树的目标到底在哪。\\

「树是想过将来和千岁……那个、呃,结婚吧?」

「不然我怎么会反抗到这种地步」\\

树之前就说过,就算不问,周也觉得应该是那样没错,但实际听到他光明正大地宣布,还是觉得有些难为情。

周同时也感到安心,原来不只是自己,这位朋友也认真地把女朋友视为将来的伴侣。\\

在高中阶段就去想着一生的伴侣,一般来说就算被嘲笑也不奇怪。虽然周得到了宫本的理解,但在大众的观念里,高中生的恋爱大抵不过是年轻时的经历,并不是会考虑到将来的东西。\\

「……树,你喜欢千岁的哪一点?」\\

周不经意地问了一句,树听了马上露出坏笑。\\

「咦?怎么怎么,周想聊恋爱话题?」

「不、不是啦。因为我不了解你们交往的时期。虽然知道你们的现在,但不知道以前发生的事情。而且我很少问这方面的事,所以只是问问看而已」\\

周是升上高中后才搬到这个地区,所以不知道树和千岁交往前后的状况。虽然有稍微听说过他们相识的契机,但两人都不会深入提起。\\

周也怕基于好奇心的追问可能会伤害到他们,所以没有多问。不过,看到树一心一意为了和千岁共筑未来而努力的样子,他真切地感受到树是真的喜欢千岁。\\

于是周很好奇,树是喜欢千岁的哪些地方,才会想和她一起走下去。\\

周摆了摆手表示自己没有要打破砂锅问到底的意思,树也没有因为这毫不客气的问题而感到不快,只是刚才疲惫的表情一下子变得柔和起来,眼神变得像在眺望向阳处一样温暖。\\

「当然全部都喜欢啊。长相、身材和性格,一旦喜欢上就会全部都喜欢。她对谁都很友好,既有兴致又充满活力的性格,还有一旦得意忘形就会搞砸事情的傻样子」

「还有只要一不注意,就会开始施展奇迹厨艺?」

「要说的话那也很可爱吧」

「我诚心祈祷受害者不要增加」

「我的胃会帮忙挡住的」

「可是她会突破你的防御,直接攻击我啊」

「啊,我没有给你那边套盾」

「你给我挺住啊」\\

周在矮桌底下轻轻踢了树一脚,但完全没有奏效。\\

「哎,只要小千过得开心,我就满足了。无论是她开朗的样子,还是感情丰富、表情多变的样子,我都喜欢。总之就是全部都喜欢……而且,我们还是相似的共犯」

「啊?」

「我和小千可能就是因为相似才会互相吸引吧」\\

周一时没听懂树的意思,僵在原地,而树似乎完全不打算解释,只是露出平静的微笑。\\

「同情、歉疚和晦暗的愉悦等等,在认识到这些的基础上,我依然喜欢她。虽然契机并不单纯,但现在我就是喜欢她的一切」\\

树没有详细说明两人之间是用怎样的感情面对彼此,但可以肯定的是,他们是在发生过争执的情况下,依然选择了对方。

至少可以确定的是,树和千岁都喜欢着彼此。\\

「我不会深入追究,不过喜欢不就好了?既然你们双方都接受了,我也没有权利说三道四」

「对啊。我们有我们自己的爱情形式,你们也有你们自己的爱情形式」

「……说得也是」\\

表达爱意的方式和爱的形式不一定与他人一致。每个人都有各自的关系和感情。周这边所珍惜的事物,和树他们所珍惜的事物不一定具有相同的性质。

然而,是否相同并不重要,重要的是彼此怎么想。不管别人怎么说,只要他们两人能够接受就行了。\\

「既然你让我说了,那你也要说吧?」

「啊?为什么会变成这样!」

「哎呀用语言表达很重要的嘛!来,你也来谈谈你对椎名的爱吧,尽情地讲」\\

树一边用湿巾擦手,一边拿起旁边的手机,脸上挂着和刚才的笑容截然不同、有些粘腻的坏笑。

周不是不能理解树的意思,但突然要他谈论爱情,要他老实地尽情说出对真昼的感情,这根本不可能。\\

「竟然要我谈论爱情」

「从你的态度就能看出你有多喜欢椎名,但你很少告诉我实际上是怎么想的吧」

「为什么我非得谈这个不可?再说,我之前不是说过喜欢她哪里了吗?」

「机会难得,再说一次让我听听」

「什么机会难得啊,笨蛋」

「有什么关系。心意还是应该说出来哦」

「我有直接告诉真昼啊。因为我不认为她光看我的态度就能明白,也不想让她感到不安」

「哦哦,该说你在这方面很认真理性吗?总之你很了解女孩子的不安」

「这不是说应该用语言表达吗?」

「哈哈,是啊」

「真昼应该没有怀疑过我的感情,也愿意相信我,但我也不想变成一个只会依赖她的信任、恣意妄为的人。虽然有一部分是为了真昼,但主要是我自己的心态问题……正因为知道我是这种类型的人,真昼才会信任我吧」\\

周自认为自己比较不擅长说话,但他不会因此而轻视表达心意这件事。坦率地传达感谢和好感,彼此都会感到舒服,真昼也会很高兴,所以周会尽量直接表达出来。\\

虽然就算不说出口,聪明的真昼也能理解周的心意,但光是期待她能感受到自己的心意而自己不采取行动,周认为这样太任性了,甚至还觉得过于幼稚。\\

虽说彼此都有察觉到对方的心意而一直在等待时机,但真昼主动制造出告白的机会,也让周深切感受到自己的不中用。\\

开始交往后,周会明确表达自己的想法,也会尽量用言语让真昼安心。如果只是传达心意就能让真昼的心不再被不安束缚,让她安稳平静的话,他当然不会吝惜这些工夫。\\

因此,周看向树,示意他和真昼之间的关系在那方面并不需要担心。树则是用手托着下巴,莫名佩服地点了点头。\\

「椎名看人的眼光真准啊。竟然喜欢上了这个明明很老实,却又性格扭曲、简单易懂又难懂的家伙」

「喂,原来你是这样看我的吗?」

「啊哈哈,刚认识的时候,你可不是有点冷淡又拒人千里的嘛?」

「别说了,那是我的黑历史」

「你真的很难为情的样子,好好笑」

「你这家伙」\\

一想起过去的自己,周就后悔不已,甚至羞耻得想在地上打滚。但也正因为有过那段羞耻的时期,周才更加感激树当时愿意向自己搭话。

不过,周也绝对不想被树拿来取笑,于是他用锐利的目光牵制树。\\

树接收到周的视线后,却毫不畏惧地挥了挥手。\\

「哎,对我来说,我也有段不想回忆的黑历史。你和椎名相遇之后,改变了很多呢」

「你那句预言成真了,真让人不爽」

「就是说你总有一天会改变。从那个时候开始,你就一点一点地改变了,只是很难看出来而已」\\

树的贼笑变得更深了,于是周又踢了他一脚,但依然没有效果。

周小声嘀咕了一句「烦死了」,树却用从容的笑容接受。为了缓和这种难以言喻的尴尬,周轻轻撩起头发,然后慢慢深呼吸。\\

「……我能改变都是多亏了真昼,也是多亏了你们。我很感谢你们」\\

当面告诉树这种话,和向真昼表白心意相比,又是另一种难为情的感觉。不过,这次应该要好好说出来。一开始说要用语言表达的是树,周也对此表示同意。

周不自然地抿紧嘴唇,但还是小心注意着,以免不小心说出平时和树开玩笑时说的话。\\

「哟,竟然坦率地夸奖我了」

「你少在那里打岔」\\

树俏皮地吐了吐舌头,那副模样以高中男生来说实在是有点太作了。周放弃吐槽,只是给了他一个白眼。\\

树似乎察觉到气氛变得冷淡,便用开朗的语气说「你在这方面还真不配合」然后毫不愧疚地用手扇着连空调也无可奈何的空气。\\

「不过,这也证明了周有多喜欢椎名。否则改变是很困难的。不仅需要耗费大量能量,也会遇到痛苦的事情。这就是爱啊,真了不起」\\

光听这句话,还以为树是在开玩笑,但从他的表情中却看不出那种感情,只是平静而感慨地低语着。

希望有所改变的树亲身体会过,改变自己需要下多大的功夫。\\

正因为如此,他才会给予纯粹的赞赏。周无法否定这一点,只能呻吟着接受,然后紧抿嘴唇,努力压下从内心深处涌出的热意。

树用温暖的眼神看着周,仿佛看穿了他的内心。周受不了地回以沉重的视线,接着便听见一阵笑声。\\

「害羞了。不过你不否认,就表示是那样吧」

「……当然,我喜欢她,也爱她。她是我最爱的人,是我想要亲手让她幸福的人。就算赌上这条命,我也要让她幸福」\\

以前的周根本无法想象自己会有这样的想法。

竟然会遇到值得赌上人生让她幸福的对象,并且与她交好。\\

「真的是爱啊。真想让椎名听听」

「刚才那些话太羞耻了,我没办法再说一遍。你也给我忘掉」

「不要~」\\

周歪着嘴角,一边瞪着树,一边送出「快给我忘掉」的眼神,希望能让这一连串的对话从树的脑海中消失。树却轻笑着说「别瞪我别瞪我」。\\

「顺便问一下,你觉得被椎名听到也没关系?」

「……被听到也没关系,不管什么听不听的,我都不会再说这种话。再说,我平时就会好好地向真昼表达喜欢」

「哦~?」

「别笑得那么贼」

「好好好,抱歉」\\

周用眼神示意「再捉弄的话我就要生气了」,树也明白他是认真的,于是嘴上说着不怎么真心诚意的道歉,一边滑着手机。

这家伙有一半没在听。周眯起眼睛瞪着他的脸,但他还是不当一回事。\\

「所以,你为了证明爱,一直在努力」

「……不行吗?」

「没有,我很尊敬你一心一意、专情地勇往直前。你有我所没有的坚定信念」

「为什么就这么确定自己『没有』?你也是『有』信念的人,而且会『坚持』到最后吧」\\

树经常说周自卑,但他自己也好不到哪里去。\\

周所认识的树虽然有些难以捉摸的轻浮,但他绝不是莽撞、缺乏计划性或没有干劲的人。一旦决定要做,他就会坚持到底,而且也拥有足够的能力和行动力去付诸实行。当他下定决心时,其强大的行动力和坚定的意志都会令人瞠目结舌。\\

真正看轻、贬低自己的人,其实是树才对。\\

为什么他在这方面就不能正确客观地看待自己呢……周毫不掩饰自己的无奈,看向正在自虐、伤害着自己信心的树,结果树却是一脸惊讶。\\

树为什么这时候会感到惊讶?难道他真的没有自觉吗?\\

「你那是什么表情?你在这方面很顽固又专情吧」

「……真想回应你的信赖啊」

「你会回应的吧?」

「当然」

「那就好」\\

树停止了刚才的缓慢自虐。周也看得出来,他那有些动摇的眼神如今带着坚定的意志回望着自己。\\

「喂,你的手停下来了,脑袋也停止运转了,别光说,快点做」\\

两人聊了很久,周再次确认时间,发现他们聊上了兴头,比预定的休息时间多花了相当长的时间。

重新确认对真昼的感情是很好,但周没想到自己会对树坦白到这种程度,羞耻心还在胸口深处隐隐作痛,不过他说的都是真心话,也是发自内心的,所以无意收回。\\

今天的晚餐是由在家的周负责,要是再继续闲聊下去,学习时间转眼间就会过去。\\

周把滚到自己这边的自动铅笔又弹回树那边,示意他赶紧重新开始学习,树则是一脸好笑地说:\\

「哎唷,好斯巴达」

「对你来说这样刚刚好」

「好过分」\\

树温柔地靠近过来,笑得肩膀都在抖。周见状也悄悄松了口气,握紧自己的自动铅笔。
