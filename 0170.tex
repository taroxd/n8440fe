% \subsection{170 餌はぶらさげるもの}
\subsection{170 餌はぶらさげるもの}

% 「つーかーれーたー」\\


% 途中軽く休憩を挟みながら必死に課題をこなしていた千歳だったが、流石に疲れてきたのか駄々をこねるようにごろごろとカーペットを転がる。\\


% 本日はショートパンツだったからよいものの、スカートなら中身が見えそうな動き方をしているので、周は呆れも隠さない眼差しを向けた。\\


% 「暴れてジュースとかこぼしたらどうするんだ」


% 「その時は土下座する」


% 「そこまでプライド捨てるくらいなら最初からこぼさないようにしてくれ。それにカーペットとか服汚れたら大変だろ」\\


% 律儀に真昼がテーブルに置いてあった二人分のコップを持っているので心配はないが、置いてあったら事故が起こる可能性もあった。\\


% カーペットに溢れても怒りはしないが、流石に染み抜きの手間を考えれば溢して欲しくはない。\\


% 真昼も「大人しくしないと駄目ですよ」と窘めている。


% その微笑みには苦笑が混ざっていて、本気で止める気はなさそうだ。息抜きをさせないと疲れるというのが分かっているのだろう。\\


% 「むむー。じゃあ転がるところないからまひるんのお膝にいくー」


% 「待て、そこは俺の指定席だ」


% 「ケチだなあ。まひるん、だめー?」


% 「……周くんが駄目って言うなら駄目です」\\


% 瞳を伏せて首をゆるりと振る真昼は、ややぎこちない。


% そんな真昼に、千歳は却下された事に不服さを一切見せない、にんまりとした笑顔を浮かべる。\\


% 「膝枕体験ならずだけどまひるんが嬉しそうなのでいいや」\\


% 嬉しそうというよりは恥ずかしそうの方が近いのだが、それでも頬がほんのりと染まりつつ緩んでいるので、千歳の言う事も間違いではないだろう。


% 指定席という言葉が嬉しかったのかもしれない。\\


% 「じゃあ私の代わりに早く堪能してよー、それ見て課題頑張るから」


% 「やなこった。からかうに決まってるだろ。俺のには違いないからお前が居ないところでしますー」


% 「するんだねえ」


% 「特権だからいいんだよ。ほら、甘いもの買ってきてやるからさっさと課題しろ」


% 「ほんと!?」\\


% 飛び起きてぱあっと瞳を輝かせる千歳に、かなり現金な少女である事を痛感させられた。


% その言葉を待っていました、と言わんばかりの笑顔に、周も真昼も揃って苦笑する。\\


% 「ご褒美だご褒美。千歳が真面目にするなら今から買ってくる」


% 「するするー! 流石周、太っ腹ー! 私行きつけの店のがいい! チーズケーキね! スフレのやつ!」


% 「注文つけんのかよ……まあそんな遠くないからいいけどさ……」\\


% 近場のケーキ屋と比べればやや遠いしお値段も少しお高めではあるが誤差だし、真昼もあの店のケーキは好きだそうなので行く事に抵抗はない。\\


% 「真昼は?」


% 「え、私ですか……?」


% 「なんならまひるんも一緒に行ってくれば?」


% 「お前が怠けるから駄目だ。それに炎天下の中歩かせるのも悪いし」


% 「私どんだけ信用されてないの……しかし周が紳士なのでここはぐっと飲み込んであげよう」


% 「お前の分だけ買ってこないぞ」


% 「それご褒美の意味なくない……?」


% 「なら黙って大人しく課題こなしてろ」\\


% 信じられないといったような眼差しを向けられたがスルーしつつ、真昼に何がいいか聞いてガトーショコラという返事をもらって立ち上がる。


% 流石に夏場はケーキの売れ行きもやや落ちているとは思うが、売り切れている可能性もなくはない。早めに行くに越した事はないだろう。\\


% 「んじゃ、行ってくるわ」\\


% 財布を携えてリビングを出れば、しずしずと後ろに真昼がついてくる。


% どうやら見送りにきたらしく、周が玄関に座ってスニーカーを履いていると真昼も側に膝をついてしゃがむ。\\


% 「どうした?」


% 「いえ、暑い中申し訳ないな……と」


% 「いいよ、俺が言い出した事だし。それより千歳ちゃんと見とけよ」


% 「ふふ、千歳さんはああいう風に振る舞ってますけど、真面目な時は真面目ですよ?」


% 「知ってるけど、それでも、だ。まあうまい事休憩挟みつつ頑張ってもらってくれ」


% 「了解です」\\


% くすりと微笑んで頷いた真昼に周も笑って、立ち上がる。\\


% 「それじゃあ行ってくるわ」


% 「あ、待ってください周くん、ちょっといいですか?」\\


% 呼び止められて振り返ると、真昼が急に周の胸にもたれてきた。


% いきなりの事に硬直すれば、真昼はもぞもぞと背中に手を回して周にぴったりと体を寄せる。


% ふんわりと香る甘い匂いと柔らかい感触に、呻き声が漏れそうになる。なんとか堪えつつとりあえず真昼の頭を撫でると、くすぐったそうに瞳を細めた真昼が顔を上げた。\\


% 「……今日は勉強でちょっと疲れたので、補給させてもらいました」\\


% 小さな囁きに、堪らず周も真昼を抱き締めると恥じらいを瞳に浮かべつつも嬉しそうな笑顔を浮かべている。\\


% 「……そういう事言われると、離したくなくなるんだけど」


% 「それは困りますね、千歳さんが悲しんじゃいます」


% 「……千歳が帰ったら、いい?」


% 「願ってもない事ですね」\\


% 頷いてもう一度周の胸に顔を埋めた真昼に、周はさっさと用事を済ませて帰ってくる事を心に誓った。

