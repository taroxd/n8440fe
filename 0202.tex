\subsection{小小的吃醋和担心}

在文化节中,咖啡厅算是个费工夫的项目,但进展却比周预计的要顺利得多。\\

最关键的因素应该是有人出借了衣服。正是因为解决了这个问题,所以才决定要开咖啡厅。

接下来,关于内部装饰和打算提供的饮料、食物方面,前者只需要摆好教室的桌椅,显得好看就不成问题了,而后者要准备起来却很麻烦。文化节有两天,既要预估出这两天的分量,准备时还必须注意卫生。\\

话虽如此,这次却没有那么麻烦。出于卫生和所需精力的考虑,最后采取的是从市场上批发的方案。

准确来说,周的班级举办的是有女仆和管家的咖啡厅,核心的乐趣就在于服务员的外表和氛围,因此食品方面只得妥协。

考虑到有多少班级在排队申请使用家庭教室,从这方面也可以说选用批发的方案是英明的。\\

「不过饮料会弄得认真点就是」\\

树带着诙谐的笑容,使着眼色说道。他还担任着执行委员长。

之前,树说自己有关系,从专卖店低价购入了咖啡。他欢快地笑着拍了拍袋子,里面装的是磨好的咖啡豆。\\

其实现场烘培是最好的,但这店毕竟是高中生临时开张的,没有功夫去做这些,于是咖啡就事先准备了。红茶叶也已经安排妥当,关于要提供的东西,可以说是准备齐全了。\\

「比预料中弄得更好哎」\\

千岁看着几乎装饰好的教室,小声嘀咕。

这里是教室,能做的装饰是有限度的。即使如此,为了让桌子看着不像是书桌而摆上的桌布、靠垫,还有储物柜上装饰的小物件都酝酿出一股氛围。

尽管这些说不上有多正规,但作为学生的活动,应该是足够了。说到底,最重要的还是换好衣服的学生。\\

「嗯,做成这样,我觉得够了」

「是啊。只是换了窗帘和小装饰,变化就很大了」

「你们干得不错啊,这窗帘很有感觉」\\

借来的窗帘上打着金色的结,很豪华。周指了指窗帘,千岁则小声说道「弄脏的话恐怕就糟了」。

窗帘旁边没有安置多少座位,如果弄脏,大概会花上不少清洗费用吧。\\

「有这些就行了吧,剩下的就是许愿顾客会来了」

「……我感觉有椎名当女仆,他们就会纷纷上钩的吧,而且奔着椎名去的怕是多得容不下」

「我女朋友可不是鱼饵啊。还有,其他女生的打扮也很合适,说他们都只奔着真昼去,是不是对那些女生有点没礼貌」\\

虽说周的关注点只有真昼,不过客观来看,穿女仆装的女生们仪容端正,衣服也很搭。刨开偏袒,真昼的可爱固然超群,但这并不代表合适的只有她一个。\\

「阿树,我觉得你该学学刚才周说的」

「疼、疼哇,你也可爱的啦」

「你夸得太没诚意了,不多夸夸的话,就罚你去上次说的那地方喝下午茶套餐」

「那边好贵啊!」

「那边每桌都配一个管家,你可以看着学学」

「学费也太贵了!」\\

周把吵吵闹闹地、要好地制定着约会计划的两个朋友放到一边,看向静静待在旁边的真昼。

不知为何,真昼的表情有点微妙。\\

「真昼?」

「……周君觉得我是……最、最可爱的,吗?」

「怎么了,这么突然,是在意我刚才表扬其他女生吗……这种事情不是废话吗,有什么好问的。你最合适,也最可爱了」

「嗯、嗯」\\

对周来说,「真昼是特别的」是前提,但真昼好像还是会感到在意。\\

周小声又确实地称赞了吃了点小醋的真昼,只是这样,真昼就接受了,并高兴地咧开了嘴。

现在还在学校,真昼不会把身子贴上来,而是腼腆地轻轻捏住了周的衣服袖子。就连这个动作都会引人注目,于是周也因为自己女朋友的可爱,心里产生了一点点烦闷。\\

(……到了当天,会聚集更多的视线吧)\\

现在这些视线是来自同学的温暖目光,尽管让周心情复杂,但还没什么问题。\\

问题是文化节当天。

可以预见,会有人投来非礼的视线,还会有人做出过分的行为。\\

(尽可能不要离开她吧)\\

周暗暗感谢为此将他们排到同一段时间的树他们。他来回看向嘴角挂着腼腆笑容的真昼和一边争吵一边又恩爱的树和千岁,轻轻苦笑。
