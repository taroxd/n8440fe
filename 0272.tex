\subsection{周围人的去向}

第二天早晨,周刚进教室,便发现树一大早地就坐在自己位置上,一眼就看得出他心情不太好的样子。

平常往往都是周更早到校,而今天树则是出门出得特别早,看他那忘掉了外面寒冷的脸色,大约是早早就飞奔出家门过来的。

恐怕是他在周之后的三方面谈中起了点争执吧。\\

「早。你怎么没精打采的」

「早。见面第一句就这?」\\

周如往常一样地打了声招呼,树便把看向窗外的视线转到周这边,无奈地笑道。

看那态度,周的猜测便成为了确信,他保持着平日里的神情,耸耸肩说「都写在脸上了」。\\

「三方面谈怎么样?」

「呃,你想知道?」

「倒不是说想知道,就是如果我避而远之、不闻不问的话,是不是会让你心情挺复杂的?怕引得我担心,反而你自己心里添堵」

「你那么懂我反而让我心情复杂哎」

「这一点你就死心吧」\\

与其客气得不到位,不如干脆敞开天窗说亮话,树就是这样的人,不合时宜的关怀反而会使他扎心。

那样的话,即便有些不礼貌,也是开门见山地问才会让他心里更痛快一些。看到树心里一块石头落了地的眼神,周便明白他选择的说辞是正确的。\\

「这个,怎么说呢,就是对牛弹琴。我爸果然很坚持想让我去某个地方,跟我意见不可能合得来。我自作主张决定了入学考试的选修科目,然后他就发火了」

「啊」\\

周做的事情其实也差不多,但一边是整体上肯定周所做之事的父母,一边是想要留住树的大辉,从而得到了完全相反的结果,这让周心里有些不好意思。\\

「不过反正已经交上去了」

「这是直接豁出去了啊」

「还能怎么办嘛。怂了就要被爸爸逼着选,就只能光明正大全力硬来了,根本没有别的法子可选」\\

比起赌气更像是豁出去的树叹了口气说自己很伤脑筋,此时他的眼神中却蕴藏着积极的光。\\

「幸好妈妈去开导爸爸说『你看吧,逼一个说不听的也没用的』『我不是说了吗,强迫得太过就会把人引爆,然后指示和建议都听不进去的』『你也够了吧,该放弃了』所以能行的、能行的」

「你妈真的很猛啊」

「说是有主见呢还是清爽呢还是坚定呢,总之就是事情都会说得很明白,讨厌不合情理的事情」\\

周感觉这是他见识过的母亲中说话最有条理的一位,在她的儿子,也就是树心里似乎也有着相同的感受。\\

「我觉得我家里跟普通的父母应该也不一样吧,妈妈虽然对我的去向并不是一点兴趣都没有,但是对此完全放任不管,或者说是让我做我想做的事,完全听凭我做决定」

「这也算是认同你,对你来说是件好事吧?」

「相应地,妈妈让我一定要努力考上,要求我对自己的发言负责并行动,既然是自己说出的事情,之后就绝对不能不求上进」

「……嗯,既然认同你了,那不也挺好吗」\\

周觉得树的妈妈说得有些过了,不过其中估计也有勉励的目的在,不是周可以说三道四的。\\

「那倒是。只要我努力就好了」

「彼此都只能努力了啊」\\

到头来还是必须得努力,这件事是明确且不可动摇的,同为明年赴考的考生,只好互相打气了。\\

「周也已经决定好要全力备战那边了吧?」

「算是吧。我虽然没有明确想去做的工作,但想学的领域是有的,也希望能做到自立。我的想法是,想做的事情日后还能去找,假如不能当作职业的话,作为兴趣爱好弄一弄也不错」

「已经那样决定了那不是挺好的嘛。话说,感觉跟真昼在一起的目标会成为你很大的动力啊」

「要你管」

「嘿嘿嘿,到了大学说不定就同居了」

「我说啊」\\

一精神起来就开始捉弄人的树让周脸上的肌肉开始绷紧,就在这时一阵温和的声音横穿过来「树你老是这样子捉弄人,之后小心被藤宫回击哦」。\\

身体转向声音传来的方向,映入眼帘的便是和往常一样表情一直都很柔和的门胁,正在放下他背着的包。\\

「优太啊,早」

「两位都早啊」

「早」\\

一如既往沉稳的门胁提醒了一声树,让他适可而止,去把包挂到自己座位上,然后回到了这边。\\

「所以是怎么说到那些的?」

「啊,我们是在说三方面谈的事,谈到未来要怎么办的时候这家伙就莫名其妙地来多管闲事了」

「说多管闲事是不是过分了点!?」

「毕竟树心思都花在捉弄藤宫上了吧,我觉得这个词很妥当」

「优太就没打算帮我是吧?」

「嗯」\\

门胁满不在乎,理所当然地点点头,树见状便是一个夸张的踉跄,宛若深受打击。周和门胁也知道那就是故意做出来的样子,互相看了一眼对方,没有理睬他。\\

「大家果然是在为三方面谈的事情吵吵嚷嚷的啊」

「是啊,有种考试真的越来越近了的感觉」

「你俩自说自话还挺过分的啊」\\

迅速从假装失魂落魄的状态恢复的树带着一些怨念说道。不过他完全没有怒意,就要来参与进两人的对话。\\

三人知道这就是小打小闹,于是才有了那样的对话,于是稍远一些的地方便传来千岁等人的交谈声「阿树其实挺喜欢那样你来我往的」「确实有点那意思」。要是能再小声点别让树听见就更好了。\\

「门胁的三方面谈是后天来着?」

「嗯,姐姐们那天不在,我真觉得实在太好了」

「感觉她们会很想跟来」

「啊哈哈……怎么说也得坚决拒绝啊」\\

周并未亲眼见过门胁的姐姐们,不过从别人口中听闻了她们十分具有个性的一面,身为独生子的周也觉得门胁很不容易,对他深表同情。\\

「门胁已经决定好去向了吗?」

「嗯,姑且是想走体育特招,不行的话再正常考」

「优太是去大赛拿到成绩的啊……感觉很有希望」\\

在目前高二这一时间点,门胁就已经参加大赛取得了成绩,好几次看到他在典礼中走上了领奖台,周并不怀疑他是有资格去争取那份名额的人才。

更何况门胁除了体育之外本身成绩也很好,有很大的选择空间。\\

「真有的话就好了。我这种水平的其实还挺多的,还得继续提升自己才行」

「感觉优太这方面有点自卑啊」

「自卑不是藤宫的专利吗?」

「喂」

「嘻嘻,开玩笑的」

「优太都觉得周自卑吗」

「吵死了吵死了」\\

周承认自己之前很自卑,但现在已经有了不少自信,即便偶尔缺乏信心,他也将自己磨练到了能够认识到当前的自卑感,并积极看待问题的程度,而且至今为止也闯过了各种难关。\\

这般捉弄也同样是小打小闹,周便只是夸张地作生气状,不再纠结于此。\\

「毕竟我能力还需提升是单纯的事实。磨练自己很有意义,而且教练也说我还有进步空间,我就感觉一定要跟学习一起努力才行」

「田径社的王牌真够努力啊」

「要是不努力的话,王牌的宝座要不了多久就要丢咯。我没打算在退役之前把这位子让出去,而且我还想要作为社长挺起胸膛带领社员呢」

「啊这样啊,社长吗……真不容易」\\

周想起暑假结束那会儿门胁就任社长之事,为他也要更忙更辛苦而心生感慨,门胁却对此无感一样,淡然地说道「大家都很靠得住,没多少事需要我去做的」。

「副社长是一哉,然后教练也在,各位社员也全都很可靠,可以说是多亏了他们。我都为自己做的事情太少而于心不安了」

「毕竟都是好好看着社长的背影长大的啊」

「是啊」

「夸我也不会有什么好处哦?」

「还打算让你难为情呢」\\

这就开始捉弄人的树并没有让门胁展现出半点动摇,他以同样笑眯眯的眼神盯着树道:\\

「哦?那我也让你难为情一下吧。对了藤宫,这阵子树——」

「对不起请原谅我」\\

过于迅速的变脸让周很傻眼,可是由于树道歉得过于诚恳,他便发觉树一定是做了一件相当不想被人知晓的事。\\

只可惜事情的内容在听到之前就被打断了,不知道是什么,但那件事无疑可以成为树的弱点。\\

「你做了啥啊,还是说你打算做啥」

「别问了没什么」

「哈哈,树已经在用眼神求我饶了他了,那就算了吧」

「最克制树的原来是门胁……?」\\

优太轻快地哈哈大笑,却完全不像是挖苦的意思。周确信这一定很能针对树,心想着今后可以拿来稍微牵制一下他了。想着这些当事人知道了肯定要摆出苦瓜脸的事情,他向笑容深不可测的门胁望了过去。
