\subsection{与朋友们的祝福}

「再正式说一次,祝昼儿生日快乐~」\\

学校里,千岁似乎是收着嗓门,避免把真昼的生日弄得过于大张旗鼓,一来到周的家里,她就高举拳头放声大喊,十分欢快。

木户和门胁欣慰地默默注视着千岁。树则是一边苦笑着说她真有精神,一边向真昼露出了柔和的笑容。

今天算是庆祝生日的第二场,专门招待了之前帮忙装饰房间的人。\\

「生日快乐,椎名」

「谢谢大家」\\

真昼已经不再是过去那样对生日本身完全无所谓的样子了,她脸上的表情是安心和淡淡的欢喜。

看到她在好朋友的庆祝下高兴的样子,周也松了口气,真昼能真心为生日高兴,可真是太好了。\\

「哎嘿,这下子昼儿也跟我一样大了~之前还小一岁的」

「大伙儿眼里本来就是真昼更成熟吧」

「你是对我的言行有什么不满意的吗?」

「没有没有~」

「……小千啊」

「那边那位,是有话想说吗?」

「没有~」\\

在身为男朋友的树眼里,他大概也对千岁的言行有些许意见。不过他也清楚,把那意见说出来的话,对面的矛头毫无疑问会转向自己,于是树只是打了个哈哈,没有再继续说下去了。\\

「好了。昨天周肯定庆祝了个够,今天就该轮到我们来为你庆祝了」

「我已经受到大家充足的祝福了……装饰也是大家帮忙的吧?非常漂亮」

「嘿嘿,要说了解昼儿的喜好,跟周比我也不会逊色的」

「那还是千岁你更加了解吧。我也就是大概知道真昼喜欢怎么样的,一时半会儿也想不到什么能直接戳中她内心的。一直以来都多亏了你」\\

周自然也是理解真昼的喜好的,但并不至于能见到一样东西就妄下定论。

即便是了解喜好,周也并没有能力从她的喜好中选出最佳的一项,或者是将她的喜好一个一个巧妙地组合起来。他没有千岁那样的品味,也不像她那样可以拿出同为女性才拿得出的方案。

在真昼的喜好方面,千岁具有卓越的品味,这一点上千岁的能力毋庸置疑。\\

「哼哼,好好依赖我,尊敬我吧」

「遵命」

「你们俩啊」\\

周坦率地低头,摆出古装剧那般磕头的姿势。从他后脑勺那儿传来真昼的声音,听上去有些伤脑筋,却又带着轻快的欢笑。\\

「总之我送的是这个,昼儿平常用的护肤品的限定包装版,超可爱的」

「谢谢千岁,你记得真清楚」

「那毕竟都留宿了好几次了~周,羡慕不?」

「咋还得瑟起来了……真昼用的什么我又不是不知道」\\

真昼是很注重自我保养的女性,在护肤这块也颇有见地。多年以来,她尝试过各种各样的产品,如今找到了适合自己的,便一直在使用。

真昼来留宿的时候,周见她用过许多次,能记住也是理所当然的。\\

「哎呀哎呀」

「哎哟」\\

本来周倒是完全没那个意思,但树和千岁听到周知道护肤品的内容,便对后续的故事浮想联翩,不约而同朝他笑了起来。周脸上抽着筋,眯起眼看向他们。\\

「别傻笑了,没有你们想的那些事」

「哎呀~」

「……小千和赤泽君都不长记性呢」

「就是啊,明明知道捉弄过了头周就会恼羞成怒的」

「你们既然知道倒是帮忙拦住他们嘛,门胁、木户」

「我觉得,拦不住」

「也是」\\

他们带着亲近的心意来捉弄周算是稀松平常的事了,其中也没有恶意,周向来是当作玩笑带过。但话又说回来,这么一直下去总叫人不是滋味,真希望在旁边观察的那两人也能帮忙挡一挡。\\

至于真昼,则是一边抚摸着从千岁那儿收到的礼物盒,一边露出笑眯眯的神色,不像是打算阻止千岁的样子。可能也是因为她知道,即便说了也不能让千岁停下。\\

「这都能买到啊……我记得是网上抽选贩卖的吧」

「那是我运气好」

「你肯定费了不少功夫……真的很谢谢你」

「不客气。其实我本来感觉口红也挺可爱的,不过考虑到你有自己喜欢的颜色,还是让周来买的话你会更开心吧。还能让他亲亲」

「千岁!?」

「开玩笑的开玩笑的」

「小千捉弄过头了哦,那边的守护者已经拿手指摆好姿势了」

「手指又能怎么」

「大概能弹你脑门吧。要不要试试?」

「算、算了」\\

周用大拇指内部抵住中指,形成一个圆形,并在中指上施力给千岁看,她便带着点嘴角的抽搐摇了摇头。

树被这一幕戳中笑点,忍不住捧腹大笑。周瞪了他一眼,告诫他管好女朋友,可惜没有效果,反而使他笑得更厉害了。\\

「这家伙」周投去冰冷的视线,不过树丝毫不以为意,眼泪都给笑出来了。他擦擦眼泪,从旁边的手提包中取出包装好的礼盒。

接着他双手捧着礼盒,小心地递给真昼,前后判若两人的模样让真昼都产生了些许动摇。\\

「这是我和优太的礼物,请查收」

「不用那么正式吧……而且还麻烦了门胁。谢谢各位」

「哪里,很荣幸能有这次庆祝的机会」

「里面的东西让周确认一下?」

「怎么,放了不该放的东西吗……不过跟门胁一起送的,那就不可能了」\\

但愿是没什么不该放的东西。树姑且也是个正经人,本来应该也是不会放什么奇怪的东西进去的。只不过若是跟千岁搭伙,也不排除万一里面会是个怂恿真昼往前迈一步的不太好的玩意儿。\\

而这次他是跟门胁商量的,所以完全可以放心。\\

「对我的信任哪去了」

「要不你看看你自己做过什么」

「……也不至于,到这地步吧?」

「哈哈,嗯」

「优太这会儿该帮我说几句吧」

「我觉得树对椎名应该是不会失礼的」

「也是,对真昼应该不会。那我放心了」

「看看这个待遇差别」

「啊对对对」

「优太你偶尔对我有点刻薄」\\

两人从以前交情就不错,也常常会轻松地闲聊,也就是因此才有了这样一段对话。

门胁平时总是笑眯眯的,浑身上下都是温厚和诚实,也就是看到这种场面,才会深切感受到他也有符合年龄的淘气一面。周偶尔也会被他打趣,但愿这代表了他认可两人之间的关系亲近。\\

「顺带一提,里面的东西是,嗯……」

「怎么支支吾吾的」

「就是不知道当礼物送合不合适。虽然是我们俩商量过后的结果」\\

送女朋友还好,送给女性朋友让两人颇为纠结。两人都面露难色,甚至几乎成了苦笑,可见两人挑选礼物时真是煞费苦心。

周不怀疑两人的品味,可是看他们过于微妙的表情,心里难免担心。\\

「……最后选了什么?」

「弄了些不错的汤汁套装」

「真、真实用……」\\

千岁不由得说道。周也是同样的感想,的确非常实用。\\

「听说以前她想要磨刀石,不过磨刀石显然不太行,就想着来点跟烹饪有关的实用物品……这样藤宫也很受用」

「等等,周君把磨刀石的事情讲出来了!?」\\

看来真昼也知道这不是女高中生会想要的东西。她脸颊泛着一抹羞红,找上了泄露信息的人,而周身为当事人,则是用丢脸的声音不停地向她道歉。\\

真昼倒不是不愿意这件事被说出去,但她也不想将它大肆声张。她眼睛半睁地瞪着周表达不满,尽管目光不算那么锐利,但也让周轻拍她的肩膀予以安抚。\\

「不过磨刀石你是确实想要吧」

「会有人不想要吗?」

「我觉得这么专业的东西对我还为时尚早了点」

「我也是~」

「我对下厨本来也没那么大的热情……」

「我只要差不多锋利就可以了」

「没、没人跟我一样吗……」\\

众人并没有把下厨当作兴趣的人,便无人赞同真昼的意见。真昼垂下眉梢,一脸失落的模样,于是周轻轻抚摸她的头来安慰她。\\

「真昼,没事还有我。不过话说回来,磨刀石给我用的话也算是委屈它了」\\

周明白磨刀的重要性,然而他并不能使用自如,所以有个简易的研磨器就好了。他也不能全方位地站在真昼那边。\\

在周的视角下,真昼那明显闹别扭的模样不禁让他要傻笑出来。在他憋着的时候,木户看出了他在憋着什么,她笑眯眯的神情让周难为情地别过头去。\\

「总之就是那么回事,方便使用、用起来又没有负担的实用消耗品是个比较稳妥的选择,所以我们就选了能做出好汤汁的套装。用这个把藤宫的胃抓得更牢吧」

「再抓要给抓烂咯」

「嘻嘻,到时候我会负起责任照顾的」

「哇哦,是爱呢」\\

真昼没有肯定、也没有否定这句话,只是面露难为情的微笑。周的内心深处有种被人用脚趾头挠痒痒的感觉,视线来回游移。\\

他知道,这是在害羞和喜悦之间来来回回所致的心痒。口中没有说出来,这一切却写到了嘴唇的动作上。「藤宫君,这里不是学校,坦率点也没事的」没有任何打趣之意的声音从木户那里传来。

「……你别多嘴」

「哈哈,抱歉抱歉,看藤宫君心里很痒的样子」

「周也真是不坦率」

「闭嘴」

「说的就是你这样的」

「你不许说话」

「哎哟,凶起来了」

「哈哈,大家都花了很多心思呢。我——应该说是跟婶婶一起决定的,纠结了会儿生日礼物,最后选择了这个」\\

木户很习惯地面带笑容忽略了一如往常的树,就在周为此感到佩服的时候,木户从包里取出一只可爱的信封。

这与前面那些都很不一样的礼物让真昼使劲眨了眨眼,而后木户俏皮地朝真昼抛了个眼神过去。\\

「这是下午茶的双人票。是一个有管家提供服务的地方。票是靠婶婶的关系拿到的」

「丝卷阿姨的人脉是不是有点广?」

「我也觉得。都有些吓人了,商量着就成了这样」\\

并无直接关系的丝卷都松籁庆祝,真昼也掩饰不住内心的些许困惑,或许是有些惶恐吧。

「这样关心我真的好吗?」她的眼神中明显流露出疑问,而木户也注意到了这点,自己也很困扰地劝慰道「毕竟是我的朋友,而且婶婶也挺中意藤宫君的」。\\

「藤宫君应该也不讨厌甜食,而且说不定也能参考一下接待客人时的举止和态度什么的。顺带一提,那家饭店提供的东西都超级好吃,椎名应该也有可以借鉴的地方」

「谢谢……明明我都还没机会去跟丝卷阿姨打声招呼」

「那是因为要等藤宫君叫上你吧?你放心,婶婶都理解的」

「呜,我尽快习惯」\\

周明白自己被委婉地挖苦也是情理之中,只好向木户低头。

接待客人这事,周已经逐渐适应,打工本身也渐渐熟悉,只不过依然会有弄砸的时候,也常常在前辈和茂野面前丢脸。

如果可以的话,希望她看到自己独当一面的样子。尽管这是周的任性与虚荣,但想让女朋友看到自己光鲜亮丽的一面,这份心情是无法妥协的。\\

「你赶紧的,我也想看」

「我也是我也是」

「你们是来插科打诨的吧!」

「这么多疑,多伤人呀」

「很难相信你……」\\

一扯上树和千岁,就十有八九——甚至99.9\
「……周君帅气的模样,可要尽快给我看看哦?」

「我尽量」\\

虽说是没办法的事,但连真昼都暗戳戳地催促起来,看来也不好让她等太久了,周下定决心,得更加努力地打工才是。\\

真昼笑眯眯的,略带着些许压迫感,引得千岁哈哈大笑。侧目看过去,千岁吐了吐舌头,丝毫没有悔改之意。周便扭过头去,将千岁的捉弄抛到一边。\\

「真的很谢谢大家,为了我这么费心……我感觉自己真的非常幸运」\\

真昼捧着礼物,腼腆地笑道。她脸上满是纯粹的喜悦。

或许是昨天哭了个够,对此有了些抵抗力,这回她没有掉眼泪,但那双比以前更加晶莹的焦糖色眼眸却颇为显眼。\\

「我觉得真昼再贪心一点也没问题」

「是啊是啊。为了吸引对方所做的任性可是完全没问题的哦,昼儿」

「……我感觉你又在灌输奇怪的事情了。不过先不说这个,我也觉得真昼可以再贪心一点,多撒撒娇也没关系的」\\

认识真昼的人,任谁都会认同:她是一个内敛又倾向于忍耐的人,不会主动去向别人要求或是索要什么。

尽管她有时会评价说自己任性,但可以肯定的是,真昼的任性都属于可爱的范畴,如果那些都算任性,那世界上就尽是贪婪的人了。

出于她自身的特点,由于什么事情都是她一个人处理过来的,她也有相应的能力,因而不擅长依靠他人。周作为男朋友,打算继续努力让她更加依赖自己。\\

「那样的话周就会更宠你吧,绝对的」

「……我、我考虑一下。而且,昨天就一定宠了很多了」

「嚯,细说。你看这儿也没外人」

「喂」\\

虽然没发生什么见不得人的事,但两人独处时发生的事情被各种打探,终究是很让人难为情的。

文化节之后的那次留宿,周相信真昼是没讲出详情的,就怕她挡不住后续千岁的循循善诱,之后得问问她有没有泄露出去。

尽管周有劝阻,但千岁依然带着轻快,甚至是清爽的笑容凑近真昼。\\

真昼露出了看起来为难,却又没有拒绝之意的含糊笑容,被千岁抱住,就这么让她带到了稍远一些的位置。「我就说吧」周作为男朋友扶额。\\

真昼会与人分享快乐之事,除非是真的需要保密的事情,只要亲近的人多问问,她就会讲出来。很担心她会说出什么……忽然,周看到坐得离真昼最远的门胁有些苦恼地垂下眉梢。\\

「门胁,怎么了?」\\

周小声问道。门胁这才发现自己刚才是什么样的表情,露出更加为难的笑容。\\

「呃,我只是在想,我被叫过来真的没问题吗?」

「门胁君啊,就是太拘谨,这会儿客气起来了。不过这种话可不适合现在说哦~」\\

尽管声音不大,但木户也听到了。她好像告诫一般看向门胁,温柔眼神中却又没有任何生气和愤怒。\\

「抱歉」

「露出这种抱歉的表情反而不好吧?」

「非要说的话,我跟椎名相处的时间比门胁君还短呢,那话要说也是我说才对」

「但木户既没什么压力,也不担心呢」

「嗯,从态度上感觉得出椎名还挺认可我的,所以也没什么担心的。一旦深入到椎名的内心,就特别好懂」\\

木户轻松自然地笑了出来,足见其对自己的发言深信不疑。

木户曾言及自己的兴趣是观察人类,本来以为是个只看肌肉的家伙,没想到她也同样掌握了这种细微之处。周很高兴有更多人能理解真昼,同时也因许多地方被看穿而感到心里痒痒的。\\

「木户在这种地方还挺自信的」

「与其说是自信,不如说是感受到了椎名的善意,也能感觉我俩是朋友关系。还有椎名知道我是爱小总的,不会有什么多余的怀疑,心里也踏实」

「原来如此」

「门胁你怎么这会儿好像懂了什么啊」

「就是那个嘛」

「就是那个嘛」

「搞啥啊」

「文化节的时候,藤宫君周围不是各种大呼小叫的嘛?」

「大呼小叫……?」

「就是,那人是不是还挺不错的?这种感觉」

「这我没法认啊,旁边就有个受欢迎到不讲道理的家伙」\\

模拟咖啡厅里,声音绝大多数都是在称赞门胁的。周虽然不至于完全没收获称赞,但都掩埋在了对门胁热烈的甜言蜜语中,并且夸周的声音也不包含甜蜜的成分。\\

尽管如此,木户却是啧了几声,摇晃着竖起来的食指,温暖的眼神似是在说「你不懂」。\\

「藤宫君,你们类型不一样啦」

「类型」

「门胁君是那种,清爽王子系优秀青年的顶层」

「我是不是这里必须点头才行啊」

「门胁君请点个头。然后,藤宫君第一眼看上去是那种不合群又扫人兴的类型,文化节把一百分满分的笑容免费送了出去,这个温差产生了作用」

「说什么呢」

「我是认真的!自称不显眼的藤宫君在文化节用端正的仪表摆出笑脸,曾经不为人知的一面产生了反差,会让一部分人心动的」

「门胁,你懂吗?」

「唔」

「为什么这句不肯同意呀~!」

「是、是说,藤宫确实比他自己以为的更加引人注目,不考虑椎名的影响也是一样。那就是椎名担惊先前担惊受怕的点,不如说现在也是一样。意思就是说,没有这种担忧,对于形成友好的关系而言,是很大的优势吧」

「对的对的。门胁君很擅长表达」\\

「给你一朵小红花」木户满面笑容地拍手,对着傻眼和困惑各占一半的周露出纯真的表情。\\

「总之我觉得椎名就是考虑到这一点,才会愿意跟我交朋友,拉近关系的」

「说的对不对先不论,我明白木户的意思了……但我觉得这并不是全部」

「不是全部?」

「就先假定存在这一点原因,我也觉得,真昼大概是因为觉得你是个好人,中意你,才和你交好的」

「咦?」\\

或许是周说出的话让她感到意外,木户发出了呆愣的声音。

周倒是很疑惑为什么这会儿木户会感到疑惑。他思索片刻,想清楚该怎么向真搞不清状况的木户解释之后,缓缓开口:\\

「木户你想,你很会照顾人,而且总是和颜悦色的,讨人喜欢,看到别人遇到困难都会帮忙。而且你还很善于观察,分得清别人什么时候需要帮忙,什么时候不需要,帮得总是恰到好处」\\

真昼的怕生,弄不好比周更加根深蒂固。\\

表面上,她能做到完美的社交,也能做出惹人喜爱的举止,但不会轻易让人踏入外壳的内侧。千岁得以强行突破,少不了她和周相互认识的成分在里面。\\

而周能与她变得亲近,完完全全就是一个奇迹。先不谈这一点,总之真昼本身就拥有拒人千里的气质。\\

即便是她放弃扮演天使大人的当下,这一气质也仅仅是缓和,并未改变。尽管她比以前更加融入同学的圈子,却也没到讲真心话的地步……终究是木户柔和且爱照顾人的气质让真昼感知到了吧。\\

「表里如一,总是面带笑容,保持适当的距离,又会不着痕迹地关心别人。碰上喜欢的事情容易用力过猛,但反过来说,也是专一到了能忘记沉迷其中的地步,并且这一切都是建立在不给别人添麻烦的基础上的。这些都是真昼所欣赏的点,所以她才会想跟你搞好关系」\\

真昼总体上是更钟爱温和、符合常识的人,而木户正完全符合这一条件。

当然,也看得出真昼会跟她保持交流并非是喜好原因,而是在相处过程中认可了她的人品。不然真昼也不会邀请她到家里来。虽然是周的屋子,但在真昼心中已经和自家没有分别,能同意她进来,显然说明真昼已经敞开了心扉。\\

「真昼经常会有些担心我交友关系太窄,不过要我说,真昼那边能交心的朋友才更少。我很高兴像木户这样的女生能跟真昼交上朋友,也很高兴能跟木户搞好关系」\\

周只不过讲出了自己的想法,木户听完却垂下眉梢看向她,一脸有话想说的样子。这又是怎么回事?周又看向门胁,他则是耸耸肩,脸上带着藏不住的无奈和佩服。

搞不懂他什么意思。\\

「……能当面说出这种话,真厉害啊,我都难为情了。能把那些话不加一点修饰地讲出来,也难怪椎名会那么喜欢你。就是椎名也难免会担心了」

「我也有同感」

「门胁你同意个什么」

「哎呀」

「哎呀」

「我说你们」\\

本以为这两人还没有那么多交集,可今天却不知为何。两个人却意气相投,异口同声,弄得周都有些困惑了。\\

「果然吧,藤宫对要好的人是不是超坦率的?」

「哪里坦率了,要说也是别扭吧」

「你的认识很扭曲呢」

「是啊是啊」

「我说你们」

「唔,藤宫君,我知道你没有别的意思,就是说,你偶尔会说得特别直白。说难听点,就算你没有那个意思,也没准就会在意料之外的地方有女生中招」

「……所以说?」

「也是有可能在你不知情的时候被其他女生喜欢上的。你自己没感觉,女朋友可是很担惊受怕的」

「……我?」

「啊,这家伙没信」\\

默默听着的树似乎是忍不住吐了个槽,周往他那边看去,正巧这时真昼和千岁也聊完来到了旁边。\\

「我觉得今天来的大家都认可周君的魅力」

「里面估计是掺了真昼百分之百的偏心」\\

真昼加入了对话中。周露出怀疑的表情,见真昼看过来,他又本能地摆正了姿势。

她,笑容非常灿烂。\\

「周君」

「嗯」

「关于你对自己评价过低这件事,之后我有很多话想说,不过现在暂且不提……周君现在一直磨练自己,至少比以前能更令自己满意了吧?」

「嗯」

「大家都认可现在的周君,就不应该随意否定他们的称赞」

「我会记在心里」\\

周可以理解真昼的说法,也没想反驳,但依旧不免觉得这些都是过誉。

只不过,真昼像是连这一点都看穿了一样,露出美丽的微笑,于是周身子一抖,赶走多余的思绪。\\

「看得出你在椎名面前很没辙啊」

「闭嘴」\\

周自己最清楚自己拿真昼没辙,没必要让树特意讲出来,这么一来反而叫人火大。他明知是自己的不是,依旧用比较尖锐的语气回应,树则是摆摆手,一副「我都懂」的样子。\\

「不过我倒是没觉得周君……会、会对其他女生感兴趣。周君眼里只有我一个」

「哇哦」

「不要打岔」

「是」\\

真昼果断地打断了千岁的起哄,周心里稍微感到有些痛快,这应该也无可厚非吧。\\

「……但话说回来,那个,有其他女生接近你的话,我会有些不舒服。周君要是能多注意一下,我就会更放心点……虽然这些都很任性」

「我怎么会主动接近别人让你不放心呢?之后我会注意距离的」\\

周重新认识到,真昼的一些言行果真是在担心这一方面。于是他也为了再次消解真昼的不安,带着无可动摇的自信做出断言。\\

被真昼之外的女性吸引是绝无可能的,至于为了考验真昼的爱情之类的原因而追求其他女性就更加不可能了。周还没有愚蠢或是鲁莽到去做出试探行为的地步。

有人吃自己的醋,要说完全没一点高兴也是骗人的,但吃醋也就代表了他令真昼感到了不安,作为男朋友,主动做这种事正可谓岂有此理。\\

「不如说,我跟真昼在交往都已经是众所周知的事实了,真会有人来追我吗?」\\

由于回想起来会很难为情,周不太会积极地去回忆过去,但要说起来,周和真昼交往的契机是体育节的借物赛跑。

在比赛时,他在众目睽睽之下被宣布是「重要的人」,恐怕全校学生已经无人不知、无人不晓了。

除此之外,周还带着些牵制的目的,堂堂正正地站在了真昼身边。而且按树的说法(虽然不太想承认),他们还「恩恩爱爱」的,客观上看,应该已经没有了能给其他人介入的空隙才是。\\

「哪怕是现在,也有人说喜欢昼儿的哦,虽然少了挺多就是了。放到周身上,应该也不无可能?」

「唔。不过目前为止还没有」\\

周之所以没有完全接受木户的说辞,一部分原因也出在这里。

真昼会夸奖他,树他们也说周变了,但要说有没有所谓的桃花运,那毫无疑问是没有的。除了真昼,周目前还没有遇到有哪个女生说喜欢他的。\\

文化节的时候,周的确听到了一些称赞之词,但在那之后没有发生什么,同学们与他相处也并无不同。\\

因此,就算被称赞,周也没什么感觉,每次都会感到疑惑。而后千岁一声伴随着叹气的声音「你真是不懂啊」打断了他的思绪。\\

「就是那个,周虽然不会像昼儿一样人见人爱……但,怎么说来着」

「有可能会被很小一部分人动真格地追求」

「想象不出。而且就算真有,我也不打算接受」

「总之,意思就是说叫你注意点。倒也完全不担心你会移情别恋就是」\\

周完全没打算关心真昼之外的人,就算别人向他表达好感,他也没办法回应。在他困扰地轻轻皱眉的时候,千岁静静地向他发出提醒,而且没有捉弄、没有笑他,也没有傻眼。周只好老实地点了点头。
