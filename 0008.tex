\subsection{天使大人既环保又平民}

「啊」\\

背后传来了一声银铃般的嗓音。\\

虽说周最近已经听惯了这声音,但这里并不是公寓,而是附近超市的零食区。\\

毕竟是有人在的地方,周没有料到真昼会对他有所反应,便困惑地回过头去,然后看见瞪圆了眼的真昼站在那里。\\

她手上提着个超市的购物篮,里面放着萝卜、豆腐、鸡腿肉和牛奶这些晚餐的食材。

按这情况看来大概是她在凑巧路过零食区时刚好遇上了周吧。\\

「先说好,这是偶然啊。我可没跟踪你」

「我知道。毕竟这里是最近的超市,这种事情我还是明白的」\\

面对周的抢先声明,真昼小声感叹着「倒是你为啥会想到那边去啊」,接着看向了手里拿着的记事本。

认认真真记下需要的东西,还真像做事一丝不苟的真昼的作风。\\

她在仔细确认了写在那有着可爱花纹的记事本上的内容后,没有看向零食区而是朝着另一边的调味料货架望过去。

「酱油和甜酒」,真昼发出了这样惹人怜爱的声音,寻找着家庭用品。尽管她的姿态确实可爱,但周更多的是觉得有些不可思议。\\

「甜酒在这边,喏」

「不是那个,我要找的是甜酒味的调料。甜酒未成年人买不了的」

「这个也算酒啊」

「甜酒是当作甜味的酒来对待的。料酒的话,因为加了盐没法直接喝,所以未成年人倒也可以买」\\

周拿起甜酒想要递给真昼,但她却摇了摇头,把甜酒味的调料放进了购物篮。

几乎不做家务的周第一次听到这种事情,不由得应和了一句,同时盯着她麻利行动的背影。\\

真昼仔细看着摆放酱油的货架,然后留意到了写着价格的标牌,低声嘟哝着皱起了眉。\\

「……大特价一人仅限一瓶……」\\

真昼似乎是想再买一瓶备用。惋惜地叹了口气后,她看向了这边。\\

「……那我也买一瓶咯」

「能懂我的意思真是感谢」\\

周察觉到真昼的话中之意,苦笑着拿起了瓶酱油,而真昼则满意地把嘴唇弯出一道微微的弧线。\\

「……意外地节约啊」

「节约……不如说是能省就省吧。冤枉钱不该花吧」

「怎么说呢,真是有日本人的气质……不过,过着拿父母钱的生活的话一般都会这样的吧」\\

周虽说是独自生活,但经济上还是要依靠父母。

周生在比较富裕的家庭,因此才能够住在那样整洁安全的公寓里。因为这样,周真的非常感谢父母。由于他不仅要交学费,生活费也要花掉不少,所以无谓的花费他都尽量避免着。\\

「……嗯。毕竟经济上还不独立,还是保持节制为好」\\

真昼淡淡地回答之后,整理起了篮子里的东西。她的声音里毫无热度。\\

真昼的声音突然变得毫无起伏,有点吓人,但等她再抬起头时,脸上的表情已恢复到一如往常。

刹那间瞥见的那黯淡的眼神,已经看不见了。\\

「……话说,你要买这个?」\\

像是要改变话题似的,真昼看向周提着的篮子里的速食米饭和土豆沙拉问道。\\

虽然从真昼那得到的晚饭确实好吃,但就凭那点量是不够的。所以,周平时就像这样,买些米饭当主食并来点沙拉当配菜。\\

「晚饭嘛」

「不健康」

「别啰嗦。不是买了沙拉么」

「虽然是土豆沙拉呢……为什么过着这样的生活都搞不坏身体呢……」

「你太关心别人了」\\

像是在表示周该多吃点蔬菜一样,真昼眯细了眼睛朝着周放出了无言的压力,而周则扭开了头当作耳旁风。\\

这样那样地说了一堆话的时候,周结完了账,并用塑料袋装好了东西,而这时真昼却从包里拿出环保袋将东西放进去。

真是个爱护环境的平民天使大人啊。\\

可是虽然买的东西都装进去了,但那个分量还是让周略微感到担心。

牛奶、酱油加上甜酒味调料,这就已经有三升了,虽说和水的密度有差异但少说也有三公斤了。这之上还买了食材,特别是那根大萝卜,想必一定很重。\\

虽说有好好地装进去并扎好口,但如果要就这么提回公寓的话,怎么说都还是件体力活。\\

(结果上看是因为有我才让她多买了这些调料和食材啊)\\

估计她是做了比平时还多的量,然后分给自己的吧。一直以来,周得到的已经接近一个人的份量了。虽然她表示只是因为做多了,但最近这段时间她应该都是特意多做的。\\

结果上看,周给她添了相当多的麻烦。如果现在还什么都不做的话,作为男人就太丢脸了吧。\\

看到真昼扎好了袋子,周便试着提了提,虽说对自己而言不算太重,但若是让女孩子提的话那就是挺重的负担了。

真昼虽然擅长运动,但纯粹比力气又是另一回事了。不如说隔着衣服都猜得到那双细手不太可能有多大的力气。\\

看见周的行动,真昼眨了眨焦茶色的眼睛。

比起惊讶,更多的是感谢的氛围。\\

「……不是要抢你的啦」

「我不是担心那个……就这点我还是拿得动的哦?」

「这种情况还是坦率地接受会比较可爱哦」

「说的好像我不可爱一样」

「你看看你在学校什么样,对我又是什么样」\\

也许是真昼对此也有认识,周看到她稍稍退缩了一点。\\

她在学校里表现出的那让所有人都认可的友善、温柔又谦虚的样子,在周面前却没有展现出来。\\

准确来说,真昼对周倒也很温柔,就是说话比较直白罢了。她似乎对周根本没留委婉的余地,说话一直是有什么说什么。

不过这总比说谎要好得多,所以周并没有多在意。\\

周看见真昼默不作声,认为机会正好,便提起了装满了食材的环保袋和自己的袋子,快步走向了出口。

背后似乎有些慌乱的动静,但周并未搭理,也不管两人拉开的距离,径直向前走去。\\

周并没有调整脚步等待真昼。

本来超市里就已经呆在一起,要是两人还并排着一起回家,并且被谁看见了的话,事情恐怕会变得很麻烦吧。

对彼此来说,这个距离才是最理想的。\\

周装作跟真昼没有任何关系的样子,提着大袋大袋的东西匆匆往前走着。这时,他仿佛听见了后面传来的一句小小的「……谢谢你了」的声音。
