% \subsection{67 天使様と新学年}
\subsection{67 天使大人与新学年}

% 「今年は勢揃いだな」\\


% 壁に張り出されたクラスの名簿を見て、樹はにやにやと笑いながら周に声をかけた。\\


% 沢山の名前が羅列された紙に視線を向けると、周と同じクラスに樹も千歳も、真昼も居た。あと門脇も居たので、天使様と王子様が揃っている事となる。


% 周りでは「椎名さんと一緒のクラスだ!」「王子とクラス同じだった!」などと一介の生徒が対象とは思えないような野太い声や黄色い声が各地で上がっていた。\\


% 樹のにやにや笑いは、真昼と周が一緒のクラスだからだろう。\\


% 「騒がしいクラスになりそうだな」\\


% からかわれるつもりはないのでそれだけ返して自分のクラスに向かう周に、樹は苦笑してついてくる。


% 内心では今年も樹が居てくれて助かった、とは思っているのだが、確実に笑われるため内心に留めておいた。褒めたら調子に乗るタイプなので、適度に流しておく方が身のためなのだ。\\


% クラスに入ったら早速と言わんばかりに天使様と王子様に人だかりが出来ていて、笑うしかない。


% 相変わらずの人気だな、と樹と並んで笑っていると、門脇がこちらが入室したのに気付いたのか相変わらずの爽やかな笑みを浮かべた。\\


% 「よ、樹と藤宮は今年も一緒のクラスだな」


% 「今年も優太と一緒だなあ。中学生の時から含めると三年連続だわ」


% 「そうなのか?」


% 「ん。同中だから」\\


% それなりに親しげだとは思っていたが、どうやら出身校が一緒だったらしい。\\


% 「今年もよろしくなー藤宮」


% 「おう」\\


% 誰にでも人当たりがいい門脇は、大して関わりもない周にも屈託ない笑顔を向けてくれるので、これがモテる秘訣か……と内心感心してしまった。\\


% 周囲を囲まれて尚平然とした態度の門脇に呆れ半分称賛半分の気持ちでいつつ返事をして、自分に宛がわれた席につく。\\


% 真昼の方は、見ない。


% あらぬ疑いをかけられても困るし、今は男女関係なく囲まれているので、彼女は彼女で忙しいだろう。


% こちらに気付いてもいないだろうし、特にアクションを起こす必要性はない。\\


% 「おはー! 今年は一緒のクラスだねー!」\\


% 席で提出書類に不備がないかの確認をしていたら、微妙に寝坊したらしい千歳がやってきた。


% 今年は千歳も樹も一緒のクラスなので、さぞ騒がしく胸焼けする日々が始まる事だろう。\\


% 「おはよう。今日は樹と来なかったんだな」


% 「うん、寝坊したー。やー新学期ってうっかり忘れかけててママに起こしてもらったんだよねー。いっくんは?」


% 「さっき自販機に向かってった」


% 「おけー。ミルクティー頼んどこ。あっ、まひるんまひるん! 今年は一緒のクラスだからよろしくねー!」\\


% 誰にでも物怖じしない千歳は、人だかりの中央に居る真昼にぶんぶんと手を振りながら突撃していった。


% まひるんというあだなに周囲が固まっていたものの、真昼が普通に受け止めて天使様の笑顔を浮かべているので、許されていると分かったのか微妙に羨ましそうな眼差しを向けていた。\\


% 「今日どうせ早いし帰りにクレープ食べよー! 駅前のクレープ美味しいんだよー」


% 「そうですね、私でよければ」\\


% 気のせいか、こちらを見られた気がしたが、周としては別に一々こちらの許可を得ずとも行ってくればいいだろうし制限する気も権限もないので、好きにしてほしい。


% 昼食はファストフードなりコンビニなりで済ませばいいだろう。\\


% 実に健全な友達付き合いをしているので、微笑ましさすら感じる。


% 千歳のこういうところはとてもよいと思っているので、あまり他人と遊ばない真昼を疲れない程度につれ回したりして楽しんでほしい。


% 千歳が同じクラスになって一番よかったのは、真昼かもしれない。\\


% 千歳の勢いに押されながらも楽しそうに微笑んでいる真昼を遠目に眺めて、周も少しだけ口許を緩めた。

