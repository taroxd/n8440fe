\subsection{天使大人的奖励}

%廊下の壁に貼られている学生名が多数綴られている紙を見て、周は「まあこんなものか」と呟いた。\\
望着走廊里贴着的写着许多学生名字的纸,「嘛也就这么回事吧」周轻声叹息道。\\

%先週行われたテストの順位が出ていたので、周は同級生と同じように見に来たのだ。
上周考试的成绩出来了,于是周便随着同年级的同学们一起来看了。

%結果としてまあいつも通り二十一位、割とよいがさほど目立たない順位に居た。手応えとしても今までと変わらないとは思っていたので、予想通りの順位に居て少し安堵した。\\
要论结果嘛第二十一名,看和往常差不多,看着还行,但不怎么显眼。写题的时候手感上跟以往没有什么变化,现在看到排名也一如既往周也稍稍安心了下来。\\

%ちなみに、真昼もいつもと変わらず一位に君臨している。\\
顺带一提,真昼依旧雄踞年级第一之位。\\

%本当に才女なのだが、努力を欠かしていない事もよく知っているので、流石としか言えない。
深知她实在是个才女,加之学习上也十分努力的周,只得叹服不愧是她。

%夕食後に勉強しているのもよく見る。\\
就连晚饭后也经常看见她学习的身影。\\

%元々の頭の出来がよいのもあるだろうが、本人のたゆまぬ努力が真昼を一位の座に置いているのだろう。\\
虽然她也确实脑子好用,但要论将她送上第一宝座的,果然还是坚持不懈的努力吧。\\

%「椎名さんまた一位だ……」
「椎名同学又是第一啊……」

%「流石天使様、頭の出来が違う」\\
「不愧是天使大人,脑子实在好用啊」\\

%喧騒の中からそんな声が聞こえてきて、周は唇を真横に結んだ。\\
听见混在喧嚣中的这般论调,周不禁撇起了嘴。\\

%「……どうしたんだ周、そんな顔して。順位悪かったのか?」\\
「咋啦周,一脸不爽的。排名不妙?」\\

%一緒についてきた樹が、周の様子を見て訝る。
和周一起的树看见周的样子,有点惊讶。

%ちなみに五十位までしか貼り出されないため、樹は自分のではなくあくまで周の付き添いといった形だ。\\
顺带一提名单只有前五十名,所以树其实只是陪着周过来看。\\

%「なんでもねえ。二十一位だった」
「没啥。二十一」

%「おお、今回は前よりよかったんだな」
「哦,这不是比上次还好了点嘛」

%「多少な。誤差範囲だろ」
「差不多吧。这点只是误差」

%「おうおう賢い人は言う事が違いますなあ」\\
「哎呀哎呀聪明人说的话感觉就是不一样」\\

%嫌みを笑いながらわざとらしく言ってくる樹には「はいはい」と軽く流すだけにしておき、改めて順位表を見る。\\
周「好好好」随意地应付着故意边笑说着烦人话的树,再次看向排名表。\\

%本当に、よく頑張っていると思う。
看起来是真的有好好努力过了啊。

%あまり見せたがらないが、隠れたところで努力をする彼女は、こんなの当たり前のように見せているが相当に頑張っているだろう。\\
虽然很少被人看见,但对在暗处默默努力的她来说,即便别人看上去理所当然,但这也是她付出了莫大的努力才得到的成果吧。\\

%周りはすごいね、と褒めてくれはするが、彼女の努力を知らないが故に労うという事はしない。
即便周围的人会夸奖她「真厉害」,但却对她付出的努力一无所知,因而也犒劳她的努力。

%それは、とても真昼にとって息苦しいのではないだろうか。\\
对真昼来说,这才是最为痛苦的事情吧。\\

%「……せめて、俺くらいはな」
「……至少,我来补足下吧」

%「ん?なんか言ったか?」
「嗯?你刚说了啥」

%「別に。ほら教室に戻るぞ」
「没啥。喂,我回教室咯」

%「ういっす」\\
「好嘞~」\\

%\\
\vspace{2\baselineskip}

%「あれ、周くんこれ何ですか?」\\
「哎呀,周君,这是什么?」\\

%スーパーから直接周宅にやってきたらしい真昼が食材を冷蔵庫に入れようとして、どうやら見慣れぬ白い箱に気付いたらしい。\\
看上去从超市回来直接就进了周家里的真昼,正打算把买来的食材放进冷藏库,结果注意到了这多出来的白箱子。\\

%「ん?ああ、ケーキ」\\
「嗯?啊,蛋糕啊」\\

%白い箱の中身は、ケーキだ。恐らく真昼も箱の形状で薄々察していたとは思うが、一応聞いてみたのだろう。
白箱子里面放着的,是蛋糕。估计看见箱子的形状真昼多少也料到了,不过还是问周确认一下。

%ちなみに、千歳がよくSNSにアップしているお気に入りのパティスリーに行って買ってきたものだ。\\
顺带一提,周是特意跑去千岁经常在社交网站上发的喜欢的糕点店买来的。\\

%「……ケーキ好きなのですか?」
「……你喜欢吃蛋糕吗?」

%「いや別に。お前に買ってきた」
「倒也不算。这是给你买的」

%「なんでまた」
「又是怎么了」

%「お前学年一位だったからささやかなお祝いくらい、いいだろ。一位おめでとう」\\
「你不是考了年级第一嘛,稍微庆祝下咯。年级第一,祝贺祝贺」\\

%自分に、というところで目をしばたかせている真昼。
听见是给自己买的的时候,真昼疑惑地眨了眨眼。

%本当に、想定外だったのだろう。\\
看来是真的很出乎意料吧。\\

%「い、一位は毎回取ってますし、そこまでめでたい事でも」
「其,其实每次都考第一,并没有什么好庆祝的」

%「それでもいつも頑張ってるし、たまにはご褒美って形もいいんじゃないか。ショートケーキだけど嫌いじゃないか?」
「就算是那样,你也一直在努力,偶尔来些奖励不也挺好的嘛。草莓奶油蛋糕不喜欢吗?」

%「え?き、嫌いではないですけど……」
「哎?倒,倒也不会不喜欢……」

%「ん、ならよかった。食後に食ってくれ」\\
「嗯,那就好。吃完饭来吃咯」\\

%呆気に取られている雰囲気が伝わってくるものの、周はそのまま会話を打ち切る。\\
即便查觉到真昼吃了一惊说不出话,但周还是就这样结束了会话。\\

%あまり真昼は気を使われ過ぎると困る様子を見せてくるので、あっさりとした態度の方がいい。\\
要是太过顾虑真昼反而会让她陷入困惑,所以态度还是干脆一点为好。\\

%彼女は他人には割と尽くすタイプの人間だが、自分の事となると非常にストイックで滅多な事では自分を甘やかさないタイプだとも思う。
在周看来,真昼这个人在对待他人上算是很尽心尽力的类型,但对待自己来则是十分的禁欲,很少让自己放松的类型。

%誰かが褒めたり労ったりしないと、真昼は研鑽ばかりで息を抜けない。恐らく、基本的に甘えるという行為を知らないのではないかと思う。\\
要是没有谁来表扬、犒劳一下她,那她就会一头扎进要做的事情里而不知休息。恐怕,她甚至连撒娇这一行为本身都不曾知晓。\\

%長く彼女と居た訳ではないが、なんとなく彼女の性質が分かってきたので、いつも世話をされている分ちょっとでも返せたらと思うのだ。\\
虽然周和她的相处不算太久,但多少也搞清楚了她的性格。一直都受她的照顾,一点也好,周也想有所报答。\\

%キッチンでまだ固まってる真昼に苦笑した周は、ゆっくりと息を吐いて彼女の再起動まで彼女を眺める事にした。\\
想着仍呆在厨房里的真昼苦笑的周,轻轻地叹了一口气,在她重启之前一直都看着她。\\

%\\
\vspace{2\baselineskip}

%食後、微妙に緊張した面持ちでケーキを皿に乗せて持ってきた真昼に、周は思わず吹き出した。\\
饭后,看着一脸微妙地紧张着把放着蛋糕的盘子端过来的真昼,周不禁笑漏了嘴。\\

%「な、なんで笑うのですか」
「为,为什么要笑啊」

%「いや、なんでもない」
「没,没啥」

%「なんでもなくない気がします」
「感觉就不像是什么也没有」

%「気にすんな」\\
「别在意」\\

%ただ、真昼が妙にカチコチと強張っていたのが面白かっただけだ。
不过是看着真昼紧张的动作发硬的怪样子感觉有些有趣。

%ただあまり笑いすぎると不機嫌になりかねないし労うという目的が果たせなくなりそうなので、ほどほどのところでやめておく。\\
但要是笑的太过了会坏了真昼的心情,那原本犒劳她的目的就达不成了,于是周笑的差不多就停了下来。\\

%一緒にコーヒーを持ってきてケーキと共にテーブルに置いた真昼が、隣に腰かけてくる。
顺带把咖啡也拿了过来和蛋糕一起放在了台子上的真昼,坐在了周的旁边。

%そこでも微妙にぎこちない動きだったので笑いそうになったものの、隣にいるので流石に控えておいた。\\
这些动作也微妙地显得不自然,令周想要发笑,但毕竟真昼本人就做在旁边,现在还是忍住为好。\\

%ちら、と真昼が遠慮がちに周を見上げてくる。\\
真昼稍稍转头,畏畏缩缩地看向周。\\

%「ん、おめでとさん」
「嗯,祝贺祝贺」

%「……ありがとうございます。でも……」
「……谢谢你。不过……」

%「いいから素直に受け取っとけ。頑張ったのは事実だろ」
「好啦好啦你就乖乖收下吧。毕竟你也确实努力过了咯」

%「そう、ですけど」
「虽然,是这么回事」

%「ほら、さっさと食べとけ。たまにはお前も自分甘やかしとけ」\\
「那就行咯,快吃吧。偶尔也让自己放松下嘛」\\

%もう買ってきてお前にやったんだから、と付け足すと、真昼も少し申し訳なさそうにしつつもちいさく頷いて、ケーキの載った皿とフォークを手に取る。\\
「反正已经买了给你了」周这么补了一句,真昼才以略带抱歉的神情微微点头,拿起了盛着蛋糕的盘子和叉子。\\

%「ありがたくいただきます」
「感激不尽」

%「どーぞ」\\
「请吧」\\

%手をひらりと振っておいたら、真昼は何だか慎重な手つきでケーキをフォークで一口大に切って口に運んだ。\\
周轻轻地摆摆手,真昼则不知为何以慎重的动作将蛋糕切成一口大小送进嘴里。\\

%女子はあまいものにうるさいというイメージがあるが、千歳もよく食べてる店のなら問題ないだろう。\\
虽然印象里女孩子对甜食十分挑剔,但既然是千岁都常吃的店那应该就没问题吧。\\

%その証拠に、口にした真昼が目を少し丸くして、それから微かに口許が緩んだ。
尝了一口的真昼,稍稍睁大了眼,然后微微放松了嘴角,恰是证明了这一点吧。

%あまり表情が変わる事のない真昼だが、最近は少しずつ分かりやすく喜怒哀楽を表現してくれるようになっている。\\
虽然真昼很少有表情变化,不过最近她也开始变得慢慢会流露出易懂的喜怒哀乐了。\\

%ゆっくり食べながら浮かぶ柔らかな表情は、食べているだけで絵になっていた。\\
真昼慢慢吃着蛋糕,脸上浮现出的柔和表情,让这不过是吃东西的场景变得好似一幅画作。\\

%「……?どうかしましたか」
「……?怎么了吗」

%「いや、なんでも」\\
「不,没什么」\\

%つい凝視してしまった事に気付いたらしい真昼が、不思議そうに首をかしげている。
突然发现自己被盯着的真昼,不解地歪起了头。

%いつもより少しだけ幼さの見える表情に、周は先程まで見つめていたのに視線をさまよわせてしまう。\\
那比起平常稍显幼稚的表情,令周方才还在盯着看的视线不自觉地迷离。\\

%入れ代わりにじっと周を見ていた真昼は、ふと思い付いたようにフォークを使ってケーキを一口分取って、周の方に向けた。\\
取而代之盯着周看的真昼,则是突然想起了什么一般,用叉子叉起一口蛋糕,朝着周这边伸了过来。\\

%いわゆる、あーんという体勢に入っている。\\
变成了所谓的啊~的姿势。\\

%「え、い、いや食べたかった訳じゃないというか」
「唉,不,我不是想要吃,是说」

%「違いましたか?」
「不是这样吗?」

%「……いや、まあ、その……もらえるなら、もらうけど、さ」\\
「……呃,嘛,那个……要是送过来了,我还是会要的,啦」\\

%流石にこれは想像していなくて見るからにうろたえてしまった挙げ句、うっかり承諾してしまった。\\
这样的场景周实在是想都不敢想,现在突然发生在眼前搞得周很是狼狈,最后含混着同意了下来。\\

%この年にもなって、それも異性に、その上とんでもなく美少女に食べさせてもらうなんて、ある意味幸運なのかもしれないが――素直に喜べるほど、周は羞恥心を捨てていなかった。\\
毕竟已经是这个年纪,更何况对方还是异性,再加之要被不得了的美少女喂食,某种意义上说不定算是幸运啊——老实说是很高兴,但周还是放不下自己的羞耻心。\\

%「元々周くんが買ってきたものですので、周くんにも食べる権利はありますし」\\
「本来就是周君买来的东西,周君你也有吃的权力啊」\\

%提案した真昼はというと全く意識していなさそうで、普段の表情のまま周の口許にケーキを差し出している。\\
如此提案的真昼似乎完全全没有意识到这些,仍以平常的表情把蛋糕伸向周的嘴。\\

%真昼を見ても不思議そうにしてるだけなので、周はええいと思いきってケーキに噛みついた。\\
就算看向真昼也只看得到她不解的回应,周便下定决心一口咬下了蛋糕。\\

%口に広がるのは、とても甘い味だった。\\
在嘴里泛开的,是无比甘甜的味道。\\

%「……あめえ」
「……好甜」

%「そりゃケーキですし」\\
「毕竟是蛋糕嘛」\\

%確実にそれだけではないのだが、真昼は気づかないだろう。
事实上并不只是那样,但真昼似乎没有注意到。

%もぐ、と噛んでもとにかく甘い。精神状態の影響がかなり大きい。\\
呜呣地嚼着,感觉总之就是甜。或许精神状态的影响也不小。\\

%「……何とも思ってなさそうなんだよなあ」\\
「……看来是什么都没感觉到啊」\\

%こちらはこんなにも甘さと恥ずかしさとむず痒さをいっぺんに食らっているというのに、真昼は至って平常通りだった。\\
这边可是甜味害羞味心痒味全部尝了个遍,真昼那边却一脸没事人样。\\

%それが地味に悔しくて、周は「ちょっと貸せ」と真昼の手からフォーク奪って同じようにケーキを差し出す。
这实在是令人不甘,周于是「稍微给我下」这么说着从真昼手上夺过叉子和刚才一样叉起蛋糕伸向真昼。

%やられたらやり返しておくべきだろう。\\
有借有还,被搞了怎能不还手。\\

%「ん」
「嗯」

%「……あの」
「……那个」

%「食べろ」\\
「吃了」

%いささか強めな口調で言ったせいか、真昼はおずおずと同じように餌を与えられる小鳥のごとくぱくりと口にした。
或许是因为周语气有些强硬的原因,真昼和刚才周一样的犹犹豫豫地,像是被喂食的小鸟一般一口吃下了蛋糕。

%じーっとその様子を見ていると、微妙に真昼の頬が赤らんでくるではないか。\\
周死死盯着真昼的脸,看见了那脸上微微泛起的红晕。\\

%「で、感想は」
「于是乎,感想如何」

%「お、美味しいです……」
「很,很好吃……」

%「ちがう。食べさせられた気分は」
「不是这个,是问的被喂的感受。如何」

%「……非常に居たたまれなくなりました」
「……感觉非常的害羞」

%「だろうな。こういうの、人にすると勘違いされるぞ。やるなら女子同士でやっとけ」\\
「是吧。所以说,做这种事可是要被人误解的啊。要做女生之间做做算了」\\

%俺の気持ちが分かったか、とぷい、とそっぽ向いた周に、真昼は消え入りそうな声で「はい……」と返した。\\
对问完「这下明白了我的感觉了吧」后呼地别过头的周,真昼以几乎听不清的声音「嗯……」地答道。\\

%安全な人間だと認識しているからあんな真似をしたのだろう。
应该是把自己当作无害的人看待,才做出那种事情来的吧。

%意識せずにやった真昼には困ったが、まあ悪い気分でもないのであまり責められたものでもない。
真昼这样无意识地做出这种事情让自己很困扰,不过感觉也不算坏,因而也没有什么特别要怪罪的。

%ただ、ひたすらに口の中に残る味が甘いだけで。\\
——只不过,那甘甜的滋味依旧在口中回荡。\\

%(油断されるのもこまったもんだ)\\
(太过不设防我也很难办啊)\\

%信用してくれるのは嬉しいものの、あんな風に無自覚に無防備にされたらたまったものではない。\\
被真昼信任本身是挺令人高兴,但那样不自觉而毫无防备地对自己做这种事实在是令人难以抗拒。\\

%そう結論付けて、周は隣で少し照れた風に縮こまってる真昼に小さくため息を送った。
得出这样结论的周,看着一边微微害羞地缩着身子的真昼,轻轻地叹了口气。\\