\subsection{天使大人的奖赏}

%廊下の壁に貼られている学生名が多数綴られている紙を見て、周は「まあこんなものか」と呟いた。\\

%先週行われたテストの順位が出ていたので、周は同級生と同じように見に来たのだ。

%結果としてまあいつも通り二十一位、割とよいがさほど目立たない順位に居た。手応えとしても今までと変わらないとは思っていたので、予想通りの順位に居て少し安堵した。\\

%ちなみに、真昼もいつもと変わらず一位に君臨している。\\

%本当に才女なのだが、努力を欠かしていない事もよく知っているので、流石としか言えない。

%夕食後に勉強しているのもよく見る。\\

%元々の頭の出来がよいのもあるだろうが、本人のたゆまぬ努力が真昼を一位の座に置いているのだろう。\\

%「椎名さんまた一位だ……」

%「流石天使様、頭の出来が違う」\\

%喧騒の中からそんな声が聞こえてきて、周は唇を真横に結んだ。\\

%「……どうしたんだ周、そんな顔して。順位悪かったのか?」\\

%一緒についてきた樹が、周の様子を見て訝る。

%ちなみに五十位までしか貼り出されないため、樹は自分のではなくあくまで周の付き添いといった形だ。\\

%「なんでもねえ。二十一位だった」

%「おお、今回は前よりよかったんだな」

%「多少な。誤差範囲だろ」

%「おうおう賢い人は言う事が違いますなあ」\\

%嫌みを笑いながらわざとらしく言ってくる樹には「はいはい」と軽く流すだけにしておき、改めて順位表を見る。\\

%本当に、よく頑張っていると思う。

%あまり見せたがらないが、隠れたところで努力をする彼女は、こんなの当たり前のように見せているが相当に頑張っているだろう。\\

%周りはすごいね、と褒めてくれはするが、彼女の努力を知らないが故に労うという事はしない。

%それは、とても真昼にとって息苦しいのではないだろうか。\\

%「……せめて、俺くらいはな」

%「ん?なんか言ったか?」

%「別に。ほら教室に戻るぞ」

%「ういっす」\\

%\\

%「あれ、周くんこれ何ですか?」\\

%スーパーから直接周宅にやってきたらしい真昼が食材を冷蔵庫に入れようとして、どうやら見慣れぬ白い箱に気付いたらしい。\\

%「ん?ああ、ケーキ」\\

%白い箱の中身は、ケーキだ。恐らく真昼も箱の形状で薄々察していたとは思うが、一応聞いてみたのだろう。

%ちなみに、千歳がよくSNSにアップしているお気に入りのパティスリーに行って買ってきたものだ。\\

%「……ケーキ好きなのですか?」

%「いや別に。お前に買ってきた」

%「なんでまた」

%「お前学年一位だったからささやかなお祝いくらい、いいだろ。一位おめでとう」\\

%自分に、というところで目をしばたかせている真昼。

%本当に、想定外だったのだろう。\\

%「い、一位は毎回取ってますし、そこまでめでたい事でも」

%「それでもいつも頑張ってるし、たまにはご褒美って形もいいんじゃないか。ショートケーキだけど嫌いじゃないか?」

%「え?き、嫌いではないですけど……」

%「ん、ならよかった。食後に食ってくれ」\\

%呆気に取られている雰囲気が伝わってくるものの、周はそのまま会話を打ち切る。\\

%あまり真昼は気を使われ過ぎると困る様子を見せてくるので、あっさりとした態度の方がいい。\\

%彼女は他人には割と尽くすタイプの人間だが、自分の事となると非常にストイックで滅多な事では自分を甘やかさないタイプだとも思う。

%誰かが褒めたり労ったりしないと、真昼は研鑽ばかりで息を抜けない。恐らく、基本的に甘えるという行為を知らないのではないかと思う。\\

%長く彼女と居た訳ではないが、なんとなく彼女の性質が分かってきたので、いつも世話をされている分ちょっとでも返せたらと思うのだ。\\

%キッチンでまだ固まってる真昼に苦笑した周は、ゆっくりと息を吐いて彼女の再起動まで彼女を眺める事にした。\\

%\\

%食後、微妙に緊張した面持ちでケーキを皿に乗せて持ってきた真昼に、周は思わず吹き出した。\\

%「な、なんで笑うのですか」

%「いや、なんでもない」

%「なんでもなくない気がします」

%「気にすんな」\\

%ただ、真昼が妙にカチコチと強張っていたのが面白かっただけだ。

%ただあまり笑いすぎると不機嫌になりかねないし労うという目的が果たせなくなりそうなので、ほどほどのところでやめておく。\\

%一緒にコーヒーを持ってきてケーキと共にテーブルに置いた真昼が、隣に腰かけてくる。

%そこでも微妙にぎこちない動きだったので笑いそうになったものの、隣にいるので流石に控えておいた。\\

%ちら、と真昼が遠慮がちに周を見上げてくる。\\

%「ん、おめでとさん」

%「……ありがとうございます。でも……」

%「いいから素直に受け取っとけ。頑張ったのは事実だろ」

%「そう、ですけど」

%「ほら、さっさと食べとけ。たまにはお前も自分甘やかしとけ」\\

%もう買ってきてお前にやったんだから、と付け足すと、真昼も少し申し訳なさそうにしつつもちいさく頷いて、ケーキの載った皿とフォークを手に取る。\\

%「ありがたくいただきます」

%「どーぞ」\\

%手をひらりと振っておいたら、真昼は何だか慎重な手つきでケーキをフォークで一口大に切って口に運んだ。\\

%女子はあまいものにうるさいというイメージがあるが、千歳もよく食べてる店のなら問題ないだろう。\\

%その証拠に、口にした真昼が目を少し丸くして、それから微かに口許が緩んだ。

%あまり表情が変わる事のない真昼だが、最近は少しずつ分かりやすく喜怒哀楽を表現してくれるようになっている。\\

%ゆっくり食べながら浮かぶ柔らかな表情は、食べているだけで絵になっていた。\\

%「……?どうかしましたか」

%「いや、なんでも」\\

%つい凝視してしまった事に気付いたらしい真昼が、不思議そうに首をかしげている。

%いつもより少しだけ幼さの見える表情に、周は先程まで見つめていたのに視線をさまよわせてしまう。\\

%入れ代わりにじっと周を見ていた真昼は、ふと思い付いたようにフォークを使ってケーキを一口分取って、周の方に向けた。\\

%いわゆる、あーんという体勢に入っている。\\

%「え、い、いや食べたかった訳じゃないというか」

%「違いましたか?」

%「……いや、まあ、その……もらえるなら、もらうけど、さ」\\

%流石にこれは想像していなくて見るからにうろたえてしまった挙げ句、うっかり承諾してしまった。\\

%この年にもなって、それも異性に、その上とんでもなく美少女に食べさせてもらうなんて、ある意味幸運なのかもしれないが――素直に喜べるほど、周は羞恥心を捨てていなかった。\\

%「元々周くんが買ってきたものですので、周くんにも食べる権利はありますし」\\

%提案した真昼はというと全く意識していなさそうで、普段の表情のまま周の口許にケーキを差し出している。\\

%真昼を見ても不思議そうにしてるだけなので、周はええいと思いきってケーキに噛みついた。\\

%口に広がるのは、とても甘い味だった。\\

%「……あめえ」

%「そりゃケーキですし」\\

%確実にそれだけではないのだが、真昼は気づかないだろう。

%もぐ、と噛んでもとにかく甘い。精神状態の影響がかなり大きい。\\

%「……何とも思ってなさそうなんだよなあ」\\

%こちらはこんなにも甘さと恥ずかしさとむず痒さをいっぺんに食らっているというのに、真昼は至って平常通りだった。\\

%それが地味に悔しくて、周は「ちょっと貸せ」と真昼の手からフォーク奪って同じようにケーキを差し出す。

%やられたらやり返しておくべきだろう。\\

%「ん」

%「……あの」

%「食べろ」\\

%いささか強めな口調で言ったせいか、真昼はおずおずと同じように餌を与えられる小鳥のごとくぱくりと口にした。

%じーっとその様子を見ていると、微妙に真昼の頬が赤らんでくるではないか。\\

%「で、感想は」

%「お、美味しいです……」

%「ちがう。食べさせられた気分は」

%「……非常に居たたまれなくなりました」

%「だろうな。こういうの、人にすると勘違いされるぞ。やるなら女子同士でやっとけ」\\

%俺の気持ちが分かったか、とぷい、とそっぽ向いた周に、真昼は消え入りそうな声で「はい……」と返した。\\

%安全な人間だと認識しているからあんな真似をしたのだろう。

%意識せずにやった真昼には困ったが、まあ悪い気分でもないのであまり責められたものでもない。

%ただ、ひたすらに口の中に残る味が甘いだけで。\\

%(油断されるのもこまったもんだ)\\

%信用してくれるのは嬉しいものの、あんな風に無自覚に無防備にされたらたまったものではない。\\

%そう結論付けて、周は隣で少し照れた風に縮こまってる真昼に小さくため息を送った。
