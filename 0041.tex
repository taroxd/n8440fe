% \subsection{41 天使様と初対面}
\subsection{天使大人与初次见面}

% 『明日周の家を訪ねてもいいかい』\\
『明天可以来周你家里吗』\\

% そんなメッセージが父親から送られてきたのは、三日、真昼が帰った後だった。\\
三号,真昼回去之后,周收到了这条父亲发来的短信。\\

% 『周が実家に帰らないのはいいのだけど、やはり私も顔くらいは見ておきたいからね。それに、志保子さんから聞いているけどお隣さんにもご挨拶は必要だと思うし』\\
『周你不肯回老家那就算了,但我还是想看看儿子的脸啊。顺便听志保子她说了感觉也得和邻居打个招呼呢』 \\

% 真昼の存在は母親からしっかりと伝わっていかに周がお世話になっているのか知っているらしく、親として挨拶をしておきたいとの事だった。\\
似乎是母亲她好好地把真昼这号人物和自家儿子受了人家多少照顾告诉了父亲,于是作为家长父亲便觉得有必要向她打一个招呼了。\\

% これが仮に志保子が知らない状態であったなら全力で拒否したのだが、もう知られている上に真昼自体が志保子とやりとりをしているため、断っても無駄なような気がする。
要是志保子她不知道这回事的话周肯定是要全力拒绝的,但现在她不但知道了,真昼自己和志保子之间还有不少来往,周便觉得就算拒绝也是无济于事了。

% 一応隠すものがなくなった今、両親が帰省しない息子の視察をする事自体には拒否感がない。\\
反正事到如今也没啥要藏着掖着的了,周对父母来视察不回家的儿子这件事本身也没有什么抗拒感。\\

% 父親――修斗が志保子と来るのなら、暴走しがちな志保子を窘めてくれる筈である。\\
父亲——修斗如果和志保子一起来的话,反而是可以让容易暴走的志保子冷静下来。\\

% どうせ断っても志保子が押しに押して真昼に会いに来る気がしたので、周は先にアポを取ってくれた父親に承諾の旨を伝えてから、真昼にメッセージを送った。\\
周料到就算自己现在拒绝了,过后志保子也会强行跑来见真昼,便向先来约时间的父亲回了一个肯定的答复,然后给真昼发了短信。\\

% \\


% 「ええと、その、私も家族の団らんの場に居てもいいのですか。邪魔では?」\\
「嗯,那个,我待在你们家庭聚会的地方没问题吗。不会打扰你们?」\\

% 翌日、朝から周の家にやって来た真昼は、少々緊張気味だった。
第二天一早便来到周家里的真昼,显得稍有些紧张。

% それはある意味当然だろう。いきなり世話している男の両親が真昼に会いたいと言い出したのだから。\\
某种意义上也是当然。毕竟正受着自己照顾的男生的父母突然说要想要见自己。\\

% 志保子とはどうやら密にやり取りを……というか志保子からよく連絡を取っているらしく、大分慣れているらしい。志保子だけならともかく父親も伴ってくるので、彼女が緊張するのも仕方ない事だろう。\\
志保子似乎是偷偷地有过联络……不如说是经常跟真昼有来往的样子,大概互相已经熟悉了吧。光是志保子那还好,偏偏这回父亲也要过来,真昼会感到紧张这也是情有可原。\\

% 「いや、父さんはお前に挨拶しに来たってのはあるし、母さんも真昼の事が気に入ってるから居てくれたらありがたい。むしろお前が居ないと駄目」
「我说啊,我爸是要跟你打个招呼才来的,我妈的话似乎很中意你的样子不如说你在她还高兴哩。这么以来反而是你不在不行」

% 「そ、そうは言われましても……」
「就,就算你这么说……」

% 「まああんま気は進まないだろうが、ちょっとだけ我慢してくれると嬉しい」\\
「好啦别那么畏畏缩缩的啦。稍微忍一忍的话我会高兴的」\\

% 両親に挨拶をさせるというシュールな事態になっているが、向こうがもう会う気なので致し方ない。\\
父母跑来打招呼这事虽然听起来有些超现实,但既然他们也有这个意思,那就没办法了。\\

% 真昼の時間をとらせるのは悪かったが、父親の性格上真昼に挨拶を済ませておかなければ気が済まないだろうし、少しの間だけ我慢してほしかった。\\
虽然占用了真昼的时间这点上有点对不起她,但从父亲的性格来看要是不跟真昼见个面怕是不会善罢甘休的,所以希望真昼能忍一忍。\\

% 「……志保子さん、私の事どう説明しているのでしょう」
「……志保子阿姨她是怎么介绍我的呢」

% 「安心しろ。父さんには恩人ってしつこく伝えてるから。間違っても母さんの楽しい妄想のお時間での役職ではないと伝えてるから」\\
「放心吧。跟父亲说清楚了是恩人的关系啦。再说,我已经说清楚了不是我妈那自我幻想时间里的那种关系」\\

% 志保子の中では既に嫁、というか可愛い娘認定をしているらしいので、全力で否定しておいた。
好像在志保子她脑子里,真昼已经是儿媳——或者说是可爱的女儿了,这点周向父亲全力否定了。

% 修斗も苦笑の後に『いつもの志保子さんの悪い癖だね』と言って納得したので、誤解されているという事はないだろう。
修斗也苦笑了会,理解般地答道『是志保子她平日里的坏毛病呢』,这么一来应该就不会有误解了吧。

%  


% ほっと胸を撫で下ろしたらしい真昼に「すまんな」と苦笑して待っていれば、ちょうどいいタイミングでインターホンが鳴った。\\
看着似是在平复心情般的真昼呼地疏口气的样子,周一边苦笑着说着「抱歉啦」一边等着,正好这时门铃响了起来。\\

% エントランス自体は合鍵で突破しているので、直通で来るのは予想していた。
大门的话他们手上有钥匙可以直接开开,因而本料着他们会直接进来的。

% 真昼がびくっと体を大きく震わせたので小さく笑って宥めつつ玄関に向かって、チェーンを外し鍵を開けた。\\
看见真昼的身体悚地一抖,周一边微笑着安慰着她一边起身走向玄关,解开防盗链拧动了门把。\\

% 扉を開けば、周にとっては見慣れた両親の姿。\\
开开门之后,门前站着的是周已经见惯了的两亲的身影。\\

% 「半年ぶりだね周」
「半年不见了呢周」

% 「久しぶり、父さん」\\
「好久不见,爸」\\

% 穏やかな笑みを浮かべた父親……修斗に、周も同じように少し安堵したような笑みを浮かべる。
看见露出平和笑容的父亲——修斗,周也露出了安心的微笑。

% ふんわりとした空気の持ち主である修斗は、なんというか居て和むタイプなので、周もつい対面していると気が緩むのだ。\\
身边萦绕着安稳的氛围的修斗,该说是性格容易相处吧,连面对着修斗的周也不禁放松了下来。\\

% 「母さんにはそんな態度してくれなかったのにぃ……」
「对你妈的时候就是一副鬼态度呢……」

% 「母さんはいきなり押しかけてきたからだろうが。事前予告すれば普通に対応したし」\\
「还不是妈你突然就不请自来。事先说一声的话我就会好好接待啦」\\

% あの時は真昼が居たからあんな対応になっただけで、周一人ならもう少し優しい対応が出来ただろう。\\
主要那时候真昼在所以周才是那样的一副态度,要是只有周一个人的话他的态度也会缓和些吧。\\

% 「とりあえず、入ってくれ。……何その荷物」
「总之,进来吧。……这提的都啥啊」

% 「色々と持ってきたのよー。まあそれは後にして、真昼ちゃんは?」
「带了各种各样的东西啦。嘛这个先放一边,小真昼呢?」

% 「奥」\\
「里面」\\

% 簡潔に返して、靴を脱いだ両親を伴ってリビングに戻れば、ほんのり居たたまれなさそうにしていた真昼がこちらを向いて――ぱちくり、と目を見開いていた。\\
周简短地答道,陪着脱掉鞋子的父母回到客厅,正赶上稍稍有些坐立不安的真昼看向这边——然后,突然瞪大了眼经。\\

% 真昼が驚くのも無理はない。\\
真昼的惊讶也不无道理。\\

% 修斗は、三十代後半とは思えないほど若々しいのだ。息子の贔屓目抜きに、三十前後の容貌を保っている。
修斗他那年轻的外貌实在难以想象已经是个三十大几的人了。就算除开从儿子眼里来看的加分,那容貌也还是三十左右的水平。

% ベビーフェイスと言ってもいい若く端整な容貌で、もう少しその血を濃く継げたらと何度思った事か。\\
看着那几乎可以称得上是娃娃脸的年轻而端正的容貌,周已经不知几次想着要是能多继承点那基因该多好了。\\

% 周とは違い柔和な顔立ちでいかにも人当たりのよい好青年(実年齢的には中年なのだが)といった男なので、血の繋がりをよく疑われた。それでも並んで歩けば年の離れた兄弟に見えるらしいが。\\
周很是怀疑这个有着与自己不同的柔和长相,看上去实在是个友善的好青年(虽然年纪上已经算是中年了)的男人和自己到底有没有血缘关系。说起来两人一起走的话几乎都能被当作是年龄差的大的兄弟了。\\

% 「真昼ちゃん、久しぶりねぇ」
「小真昼,好久不见呢~」

% 「久しぶりって、一ヶ月も経ってないだろ」
「啥好久不见啊,还没一个月吧」

% 「私の中では久しぶりよ」\\
「在我心里已经好久不见了哦」\\

% 真昼に駆け寄ってにこにこと満面の笑みを浮かべる志保子に、真昼も居住まいをただして「お久しぶりです」とほんのり外行き用の笑みを浮かべている。
向着满脸笑容跑向自己的志保子,真昼也摆正了坐姿微微露出外出用的笑容「很久没有见过您了」这么答道。

% ただ、視線は困惑気味に修斗に向けられていて、その視線に気付いた修斗も穏やかな笑みをたたえて志保子の隣に立った。\\
不过,真昼还是以困惑的眼神望向修斗,而注意到这视线的修斗则保持着一脸平和的笑容站在了志保子身边。\\

% 「初めまして。周の父の藤宮修斗と申します。椎名さんの事は志保子さんから伺っているよ。いつも息子がお世話になっています」
「初次见面。我是周的父亲,藤宫修斗。椎名的事情我已经听志保子说过了。一直都在照顾着我家儿子呢」

% 「初めまして。椎名真昼と申します。こちらこそ周くんにはお世話になっています」\\
「初次见面。我是椎名真昼。我才是,一直都在受周的照顾」\\

% 綺麗にお辞儀した修斗に合わせて、真昼も折り目正しく挨拶する。\\
真昼彬彬有礼地回应着修斗正式的辞令,做着初次见面的问候。

% 真昼が心配していたのは、修斗が志保子のようなタイプかどうか、という点だったのかもしれないが、修斗は温厚な常識人なので真昼には是非安心していただきたいところである。
真昼担心的,大概是修斗他会不会性格跟志保子一样,但看见修斗是个温厚而有常识的人之后,总算是安心了下来。

% 志保子のストッパーをこなせるのは修斗だけであり、志保子も修斗には弱い。ベタぼれ、という理由もあるのだが。\\
毕竟修斗算是志保子的控制器,志保子也对修斗强硬不起来。毕竟,是喜欢的一塌糊涂的关系。\\

% 「あら、そんな謙遜しなくていいのよ? どうせ周はだらしないからねえ」
「哎呀,没必要那么谦虚啦。反正周是个邋遢仔啦」

% 「だらしなくて悪かったな」
「身为邋遢仔真是对不起了」

% 「こら志保子さん、そういう事を言わないの。……周、日頃お世話になっているんだからちゃんと彼女は労ってるね?」
「好啦志保子,这种话不能说啦。……周,平常一直受人家照顾,有好好地感谢过人家没?」

% 「出来うる限り」
「尽我所能」

% 「よろしい」\\
「那就好」\\

% 女性は大切にするもの、という教育方針の修斗は、息子である周が真昼を労っているのか心配していたらしい。
以「应好好对待女性」为教育方针的修斗,似乎是在担心周有没有好好感谢过真昼。

% 流石に、尽くさせるだけ尽くさせて自分は楽している、というのは周の心情的にも無理なので、当然真昼に最大限気をつかっているつもりだ。\\
再怎么说,把事全部丢给真昼,自己在一边享受这种事情,周自己心里也过意不去,因而周的打算也是尽自己所能关照真昼。\\

% 周の返事に安心したらしい修斗は、改めて真昼の方に視線を合わせる。\\
停了周的回复,安心下来的修斗,再次看向了真昼。\\

% 「……本当に、何とお礼を申せばいいのか。日頃から料理をつくってもらっていて、加えておせちまでつくってもらっているようだし……」
「……实在是,本应该给你一些回礼的。好像不但平常做饭都是靠你,连年菜都麻烦你来做了吧……」

% 「いつも感謝してるし、なるべく真昼を労ってるから」
「我一直都很感谢人家,也尽我所能慰劳她啦」

% 「はい。……周くんは、案外気を使ってくれますので」
「嗯。……周君也意外地很会关心人呢」

% 「案外ってなんだよ案外って」
「意外是什么鬼啦意外」

% 「だって……」\\
「毕竟嘛……」\\

% 大雑把なようで結構細かく見てますよね、と言われて、大雑把なのは反論出来なかったので言葉に詰まれば、修斗が柔らかい笑みを浮かべている。\\
看上去一副大大咧咧的样子却很能注意到细节呢——被这么说了的周,因为大大咧咧是事实,结果一时语塞,而看着的修斗则露出了柔和的笑容。\\

% 「仲良いようで何よりだ。周も、椎名さんにはあまり迷惑をかける事のないようにするんだよ」
「关系好是最重要的啊。周你也不要太过给人家椎名添太多麻烦哦」

% 「……分かってる」
「……知道了」

% 「椎名さんも、周に悪いところがあればきっちり言ってあげてほしい。この子は素直ではないようで案外素直だから、嫌なところはすぐに直してくれると思う」
「椎名的话,要是周有什么地方你不喜欢的话希望你能好好指出来。虽然看上去不像,但这孩子其实还是很坦率的,要是有你不喜欢的地方应该会很快改正过来的」

% 「……周くんは優しいですから、嫌なところなんて……その、少ししか」
「……我觉得周君很温柔,所以,不喜欢的地方……那个,只有一点点」

% 「あるんだな」
「有呢」

% 「……嫌というか、……だめなところです」\\
「……不喜欢,……说是废废的地方更准确吧」\\

% もじ、とほんのりと恥じらいながら言いにくそうにしている真昼に、そんな恥ずかしそうにして言う駄目なところってなんなんだ……と問い詰めたくなった。\\
略微害羞,支支吾吾地说着的真昼,搞得周都想问问自己到底是哪里废废的会让真昼说起来这么害羞了……

% 志保子は何故か知らないが「ははーん」と心当たりがあるらしく、にやにや笑いでこちらを見てきたので、何なんだと睨んでやるくらいしか出来なかった。
志保子则是不知为何,似是有了头绪般「哈哈」地咧着嘴边看着这边边笑着,让周不知为何除了盯着她以外毫无办法。
