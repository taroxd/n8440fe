% \subsection{41 天使様と初対面}
\subsection{}

% 『明日周の家を訪ねてもいいかい』\\


% そんなメッセージが父親から送られてきたのは、三日、真昼が帰った後だった。\\


% 『周が実家に帰らないのはいいのだけど、やはり私も顔くらいは見ておきたいからね。それに、志保子さんから聞いているけどお隣さんにもご挨拶は必要だと思うし』\\


% 真昼の存在は母親からしっかりと伝わっていかに周がお世話になっているのか知っているらしく、親として挨拶をしておきたいとの事だった。\\


% これが仮に志保子が知らない状態であったなら全力で拒否したのだが、もう知られている上に真昼自体が志保子とやりとりをしているため、断っても無駄なような気がする。


% 一応隠すものがなくなった今、両親が帰省しない息子の視察をする事自体には拒否感がない。\\


% 父親――修斗が志保子と来るのなら、暴走しがちな志保子を窘めてくれる筈である。\\


% どうせ断っても志保子が押しに押して真昼に会いに来る気がしたので、周は先にアポを取ってくれた父親に承諾の旨を伝えてから、真昼にメッセージを送った。\\


% \\


% 「ええと、その、私も家族の団らんの場に居てもいいのですか。邪魔では?」\\


% 翌日、朝から周の家にやって来た真昼は、少々緊張気味だった。


% それはある意味当然だろう。いきなり世話している男の両親が真昼に会いたいと言い出したのだから。\\


% 志保子とはどうやら密にやり取りを……というか志保子からよく連絡を取っているらしく、大分慣れているらしい。志保子だけならともかく父親も伴ってくるので、彼女が緊張するのも仕方ない事だろう。\\


% 「いや、父さんはお前に挨拶しに来たってのはあるし、母さんも真昼の事が気に入ってるから居てくれたらありがたい。むしろお前が居ないと駄目」


% 「そ、そうは言われましても……」


% 「まああんま気は進まないだろうが、ちょっとだけ我慢してくれると嬉しい」\\


% 両親に挨拶をさせるというシュールな事態になっているが、向こうがもう会う気なので致し方ない。\\


% 真昼の時間をとらせるのは悪かったが、父親の性格上真昼に挨拶を済ませておかなければ気が済まないだろうし、少しの間だけ我慢してほしかった。\\


% 「……志保子さん、私の事どう説明しているのでしょう」


% 「安心しろ。父さんには恩人ってしつこく伝えてるから。間違っても母さんの楽しい妄想のお時間での役職ではないと伝えてるから」\\


% 志保子の中では既に嫁、というか可愛い娘認定をしているらしいので、全力で否定しておいた。


% 修斗も苦笑の後に『いつもの志保子さんの悪い癖だね』と言って納得したので、誤解されているという事はないだろう。


%  


% ほっと胸を撫で下ろしたらしい真昼に「すまんな」と苦笑して待っていれば、ちょうどいいタイミングでインターホンが鳴った。\\


% エントランス自体は合鍵で突破しているので、直通で来るのは予想していた。


% 真昼がびくっと体を大きく震わせたので小さく笑って宥めつつ玄関に向かって、チェーンを外し鍵を開けた。\\


% 扉を開けば、周にとっては見慣れた両親の姿。\\


% 「半年ぶりだね周」


% 「久しぶり、父さん」\\


% 穏やかな笑みを浮かべた父親……修斗に、周も同じように少し安堵したような笑みを浮かべる。


% ふんわりとした空気の持ち主である修斗は、なんというか居て和むタイプなので、周もつい対面していると気が緩むのだ。\\


% 「母さんにはそんな態度してくれなかったのにぃ……」


% 「母さんはいきなり押しかけてきたからだろうが。事前予告すれば普通に対応したし」\\


% あの時は真昼が居たからあんな対応になっただけで、周一人ならもう少し優しい対応が出来ただろう。\\


% 「とりあえず、入ってくれ。……何その荷物」


% 「色々と持ってきたのよー。まあそれは後にして、真昼ちゃんは?」


% 「奥」\\


% 簡潔に返して、靴を脱いだ両親を伴ってリビングに戻れば、ほんのり居たたまれなさそうにしていた真昼がこちらを向いて――ぱちくり、と目を見開いていた。\\


% 真昼が驚くのも無理はない。\\


% 修斗は、三十代後半とは思えないほど若々しいのだ。息子の贔屓目抜きに、三十前後の容貌を保っている。


% ベビーフェイスと言ってもいい若く端整な容貌で、もう少しその血を濃く継げたらと何度思った事か。\\


% 周とは違い柔和な顔立ちでいかにも人当たりのよい好青年(実年齢的には中年なのだが)といった男なので、血の繋がりをよく疑われた。それでも並んで歩けば年の離れた兄弟に見えるらしいが。\\


% 「真昼ちゃん、久しぶりねぇ」


% 「久しぶりって、一ヶ月も経ってないだろ」


% 「私の中では久しぶりよ」\\


% 真昼に駆け寄ってにこにこと満面の笑みを浮かべる志保子に、真昼も居住まいをただして「お久しぶりです」とほんのり外行き用の笑みを浮かべている。


% ただ、視線は困惑気味に修斗に向けられていて、その視線に気付いた修斗も穏やかな笑みをたたえて志保子の隣に立った。\\


% 「初めまして。周の父の藤宮修斗と申します。椎名さんの事は志保子さんから伺っているよ。いつも息子がお世話になっています」


% 「初めまして。椎名真昼と申します。こちらこそ周くんにはお世話になっています」\\


% 綺麗にお辞儀した修斗に合わせて、真昼も折り目正しく挨拶する。\\


% 真昼が心配していたのは、修斗が志保子のようなタイプかどうか、という点だったのかもしれないが、修斗は温厚な常識人なので真昼には是非安心していただきたいところである。


% 志保子のストッパーをこなせるのは修斗だけであり、志保子も修斗には弱い。ベタぼれ、という理由もあるのだが。\\


% 「あら、そんな謙遜しなくていいのよ? どうせ周はだらしないからねえ」


% 「だらしなくて悪かったな」


% 「こら志保子さん、そういう事を言わないの。……周、日頃お世話になっているんだからちゃんと彼女は労ってるね?」


% 「出来うる限り」


% 「よろしい」\\


% 女性は大切にするもの、という教育方針の修斗は、息子である周が真昼を労っているのか心配していたらしい。


% 流石に、尽くさせるだけ尽くさせて自分は楽している、というのは周の心情的にも無理なので、当然真昼に最大限気をつかっているつもりだ。\\


% 周の返事に安心したらしい修斗は、改めて真昼の方に視線を合わせる。\\


% 「……本当に、何とお礼を申せばいいのか。日頃から料理をつくってもらっていて、加えておせちまでつくってもらっているようだし……」


% 「いつも感謝してるし、なるべく真昼を労ってるから」


% 「はい。……周くんは、案外気を使ってくれますので」


% 「案外ってなんだよ案外って」


% 「だって……」\\


% 大雑把なようで結構細かく見てますよね、と言われて、大雑把なのは反論出来なかったので言葉に詰まれば、修斗が柔らかい笑みを浮かべている。\\


% 「仲良いようで何よりだ。周も、椎名さんにはあまり迷惑をかける事のないようにするんだよ」


% 「……分かってる」


% 「椎名さんも、周に悪いところがあればきっちり言ってあげてほしい。この子は素直ではないようで案外素直だから、嫌なところはすぐに直してくれると思う」


% 「……周くんは優しいですから、嫌なところなんて……その、少ししか」


% 「あるんだな」


% 「……嫌というか、……だめなところです」\\


% もじ、とほんのりと恥じらいながら言いにくそうにしている真昼に、そんな恥ずかしそうにして言う駄目なところってなんなんだ……と問い詰めたくなった。\\


% 志保子は何故か知らないが「ははーん」と心当たりがあるらしく、にやにや笑いでこちらを見てきたので、何なんだと睨んでやるくらいしか出来なかった。

