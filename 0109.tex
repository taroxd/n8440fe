% \subsection{109 天使様の借り物}
\subsection{109 天使様の借り物}

% 周の出番は基本的に出場種目である玉入れと借り物競走、そして男子全員参加の騎馬戦くらいで割と暇である。


% 他のパッションに溢れた生徒は二種目より多くを希望しているが、周はそういった熱意はなかったので二種目と全体競技のみに抑えていた。\\


% 因みに玉入れは既に終わっている。\\


% 本当に盛り上がりのない競技というか、まあ玉を高い位置にある籠に入れるだけの作業だ。\\


% 中に入れる玉こそ奪い合いであるが、元々量が多い上にそんなムキになってするものでもないので、終始和やかな争いをしていた。\\


% 活躍をしろと千歳に背中を押されて出場したものの、玉入れに活躍もなにもないだろう。


% 普通に玉を幾つか拾って向きを変えて重ね、まとめて投げるという地味な作業の繰り返しなので、目立つ事はなかった。\\


% ただ、狙いが正確だったのと玉をまとめた事が功を奏したのか、白チームより玉の数は多かった、くらいだろう。\\


% 「ほんと地味な種目行くよねえ周」


% 「うるせえ。お前そろそろ交代の時間だろ、行ってこい」


% 「あ、そうだった」\\


% スケジュールを見ながら「実行委員って結構忙しいのよねー」とこぼしつつ彼女は運営のテントに向かっていく。


% なら何で立候補したんだよ、と思わなくはなかったが、今更だろう。\\


% ぱたぱたと小走りで向かう千歳の背中を眺めながら、テントの支柱に貼り付けられた本日の日程を確認する。\\


% 午前中の日程は、あと数種目で終わる。周が個人種目としては最後に出場する借り物競争も、その数種目に含まれていた。


% 残る種目が終わればお昼休憩を挟み午後の部に移る。\\


% とりあえず、周は借り物競走が終われば後は午後の騎馬戦で出場種目はなくなるだろう。\\


% 「……つーか借り物の時あいつ運営じゃねえか」\\


% 千歳がこのタイミングで交代にいったという事は、残る種目は恐らく千歳が担当する事になる。確実に借り物競走の判定員も彼女になる……というか狙っている気がする。


% 誰が借り物のお題を考えているのか知らないが、あまりロクなお題がなさそうでやや怖かった。\\


% 微妙に気が重くなりつつも次の次に控えた借り物競走の集合場所に行くと、同じく希望が通ったらしい真昼が静かに佇んでいた。


% 別に話しかける用事がないので周も口を閉じていたが、真昼と視線が合うと淡く微笑まれて目礼される。 \\


% 外では他人として接しているのだが、少しだけいつもの笑みが滲み出た表情に、少し心臓が跳ねた。


% 周も無表情で同じように返したものの、なんというか居心地の悪さは否めない。\\


% そんな周と真昼を、体育祭の運営として集合をかけていた千歳は愉快そうに見守っていた。 \\


% \\


% 借り物競走の番になり、係員……この場合は千歳の指示に従って、グラウンドに入場する。


% 既にグラウンドには折り畳まれた紙が幾つも散らばっており、スタートの合図が出たらその紙を拾ってそこに書かれたお題のものを持ってくるだけ。\\


% 借り物競走は他の走る種目と違い息抜きに近いような種目だし、借りものを楽しむといった目的があるので、さほど真剣みはない。


% ただお題によっては晒し者になる場合もあるので注意が必要だろう。\\


% 「出場する選手の皆さんはスタートラインに並んでください」\\


% マイクを使ってはきはきと指示する千歳は、ふざけなければ本当に司会向きな少女だ。明朗な人柄もそうだが、空気を読む事も状況を読む事も出来るし、聞き取りやすく高すぎない澄んだ声は、耳を傾けさせるに充分なものだろう。\\


% 全校生徒と職員に見守られているので、今のところおふざけは一切なしの千歳が「位置について」と合図する。


% 号砲自体はもう一人の係員の男子が持っているので、あくまでカウントをするだけだろう。\\


% 千歳の「用意」という言葉の後、一拍置いて空砲の音が響く。\\


% この音はいつになっても心臓に悪いが、それはおくびにも出さずに緩く走って落ちている紙の元に向かう。


% 既に早い選手は開いてお題を確認しており、周も彼らに続くように一つ折り畳まれた紙を拾い上げて、中身を確認する。\\


% 中には、几帳面そうな文字でこう書かれていた。\\


% 『美人だと思う人』\\


% こういうパターンも想定してはいたが、物ではなく者を借りてこいというお題だった。\\


% 本当に誰がこのお題を考えたのかと突っ込みたくなったが、ギリギリこのお題は周でもクリア出来る。


% 一番の困る『好きな人』とかでもないし、客観的に見て美人な人を連れてくればいいのだ。\\


% つまり、誰もが認める美人……真昼を呼べばいい。真昼の借り物が終わってゴールするついでに一緒にゴールすればいいだけだ。\\


% 真昼を連れていくのはかなり目立ちそうではあるが、お題がお題なので中身を知れば妥当なところだと判断してもらえるだろう。\\


% そう思って同じようにお題を拾っているであろう真昼を探そうとして……横から、Tシャツを掴まれた。\\


% 掴まれたというよりは摘ままれた、と言った方が正しいのだが、裾の部分をくいくいと引っ張られて、周がなんだと振り向く。\\


% そこには、今求めている人間が遠慮がちにこっちを見ていた。\\


% 「藤宮さん、借り物が藤宮さんなので藤宮さんの借り物が済んだらご同行願いたいのですが」


% 「え、俺?」


% 「はい」\\


% まさか互いが借り物だったとは思うまい。\\


% ある意味好都合であったが、非常に目立ちそうな気がする。


% グラウンドのど真ん中で真昼に話しかけられている時点で目立つもなにもないが。\\


% ゴールラインの向こうでは、判定員の千歳がにやにやした様子でこちらを見守っている。\\


% (あいつ後で覚えてろ)\\


% お題の文字がそもそも千歳の書いた文字なので、彼女がある程度狙ったお題もあるだろう。真昼が何を引いたかは知らないが、わざわざ周を指定するのだから真昼にとって譲れないお題がきたに違いない。\\


% 「あー。……ちなみに借り物は?」


% 「秘密です」\\


% ゴールしたら読み上げられるというのに、真昼はお題を口にしようとはしなかった。


% なので、ため息をついてゴールに向かう。\\


% 「俺も借り物お前だからゴールするぞ」


% 「……藤宮さんこそ借り物何なのですか」


% 「秘密」\\


% 真昼と同じような答えを返すと、小さく笑われた。\\


% 「そうですね、ゴールしてからのお楽しみです」\\


% 囁いて、真昼は周の手を取った。


% 周囲がざわつくのもお構いなしに、真昼は周に触れてゴールに向かう。\\


% 周としては微妙に胃が痛かったが、上機嫌そうな真昼を見ているとまあ仕方ないかと思えてしまうのだから、惚れた弱味だと自覚していた。\\


% 周にとって微妙にアウェー感の漂うグラウンドを駆け抜けてゴールラインまでたどり着くと、実に上機嫌そうな千歳に迎えられた。


% 思わず舌打ちをしたものの、気に留めた様子もない。\\


% 「おっと、これは二人でゴールー? 双方借り物競走の走者だったと思うんだけど」


% 「千歳この野郎、にやにやしやがって。互いが借り物だったんだよ」


% 「ははーん。じゃあお題の確認するけどどっちからで?」


% 「藤宮さんからでお願いします」\\


% きっぱりと真昼が指定して驚いたが、千歳が心得たと言わんばかりに周の持つ紙を指で示す。見せろ、という事だろう。\\


% 特に隠すものでもなかったので、あっさりと彼女に向けてお題を見せる。


% お題の中身に微妙に千歳ががっかりしたような表情だった。


% 何を期待していたのかは知らないが、お望みのものではなかったのだろう。\\


% それでも気を取り直してにこやかな表情でマイクを口許に寄せる。\\


% 「ただいまお題確認中です。赤組一着のお題は……『美人だと思う人』ですね」\\


% 群衆は読み上げられたお題に、どこか安堵したような空気をかもしている。\\


% 実に無難なチョイスだろう。学内で真昼以上の美人は周が知る限り居ないし、周にとってはやはり真昼が一番可愛いのだ。


% 周の個人的な意見を除いても、真昼を連れてくる事はなんらおかしくない。\\


% 真昼と二人でゴールした事で周に敵意が飛んできていたのだが、お題の内容で多少和らいだように思える。\\


% 問題は真昼側のお題だろう。\\


% 何が書かれているかは周は知らないのだが、わざわざ周を指定する辺り周の平穏な学生生活的によろしくないものな気がしてならない。\\


% 千歳は真昼からお題が書かれた紙を受け取って、ぱちりと目を瞬かせて、それから真昼を窺う。


% 周の方からは何が書かれているのか見えなかったが、千歳の表情からは「言ってもいいんだよね?」といった色が見えた。\\


% (一体何のお題で俺を連れてきたんだ)\\


% 千歳の反応で、ますます分からなくなる。


% 真昼は穏やかな微笑みをたたえたままだ。つまり、そのまま読み上げても問題ないという意思表示である。\\


% 千歳はそれを確認して、いつもの笑みに戻る。\\


% 「えー、続いて同着ですが白組一着のお題確認です。白組一着のお題は……『大切な人』です」\\


% 千歳の声がグラウンドに響いた瞬間、生徒の待機所の方からざわつきが生まれる。\\


% 反射的に真昼の方を見れば――彼女は、こちらに視線を合わせて薄紅の唇に弧を描かせた。\\


% それは、悪戯に成功したような子供の笑みにも、照れを含んだはにかみにも、見えた。


% 間違いないのは、周がこのお題を知った時の反応を見るためにこちらを見ていた、という事だろう。\\


% (小悪魔め……)\\


% 思慮深い真昼なら、お題が公になった時点で周囲がどう反応するか予想する事くらい容易だろう。\\


% それでも真昼は周を借り物として選んだのだ。関係に変化をもたらすために。


% これからは、中途半端な他人としては居られない。\\


% いつも学校で見せるような美しい笑みではなく、周に見せる素の微笑みに、周は「絶対後で周りに問い詰められる」とぼやいてぐしゃりと掌で髪を掻いた。

