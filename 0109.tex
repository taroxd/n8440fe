% \subsection{109 天使様の借り物}
\subsection{109 天使大人与借物}

% 周の出番は基本的に出場種目である玉入れと借り物競走、そして男子全員参加の騎馬戦くらいで割と暇である。
周要参加的项目,数起来也只有投篮、借物赛跑、再加上所有男生都要参加的骑马战这三样,因此还挺闲的。

% 他のパッションに溢れた生徒は二種目より多くを希望しているが、周はそういった熱意はなかったので二種目と全体競技のみに抑えていた。\\
有些干劲满满的学生报了两个以上的项目,但周对体育祭则不那么上心,于是便只参加了最低线的两个项目和团体项目而已。\\

% 因みに玉入れは既に終わっている。\\
顺带一提投篮已经结束了。\\

% 本当に盛り上がりのない競技というか、まあ玉を高い位置にある籠に入れるだけの作業だ。\\
毕竟说白了,也就是把球往高高挂着的篮里扔这么一回事,令人提不起劲来也是自然。\\

% 中に入れる玉こそ奪い合いであるが、元々量が多い上にそんなムキになってするものでもないので、終始和やかな争いをしていた。\\
虽说抢投过了的球倒是有些比拼,但本来球就很多,实在是没必要觍着脸去抢别人的球,因而整个比赛自始至终都很和平。\\

% 活躍をしろと千歳に背中を押されて出場したものの、玉入れに活躍もなにもないだろう。
虽然出场的时候千岁喊着要活跃点,但讲道理投篮哪来的活跃啊。

% 普通に玉を幾つか拾って向きを変えて重ね、まとめて投げるという地味な作業の繰り返しなので、目立つ事はなかった。\\
不过是重复着简单地捡几个球,让它们滚向一处收集起来,然后投出去这样枯燥的任务,实在是没什么起眼的。\\

% ただ、狙いが正確だったのと玉をまとめた事が功を奏したのか、白チームより玉の数は多かった、くらいだろう。\\
不过,大概是投的比校准,再加上把球集在一处起了作用的缘故,最后投进的球数比白组要多,倒是值得一提。\\

% 「ほんと地味な種目行くよねえ周」
「我说周你啊,还真是净捡些没意思的项目啊」

% 「うるせえ。お前そろそろ交代の時間だろ、行ってこい」
「多管闲事。倒是你差不多该要换班了吧,还不快去」

% 「あ、そうだった」\\
「啊!对对对」\\

% スケジュールを見ながら「実行委員って結構忙しいのよねー」とこぼしつつ彼女は運営のテントに向かっていく。
千岁看着自己的日程表,一边嘟哝着「执行委员还真是忙耶~」,一边朝着运营组的帐篷动身走去。

% なら何で立候補したんだよ、と思わなくはなかったが、今更だろう。\\
听她这么报怨,周实在是想想吐槽「那你当时干嘛要报名啊」,不过现在再说也都是马后炮了。\\

% ぱたぱたと小走りで向かう千歳の背中を眺めながら、テントの支柱に貼り付けられた本日の日程を確認する。\\
周一边望着一路小跑的千岁的背影,一边浏览着贴在帐篷柱子上的今日日程。\\

% 午前中の日程は、あと数種目で終わる。周が個人種目としては最後に出場する借り物競争も、その数種目に含まれていた。
上午的项目已经不剩几个了,而周参加的最后一个个人项目借物赛跑,正好在这里面。

% 残る種目が終わればお昼休憩を挟み午後の部に移る。\\
等这几个项目都结束,中午休息完之后便是下午的日程了。\\

% とりあえず、周は借り物競走が終われば後は午後の騎馬戦で出場種目はなくなるだろう。\\
总而言之,一会比完借物赛跑之后,周剩下的项目就只有下午的的骑马战了。\\

% 「……つーか借り物の時あいつ運営じゃねえか」\\
「……话说回来借物赛跑的时候那家伙负责运营来着」\\

% 千歳がこのタイミングで交代にいったという事は、残る種目は恐らく千歳が担当する事になる。確実に借り物競走の判定員も彼女になる……というか狙っている気がする。
卡在这个节骨眼上换班,那恐怕负责接下来的项目的就是千岁了。这么一来,借物赛跑的裁判员也成了她……总觉得这里面有什么阴谋。

% 誰が借り物のお題を考えているのか知らないが、あまりロクなお題がなさそうでやや怖かった。\\
虽然不知道这回借物的题目是谁想的,但估计就不是什么正经题目,感觉有点吓人。

% 微妙に気が重くなりつつも次の次に控えた借り物競走の集合場所に行くと、同じく希望が通ったらしい真昼が静かに佇んでいた。
虽然莫名地有点不安,但周还是朝着下下个项目的借物赛跑集合点走去——而似乎也报了这个项目的真昼,静静地站在那里。

% 別に話しかける用事がないので周も口を閉じていたが、真昼と視線が合うと淡く微笑まれて目礼される。 \\
反正也没有交谈的必要,周便也没有开口,但和真昼对上眼的时候,她还是微微笑着向周点头示意了一下。\\

% 外では他人として接しているのだが、少しだけいつもの笑みが滲み出た表情に、少し心臓が跳ねた。
虽然在外面两人还是保持着一般人的距离,但望见真昼微微流露出平日里的笑容,周还是不禁心脏加速了。

% 周も無表情で同じように返したものの、なんというか居心地の悪さは否めない。\\
周也抑制着自己地表情淡淡回意,不过心里还是感觉有点痒痒的。\\

% そんな周と真昼を、体育祭の運営として集合をかけていた千歳は愉快そうに見守っていた。 \\
而此时此刻,正负责着体育祭运营,召集着选手的千岁,则是一脸愉快地望着这俩。\\

% \\


% 借り物競走の番になり、係員……この場合は千歳の指示に従って、グラウンドに入場する。
终于到了借物赛跑的时间,选手们依照着负责人……也就是千岁的指令,进入了操场。

% 既にグラウンドには折り畳まれた紙が幾つも散らばっており、スタートの合図が出たらその紙を拾ってそこに書かれたお題のものを持ってくるだけ。\\
操场上已经散落着一些叠起来的纸片,而接下来只要一声令下,选手们就可以去捡起这些纸,然后照着里面写的题目把东西拿来了。\\

% 借り物競走は他の走る種目と違い息抜きに近いような種目だし、借りものを楽しむといった目的があるので、さほど真剣みはない。
借物赛跑与其他的跑步项目不同,算是比较休闲的项目,享受借东西过程中的乐趣也是目的之一,因而选手们也没有太过认真。

% ただお題によっては晒し者になる場合もあるので注意が必要だろう。\\
不过根据拿到的题目,也有可能落得公开处刑的下场,这一点还是要小心的。\\

% 「出場する選手の皆さんはスタートラインに並んでください」\\
「请各位比赛选手站到起跑线前做好准备」

% マイクを使ってはきはきと指示する千歳は、ふざけなければ本当に司会向きな少女だ。明朗な人柄もそうだが、空気を読む事も状況を読む事も出来るし、聞き取りやすく高すぎない澄んだ声は、耳を傾けさせるに充分なものだろう。\\
千岁熟练地拿着话筒指示着选手们。她要是性格认真点的话那就真的很适合当主持人了。不但性格明朗,善于察觉气氛和状况,而且嗓音也很清澈,不会太尖,听得很清楚,用来集中他人的注意力是再适合不过了。\\

% 全校生徒と職員に見守られているので、今のところおふざけは一切なしの千歳が「位置について」と合図する。
被全校师生看着的千岁,也收起了平日里的玩笑话,发出了「各就位」的信号。

% 号砲自体はもう一人の係員の男子が持っているので、あくまでカウントをするだけだろう。\\
当然发令枪是另一个负责的男生拿着的,因而千岁负责的只是倒数。\\

% 千歳の「用意」という言葉の後、一拍置いて空砲の音が響く。\\
在千岁喊出「预备」口令的一拍之后,砰地响起了一声枪声。\\

% この音はいつになっても心臓に悪いが、それはおくびにも出さずに緩く走って落ちている紙の元に向かう。
虽然这枪声一直都对心脏很是不好,但周还是忍了下来,慢慢跑向放着纸的地方。

% 既に早い選手は開いてお題を確認しており、周も彼らに続くように一つ折り畳まれた紙を拾い上げて、中身を確認する。\\
跑得快的选手已经展开了纸片,看过了里面的题目。周跟在他们后面,捡起了一张纸,读起了里面的内容。\\

% 中には、几帳面そうな文字でこう書かれていた。\\
纸片上,工工整整地写着——\\

% 『美人だと思う人』\\
『认为是美人的人』\\

% こういうパターンも想定してはいたが、物ではなく者を借りてこいというお題だった。\\
周也料想过这种要借的不是物品而是人的情况。\\

% 本当に誰がこのお題を考えたのかと突っ込みたくなったが、ギリギリこのお題は周でもクリア出来る。
虽然说真心话,周很想吐槽到底是谁想出来的这么一个题目,但幸亏这道题周勉强还是能解决的。

% 一番の困る『好きな人』とかでもないし、客観的に見て美人な人を連れてくればいいのだ。\\
反正也不是最难搞的『喜欢的人』这种题,那么只要找一个大家公认是美人的人带来就好了。\\

% つまり、誰もが認める美人……真昼を呼べばいい。真昼の借り物が終わってゴールするついでに一緒にゴールすればいいだけだ。\\
也就是说,大家公认的美人……那把真昼叫来就行了。等真昼借来了东西之后,跟她一起冲过终点就好。\\

% 真昼を連れていくのはかなり目立ちそうではあるが、お題がお題なので中身を知れば妥当なところだと判断してもらえるだろう。\\
虽说跟真昼一起走实在是显眼,但总归是题目要求所迫,等大家知道了题目之后大家应该也会理解这是个妥当的选择吧。\\

% そう思って同じようにお題を拾っているであろう真昼を探そうとして……横から、Tシャツを掴まれた。\\
周这么思考着,正打算去找大概正在捡纸片的真昼——结果,T袖被人从一旁拽住了。\\

% 掴まれたというよりは摘ままれた、と言った方が正しいのだが、裾の部分をくいくいと引っ張られて、周がなんだと振り向く。\\
准确来说,并不是拽,而只是被轻轻地拉住,但衣摆被人轻轻地拉了几下,周疑惑地扭过了头。\\

% そこには、今求めている人間が遠慮がちにこっちを見ていた。\\
周打算去找的人,现在正在面前瑟瑟缩缩地看着自己。\\

% 「藤宮さん、借り物が藤宮さんなので藤宮さんの借り物が済んだらご同行願いたいのですが」
「藤宫同学,因为我要借的是你,所以等你借到了要借的东西之后,可以让我也一起跟着吗」

% 「え、俺?」
「诶,我吗?」

% 「はい」\\
「嗯」\\

% まさか互いが借り物だったとは思うまい。\\
居然互相都是要借的东西,这让周始料未及。\\\

% ある意味好都合であったが、非常に目立ちそうな気がする。
虽然某种意义上也算是巧,但这么一来想必要变得十分显眼了。

% グラウンドのど真ん中で真昼に話しかけられている時点で目立つもなにもないが。\\
虽说都已经站在大操场上跟真昼说话了,再谈什么显眼不显眼的也没啥意义了。\\

% ゴールラインの向こうでは、判定員の千歳がにやにやした様子でこちらを見守っている。\\
在终点线的对面,负责裁判的千岁,一脸坏笑地一直看着这边。\\

% (あいつ後で覚えてろ)\\
(好你个家伙,一会让你吃不了兜着走)\\

% お題の文字がそもそも千歳の書いた文字なので、彼女がある程度狙ったお題もあるだろう。真昼が何を引いたかは知らないが、わざわざ周を指定するのだから真昼にとって譲れないお題がきたに違いない。\\
况且纸上的字也是千岁写的,估计她肯定是打着什么小算盘的。虽然不知道真昼抽到了个什么题目,但从她还特意选择周来看,那题目肯定是有什么对真昼来说无法让步的东西在吧。\\

% 「あー。……ちなみに借り物は?」
「啊——。……话说你要借的是什么?」

% 「秘密です」\\
「保密」\\

% ゴールしたら読み上げられるというのに、真昼はお題を口にしようとはしなかった。
明明过了终点线就要被读出来的,真昼却不肯把要求说出口。

% なので、ため息をついてゴールに向かう。\\
于是,周只好叹了口气,向终点走去。\\

% 「俺も借り物お前だからゴールするぞ」
「我要借的东西正好是你,那就出发吧」

% 「……藤宮さんこそ借り物何なのですか」
「……藤宫你才是,要借的是什么啊」

% 「秘密」\\
「我也保密」\\

% 真昼と同じような答えを返すと、小さく笑われた。\\
周照着真昼的回答原样奉还,真昼却微微笑了出来。\\

% 「そうですね、ゴールしてからのお楽しみです」\\
「嗯,那就等到冲过终点线再来揭晓吧」\\

% 囁いて、真昼は周の手を取った。
真昼轻轻说到,然后牵起了周的手。

% 周囲がざわつくのもお構いなしに、真昼は周に触れてゴールに向かう。\\
丝毫不理会四周的嘈杂,真昼拉着周朝着重点出发。\\

% 周としては微妙に胃が痛かったが、上機嫌そうな真昼を見ているとまあ仕方ないかと思えてしまうのだから、惚れた弱味だと自覚していた。\\
虽说周稍稍地感觉有些胃痛,但看见真昼一副满心欢喜的样子,自知是自己先被攻陷处于弱势,只好放弃,无奈地随了真昼。

% 周にとって微妙にアウェー感の漂うグラウンドを駆け抜けてゴールラインまでたどり着くと、実に上機嫌そうな千歳に迎えられた。
穿过这个周觉得略微有些呆不下去的操场,两人走向终点,而在那里,心情大好的千岁正等着迎接两人。

% 思わず舌打ちをしたものの、気に留めた様子もない。\\
周不禁啧了一声,而千岁却丝毫没有表现出在意。\\

% 「おっと、これは二人でゴールー? 双方借り物競走の走者だったと思うんだけど」
「哎~呀,这是两位一起冲线嘛~?没记错的话你们两位都是参赛选手呢」

% 「千歳この野郎、にやにやしやがって。互いが借り物だったんだよ」
「好你小子一脸坏笑的。我们互相是对方要借的东西」

% 「ははーん。じゃあお題の確認するけどどっちからで?」
「哈哈~。那么就来确认下题目吧,你们哪位先来?」

% 「藤宮さんからでお願いします」\\
「那就藤宫先来吧」

% きっぱりと真昼が指定して驚いたが、千歳が心得たと言わんばかりに周の持つ紙を指で示す。見せろ、という事だろう。\\
真昼不假思索的答复令周吃了一惊,可千岁却像早已有数一般,朝着周拿着的纸伸出了指头。让我看看——千岁这么示意着。\\

% 特に隠すものでもなかったので、あっさりと彼女に向けてお題を見せる。
周想着,反正也不是什么非得藏着掖着的内容,于是便痛快地把题目给了千岁。

% お題の中身に微妙に千歳ががっかりしたような表情だった。
看完了题目,千岁的表情稍稍显出一丝失望。

% 何を期待していたのかは知らないが、お望みのものではなかったのだろう。\\
虽然不知道千岁期待着什么,但看起来并不是她想要看到的结果。\\

% それでも気を取り直してにこやかな表情でマイクを口許に寄せる。\\
尽管如此,千岁还是恢复了气势,满脸笑容地拿起了话筒。\\

% 「ただいまお題確認中です。赤組一着のお題は……『美人だと思う人』ですね」\\
「现在开始检查题目。红组第一拿到的问题是——『认为是美人的人』呢」\\

% 群衆は読み上げられたお題に、どこか安堵したような空気をかもしている。\\
一旁的众人听完了读出来的题目后,不知为何散发出了一股安心的气氛。\\

% 実に無難なチョイスだろう。学内で真昼以上の美人は周が知る限り居ないし、周にとってはやはり真昼が一番可愛いのだ。
自然是个十分稳妥的选择。就周所知,学校里没有比真昼还称得上是美人的人,况且对周来说真昼确实是最可爱的。

% 周の個人的な意見を除いても、真昼を連れてくる事はなんらおかしくない。\\
就算抛开周个人的看法,把真昼带过来也是十分正常的选择。\\

% 真昼と二人でゴールした事で周に敵意が飛んできていたのだが、お題の内容で多少和らいだように思える。\\
虽说和真昼一起到达终点想必会引来旁人的敌意,但知道题目的内容之后多少应该会缓和一些吧。\\

% 問題は真昼側のお題だろう。\\
但问题的关键是真昼这边。\\

% 何が書かれているかは周は知らないのだが、わざわざ周を指定する辺り周の平穏な学生生活的によろしくないものな気がしてならない。\\
虽然纸上写了什么周不得而知,但想到这题目让真昼非指定自己不可,周便不禁觉得自己安宁的学生生活将要不保。\\

% 千歳は真昼からお題が書かれた紙を受け取って、ぱちりと目を瞬かせて、それから真昼を窺う。
千岁从真昼手上接过写着题目的纸,然后突然睁大了眼,然后瞄了真昼一眼。

% 周の方からは何が書かれているのか見えなかったが、千歳の表情からは「言ってもいいんだよね?」といった色が見えた。\\
从周的角度虽然看不清纸上写了什么,但千岁的表情似乎是在说着「我真的念了啊?」一般。\\

% (一体何のお題で俺を連れてきたんだ)\\
(到底是拿到了什么题才要把我带来啊)\\

% 千歳の反応で、ますます分からなくなる。
千岁的反应让周更加摸不着头脑。

% 真昼は穏やかな微笑みをたたえたままだ。つまり、そのまま読み上げても問題ないという意思表示である。\\
而真昼的脸上却依旧是那安宁的微笑。也就是说,真昼觉得就算读出来也没有关系。\\

% 千歳はそれを確認して、いつもの笑みに戻る。\\
千岁确认了之后,恢复了平时的笑容。\\

% 「えー、続いて同着ですが白組一着のお題確認です。白組一着のお題は……『大切な人』です」\\
「嗯~,那么来看一看同时冲线的白组第一拿到的问题。白组第一拿到的问题是——『重要的人』」\\

% 千歳の声がグラウンドに響いた瞬間、生徒の待機所の方からざわつきが生まれる。\\
在千岁的声音响彻操场的同时,学生休息处迅速响起了一阵喧嚣。\\

% 反射的に真昼の方を見れば――彼女は、こちらに視線を合わせて薄紅の唇に弧を描かせた。\\
周下意识地看向真昼——映入眼帘的她也与周对上了实现,弯起了淡红色的嘴唇。\\

% それは、悪戯に成功したような子供の笑みにも、照れを含んだはにかみにも、見えた。
那表情,既像是恶作剧成功的孩子的笑容,其中却又透出几分羞涩。

% 間違いないのは、周がこのお題を知った時の反応を見るためにこちらを見ていた、という事だろう。\\
但可以确定的是,真昼看向这边,是为了看到周知道了题目内容之后的反应。\\

% (小悪魔め……)\\
(小恶魔啊……)\\

% 思慮深い真昼なら、お題が公になった時点で周囲がどう反応するか予想する事くらい容易だろう。\\
思维缜密的真昼想必已经预想到了,要是把题目公开之后,周围的反应会是如何吧。\\

% それでも真昼は周を借り物として選んだのだ。関係に変化をもたらすために。
即便如此,真昼还是决定,选择周,作为自己借的东西。为了改变。两人的关系。

% これからは、中途半端な他人としては居られない。\\
从今往后,两人的关系,将不再是,不明不白的别人。\\

% いつも学校で見せるような美しい笑みではなく、周に見せる素の微笑みに、周は「絶対後で周りに問い詰められる」とぼやいてぐしゃりと掌で髪を掻いた。
映入眼中的,不是学校里见到的完美笑容,而是在家中露出的真心微笑。「这之后肯定要被周围的人问来问去了啊」,周这么低声叹道,胡乱地用手挠了挠头。
