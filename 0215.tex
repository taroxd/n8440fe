\subsection{现场表演}

人这么多,要聊太久也不合适,于是周找个机会结束话题,离开了咖啡馆。他叹口气,不知去哪里好。\\

文化节开到4点,还有一个半小时左右就要散场。

那会儿就又要忙着统计销售额、进行报告和第二天的准备工作了。周想在此之前再玩一波文化节,可主要的地方都已去过。\\

「真昼有其他地方想去吗?」

「说到这个……我们也逛了好些地方了,要不去体育馆的舞台看一会儿?」

「舞台啊,现在是什么活动?」\\

从下午开始,文化节就有舞台的活动,学生会志愿表演各种节目。周记得日程上有写现场表演和戏剧。

他看了看手册,现在是轻音乐社团在上台表演。\\

「现在是音乐表演,你有兴趣吗?」

「我不怎么听音乐,有这个机会就来听听吧」

「也是,你不怎么放背景音乐,要放也只放西洋的来着」\\

真昼对潮流很敏感,却不太熟悉音乐。事实上,出于个人爱好,比起流行的日本音乐,她更喜欢古典的西洋乐。

就连电视里常常露脸的知名男性偶像,她也只不过能把脸和名字对起来而已。\\

「既然你想听那就去吧,我也挺好奇的」

「嗯」\\

毕竟没什么想逛的店,周便牵起真昼的手走向体育馆权当消遣。\\

体育馆的灯光大部分都关了,还在工作的灯把舞台照得亮堂堂的。

外头也听得到声音,走进体育馆后,声音又响了许多,动人心魄。周感觉痒痒的,轻轻带上门以免打扰其他观众,然后迅速坐到空着的地方。\\

他抬起头,现在是志愿的团体站在台上展示歌曲,其中还有个熟悉的面孔。周眯起眼睛看向那人的脸。

站在麦克风前的,是周从早上开始就经常见到的人。\\

「……哎,这不是门胁吗,他可没说过要表演啊」\\

由于一起去过几次卡拉OK,周也深知门胁唱歌唱得好,然而他却实在没想到门胁会站到台上,更何况他还没听到过这样的传闻。

社团加上文化节的准备,还要站上舞台,这旺盛的活力让人惊叹。\\

只不过,门胁自己并不喜欢显眼,所以才出乎预料。\\

「门胁真的是什么都会哎」

「这话可不该由你来说」\\

真昼似乎感到钦佩。但其实真昼也基本什么都会——学习、运动、家务样样精通,少有能像她这么能干的人。\\

「……我也有不会的事情」

「比如说?」

「……游泳」

「那倒是,到最后你还是不会游泳」

「如果觉得一天就能学会,那也太小瞧游泳了。我可是不管怎么练都掌握不了……」

「对不起啦」\\

听到「还是不会游泳」真昼或许有些不服气,拿拳头轻轻捶起了周的胳膊。周一边苦笑,一边重新看向舞台。\\

尽管不喜欢显眼,但门胁对于显眼似乎已经习惯。面对大量观众,他不露怯色,落落大方,带着柔和的笑容,甚至还轻轻挥手回应粉丝,看得出他是见过世面、处变不惊的。\\

接着,碰巧前面空了出来,不怎么挡住视线了。和周对上眼神后,门胁脸上微微抽起了筋。

他似乎没想到周会来。\\

「稍后再聊吧」周一边走近一边挥手。门胁眨了眨眼,然后露出了和刚才不同的笑容。

那一笑引得女生发出尖叫声,由于这一如既往的情形,周和真昼都没能憋住,不禁笑了出来。
