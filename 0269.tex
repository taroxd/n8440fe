\subsection{面向未来的面谈}

「三方面谈啊」\\

周一边感谢树和千岁用最大限度的委婉和自然去打听真昼相关的消息,一边也在勤于打工的同时一点一点为真昼的生日做准备,尽可能不让真昼注意到。\\

就在这样的一天里,发下来了一纸不太受学生待见的通知。

在文化节刚过的时候就向监护人确认过日程,也做过第二次的去向调查。伴随11月的到来,参加考试的考生和学校两边也到了需要正式会谈的时候了。

此次将是在父母陪同下正式确认志愿的去向,并结合学习水平和生活态度一同讨论。\\

周看过文件确认一番,他的面谈安排在最前面的一块,得今早告诉志保子才好。

由于三方面谈预计会开展的期间附近,志保子的工作安排有一定的调整空间,所以打从一开始就决定是志保子来参加。周很感谢她在远方还来配合自己的安排,但说实话,他对这事并不积极。\\

(妈妈怕是要兴高采烈地过来了)\\

对于儿子——不如说是对于真昼,志保子基本上是想疼爱和折腾的。很容易想象得到,既然准备要来,她就会双手赞同、洋洋得意地过来。\\

「噫,我妈这会儿来不了,不得不找爹了,真糟糕」\\

而这位由于完全不同的理由觉得麻烦——不如说是甚至带着厌恶感地让灯光穿过这纸通知,一副无精打采的样子的则是树。

早会结束解散的时候,他都还留在座位上一脸苦涩之相,其心中的不悦可见一斑。

看他脸上明明白白写着『不爽』二字,周自己心中的拒绝感则是没有那么强烈,他便只好垂下眉梢笑道:\\

「你还真是一扯上大辉叔叔就很拒绝啊」

「这次没办法吧,三方面谈之后我肯定得挨唠叨,说成绩怎么怎么的,品行怎么怎么的,去把志愿改成这个之类的」\\

周心目中的大辉和树心目中的大辉,由于蕴含的感情和所见的人格不同,自然是大相径庭。但不得不接受的是,树心目中的大辉就是那样的一个人。

千岁也来到旁边沉吟一声,露出稍显困扰的样子。\\

「我这边是我妈过来,估计会使劲打扮」

「我也是……就是说为什么爸妈都那么有干劲啊。还有些人穿得跟战斗服一样」\\

就算室内便服显然是太过随便,但若是穿着打扮凸显出浑身的干劲,则会让走在旁边的孩子不知所措,没那么熟悉的模样有时也会叫人尴尬。\\

「实质上确实跟战斗差不多吧?孩子们本来也是投身到炽烈的竞争里去的」

「考试是战争,倒也能理解」

「而且面子总归还是要的~在学校可能会被同学看见,要是在一起的时候被说些什么,孩子心里也不舒服吧?我感觉应该也有不让孩子和自己丢脸的因素在」

「这倒是说得通……不过我妈估计会超有干劲地过来」

「啊哈哈,差不多想象得出」

「求求你普通点……」\\

志保子会穿的无疑会是与场合相符的衣装,但可悲的是,考虑到能有机会见真昼、面谈是为了儿子的将来、同时学校又是周的父亲修斗的母校,这几项因素叠加上去,无论怎么想,得到的结论都是她一定会鼓足干劲。\\

想到这些就有点没劲了,于是周暂且忘记志保子的事情,悄悄瞄了一眼目前离席的真昼的座位。

真昼去图书馆有事,离开了座位。要是让她听见了刚才的对话,心里恐怕得起一阵波澜,周甚至心里不由得为她的缺席而庆幸。\\

(……这种事情可不能随随便便就触及)\\

从没听说过真昼的父母出现在这种场所。如果他们来过,免不了会有人目击到并引发一阵话题,所以料想十有八九是没来过的。\\

说到底,真昼有没有告知他们三方面谈的事都难说。\\

考虑到真昼对她父母的感情,以及父母方对她的感情,她恐怕会选择不提供任何信息。

或许,把事情告诉她的父亲朝阳的话,朝阳是会过来的,但也能想到真昼会拒绝。如今真昼早已割舍了朝阳的存在和干涉,她还是会做出不告知的选择。\\

「好了先不想啦,再想要惹人生气了!说起来官老爷,那件事情啊,我去打听到了哟~嘿嘿嘿」\\

像是为了打破现在的气氛,千岁用明朗的声音喊着,随后她一边慢慢降低音量,一边露出有所企图似的坏笑。周一边吐槽「脸、脸太近啦」,一边为在真昼回来之前换了话题而安心,而后看向千岁手中的笔记。\\

\vspace{2\baselineskip}

由于事先就已参考过大家的意愿,三方面谈的日子比预期中更早来临。

三方面谈是在下课后进行的,周便让志保子也在下课后规定时间之前过来。远远看到她站在来宾用的门口,那一刻周就发觉她还是相当积极的。\\

只要不说话,志保子看上去就是个温柔稳重的女人,而今天她的一身西装则是更加凸显了庄重而非柔和,要说的话,就是比她去工作时更加庄重。

仪容端庄,姿态挺拔,志保子带着一身难以接近的气场,这甚至无法从她平时的模样想象出来。\\

虽然是自己亲妈,站姿却又有种没什么用的偏年轻的年龄不详之感,路过的社团里还没走的学生,又或者是同时间段面谈的学生,都投去几抹视线,弄得周非常难以朝那边接近。\\

只不过,即便在这里踌躇,面谈的时间也改不了了。于是周下定决心喊了一声「妈妈」,志保子便一下子绽放出明朗的笑颜。\\

「哎呀周,差不多一个月没见了吧。看你活蹦乱跳的就好」\\

一旦笑起来,方才的表情和氛围就都烟消云散,很有志保子的风格。

或许是觉得不由得脱力很有意思,志保子很夸张地说着「哎呀哎呀,见到妈妈高兴得没力气了吗?」「咋可能哦」周朝她翻了个白眼。\\

自然地,即便充满干劲认真梳妆,她的内在也没有丝毫改变。发出银铃般的轻笑后,志保子缓步从走廊中出发。

距离安排的面谈时间还有好一阵,志保子应该也几乎不了解教学楼的结构,她之所以会走起来,是因为知道周会适度地为她带路。

周叹了口气,往志保子身后追去,距离不长,很快就追上了。\\

「周你啊,没什么事情就不联系,多让人头疼呀」

「没事还联系什么……?」

「哎,陪着聊聊天不好嘛」

「感觉全都是些没什么所谓的内容」

「闲聊不就是这样的嘛。主要是为了交流交流」

「好歹控制一下吧。还有也别偷偷把照片传给真昼」

「哎~」

「哎什么哎」\\

明明都说过一次了,志保子却毫无悔意,依旧把照片偷偷发到真昼手上,有关这一点实在是必须好好讲个明白。\\

「那就我、周和小真昼拉个群,发到里面,就不是偷偷的了」

「我的意见呢!」

「开玩笑的」\\

志保子平淡地开了个完全听不出是玩笑的玩笑。周使劲皱起眉头,结果志保子还宣称「哎呀,年轻时就这样,之后皱纹会留在脸上的哦?」周觉得,等将来自己老了,要是脸上的皱纹多得异常,那都得是志保子的责任。\\

「还有,在好好学习吧?」\\

等到脸上的皱纹终于散开的时候,志保子依旧是直白地问道。\\

「看最近的成绩单也看得出吧」\\

考试的分数、排名和成绩单,这种东西周是都会发给父母的,完全没有隐瞒,志保子不可能不知道。\\

「话是这么说啦。不过你的视角跟老师的视角关注的东西肯定是不一样的,所以也得问问你感觉如何」

「……我有在好好学习的,至少从来没落下过努力。活得对得起自己——可能还算不上,但我在朝这个目标努力」\\

高一那会儿,由于本身认真的性格,成绩也还尚可,但那只是因为被要求保持成绩所以才努力了一番,并没有目标。没什么特别想做的,也没什么需要做的,因为是学生,所以才学习,仅此而已。\\

而认识有所变化,大约是从高二开始的。\\

为了站在真昼旁边也不致真昼遭受非议,也为了能自发地感到自豪,周开始着眼未来,有坚定的意向去努力了。

或许,说是心态发生了变化也不为过。\\

并非浑浑噩噩地维持成绩,而是能够为了自己而明确地努力,或许这可以说是最大的变化了。\\

动机和心情变得积极,干劲也不同了,这一学年的成绩目前来说比高一更优。顺着这个势头的话,学年末的总评料想也会相当不错,这更加激发了周的干劲。\\

「嗯,倒是本来也知道」

「我说啊……」

「周是一旦决定就会坚持到底的人吧」\\

周正要往下说,那直白而又不带半点疑虑的话语,便足以打消他内心的不满。\\

「毕竟是我的儿子嘛,17年来都看在眼里呢。你这种地方都很守规矩,至今为止认真做的事情都能做点成果出来,而且——」

「而且?」

「有小真昼在,你也马虎不得吧?男生总是喜欢表现表现的」\\

志保子带着一些玩笑般淘气的笑容抛了个眼神,周则抿紧嘴唇,扭头转向一边。\\

「好烦啊。好了妈妈时间快到了赶紧走吧」

「哎呀哎呀」\\

「是不是给说中了呢」似乎听到后面还加了句多余的小尾巴,周将这些全部无视,领在在欢快地笑着的志保子前头,较之先前更加快了步伐。\\

\vspace{2\baselineskip}

原本的安排中,三方面谈也就只占用10到15分钟的一点点时间,很快就散会了。

周自身的品行就属于优等生那边,成绩也没有问题,志愿的大学也大致符合他现在目前的定位,谈话进展非常迅速。

快到高三了,这是准备考试的高二学生最后一次三方面谈,本以为会更耗时间,谈得更为深入,结果只是班主任、周和志保子确认下想法就结束了,不得不说有点扫兴。\\

行礼离开面谈室,再走了一段后,志保子摘下了先前正经母亲的面具,露出平时轻快的笑容。

原本她还作为一个母亲严阵以待,而听到班主任给出的好评,便更加放心了吧。\\

「辛苦了。在老师眼里你在学校表现也不错,那就再好不过了。虽然本来也没担心,但听老师说的你比我想的还要努力,我还是很开心的」

「努力学习本身也是一个人住的条件吧」\\

要说动力,高一时和高二时是一个天一个地,但就成绩本身来说,高一时也是足够的。

到了现在这样,周也完全不觉得自己会被带回去。但毕竟是曾经许下的约定,这方面于情于理也是应该做到无可挑剔才是。\\

「啊,那个是觉得这么说才能让你有点压力,我觉得不那么说的话你也一样没问题的啦。毕竟你性格还是挺认真的」

「说得好像我平时不认真似的」

「哎呀,虽然平时就认真吧,但怎么说呢,都是些平平无奇地持续在努力,乍看上去看不出干劲如何的那种。现在就很积极,一个目标还不够,会去找到并把握住一个接一个的目标,特别认真的那种?算是升级了?我觉得是件好事」

「……挺好」

「比起高一的时候成绩也上涨了很多,我是没什么意见的啦。而且启动干劲的开关也在身边」

「也不是为了真昼什么的,都是就为了自己。不过看着真昼我会感觉不得不努力,好像燃起来了一样,倒也是事实」\\

要说的话,周虽然自认很认真,但一旦对上真昼,两者的自律就可谓是天壤之别,简直不配放在一起比较。

周从未见过真昼那般严于律己的人,也知晓她那名副其实的能力所需要花费的努力非比寻常。

她已经粗学完了高中范围的知识,进入了迎考和巩固阶段,定然是付出了难以估量的努力。\\

按她自己的说法,「为了今后的轻松付出的努力其实并不算很吃苦」,态度非常平淡,甚至让周担心她是否太勉强自己了。\\

或许可以说,就是在那样的真昼旁边,周的动力才更被激发出来。\\

真昼在努力,只有自己差不多就凑合,这样不求上进的心态是周所不喜的,他自然也就被真昼带着更加热衷于学习,同时也当作是磨练自身。\\

「能够相互切磋进步的关系真好呢,各方面都很热火朝天」

「我说啊」

「哎哟,别瞪我了,在夸你呢夸你呢。跟小真昼关系好不是挺好的嘛,有什么不满的」

「不满的就是妈妈对我的认识和折腾人的地方」\\

要说的话,周是当今年代稍有的跟母亲关系极为良好的人,但也并非完全没有不满。\\

(妈妈就是经常添上多余的事情)\\

也不知是故意还是缺根筋,又或者是无论如何都想把真昼收为女儿。

不管是哪个,又或者是以上全部的原因,志保子一旦扯上真昼,就常常会催促周,或者说是故意逗弄,煽风点火。\\

「哎呀真过分,不就是点小小的交流嘛」

「不停做别人讨厌的事情可不是什么交流」

「知道了知道了,是我不对」\\

嘴上这么说,志保子也没什么愧疚的样子,周皱起眉头,再做作地叹了口气,给志保子带去一点罪恶感,也就算是和解了。

志保子用轻快的步伐在走廊中穿行,没什么反省的表现。周按着额头原路返回,忽然志保子停下脚步看向窗外。

周也跟着站定,便听到了先前没怎么放在心上的社团活动中学生的呼喊声。\\

为了传清楚指示而饱满的声音、为了协同配合喊出的声音、由于获得成果而叫出的兴奋的欢呼声、当作信号的哨声,混在这些声音之中,还有某处教室传来的吹奏乐社团的演奏声,就仿佛在为他们加油似的。\\

「真好啊,青春的声音」\\

志保子看着远处学生小小的身影,似乎觉得有些晃眼的样子。她静静地笑道。\\

「先不说这个了。周之后是准备认真学习备考了吧?」\\

在周正打完问是有什么心事的时候,志保子又恢复成了往常的表情,用一如既往的眼神盯着周。这时候再问,恐怕也得不到回答了。

他决定忘掉刚才那仿佛融合了乡愁和艳羡的眼神。\\

「……那肯定。学校推荐组到一年后就已经在考试中了,现在甚至都有学生考完了。还剩的时间就只有一年了」\\

那还打工是不是有点太莽了?这一问题浮现在脑海中,但这问题毫无意义,很快便抛到脑后去了。自己决定的要两头开花,那自然是硬着头皮也得上。\\

「那明年要忙起来了呢」

「从高二到高三差不多就是这样的吧。虽然有点不太想去把日程排满」

「真可怜啊,这就是考生要走过的路」\\

也没什么人是主动去拼命学习的,是因为不得不做,认为为了自己应该这么做,所以才会认真备考。

周已经做好了心理准备,接受了自己之后会很忙的事实。志保子笑着说「真不错,这些也都考虑在内了啊」。\\

「总之,今年冬天回家一趟吧。明年要考试,应该没什么空闲」

「……虽然早有准备,但现在就要考虑未来,总觉得有点扫兴啊」

「嘻嘻,好一张苦瓜脸。确实不是什么轻松的事,我上学那会儿就是见过地狱的」

「妈妈之前很聪明的吗?」

「意思是现在很笨咯」

「哪听出这个意思的啊!说的是当时的成绩!」\\

至少现在的志保子属于智商比较高的,而且知识广博(包括一些多余的知识),谈吐也很理性。\\

按正常标准来说,她应该能算是聪明的那类,但还是想象不到其学习成绩如何。

从出生开始的这17年间,周深知一旦惹得志保子不开心,短时间内是救不回来的,于是他慌忙继续说下去以解开误会。志保子一瞬间露出了冷冰冰的眼神,最终也还是一句「真是的」将这事情带过去了。\\

「嗯~跟修斗比的话我没他那么聪明,要说当时的成绩的话应该算是普通吧。我也没什么特别突出的技能,应该就是个平平无奇的学生啦?」

「平平无奇啊……」

「露出那么怀疑的眼神是干什么。别看我这样,我当时也是个朴实无华的乖女孩」

「朴实无华的乖女孩啊」

「从刚刚开始有什么想说的可以说明白哦?」

「没什么」

「你啊」\\

虽然被瞪了,但要是周说出多余的话,显然会更惹得志保子生气。在对待母亲方面有一定心得的周选择了沉默。\\

见周背过脸去,志保子知道再多问也没意义了,嘀咕了一句「真拿你没办法」,而周则依旧保持无视。\\

「总之,我没有特别优秀,也没认真到能被夸奖的程度。虽然定下了去向,但是比较突然,为了备考急忙临时抱佛脚,当时可厉害了,面相应该都变了」

「面相是说」

「总之就是人都要死了,脸上一点从容都没有。当时问朋友,都说我好像被魔鬼上身了一样,简直像是要把自己逼疯似的」\\

面貌看着有种温厚氛围的志保子能让朋友说出被魔鬼上身这种话,着实是让人意外。周不由得又往志保子那边看去,志保子则一点没有那样的迹象,只是表情如常地淡淡点头「现在也可以说是缺少计划性」。\\

孩子眼中的母亲与朋友眼中的母亲不同是常有之事,但周怎么都想象不出志保子发疯一般的表情。

现在的志保子正展露着明快的笑容,面对周的视线耸了耸肩。\\

「好啦。跟我不一样,需要事先准备的事情你都做了,平时也都认真扎实地打下了基础,我也没有那么担心。我觉得你应该不会搞砸,也知道你是清楚自己的实力做出的选择」

「那自然是」

「那我就不担心了。不过关于去向,倒是希望你能事先跟我商量一下」

「那个真的很抱歉」\\

当然,周完全没打算擅自挑个大学去考试,所以事先跟修斗和志保子说过志愿的大学和院系。

毕竟是父母出学费,这些还是需要征询父母的意见的,若是资金不足,周还想过选择其他大学或是利用奖学金制度,但父母十分轻巧地答应下来,没有起一点波折,事情也就顺利地进展下去了。\\

「周你自己喜欢的话,去哪里都可以,资金也不是问题,可以放心地去选择,这些也都是我们讲过的就是了。我们也知道要细说未来规划是有些难为情的一件事,常常跟你说,只要是考虑清楚之后做出的决定,我们就不会阻止。不过作为父母,我们肯定还是很在意的」

「……对不起」

「你不想对我们说也很正常,不过既然有个很好的目标,当面说出来我们才能给你加油哦?」

柔软、圆滑的声音轻轻传来,并没有责怪之意。周知道这事是自己做得非常不好,一股愧疚之感扎向心中。\\

即便现在周不说,志保子应该也不会责备,但周作为儿子也知道母亲那么说是为他担心,于是带着些犹豫,缓缓垂下眼帘,整理自己的心情。\\

「……说目标吧,其实我也不是特别挑剔,只要可以从事一个能支持跟真昼体面地生活,既充实又有适当闲暇的工作就可以了」\\

周选择的大学和院系,虽说姑且是跟父母商量过,但基本是周仅凭自己的意见决定的,他并没有非去不可的执念。\\

「我知道我是考虑了自己的实力,选择了最有利于就业的大学院系,而并非为了做自己喜欢的事。当然,得是自己感兴趣的领域,这是大前提」\\

周是在他自己想学且学得来的领域中,挑选了一个通过自己目前的学习水平和今后的努力可以考上的,并且还有助于就业的大学。

与其他一些对将来想要从事的职业、大学想做的事情清晰明了的学生相比,周的决定方式就更偏向于随意那一边。\\

而周不太愿意主动说出来,一定程度上也是认识到了这一点的缘故。\\

他对报考大学以及为之付出努力自是有心理准备的,也打算用出全力,这样才对得起自己,但要说未来想做什么,他便会一下子没了自信。\\

「第一是步入社会时要维持体面的生活,其次是时间要有自由,最后是对职业的个人喜好,总之我想要拥有健全的生活,这就是我心里的条件。大学原本是去学习专业知识的地方,但我的热情还不足以支撑我单凭这一点做出选择……总会优先考虑在那之后的事情」\\

为了考上大学,周倒是满腔热血,而他目前却还没有今后要怎样怎样的理想,也没有足够的热情让他下定决心去特定的地方学习。

热情存在却又缺失,这番矛盾让周在内心呻吟,志保子则是不怒不悲,只是静静凝视着周,仿佛在倾听、应和。\\

「看不出有没有梦想呢。这种现实主义倒是很符合你的性格」

「也不算是现实主义吧,就是没有完全决定好,所以列了下条件」\\

打从一开始就有想做的事情,想好了毕业后选择这方面的职业,再比较企业的条件做出选择,这是周现在做不到的。\\

「很羡慕已经明确决定好了想做什么的人。我是因为想在远离家乡的地方静静生活,才来了爸爸的母校这边,现在渐渐习惯,也构建出了自己的栖身之所……但说到底,要说自己想成为怎么样的人,又或是想要做什么,这种理想我是不太有的」

「我也是一时兴起就去了艺校,也不好对你多说什么。不过你可要想清楚了再选,这可是你自己的人生」

「知道的,这种人生大事的选择」\\

学生时代就一定程度地打下了人生的基础,这一点周心知肚明,也是因此才摇摆不定,难以抉择。

这种时候,父母会尊重周的自主性,一切的选择权都交由自己和自己的实力,故而他会感到不安。\\

与去向被父母决定,又或者是由于钱的问题而放弃升学的学生相比,这或许是相当奢侈的烦恼,但他也更加意识到,自由也就意味着相应的责任。\\

亲手做出的选择,即便结果是惨痛的失败,也都是自己的责任。\\

「我们也就会干涉到你独当一面为止,之后你们还要两个人生活下去吧?这是你今后要开拓的道路,好好烦恼后再做决定吧」

「我知道的」

「还有,想成为怎么样的人,想做什么事情,这些都是可能变来变去的。但现在就该去掌握一些知识和技术,等到那时候选择道路才能免于辛苦。趁着还是学生,要优先拓展自己的技能,到了之后再想要拓展,时间和金钱可能就不允许了。还能依靠父母的时候,可要多多依靠」

「……嗯」

「你放心,别看我和修斗这样,两个人一起工作也是有积蓄的,为了你能一个人平安自立,我们积攒下了许多东西,就尽管依靠我们吧」\\

志保子的态度始终是尊重周的自由,相信他,为他助力,她理解周所烦恼的事情,在此基础上,仅仅是在周的背后推一把。

这种时候,即便平时觉得母亲让人头疼,但周也不得不明白,志保子本质上还是个包容的好母亲。一股暖意逐渐渗透周的内心。\\

而志保子自己,则是保持着平时的微笑,充满自信地拍着胸脯,也不好说她知不知道周的感动和感谢。\\

「嘻嘻,周你总是自顾自地努力,稍微可以依靠我们的。啊,不过学习方面的事情我也有点不放心,这方面就去找修斗吧」

「这方面话不敢说满倒是很有妈妈的特色呢」

「没那金刚钻,不揽瓷器活嘛」

「这其实也就是表示对学习没有自信咯」

「你刚才有说什么吗」

「没有」

「真是的。啊,相应的,关于打扮自己的问题都可以问我哦?妈妈为了周可是能打起十八分精神的」

「免了」

「我说你呀!」\\

咚的一声,背后传来沉重的声音和冲击,不过那并没有带来疼痛,反倒是如字面意思般从背后推了蜷缩不前、心生怯意的他一把,仿佛有一股风不经意间扫荡了心中积下来的沉淀。\\

本觉得自己还挺胆大,看来也会有幼稚的时候。周这时候有了一些能对自己感到无奈的从容了,见志保子始终一脸轻快的笑容,他也跟着咧嘴微笑。\\

「也该去小真昼那边了吧。她的面谈是今天吗?」

「真昼是明天」\\

志保子言者无心,周却是无法再说什么。\\

他想得到,名为三方面谈,对真昼而言实则是一对一的面谈,提及这个话题,就怕会在真昼心中扎下一根小小的刺。

志保子可能也对情况有一定程度的了解,她并没有说出周所担忧的事情,而只是露出遗憾的模样「哎呀是吗,要是同一天的话,就能一起回去顺便买东西了」。\\

「要不就在家里跟她打招呼吧。明明才见到不久,就像是很久没见到了一样」

「就这样吧,真昼也会高兴的」

「嘻嘻,不拦着我见她啊」

「拦也没用。况且真昼和妈妈见一面,两边都开心的事情,我怎么可能拦着呢」\\

志保子会灌输多余事情的担心、真昼能见到她坦率地敬慕的志保子所带来的喜悦,这两者之中自然是选择后者。

原本就爱撒娇的真昼能够坦率撒娇的对象,除了恋人关系的周,就是如母亲般敬慕的同性的志保子了,周自是不会拒绝真昼跟她在乎的人之间的重逢。\\

话虽如此,周仍然担心志保子会不会向真昼传授奇怪的知识,他是确定要在旁边监督的。\\

(毕竟妈妈看真昼纯真,偶尔会灌输奇怪的事情)\\

要是没有修斗的救场,志保子就会刹不住车,兴奋地往真昼耳朵里发送一些对真昼而言还为时尚早的内容和一些和周有关的多余知识,在这方面周是完全信不过志保子的。\\

「你温柔了不少啊」

「妈妈能沉稳点的话,我有信心能更加温柔」

「好过分,居然说我不沉稳」

「求求你小点声,还有手势也收敛点。不然说什么沉稳呢」\\

在儿子面前总是做出比真实年龄年轻太多的举动,要是能收敛一些的话,自己还能更加尊重一点——周没有说出口,心里却是这样想着,而志保子则是耸耸肩,看那表情好像是在说周过于神经兮兮似的。\\

「……变得不可爱了啊」

「本来就不可爱,随你怎么说吧」

「你看,说的就是这种地方」\\

志保子就像是有些无奈的样子,却又欢快地笑着戳了戳周。周故意地长叹一口气,然后加快了脚步。
