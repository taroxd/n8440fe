\subsection{烦恼之源是回礼}

情人节过后,接着就是月底的期末考在等着。浮躁的气氛过了几天后,又转变成沉重的氛围,学生们也各自开始为集一年之大成的考试做准备。\\

「藤宫,可以打扰一下吗?」

「嗯?」\\

由于考试将近,周今天打工休假,打算去书店买参考书和备用的笔记本,所以放学后和真昼分开,各自行动。当他在教室准备回家时,突然被班上同学叫住了。

周和对方不怎么说话,所以心里有些困惑,不知道他有什么事,而对方不知为何用求助的眼神看着这边。\\

「太刀川,有什么事吗?」

「那个,上周上课的笔记可以借我复印一份吗?」\\

还以为对方会提出什么要求,结果没什么大不了的。

周的笔记有把黑板上的内容抄得很整齐,给别人看也没问题。借出笔记本身没什么,但周总觉得太刀川的个性比较认真,所以对他没记笔记这件事感到意外。\\

太刀川似乎察觉到周的疑惑,垂下了眉梢。\\

「只要最近的就可以了。我得了流感,这段时间请了假。本来想拜托其他人,可是他们都说没记笔记」

「哦,你是请假了来着。可以是可以,不过为什么找我?」

「因为现在在这里的人里面,我方便找的就只有你,而且你看起来很认真地在记笔记。再说,藤宫你成绩很好嘛」\\

的确,现在留在教室里的有周、树,还有几个女生以及太刀川的朋友。真昼说要和千岁进行一对一的读书会,所以和周分开行动。

同性和异性之间会选择前者,而且太刀川也不是会积极地去和女生搭话的那种人,那么人选必然会缩小到周和树身上。从他本人的发言来看,选了周应该是因为成绩吧。\\

「谢谢你的夸奖,可是找我借真的好吗?」

「就是你才好。你好像很经常照顾白河,感觉也会把考试可能会出的重点都记下来。就算椎名现在在这里,我也不好意思拜托她,而且这也不太对得起你……」\\

周并不会因为真昼和别人接触而吃醋,他觉得只要真昼觉得没关系,那也没什么不好。不过,现在拿不在场的人说事也没有意义,既然太刀川说想要借周的笔记,周也不会拒绝。\\

「我会给你谢礼的,拜托了!」

「不用什么谢礼啦……来,拿去」\\

太刀川这么拼命地拜托,大概是因为考试迫在眉睫,时间紧迫的缘故吧。

周不打算让他着急,于是从书包里拿出活页笔记本,连同整个活页夹一起递给她。太刀川的表情随即变得明朗起来。\\

「感激不尽!我马上去复印!」\\

太刀川大概是不好意思让周等太久,便抱着活页夹冲出了教室。周目送着他的背影,一边笑着感叹他还真有精神。这时,做好回家准备的树轻快地凑了过来。\\

「咋了?」

「太刀川来借笔记,我就把笔记给他了。他应该去一楼的复印机那里了,很快就会回来的吧」

「原来如此。我知道他为什么选你了。毕竟你在这方面很认真嘛」

「说得好像我其他方面都很随便一样」

「你在和椎名变得要好之前,房间不是乱七八糟的吗?」

「……那是两码子事」

「好好好」\\

虽然从现在的情况想象不出,而且周也不愿意去回想,但刚认识真昼的时候,周的房间简直可以用混沌来形容。\\

(……现在都有好好收拾)\\

自从真昼教他打扫和整理之后,周的家里就一直保持着整洁有序的状态。

由于周有主动打扫和整理的习惯,所以没有发展成真昼频繁指指点点的情况。\\

「毕竟大家都知道你的笔记整理得比较干净,而且有重点,没有多余的废话,很容易看懂」

「我是很高兴能得到这样的评价,不过为什么大家都知道?」

「因为大伙儿有时会瞅见你把笔记给小千看吧。椎名也夸奖过你,所以大家才有这种认知」

「是没差啦。对了,千岁学习这块没问题吗?期末考试快到了,考试范围是一整年,还挺大的吧」

「她都在哀号了」

「我想也是」\\

经过年初的树离家出走事件,两人都有了心境上的变化,最近非常认真地投入学习,可是在那之前的课程内容并没有学得很好。

说起来也是理所当然的,他们正在重新学习之前敷衍了事的部分,而且考试范围也很广。

如果只是改变心态、勤奋向学就能轻松记住的话,那就不用那么辛苦了。

这种时候,树总是能找到合适的平衡点,看起来不像千岁那么辛苦。\\

「今天她好像要跟椎名一起好好学习」

「是这么说的。所以今天放学后我们不会一起回去……真昼是两人独处的时候才会斯巴达教育,希望她好好加油」

「嗯,感觉她会发信息来哭诉」

「她又不是要逃走,应该没问题吧。先不说这个了,你今天不用打工吧?」

「不然我就不会留在班上了。我打算回家路上去一趟书店」

「那我也一起去。我想买高三的参考书」

「你要来是没关系,话说你也变得很认真了呢」

「是啊。总不能一直逃避下去,我已经下定决心了」

「那就好」\\

在那之后,树豁然开朗的态度让周瞠目结舌,他认真念书的样子甚至感化了班上同学。\\

或许是因为这样,周的班级原本就被评价为比较认真的班级,现在老师们对他们的评价又比以前更高了。周感慨地想,事情的发展真是难以预料。\\

「喂——藤宫,谢谢!」\\

这时候,太刀川以高速球的速度完成复印,冲了过来。周从笑容开朗的太刀川手中接过活页笔记本。\\

「还有,对不起!」

「为什么要向我道歉?」

「呃,因为复印的范围比说好的更多……对不起」

「啊,这意思,这点小事没关系啦」\\

复印费是太刀川出的,周并没有损失。他大致浏览了一下,活页笔记本没有破损或污损,所以周个人也没什么好抱怨的。

老实说,周反而对太刀川的诚实感到欣慰。明明不说周也不会知道,他却特地向周道歉,可见他的为人有多好。\\

「真的很感谢你。这份恩情我一定会……!」

「不用啦,别放在心上。我又没有出什么钱」

「话是这么说没错啦」

「没关系、没关系。啊,角落的备注栏有用红笔写的地方,是老师说会考的,复习的时候最好重点记住那些部分。听说去年和前年都考过」

「帮大忙了!」

「哎,我也要看、我也要看」

「你上课有认真在听吧」

「我想看看我抄的笔记跟你的有什么不同」

「真拿你没办法,给你」

「为什么对太刀川就这么温柔,对我却这么严厉!好过分!」

「人家是因为不可抗力才请的假,态度还那么好,那能一样吗」\\

一边是出于无可奈何的理由请假而没记到笔记的人,一边是出于好奇心而纠缠的人,两者区分对待也是合情合理。

树故意装哭,太刀川哈哈大笑。周看着他们,也跟着笑了起来。\\

\vspace{2\baselineskip}

「……世事真是变化无常啊。情人节才刚过,马上就出现白色情人节的特辑了」\\

周与太刀川分开后,跟树一起来到书店所在的购物中心,里面的装饰跟上次来的时候已经截然不同。\\

话虽如此,这家购物中心的装饰风格还是老样子,只是宣传标语换成了与白色情人节相关的内容。活动场地的展示柜也一直摆在那里,只是里面的东西从巧克力变成其他点心与杂货。\\

周他们在书店买了想要的东西,顺便在购物中心闲逛,食品类的店家处经常能看到白色情人节的宣传标语。\\

「别忘了女儿节啊」

「在现代,企业的营销战略比传统节日更重要吧」

「毕竟对商家来说,大笔的金钱流动是好事。从可以吃到美食的角度来看,我们或许也应该感谢」

「这是能收到的人该说的话~」

「不是,也有些人会自己买的吧……」\\

就算被这样调侃,周也无可奈何。最近在活动会场有知名店家入驻,推出巧克力庆典,为此感到期待的人也逐渐增多。

很多人在理解这是企业战略的前提下,依然愿意参与活动,周对此也没什么要吐槽的。再说,受惠的人也没资格说三道四。\\

去年周也受惠于现在的白色情人节特辑,所以更是如此。\\

「对哦,白色情人节快到了」\\

情人节收到巧克力的人,要在这一天回礼。\\

「这是优太每年都很伤脑筋的节日」

「感觉会花很多钱」

「应该会吧。即使如此,他还是会规规矩矩地回礼,真的很厉害」\\

不少同学都很羡慕优太,但周一点都不羡慕。

收到那么多情人节巧克力,别说吃完了,光是带回家都很辛苦。周就见证过他辛苦的样子,树也帮过忙。\\

而且还必须全部吃完。以优太的性格来说,他不可能丢掉,如果要全部吃完,就必须在热量和营养上进行管理。

不仅如此,他还必须记住送礼对象的名字并准备回礼,考虑到这些劳力,周绝对不会说,也说不出想像优太一样这种话。\\

「我知道他在这方面很认真,真的很了不起」

「他从初中开始就很注重这些」

「仔细想想,校规呢?」

「只有情人节和白色情人节是默许的,因为学生施加的压力太大。以前想管的时候,好像还引起了强烈的抗议」

「这种时候的团结力真可怕」

「女生的力量真是惊人。顺便问一下,你白色情人节打算怎么办?」

%
「就是啊,白色情人节的回礼最难选了。考验品味和平时的观察力」\\

周也一样,不知道该怎么回礼。

周和优太不同,收到的基本上是义理巧克力,回礼也相对简单,但问题在于送给真昼的回礼。

去年他和店员商量后,买了手链和三张不管什么事情都会答应的券,今年还在犹豫要送什么。

去年他们还没有交往,周也没有明确意识到自己是把真昼当作异性来喜欢的,但今年不一样了。\\

今年的前提是,要送给女朋友。\\

虽然大致有了一些想法,但他还是会烦恼。\\

「除了椎名之外,你今年要好好回礼的量也增加了啊」

「真昼以外基本上是义理巧克力」

「基本上,是吧」

「无可奉告」

「我不会过问你的私事啦。太烦人的话会被你骂的」

「既然知道,就请你平时就彻底遵守」

「喵哈哈」

「那是你敷衍我的笑法」

「先不说这个了」

「喂」\\

关于那次告白,周考虑到小西的心情,所以没有告诉树,但树似乎也隐约察觉到了,这就是树可怕的地方。\\

「适当挑个不错的点心包装一下送出去,这样应该最稳妥吧。反正对方送的也是义理巧克力,送消耗品比较保险。我也会收到班上女生送的巧克力,回礼都是些不会有残留的东西」

「消耗品的话,最好选耐放一点的东西吧。如果是食物的话,最坏的情况下还可以丢掉……顺便问一下,千岁有什么想要的东西吗?不管怎么说,她花了很多心思,我也想回赠相应的礼物」

「她在很多方面都费了不少工夫啊」

「那家伙为什么那么认真地想要捉弄我……」\\

周很想吐槽她在不必要的地方花太多工夫了,不过这大概是千岁深思熟虑后的结果吧。考虑到去年的情况,除了千岁的『中奖』之外,其他巧克力的味道应该都很好,而且还有真昼的保证。

比起一般的手工巧克力,千岁的巧克力更费心费力,周打算把她的热情也纳入回礼的考量。\\

「啊~我觉得她是因为你的反应很有趣,又想发泄学习压力,顺便满足兴趣。毕竟她自己也吃得很开心」

「不,基本上是很好吃啦,基本上。只是有几颗特别糟糕的巧克力想给我的味蕾来点打击而已。不过,我还是会好好吃下去的」

「椎名应该会帮你清清口吧」

「要你管。千岁没有什么想要的东西吗?」

「大概是学习的动力」

「那得靠她自己找出来」\\

虽说千岁从新年过后就开始认真读书,但心情难免会偏向不想读书的那一边。尽管如此,她还是有在努力,这点很了不起。\\

「如果没有想要的或者喜欢的东西,那就送她喜欢的糕点店的点心礼盒吧。毕竟学习也需要补充糖分」

「要是让小千听到这句话,她应该会泪目吧」

「那只是不想考虑读书的事情吧」

「哈哈哈。不,还是有在努力的啦」

「我知道……至少我知道她在积极地努力着。你也是」

「嗯,到了这个时期,可能任谁都会这样吧」

「即便这样还是有好好努力,我觉得很了不起」\\

一旦下定决心就会坚持到底,这就是树这个男人的作风。

尽管对父亲有些意见,但为了争口气和取得谈判的立场,树依然认真地努力着。看着他那副模样,周也绷紧了神经。\\

「呃,别突然进入娇羞期,我会害羞的……所以呢?你打算回送什么给最重要的椎名?」

「啊,关于这个。从之前的生日和圣诞节等活动就能看出,真昼她不会说想要什么东西」

「感觉她没什么物欲呢」\\

正如树所说,真昼没有明显的物欲。

也不是完全没有,应该说她对东西没有执着吧。\\

对真昼来说,想要的东西并不等于必需品。如果是必需品,她就会毫不犹豫地买下来,但她没什么物欲,甚至会烦恼自己想要的东西是什么。

对男朋友来说,给她挑选礼物是比送其他人礼物更困难的事情。\\

「比起礼物本身,她更看重的是为她着想的心意。不对,她收到礼物也会很高兴,但感觉上来说,花在挑选礼物上的时间和心意更重要?」

「我懂你的意思。话说,你收到椎名的礼物时也差不多是这样吧。说起来,你也没什么物欲」

「要你管」

「你们两个还真像啊」\\

周并不是没有物欲,但他更在乎现在的日常生活,比起有形的物质更想要美味的饭菜、整洁的房间和安稳的时间,所以不太会想要什么东西。

真昼的性格恐怕也差不多,所以周不否认他们很像,但树调侃的语气实在让人火大,于是周瞪了他一眼。

果不其然,这一瞪对树是完全没用的,周夸张地叹了口气给他看,然后用手肘轻轻撞了他一下。\\

「那你有什么主意吗?」

「也不是主意,只是想到一个她可能会喜欢的」

「嗯嗯」

「让她来我打工的地方,她应该会喜欢」\\

周想到的真昼可能会喜欢的事情,就是让她看看自己打工的样子。\\

因为周还说不行,所以真昼就老实地退让了,但她本人好像非常想去,一直按捺不住地想要看周穿打工制服的样子。

周自己也疑惑,这真的是那么让人想看的东西吗?不过,假如真昼也在同样的地方工作,周应该也会想去看看吧。所以他并不否定她的欲望。\\

(话说回来,她也太心神不宁了吧)\\

看到她那么明显地表现出期待和爱意,周反而担心自己能不能拿出配得上她的心意的工作表现。

因此,周一直请她等到自己习惯这份工作为止,但她好像快要等不下去了,一副快受不了的样子。所以周才想,趁这个机会邀请她去看看应该比较好。\\

「我反倒想问,为什么你还没找她去?」

「因、因为很忙,而且我还没习惯……还有,就是很害羞吧。穿着打工的制服接待客人,不是比被看到奇怪的表情或者刚睡醒的样子还要害羞吗?」

「意思是被看到刚睡醒的样子是常有的事」

「闭嘴」

「你不否认啊」

「……只是偶尔留宿而已。而且不是你想象的那样」\\

树的嘴角勾得越来越高,周再次用手肘顶了他一下,低声说道。

留宿的频率虽然不高,但会定期发生。

树胡思乱想的那种事情都没有发生过,这是周和真昼之间的约定,或者说是周许下的誓言,但树相不相信又是另一回事了。\\

(在这家伙的眼里,我可能是个胆小又因为太喜欢对方而不敢出手的男人吧)\\

不是胆小,绝对不是。\\

「好好好,我知道了,你就是个害羞鬼。所以,你差不多习惯到可以让她看了?」

「都四个月了,就算不能说已经独当一面,也多少习惯到可以给别人看了……真昼应该也等不及了吧」

「只要是关于你的事情,椎名都会很感兴趣。你够被爱着呢」

「我知道。我很感激」\\

真昼全心全意的爱,身为当事人的周最能切身体会。不只是从她的话语,也能从眼神、态度和一举一动中感受到她的爱意。

正因为喜欢,才会尊重周的意见和心情,这让周在她面前抬不起头来,也想回报她为自己着想的体贴。\\

「变得能坦率地接受,是你最大的变化吧」

「因为有人一直骂我太自卑」\\

以前的周经常被说成是自卑的人。

亲生父母也说过,但最常这么说的就是眼前这位最好的朋友和最爱的人。他们总是敦促着消极的周,从背后推着他前进。

周不认为自己完全改掉自卑的习惯,但努力过后,他对自己有了自信,也能好好向前看,衡量自己的位置。

回想起来,以前的周真是个阴沉自卑的边缘人。他身上的变化,已经足够让他感慨万千。\\

「至少有三个人骂过你吧」

「……真是非常感谢」

「哪里,不用客气」\\

虽然也有很强势的推动力量,但多亏了他们,周才能成为现在的自己,所以周感谢不已。

不过,因为树真的踢过他的屁股,所以周在心里发誓,以后树要是再忸忸怩怩的话,一定要踹他一脚。要是跨年那阵子踹他一次就好了。但周也觉得,那个机会已经不会再来了。\\

「总之,现在我能想到最能让她高兴的回礼就是这个了。虽然不知道算不算得上回礼」

「不错啊。椎名应该也会觉得你想怎么做就怎么做最好」

「……但愿如此」

「为什么这时候就没自信了?你刚才不是还很得意吗?」

「哎,打工时的举止和工作方面是完全没问题。只是真昼对我打工的样子抱有太大的期待了。我就是很普通啊」\\

周是打算让真昼看看自己认真工作的样子,但还是有点担心她会不会过分期待了。

在打工的地方,周穿的是白衬衫和黑色背心,同色的男用围裙和长裤,简单来说就是服务生的制服,没什么稀奇的。\\

她真的会因此满足吗?\\

「我觉得椎名会很兴奋就是了」

「很兴奋?」

「椎名在学校里虽然很稳重,但和熟人在一起的时候,她会表现出更真实的一面吧。只要是和你有关的事,她都会天真地感到高兴,所以打工的样子对她来说应该也是很刺激又兴奋的事。还有,椎名可能有制服癖」

「真昼身上本来就有各种各样的嫌疑了,拜托别再给她增加新的嫌疑」

「嫌疑……?」

「为了真昼的名誉,我就不说了」\\

最近,不知道该说是多亏了某个超喜欢肌肉的少女,还是该说被她影响了,真昼对周的肉体表现出了更多的兴趣。\\

她会用毫无邪念的眼神触摸、观察周的肌肉,然后露出心满意足的样子,所以周也无法责备她,只能任由她摆布……要是再追加制服癖这个要素的话,事情就麻烦了,希望只是杞人忧天。\\

周希望她只是因为喜欢自己才看,才会感到好奇。\\

「别说得那么让人在意啊……反正都是能用喜欢你来解释的吧。真的有那么在意吗?」

「别、别再聊这个了」\\

本人不在场,而且还是在猜测中被下定论,这不是什么好事。周挥挥手打断话题,不让树继续追问下去。\\

「总之,我打算送这个当回礼,你先别告诉她。我会好好告诉她的」

「谁会做那种会被你记恨一辈子的事情啊。又不是开玩笑的,我可不会做别人叫我不要做的事,我才没那么蠢」

「如果是开玩笑的你就会说出去吗……」

「那种情况本来就不是真心不让说嘛。啊,顺便问一下,你什么时候会叫我?」

「你不用来」

「好过分!」\\

周故意冷淡地回答,树则是笑得很开心。周也稍微缓和了冰冷的眼神,轻轻笑了起来。
