% 30 天使様と困惑

%  最悪だ、と隣に姿勢よく座る真昼を見ながら、周はため息をついた。\\


%  ベランダで遭遇するという大惨事を迎えた周は、仕方なく真昼を一旦家に招いた。


%  どうせ誤魔化そうとしても、この二人は確実に邪推をしてくる。ならば正直に言った方がまだ余計な憶測と勘違いは防げるだろう。\\


%  それに、しっかりと口止めをしておかなければあとが怖い。\\


% 「……あの、本当にすみません……」


% 「……お前が悪い訳じゃない……」\\


%  非常に申し訳なさそうなか細い声で謝られるものの、こればかりは真昼も悪くない。


%  ホワイトクリスマス、それも今年初雪だったので、ついベランダに見に行ってしまったのだろう。\\


%  周も窓を開ける音が聞こえていたなら恐らく止めただろうが、部屋に音楽をかけていたせいで聞こえなかったのだ。


%  それに、真昼もなるべく音をたてないようにしていたのだろう。全く気付かなかった。\\


%  反省しあう二人を見ながら、千歳が目を輝かせながらぐいっと顔を近付けてくる。\\


% 「で、周のお隣さんって天使様だったの!?」


% 「あの、天使というのはやめていただけると……」\\


%  流石に天使様と目の前で呼ばれるのは嫌だったらしく控えめに拒んでいるのだが、千歳はにこにこにしながら聞いているのかいないのか分からない。\\


%  樹はというと、頬をかきながら周と真昼の姿を交互に見ながら眉尻を下げている。\\


% 「えー。じゃあ……今までの流れでいくと、椎名さんは周の隣に住んでいて、よく周にご飯を作ってると。この見解はあってるか?」


% 「……おう」


% 「ま、まあ……その、恩がありましたし、藤宮さんが見るからに不健康な食生活を送っていたのが気になったので……」\\


%  関わるきっかけもさらっと話してどうして交流が続いているのかも説明すると、樹は「なるほど」と言いつつも微妙に納得していなさそうな表情でもある。\\


%  周が樹の立場でも、納得出来ないだろう。


%  周のような一般男子が、真昼のような優秀な女性に世話を焼かれているなんて。\\


% 「うーん、事情は把握したんだけどさあ。椎名さんはともかく周に他意がない方が不思議なんだよなあこの状況。ほぼ通い妻じゃん」


% 「ぶっ」\\


%  普段全く聞きなれない単語を使われて、思わず吹き出した。\\


%  通い妻。


%  言われてみれば、状況だけならそう見えるのかもしれない。夕食を毎日作ってくれて、たまに休日の昼ご飯もご馳走になっている。その上たまにお掃除も手伝ってくれているのだ。聞いた感じそう聞こえてしまうのかもしれない。


%  違うのは、互いに愛情というものは持ち合わせていないというところだろう。\\


%  真昼も樹の言葉に僅かに目を丸くしたものの、すぐに外行き用の笑顔で「そんなつもりはありませんし、あり得ません」ときっぱり否定をしていた。


%  樹や千歳には普段の学校と同じように接するんだな、と思うと、何だかややくすぐったいものを感じる。\\


% 「こっちにやましい思いは一切ないし、だからこそ椎名も俺を手助けしてくれるんだろう」


% 「周がそういうなら良いんだけどさ。ほんと、謎の組み合わせだわ……あの才女が周に料理なあ。……ぬいぐるみの贈り先も椎名さん?」


% 「……まあ」


% 「へえー」


% 「うるさい」


% 「まだ何も言ってないよ?」


% 「顔がうるさかった」


% 「ひどい!」\\


%  千歳のにこにこ……というかにやにや笑いは、非常にささくれだっている心に悪い。\\


%  今のところ事実確認だけでさほどからかわれてはいないが、からかわれるのは御免なのだ。真昼にも影響が出るので、出来れば千歳は無視したいところである。\\


% 「まあまあ落ち着け二人とも」\\


%  最初から周の様子が変わった事を気付いていた樹は、千歳のようにからかう様子ではない。


%  本気でいやがる前にやめてくれるので、何だかんだ空気が読めて気遣いも出来る男なのだ。出来れば詮索する前に止めてほしくはあったが、仕方ないだろう。\\


%  微妙に睨む周と謎がとけてご満悦の千歳を諌めた樹は、何故か姿勢を正して真昼に体ごと向けて頭を下げている。\\


% 「……えーと椎名さん、うちの周がお世話になってます」


% 「いつからお前んちの子になった」


% 「こちらこそ藤宮さんと仲良くしていただいてありがとうございます」


% 「そこ便乗すな。俺が駄目なやつみたいだろうが」


% 「実際駄目なやつだし」


% 「このやろ」\\


%  確かに、散々樹にも言われていたし、自覚もしていたが……指摘されるというのは、ものすごく複雑なものである。\\


%  こういう冗談には乗れるらしくちゃっかりボケた真昼は、周と樹のやり取りを見てくすりと微笑んでいた。\\


%  周だけに見せる素、というほどではないが、少しだけ被った猫がお出かけした笑顔であり、樹もどこか呆けたような表情である。


%  彼女持ちがみとれてんな、と樹を小突いたら、不機嫌な千歳が同じように……いや、ちょっと強めに小突いたのが何だか面白かった。\\


%  ただ、真昼が不思議そうに小首をかしげたので、周は何でもないように元の体勢に戻る。\\


% 「……で、だ。別に、俺らはお前らみたいな甘ったるい関係ではないが、他の連中に漏れると確実に面倒くさい事になるのは分かるな」


% 「分かってるよ、人に言ったりはしない」\\


%  暗にしゃべったらどうなるか分かるよな、と脅しをかけるのだが、樹があっさりと頷いたのが意外だった。\\


% 「千歳もだぞ」


% 「私もそこまでおしゃべりじゃないよー。それに、こんな可愛い子が周にご飯作ってるとかって信じてもらえなさそうだし」


% 「不相応ですまんな」


% 「そこまで言ってないのにー」\\


%  千歳の言う通りだしそれは自覚している。


%  普通の男子生徒が、学園のアイドルと言ってもいい天使様にお世話をしてもらっているなんて、誰も信じないだろう。


%  仮に信じられたとしても相応しくないと罵られるに違いない。\\


%  別にそれは予想出来るし、だからこそ外にはこの事実を漏らしたくない。面倒事はごめんだった。\\


%  卑屈だねぇ、と笑った千歳は周を見ていたものの、途中から吸い寄せられるように真昼に視線が移っている。\\


%  じい、と熱心に見つめたかと思えば、ほう、とため息をこぼしてはまた見つめたり。


%  真昼も居心地が悪そうにしていて、どうしていいのか分からないようだった。\\


% 「あの、なんですか?」


% 「……改めて思ったんだけどさ。椎名さんってめちゃくちゃ可愛いよね」


% 「え? ありがとうございます……?」\\


%  正面から褒めた千歳は、そのまま真昼の容姿をじろじろと眺めている。\\


% 「こんな近くで見たの初めてなんだけど、やっぱ天使様って言われるくらいに美人なんだよね。顔立ち整ってるしすごく肌白くて綺麗だし睫毛長いし髪さらっさらだし華奢なのに凹凸あるし」


% 「あ、あの……?」\\


%  千歳の悪い癖が出そうで、周は大きくため息をついた。\\


%  周は、千歳が苦手だ。


%  嫌いではない、人柄は割と好ましくはあるのだが……どうしても、苦手なところはある。ハイテンションだとかたまに無駄に踏み込んでくる所とか、そういった所が苦手だ。身内に似たような人間がいるから余計に苦手意識がわくのだろう。\\


%  つまるところ、その母親にどこか繋がっているところが苦手なのだ。\\


%  千歳は周の母親に性格もそうだが嗜好も似ていて……綺麗なものや可愛いものが、大好きだった。\\


%  流石に止めないと真昼が可哀想な気がしたので、ったくと悪態をつきながら手を伸ばしかけた千歳の頭をぺしりと軽くはたいておく。\\


%  制止と突っ込み目的なので本当に軽くではあったものの、衝撃を受けた千歳は「あいたっ」と小さく声を上げて真昼へ伸びた手を引っ込めた。\\


% 「何もはたく事ないと思うんだけど」


% 「こいつは人見知りなんだから慣れない内からスキンシップはやめろ」


% 「慣れたらいいの?」


% 「それは椎名に聞いてくれ。まずは段階踏め段階」\\


%  明らかに真昼は逃げの体勢に入っていたので、止めて正解だっただろう。\\


%  ほんのり、というかかなり困っている真昼を見て千歳も止めた理由は納得したらしい。\\


% 「ごめんなさい、興奮のあまり触りそうになりました」


% 「は、はあ……」\\


%  いきなり触りそうになったと暴露されても真昼は困っているようで、どうしていいのか分からなさそうにこちらに助けてと視線で求めていた。\\


% 「あー。椎名、千歳は勢いのある変人だが悪いやつじゃない……と思う」


% 「ねえそれ庇ってるの? 庇ってないよね貶してるよね?」


% 「今の言動見て否定できるか?」


% 「できません!」\\


%  堂々と自ら否定した千歳はじーっと真昼を見たのち、とても大真面目な表情で真昼に再度手を伸ばしていた。


%  今度は、真昼に掌を差し出す形で。\\


% 「ならばお友達からよろしくおねがいします」


% 「え? は、はい、よろしくお願いします……?」\\


%  握手を求められて、真昼はおろおろとしつつも差し出された掌を握った。\\


%  おそらく一度気に入ると仲良くなろうとする千歳の性質上真昼が振り回されそうな気がしなくもなかったが、流石に普通の友達付き合いならこちらが口出しする事でもないだろう。


%  節度を持った付き合い方をしてほしいものである。\\


% 「うんうん、新たな友情が育まれたなあ」


% 「お前は彼女の手綱をちゃんと握っとけ」


% 「頑張る」\\


%  毎回暴走させかけている樹に鋭く突っ込んで、真昼の手を握ってにこにこしている千歳を見てまたため息をついた。

