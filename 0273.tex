\subsection{毕业后的一个约定}

真昼的三方面谈的时间安排得比较晚,于是周在图书室自习打发着时间,在收到真昼发来的面谈结束的讯息后,他便朝约好碰面的学校鞋柜走去。

由于不想让真昼一个人回家,周今天也请了兼职的假。\\

透过窗户可以看到太阳已经西沉,周走在变得相当安静的校园里。当他到达楼门口时,真昼已经先到了,她换完了平底便鞋,手上拿着手机。

朱红色的晚霞从敞开的门里照射进来,将真昼亚麻色的头发染上鲜艳的色彩。

周围并没有其他学生,站在那里的真昼显得有些落寞。\\

「辛苦了」\\

周忍不住出声后,一直低头看着手机的真昼抬起头,露出了温柔的微笑。\\

「让你久等了,谢谢你等到现在」\\

啪嗒啪嗒,真昼小跑着到了穿鞋能进入的边界,周仿佛能看到她晃动着的小尾巴,差点就要发出让人觉得可疑的声音了,他只好咳了几声来掩饰,轻轻地抚摸真昼柔软的头发。\\

「好啦好啦,明明是我自作主张在等你。抱歉还让你在这等这么久,这里很冷的吧」

「周君就是这样,总把责任揽到自己身上呢,而且还不想让我有负担」

「这种事就不要拆穿了吧」

「呵呵,还是被我看穿了,总之谢谢你啦」

「好嘞」\\

周很开心真昼能理解自己的思维方式,但是如果被她看穿太多周也会感到困扰,这次是被看穿的羞涩占据了上风。\\

但似乎连这一点都看穿了的真昼发出了含蓄却又愉快的笑声,周觉得有些尴尬,扭过头去打开了自己的鞋柜门。\\

\vspace{2\baselineskip}

「……怎么样?」\\

两人一起慢悠悠地回到家后,周犹豫着向真昼问道。\\

真昼马上就明白了「怎么样」是在问什么,伤脑筋地发出了「唔~」的声音,但那声音中并感受不到有任何苦恼,毕竟她已经心中有了分别,因而身周的氛围都很轻松。\\

「怎么说呢,这个问题好难回答啊。关于父母不会来这件事,老师在这一年半的时间里怎么也都清楚了,讲一声也就过去了,顶多就是老师脸色有点不好看」

「那是当然的吧」

「我也一点办法也没有,要我说的话」\\

「毕竟怎么都不会来的嘛」轻描淡写地说完,真昼便垂下眼帘,疲惫地叹了口气,似是在表达她的无奈。\\

「坦白说,搞得那么在意反而让我头疼,这也不是一天两天的事了。明明事先和老师说过,实际面谈时却还是散发出一股阴郁的气场……反而弄得我心里很在意」

「我觉得老师只是认为这件事情很敏感,不知道该怎么处理吧」

「这一点我理解,但是被这么避而远之地对待,心情终究是好不了的。特别是我自己对事情本身还不怎么介意」

「话是这么说,但老师可能还是有点介意的吧。面谈本身没问题吗?」

「毕竟我学习得很努力,学业方面完全没让老师操心。老师说我成绩和品德都没有什么问题,就算把目标大学的要求考虑在内,也是相当有把握的。我希望尽量走学校直推的渠道早日录取,失败了的话就正常参加考试吧」\\

真昼要是有问题,那就没多少学生没问题了,老师的评价颇为妥当。硬要说也就是没参与社团了(周也被老师指出了这一点),而真昼还考取了一些证书,并且积极参加模拟考,社团活动也不会成为她的软肋。\\

令周在意的,是迄今为止彼此刻意没有多问的,真昼的未来规划。\\

「真昼未来的打算是?」

「可以的话,最好是周君现在的目标大学,不过院系会不太一样」\\

这么干脆的回答,反而让周不知所措了。真昼又浅笑着道:\\

「啊,不是因为想和周君在一起才这么打算的哦。我自己就是这么考虑的,再怎么也不至于为了爱情来决定去向」

「嗯,我知道真昼不是那种会把自己的去向交给别人的类型」

「嘻嘻,黏在一起到这个地步肯定是不行的……不过我也有点迷茫」

「迷茫?」

「那个,假如我们,都去了志愿的大学……这个公寓会有点不太方便吧。那个,离大学的位置有点远了。虽然我很喜欢这个公寓的地段」

「唔,单程随随便便就是一个多小时啊。说长也不长,但只要能缩短上学的时间,就能享受到很不同的乐趣了」\\

周的目标大学虽然位于市区,但是位于东京23区中,而周不仅住在23区外,还必须老老实实走到车站,从这里去上学是需要花一些时间的。

虽说比起从其他县去上学的情况在时间上宽裕得多,但说实在的,周还是希望上学时间能尽量缩短。这关系到上大学的精力问题,如果能住在离学校近的地方的话,心态也会比较从容吧。\\

「话又说回来,如果住宿舍的话,就不能随便和真昼见面了,我不太习惯集体生活,不想和别人共用厕所和浴室,也不喜欢太吵闹的环境,是真的不想住宿舍啊」

「我也一样,不能和周君碰面的话,会觉得很寂寞」

「那就是一起搬去另一个公寓了……不想和真昼分开,这会不会太任性了呢」\\

两个人好不容易交往,还住在隔壁,却由于升学要分开,周是打死也不愿意的。真昼似乎也有着相同的想法,轻轻摇了摇头,亚麻色的头发随之产生了几道波纹。然后她腼腆地说道:\\

「那、那个,不如说是我该这么想才对……可以的话,我也想待在你身边」

「嗯,太好了」\\

周一边为真昼不愿与自己分开而感到幸福,一边开始做现实的考虑:如果要搬家,就要先去跟父母商量了。\\

在这里继续读大学的事已经和父母说过了,他们也同意周继续一个人生活,所以如果房租不变的话,搬家的事也应该会比较容易得到批准吧。\\

话虽如此,在相同租金范围,安全也有保障的范围内,即便是在房间大小上做出妥协,也不太能保证可以控制在预算之内。即使是选择稍微远离大学的低价地段,在23区内外也会有很大的差价吧。\\

这么一想,周觉得就不能轻易提搬家的事了,正想着该怎么办,这时真昼抬头担心地看向手捂嘴边、沉吟着的周。\\

看到她的样子,周突然想到一个主意。\\

「干脆我们住在一起应该会更方便吧」

「哎?!」\\

听到真昼发出的惊愕声,周依然继续说了下去。\\

「我觉得两个人一起住的话,房租就能相应地节省一些。而且这样也能尽可能地方便接送」\\

相比起分开租两个房间,两人一起合租一个大一些的房间,包括水电费在内的各种费用都会更便宜。周觉得虽然只是个很简单的想法,不过也不是什么馊主意。

如果是和真昼合租的话,周觉得自己的父母说不定——不如说是会举双手赞成。\\

周边用手机简单调查着目标大学附近的房屋租金,边做着计算,真昼则是含糊其辞地说「……是,是的呢」,像是肯定,又像是没在肯定。\\

「真昼?」\\

周觉得这个主意很好,但真昼的表情却显得有些僵硬,有些发愁的样子……不如说,充满了困惑和羞涩。\\

「周君,是觉得跟我住在一起也没问题呢」\\

听完真昼细声的喃喃,周一个没拿稳,手机落到了大腿上。\\

(……那个,我,刚刚是不是在邀请同居?)\\

由于确实是无意间说的话,周完全没往那方面去想,但事实上就是那么回事,真昼也是那样去理解的。

等反应过来,思绪便一下子如沸腾般地搅动、杂糅在一起,满溢的羞涩、对自己迟钝的傻眼,以及让真昼乱了阵脚的愧疚使周连忙使劲摆手。\\

「对、对不起,刚才说了很胡来的事!真昼也需要私人空间,不是说我一个人替你决定什么的!?我是考虑到未来的事,不,那个,唔,两个人一起的话,会更幸福吧,大学生活也会更加努力,这些也仅仅只是我个人的想法而已……那个,对、对不起」\\

周认为自己必须好好反省刚刚只顾着自说自话,而没有确认过真昼的想法,所以他慌忙连说带比划地向真昼表达歉意,真昼见此态度则是微微眯起了眼睛。\\

那副模样与其说是生气,反而更像是无语。\\

「这个时候道歉,听起来倒像是我在指责你了」

「我、我不是那个意思。那个,毕竟我确实说了一些十分自私的话」

「周君所谓的自私,是指无视我的想法,自己擅自安排事情的意思吗」

「是的」

「……那么,这就不是自私」\\

听到那实在太过符合自己心意的话,周差点怀疑自己是不是出现了幻听;再迅速往真昼那边看过去,发现她红透了的脸蛋上,一双湿润的眸子正抬起来看着周,像是在观望,又像是在期待。\\

周并没有迟钝到认为真昼这样是自己惹她讨厌了,一想到真昼是希望能和他在同一屋檐下一起生活,周内心里火焰般的情感迸发而出,他的眼眶也逐渐发烫。\\

「我接受这份邀请,也没有问题吧」

「……嗯」\\

真昼强忍着羞意,含蓄地回应道。周感觉心脏跳动的声音俨然是在催促着自己的身体,静静地以点头回应。\\

「好开心」

「我也是」\\

无论周和真昼一起相处,接触了多长的时间,都无法缓解此时两人间的窘态。毕竟是互相确认到对方都有打算住到一起去的想法,两人会觉得尴尬也是理所当然的。\\

即使周现在大部分时间都会以真昼来他家里的形式一起度过,但这和同居又完全是两回事。\\

都否认了那么多次树对他是不是在同居的调侃,周却发现自己无意识间也有着这个愿望。他感到十分羞耻的同时,一想到真昼接纳了他,超过那羞耻的欢喜又将一切都冲走了。\\

真昼被周注视着也不太好意思,露出了略带羞涩的纯真笑容。\\

「现在就已经够幸福了,之后更是可以每天迎接周君回家,或者到家被周君迎接,睡前道个晚安,临行前还能跟留在屋里的人打一声招呼……想想都觉得,特别棒,很幸福」\\

真昼嘿嘿地傻笑着,那副心满意足的模样俨然是把她刚才的话又原样说了一遍。在周看得入迷的时候,真昼好像突然注意到了什么似的,朝他露出了一丝不安的神情。\\

「啊,是不是应该最好先跟志保子阿姨他们打个招呼呢?私自做决定可不行啊,你毕竟是他们的宝贝儿子……」

「唔,你说的也有道理,不过我觉得妈妈他们应该会很高兴的。这么一来,我是不是最好也跟小雪阿姨打声招呼……?」\\

关于真昼的亲生父母,虽然父亲那边不好说,但母亲对她漠不关心。周觉得没必要让真昼感到沮丧难过,所以故意没提她的父母,幸好真昼并没有注意这点。

真有个万一的话,周是打算去跟能联系上的朝阳放话说自己要拐跑真昼的,他希望真昼只用考虑幸福的事就好。\\

「小雪阿姨肯定还是很关心你的,听说你和一个来路不明的男孩子一起生活的话,她会感到不安的吧。不如说最好是现在就过去打声招呼」

「那个,我也想去和她见一面,想介绍周君给她认识,听她聊很多很多事情……一定得找个机会去一趟」

「是、是啊,那就这么办吧」\\

虽然周和真昼忙里忙慌地讨论了半天,但稍微动动脑子就会发现,都还没被录取就考虑这些未免也太过心急了。当两人意识到这些畅想都过于遥远后,都忍不住笑了出来。

然而,两人之间就此确立下未来的约定,也足以给彼此的心中带来巨大的希望和无比的幸福。\\

「彼此都一定要努力准备考试了啊」

「嗯,我一定会努力考上的。要做的事可真多」

「不过周君那都是给自己找的事情」

「是啊。这毕竟是把应试考虑在内,自己做出的选择,所以我会负责任地努力工作实现目标,学习也不会懈怠的」\\

关于打工这件事,周认为这是他应该做的事情,即便知道可能会影响升学,他依旧坚持要去做。因此,他不打算以此作为理由而懈怠。周相信自己可以做到,所以才选择了这条路。\\

「周君作出的决定,我也不会说三道四,我能做的也就是支持你和日常提供些帮助罢了」

「别,不用为我那么上心的。我这都是一些个人的缘由」

「我也是在力所能及的范围内自说自话去做而已啦」

「……这点你倒是不肯让步」

「嘻嘻,我就是这样的人嘛」

「知道的」\\

在这一年里,真昼和周都慢慢地了解到彼此是怎样的人,并且互相理解了对方的想法,也都明白一旦对方决定好了的事,就不会作出让步。

因此,尊重对方的选择并彼此珍视是至关重要的,这也是在共同生活中愉快相处的秘诀。周重新真切地认识到这点,握住了像撒娇一样靠过来的真昼的手。\\

(……交谈的机会么)\\

周想起刚才的对话,在心中默念道。\\

等真昼回去后,就去续写那封草稿箱里写到一半的邮件——他在心中做下了决定。
