\subsection{卡拉OK中的谈话}

或许是庆功使所有人都很high吧,周围人催着周唱各种各样的歌曲,等到唱完,周已经累得不行了。

一起唱的门胁依旧平静,大概是基础体力的差距造成的。\\

「辛苦了,唱得不错」\\

真昼平静地微笑着迎接周回来。她同样比平时更加神采奕奕,大概也是在high着。\\

「……真昼也挺来劲的嘛」

「毕、毕竟……周君唱歌的样子帅啦」

「谢谢夸奖,那接下来轮到你了」

「嗯?」

「千岁,真昼借给你,接下来你和她一起唱吧」\\

周向千岁喊着,把心情大悦的女友作为祭品献了出去。

千岁听了,露出怀疑的眼神,但还是满意地笑了笑,开心地回答说「交给我啦~」\\

「不是,等等」

「真昼开心了,我也想听真昼唱歌乐一乐——」

「这、这个……」

「千岁选歌的话,估计是真昼会的,没问题没问题」

「有、有问题吧……千、千岁——」

「来来,昼儿你就上吧,反正大家唱着唱着就热闹起来了」\\

周挥手目送起了兴致的千岁牵起真昼的手。尽管真昼送来怨念的眼神,但这条路是周也走过的,希望她能放弃。\\

「这事也该经历一下」周感慨地点头,望着话筒在手、不知所措的真昼,满足地眯起眼睛。这时门胁在旁苦笑着吃起薯条。\\

「过后椎名不会报复?」

「最多就是被捶几下」\\

要报复也是可爱的报复,可爱得周都想主动承受,看看她的反应了。

看到周无所谓的态度,门胁耸耸肩,然后望向慌慌张张开始唱歌的真昼,仿佛很耀眼似的。\\

除了游泳,真昼几乎什么都会,唱歌也不例外。好在选曲选的是婉约的日本音乐,清亮的声音唱出优美的歌曲,大家纷纷停止闲聊,沉醉于其中。若是叫她在晚上唱首摇篮曲,很快就能让人进入梦乡吧。周的表情也变得舒缓下来。\\

千岁也和着真昼柔声唱着,同样唱得不错。千岁熟门熟路,比真昼更有顺应歌词和音乐的抑扬,要说技艺,还是千岁更高。

她满面欢喜,恐怕这首结束后也不会放走真昼。\\

(算了,反正看真昼也挺开心的)\\

原本那遭受抛弃后不满的表情尽管还含着羞,却也柔和了下来,看上去很愉快。真昼没有这么多人唱卡拉OK的经历,如今乐在其中,周也为之感到满足。\\

「……说起来,你们两个之后就回去是吧?」\\

正当周平静地望着握着麦克风的真昼时,门胁靠过来,以只有周听得见的声音轻声询问。\\

「嗯。我爸妈过来了。真昼也是大致准备好晚饭才来上学的」

「哎,怎么说呢,简直像是同居了吧,讲真的」

「要你管」\\

真昼只有睡觉、打扮和洗澡时会回自己家,大部分时间都在周的家里。这一切已经变得理所当然,没有一点别扭之处,可见真昼已经方方面面融入了周的生活。\\

「意思是唱完卡拉OK你们两个就先走对吧,好的。其他人可能会觉得遗憾,也没办法了」

「真昼不在,是会有人遗憾的吧」

「啊哈哈,你真就不考虑自己啊」\\

门胁苦笑着戳起周的肩膀,周则往门胁的肚子边上戳了回去,表示自己不是真昼、门胁那样的人。

尽管最近渐渐融入了班级里,但周并没有他们两个的人气,就算有人感到遗憾,那也是因为把他和真昼放在一起了。不知为何,同学们会温暖地守望他,周感觉那就是原因。\\

「你家人来了,也没办法。他们很顾家吧,大老远跑过来」

「……嗯,他们的确是关心着我」

「有这样的爸妈真不错。和椎名也处得融洽,多好」

「是啊,重视她都超过我这个儿子了」

「啊哈哈,不过我觉得,这到底是因为他们重视你啊」\\

门胁笑着说完后,周稍稍睁大了眼,然后带着点难为情的感觉,小声嘟哝「我知道的」。

看到周害羞起来,门胁再次笑道「多好的父母」又轻轻戳起了周。
