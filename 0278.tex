\subsection{曾经的幸福与现在的幸福}

真昼沉浸在周为她庆生的余韵中,静静地闭着眼睛。周看着她这副模样,对目前为止的成功感到自豪,但同时他也没有把握:不知道真昼会对接下来的最后一项大礼产生什么想法?现在还没到松懈的时候,周抿紧了嘴唇。\\

朝墙壁瞥一眼,就能看到在缤纷的装饰中变得很不起眼的挂钟,正一分一秒地朝约定的时间走动。\\

(……差不多了吧?)\\

在真昼生日的最后,恐怕是最大惊喜的时刻就要到来了。

真昼深信今天的庆祝活动已经结束,以为接下来就只剩下慢悠悠地品味幸福了。她坐在沙发上戳着装在猫咪玩偶上的肉球,看起来实在是非常放松。\\

尽管周对于打扰她放松的心情很过意不去,但毕竟现在已经不能回头了,所以他也只能做好心理准备。\\

「……呃,真昼,那个啊……」

「嗯」\\

真昼丝毫没有起疑地抬头仰望周,她的脸上满是染上幸福色彩的神情,程度比平常多出五成,看起来既天真又无邪。她向周投以满怀爱情的眼神,这副模样可爱到周根本按捺不住,他感觉自己的脸颊内侧就要被咬得发炎了。\\

他一边用咬住脸颊的疼痛抑制住想要马上上前疼爱真昼的念头,一边佯装平静地继续说道:\\

「我还有一个惊喜要送给你」

「这样已经很够了吧?惊喜会不会太多了?」

「如果要连同这一路以来的份一起庆祝,现在这些还远远不够呢,我觉得你可以再贪心一点没关系啊?」

「这、这怎么可以……那个,这样不会撑破吗……应该说,不会过载吗?」

「那么到时候就请你原谅我啰」

「咦?」\\

不如说,周是已经设想到可能会发生的状况,准备出这份最后惊喜的。\\

真昼完全没有进入状况,满脸讶异地凝视着周,因此周也故意不作解释,他只是静静起身,去完成那份最重要的礼物兼惊喜。\\

「虽然我之前姑且准备了一下,但现在还差最后的步骤没完成,所以得麻烦你在那边坐好」\\

周不管面带困惑的真昼,兀自回到自己房间,把处于睡眠状态的笔电拿回客厅。\\

看似完全没有关联的笔电突然出现,让真昼疑惑的视线变得更强烈了,但周还是没有给真昼答案,只是始终维持着敬请期待的态度,把笔电放到矮桌上。

屏幕中是一个可以视频通话的会议应用,目前正处于待接通的状态。\\

「我想,你可能会觉得我非常多管闲事」\\

周凭着自作主张和自以为是的想法,决定送出这份今天最重大的、只凭他一个人无法实现的惊喜。\\

不知道真昼会不会高兴?期待占据了他的胸膛,但另一方面,他也担心自作主张会不会惹她反感,两种情绪纠缠在一起。对他来说,这就像一场豪赌。

当然,就算没有这份惊喜,真昼也已经非常开心,而实际上她也确实高兴到哭了。所以,或许从一开始就不应该做这件事。\\

「你可能会觉得我多此一举,但想为你庆祝的人不是只有我而已,这边就有一位很想念你的人」\\

今天已经有很多人为了真昼的诞生而欢喜,祝福她生日快乐。与她来往密切的那些重要的朋友,还有她当成亲生父母来敬仰的志保子与修斗,也都纷纷献上了祝福。\\

她一定开心到没有任何怨言才是。\\

然而,周依旧觉得美中不足。

不是只有现在的朋友们和宛如双亲的大人祝福她就好。那位在她小时候无比仰慕的重要之人,是不是被完全遗漏了?\\

应该还有一个人会由衷地祝福真昼生日快乐,不是吗?\\

周打字告诉对方准备就绪,他缓缓深呼吸,让越来越快的心跳缓和下来,接着用鼠标点击麦克风图案的按钮。\\

随后,画面就从空无一物的漆黑,转为一片鲜丽的色彩。\\

『大小姐』\\

一道声音从电脑的扬声器中缓缓流出。\\

虽然周没有听过这道不高也不低、既柔和又平静的温和嗓音——但对于真昼而言,想必不是如此。

「咦?」真昼的口中冒出尖细的惊呼声。\\

接着,她就像被弹起来似地从沙发上跳了下来,以这股劲头站到地板上,将脸凑向矮桌上的笔电,目不转睛地凝视着画面,这时的她脸上满是惊愕,与平常的沉着冷静相去甚远。\\

她的眼眸睁得不能再大,像是在表达无法置信一般;嘴唇微张,看起来有点呆呆的。她这副表情就是如此地不平静,用困惑来形容也并无不可。\\

那边多半也察觉到真昼与平时大相径庭的反应了。

只见画面中那位年龄比周的父母大上十来岁的女性,露出了在贤淑笑容中加入些许惊讶与愉悦的表情,看向真昼。\\

『我已经卸下了这份职责,这样的称呼是不是不太对呢……』\\

面对仍然浑身僵硬盯着屏幕的真昼,那名女性——小雪隔着画面和她四目相会,露出微笑。\\

『……真昼,好久不见』\\

小雪把手放在胸前,带着高雅的微笑呼唤真昼的名字。见真昼依旧思绪停顿,她还是继续说了下去:\\

『忽然叫你的名字,真是抱歉,不过我已经不是你家的管家了,你可以允许我叫你的名字吗?』

「为什么,咦,骗人的吧……」

『看来这个惊喜很成功呢,嘻嘻』\\

用很成功来形容也丝毫不夸张,不只如此,应该可以说是成功过了头,周都开始担心真昼会不会因此心脏骤停了。这份大礼想必就是这么超乎她的预料。

小雪带有一些捉弄意味的声音听起来并不低俗,而是流露出一种成熟的高雅气息。离真昼有些距离的周也看得出来,小雪这样的态度反倒让真昼更加混乱了。\\

「咦、咦?你、你为什么会……」

『你问为什么,是指我为什么会像这样和你视频的意思吧?我认为问一下你身旁这位会比较快』\\

居然现在把话题丢给我吗!周这么想着,却也不得不向猛地转头看他的真昼解释清楚,他只好轻声笑着,请真昼在沙发上重新坐好,好让她暂且冷静下来。片刻之后,他抬头仰望做好了心理准备的真昼。\\

「啊……我先跟你道个歉。对不起」

「咦?」

「我做了一件坏事」

「坏、坏事……?」

「你都不好奇为什么我可以和小雪阿姨取得联系吗?」\\

在真昼和周邂逅之前,小雪是负责照顾真昼的管家兼家庭教师,她和周没有直接的关系,周甚至没有和她说过话,也没有听过她的声音。

这样的话,周当然无从知晓小雪的联系方式。这一点真昼很快就能够想明白。\\

瞬间明白状况的真昼「啊」了一声,对此周只能怀着非常过意不去的心情,在脑中整理事先构思好的解释,同时缓缓开口向真昼说明事情的经过。\\

周想要让真昼在她的生日当天尝到无上的喜悦,为此周思考了各种方法,希望能使真昼感到幸福。那会儿恰好是三方面谈,于是周就想起了算得上是真昼家长的小雪。\\

「之前啊,那个,我们不是说要跟小——久慈川阿姨打声招呼吗?」\\

在文化节那阵子,真昼和周提到了小雪的事情,周因此知道小雪是个多么善良的人,真昼后来还带着掺杂怀念、喜爱与苦涩的微笑告诉周说,小雪在她们分开后寄了那么多信给她。\\

「后来你给我看了一下久慈川阿姨寄来的信,我就是在那时看到了联系地址。再后来,我们在讨论一起写信过去和她报告的时候,我把那张纸条上的电子邮件和电话号码给记住了」\\

就是在那时,周才有机会接触到小雪的个人信息。

虽然她们平常都是写信联系,但小雪还留下了许多其他能联系到她的方式。在真昼爽快地把信拿给周看的时候,放在一起的还有张写着电子邮件、住址和电话号码的纸条。\\

因为那串邮箱名字很有特征,又很好懂,周不用特别盯着它看,只是瞥上一眼,就立刻把它记下来了。\\

说实在的,周在发出邮件的时候,也反复自问自答和自责了好几天:这样做真的好吗?别说违反礼仪,这根本就是违反道德了吧?这种想法至今仍未改变,发出邮件的胃痛依旧记忆犹新。\\

他自己也想要尽早和小雪打个招呼,但他明白不应该是靠这种方式擅作主张。

他非常清楚,将小雪牵扯进来,也全都是他的自作主张。\\

即使如此——周无论如何都要借用小雪的力量。\\

「真昼和久慈川阿姨,真的非常抱歉,我不应该擅自得知还利用个人信息的。请允许我表达诚挚的歉意」\\

周知道对面也看得到自己这边,于是他移动到镜头前方,深深地低下头,而小雪则是露出了苦笑,像是在说「真拿你没办法」。\\

其实从第一次取得联系开始,周就不断在向对方道歉,但就算这样,他还是觉得不够,所以现在他也是一直低着头。然后他就听到头顶前方传来声音『已经够了吧,麻烦你把头抬起来』。\\

『一开始我还以为是诈骗邮件,跑去找儿子女婿商量呢』

「真是非常抱歉」

『算了,看你这么拼命,我就原谅你吧。毕竟我也感受到,你可以为她豁出一切的气魄,还有你对这件事抱持的内疚了。下次别再这样做了啊』

「是的,我发誓绝不再犯」\\

周不打算再做出这种一次跳过好几个步骤、可谓失礼至极的行为。

就小雪来看,素未谋面的周想必和可疑人物毫无二致,但她不只听了周的请求,还答应给予协助,让周怎么感谢和道歉都不够。\\

『真昼,看在我的情份上,你要不要也原谅他呢?为了你,他拼命向我解释,问我能不能答应他无礼又冒失的请求,头都没有抬起来过呢』

「我、我怎么会生气呢!周总是为了我这么拼命,看到他这样做也是为了我,我真的觉得很不好意思,很高兴,也很感谢他!」

『他非常热情地告诉我,他想要让你度过人生中最开心的生日,为此我有着举足轻重的地位。既然他都那么拜托我了,我当然也想要帮助他了呀?而且,如果我可以成就你的一部分幸福,我也会觉得很光荣』\\

随着这番柔和的嗓音编织出的话语,泪滴又从真昼的眼眸中,沿着她一抹桃红的脸颊,哗啦哗啦地垂落。周自然而然地明白,那喷涌的感情凝聚而成的东西究竟是什么。\\

周自己也知道,真昼的泪腺已经因为今天一连串的庆祝活动而彻底松弛下来,于是他把放在旁边的手帕递给真昼。哭得有点花脸的真昼则是坦率地收下了。\\

『容我再说一次。好久不见了,真昼』\\

小雪看着真昼用手帕拭去她所溢出的情感后,便用平静的语气向她搭话。周心想他这外人还是别介入这久违的邂逅比较好,就在他正想稍微退开时,真昼的指尖停下了他的动作。

稍微捏住袖子这点小小的阻力,就自然而然地将周固定在原地。可就算真昼不介意,对小雪来说,自己在场也许会妨碍两人的重逢。周观察小雪的意思……她却微笑着用眼神指示周留下来。\\

真昼拍了拍沙发旁边的位置,催促他过来。

周看向电脑屏幕,画面之中的小雪依然是看不出内心想法、笑容满面的模样。尽管他迟疑了一下,但最终他还是相信小雪没有拒绝,于是他犹豫地在真昼旁边坐下。\\

『尽管我和真昼书信来往过好几次,但已经很久没有像这样见面了……不知道隔着屏幕算不算?至今我都没有机会看到你长大后的模样……能有这个机会,我真的很高兴』

「好、好久不见了,小雪阿姨……」

『哎呀,比起哭泣的表情,我更想看看你的笑容。难得可以看到你的脸呢』

「好的」\\

真昼用手帕接住从白皙的眼皮下一颗接一颗涌出的大颗珍珠,终于露出了开朗的笑容。小雪见状,也露出了安心的微笑。\\

『不过,能坦率地哭出来,也是一件值得高兴的事。这证明你已经坚强到可以在别人面前展现出自己软弱的一面』\\

真昼从小就不愿表露出软弱的一面,哪怕是在小雪面前,她也几乎不会露出哭泣的模样。这样的真昼想必很坚强,但同时也很脆弱。

无法依赖任何人,只能独自承受;无论多么难受,都没有办法向人撒娇。

现在的她多半抛弃了固执,变得更加柔弱,同时也更为坚强、有韧性了。\\

『你真的变成一个很优秀的人了,真昼』

「……谢谢小雪阿姨」

『现在的你,表情比最后一次见到你时开朗了许多,眼中的光芒也不一样了。看来你现在的环境确实不错,真是太好了』

「是的……」\\

小雪应该是为了真昼着想,才会用这种听起来像是彻底放心下来的声音吧。

她柔和的微笑中也带着担心与安心,看得出来当时的她的确相当担心真昼。

虽然真昼停止了哭泣,她却不知为何缩起身体,可同时又挺直背脊,摆出颇为端正的姿势。小雪见状便用手掩住嘴边,优雅地露出微笑。\\

『嘻嘻,我们已经不是雇佣关系了,你不用那么拘谨。我不过是个普通的阿姨罢了』

「这、这是因为,那个,很久没见到你了,我有点紧张」

『这样啊。嘻嘻,我也因为太高兴,一不小心就跟你太亲昵了,算是彼此彼此吧』\\

听到小雪这么说,真昼明显害羞了起来。对此,小雪则是露出比刚才气势更足的笑容。\\

『说点有的没的吧。你过得好吗?有时候你是会寄信过来没错,但我还是想直接听你亲口说』

「嗯,好着呢,真的特别好……」

『你真的很紧绷呢。别那么紧张,我不会生气,也不会跑掉的』

「好的」

『喏,赶快放轻松』

「呜呜……」

『这只能靠你自己去习惯了』\\

即使小雪再次点破,久违的邂逅似乎仍旧让真昼无比紧张,她现在还是把挺得比平时更用力。

不过就算是这样,她看着屏幕另一端的眼神中也不全是紧张,而是饱含信赖与亲昵,看来她早晚可以习惯的。\\

『看你这么有精神,实在是太好了。就算我不问,从你的表情也看得出来……你遇到了一个良人呢』

「是的」\\

猝不及防地听到「良人」这个字眼,周随即挺直腰杆,结果真昼立刻给予肯定,害他难为情得视线乱飘。\\

『既然真昼这么说,那就不需要我担心了。哎,毕竟都可以为女朋友四处奔走了,我本来就不觉得他是坏人』\\

「活的时间长了,自然就能分辨人好不好了」小雪优雅地发出乐呵呵的笑声。周感觉胃有些隐隐作痛,但毕竟是他擅自与小雪接触的,所以他面带微笑,没有表现出什么不满。

不知道小雪有没有察觉到周的复杂心情,她仍然维持着优雅的表情。\\

「……我真的对周君感激不尽,为了我准备那么多,还为我和小雪阿姨牵线」

「你不用在意,这些都是我的自作主张」\\

不如说该拼命低头道谢的是周这边才对,但真昼为此感到高兴的事实还是让他觉得很开心,也很自豪。

不过他用的方法大致上属于灰色地带,他也不能因此就沾沾自喜。\\

「难得碰到你出生的日子,还是让你的幸福多到双手抱不住比较好吧。虽说我自己也说不准能不能成功就是了」\\

是周自己承担了受到厌烦、引人发怒、遭到回避、惹人厌恶的风险,所幸小雪和真昼都干脆地原谅了他。\\

「如果只需要稍微跑几步就可以让你高兴,那我当然不会嫌麻烦……你为了我努力这么多,我也想要好好报答你。我希望你能笑得更开心点,虽然刚刚才把你弄哭就是了」\\

周每说一句话,真昼的眼角就会跟着流出漂亮的泪滴,于是周连忙拿起手帕,轻轻地把它们吸进布里。

真昼的眼眸中流露出了太多太多的情感,多到从以前的她根本无法想象,甚至令人怀疑这条手帕是不是也要跟着哭出来了。周感觉自己这次算是成功狠狠地触动了她的心弦。\\

「那个,这样有让你高兴吗?」

「嗯,当然」\\

真昼的脸上浮现出符合年纪的纯真笑容,雀跃得就像是在说不能只顾着哭似的。周和小雪见状,都纷纷放下了心来。\\

『关于你身边的这位,我想听你再介绍一次。你们正在交往对不对?』

「……是的。他是我第一个喜欢上的人,只要和他待在一起,我就能平静下来,胸口也会暖洋洋的。不管是戴上面具的我,还是真实的我,他全部都喜欢,在了解了我的情况之后,也还是很珍惜我,愿意和我一起展望未来、一起向前迈进」\\

周的正前方——准确说来是正前方的画面中——就是曾经待在真昼身旁,让真昼宛如母亲般仰慕的对象,而真昼就在这样的状态下,诚挚、喜悦、怜爱地称赞周这个人的人格,周对此难为情得不得了。\\

即使如此,真昼这么为他着想,还是让周感到非常高兴。这种想法格外强烈,使他的视线还在屏幕上,手指却自然而然地寻找起身旁的真昼,结果他就和对方同样在摸索的指尖相遇了。\\

两人从指尖再到手掌不约而同地贴合在一起,然后手指交缠,比平时温暖许多的体温,逐渐融入了周的手指之中。\\

『这样,我很高兴你可以和这样的人相遇。真的是太好了』\\

周从相系的手感受到了信赖与幸福,他的嘴角因而不禁上扬了起来。小雪注意到了周的表情变化,脸上浮现出比他们牵着的手掌还要温暖,而且非常慈祥的表情。身为被介绍的男朋友,那有点——不对,应该说是非常害羞的心情又回来了。\\

不幸中的大幸是,小雪似乎不打算像周的母亲那样捉弄他,她始终一脸温和,只是带着微笑旁观。\\

『真昼从小就比别人聪明许多,所以我一直都很在意,她是不是已经对别人感到失望。现在看来是我杞人忧天了』\\

这种担心,是自幼就关注着真昼才会产生的,周也很能理解,他甚至还冒出了一些让真昼听见会惹她生气的想法:真亏她会选择这样的自己呢。\\

『顺便问一下,你抓住他的胃了吗?』

「……抓住了吗?」

「已经牢牢抓住了」

『哎呀哎呀』\\

这个问题自然是想都不用想,周用力点了点头,然后便见小雪露出了不出所料的微笑。\\

周经常听说,真昼的厨艺来自小雪的教导,因此平时总是能享用到美味餐点的周,是不是该现场对形塑出如今真昼的小雪下跪磕头呢?\\

本打算先鞠一个比刚刚更深的躬,真昼却不知为何连忙摆起了没牵着的那只手。\\

「啊,不、不过周君不会把事情都抛给我。他也是会做饭的!我们会一起做饭,周还会靠自己做饭给我吃!我不知道这该说是轮班制还是交替制,总之我们是认真在一起生活的!」

『嗯、嗯。你不用那么慌张,我也很清楚。他就是你理想中的人吧?』

「是的!」\\

尽管真昼眼神闪烁,但她坚决地点了点头,毫不犹豫地继续说了下去。\\

「小雪阿姨,你以前不是说过,要我选择愿意注视着我的人吗?」

『嗯,我的确说过』

「我觉得说的果真没错。虽然你还在的时候就是这样了,不过自从你不在之后,我接触了各种不同的人,我因此重新认识到……让我幸福的人,不会强行改变我,不会只用表面来判断,也不会轻视我的想法」\\

正因为真昼至今为止一直被许多人围绕着,所以她与人相处时最核心的基准才会是能否尊重他人吧。

人之为人,似乎理当如此,却又不容易做到。\\

「周君会以我的想法为优先,尝试去理解我在想什么;知道我的内在后也依然喜欢我,还愿意去了解、接受我的出身背景。可以得到周君的尊重,我觉得非常、非常幸福……虽然他有时候会因为尊重过头而畏首畏尾的就是了」

「都是为了你啦!」

「我、我知道的!……我十分清楚你有多么珍惜我。而且这也正是尊重我的体现」\\

周感觉真昼无意间说了他胆小,但他的胆小应该是建立在双方共识之上的才对。她是有哪里觉得不满吗?\\

周凝视着真昼,随后她便害羞地补充道「我、我没有不满哦!我只是觉得你可以不用那么顾虑我,应该以你自己的想法为优先」。可惜她好像不明白自己在说什么,令周身为男朋友感到头疼的事情又多了一样。\\

(要是真以我的想法为优先,真昼一定会烧坏的)\\

尽管周完全没有打算打破当时的誓言,但真昼这个说法,岂不就是让人觉得,只要不违背誓言,做什么都可以吗?

周满怀顾虑地看着真昼,心想:对没什么抗性的真昼这样触摸、撒娇好像不太好。至于真昼本人,她只是因为说出劲爆发言而满脸通红,似乎没有注意到周在想什么。\\

『看你们感情很好,我放心了。不过一码归一码,一起生活这句话让我蛮在意的』\\

小雪的语气听起来不像是在责备,却带有一丝傻眼的感觉。听到这副语气,周才发现他们刚才在真昼像母亲一样仰慕的女性面前说了不该说的话,不禁绷紧脸颊。\\

「咦?啊,不、不是的!周君是我的邻居,应该说是我们刚好住同一栋公寓!绝对没有发生什么小雪阿姨担心的状况!」

「我发誓,我没有做出什么伤害真昼的事情」\\

站在小雪的角度来看,听上去不免有种可爱的女儿被素昧平生的男人玩弄的感觉,所以才会这么担心。

要是不知情,她会有「这个年龄就在同居?」的怀疑也是很正常的。\\

周深深反省着自己的粗心大意,对此小雪则是散发出比起刚才更困扰、更傻眼了一些的氛围,无奈地叹了口气,接着用柔和的视线望向真昼。\\

『这部分我没资格说三道四,但好在你们相处这么久之后,感情还是这么和睦。毕竟相处的时间越久,就会看到越多彼此不喜欢的地方』

「不、不喜欢的地方倒是……那个,就算真的有,我们也可以互相讨论怎么改进」\\

常常听说有些情侣等到开始同居之后,双方的生活习惯、金钱观、卫生观、常识与道德观之类的都会逐渐外显,让彼此的感情出现裂痕。不过即使在近乎一起生活的情况下,周也几乎看不出来真昼有什么让他讨厌,或是令他觉得无法接受的地方。\\

硬要说的话,第一个应该就是她常常把事情憋在心里,第二个就是她会为了讨周高兴而做出千岁灌输给她的大胆举动。关于前者,真昼已经变得坦率许多,这种情况逐渐有了改善;至于后者,问题反倒出在千岁身上,还是从她身上下手为好。\\

再说另一边,真昼也不怎么会指出她对周的不满。不对,在他们刚认识的时候,真昼倒会毫不客气地指指点点,现如今或许是真昼希望他改正的地方已经大部分都改好了吧。\\

但就算是这样,周仍然觉得自己可能还有一些缺点是真昼希望他改掉的,于是他郑重其事地告诉真昼「如果我有什么不好的地方,你尽管说,我也不是想让你伤脑筋。我希望我们彼此都能过得开心点,所以只要是我能改的,我都会想办法改」结果真昼听了,却慌张地连连摇头。\\

「是你太关心我了,真的不用这样啊!?你已经很棒了哦!?」

「客套话就免了吧」

「……那么,这就是你该改掉的毛病了。请你在得到夸奖的时候坦然接受」\\

真昼明显地噘起嘴,拍了拍周的大腿。周见状,心想再让她闹别扭也不好,便答了一句「我知道啦,谢了」以免她嘟起脸颊。\\

『看来真昼已经完全信任你了呢』\\

听到小雪感慨的低语,周把视线转回屏幕上。只见小雪默默观望的视线正落在刚才他俩嬉闹着牵起手的地方,多半是给她看得一清二楚了。\\

真昼似乎也很害羞,她缩起肩膀,脸颊泛红;周则是拼命忍耐着涌上心头的羞耻感,不让它显露出来。

小雪看着周和真昼这副模样,愉快地咯咯笑了起来,接着带着笑容缓缓把视线向周倾斜。\\

『在藤宫你看来,真昼是个怎么样的人?』

「怎么样是指……」

『哎呀,这样弄得好像我在约谈你一样。不该这样问……我想知道的是你眼中的真昼』\\

小雪用打量般的眼神温柔地问道,周慢慢在心里思考着如何回答,便没有立刻将回答讲出。\\

如何看待真昼?\\

她的意思就是,在周看来,真昼是个怎么样的少女。周说过他喜欢的是真昼的内在,那么周真的理解她吗?小雪的这个问题就像是为了消除这个疑问一般。\\

而且,从小雪的态度可以看出,她做这一切都是为了真昼。\\

理解小雪的意图后,周开始烦恼该怎么回答。\\

(我眼中的真昼……)\\

周静静移动视线,看向身旁的真昼。她似乎很在意周的想法,与他四目相对的双眸中流露出些许期待与不安。

面对那窥探般的眼神,周决定不加修饰,坦率地说出自己的想法。\\

「……她很爱逞强,又很会忍耐,但其实很怕寂寞,也喜欢撒娇」\\

这就是周眼中的真昼。\\

「周君!?」

「不是啦,就是那个,你好像很喜欢向我撒娇嘛」

「是喜欢没错!但是请不要在小雪阿姨面前光明正大地说出来!」\\

周突然暴露了真昼不想让人得知的一面,让真昼的脸比刚才更红了,她不断拍打着周的上臂,但周完全没有收回这句话的意思。\\

「基本上,真昼什么事都能靠自己做到,也不想让别人进入自己的内心,所以我认为她是那种什么事都想自己做的人。她不太会依赖别人,偶尔会因此作茧自缚,或者说是因为自己设下的限制而伤透脑筋」\\

真昼总是谦虚客气,无法以自己为优先,不会放纵自己。她多半是下意识地不想造成别人麻烦,不想让别人抛弃自己,这种想法使她在面对周的时候,也不想完全依赖对方。\\

面对周以外的人时,就更是如此了。她本来就无法完全信任别人,再加上她会下意识地认为自己应该当个好孩子,因此她讨厌表现出自己软弱的一面,进而会在外人面前戴上面具,装成完美的少女,甚至让别人以为这才是她普通的样子。\\

这就是大家所看到的「天使大人」。\\

可是,现在的真昼不一样了。\\

「现在的她学会了依赖别人,学会了依靠别人,愿意让我陪在身边,相信我,让我看到最真实的她。我认为这对她来说肯定是个极为重大的决定,也证明她对我寄予了庞大的信赖与爱情」\\

正因为真昼相信自己不用伪装,可以依赖周,可以向周撒娇,她才会像现在这样感情丰富、怕寂寞、用坦率的心灵渴求着周。\\

这让周感到非常自豪。\\

「在我面前,她不用当个乖孩子,也不用努力,会毫不掩饰地向我撒娇,实在非常可爱……让我也忍不住想要宠她」\\

最一开始的真昼,总是建起一道透明的墙,柔和地将一切弹开。不过她渐渐能够抛下隔阂与客气,尽管仍然有点拘谨,却还是能坦率向周撒娇。正因为周见证了她改变的过程,他才更加觉得真昼撒娇的样子非常可爱。\\

当然,不论是平时独立自主的真昼,还是企图让周变成废人的小恶魔真昼,两者的可爱都不需多言,但那又是另一种方向的可爱了。\\

周心里也不是没有想要疼爱她,让她融化、沉溺其中的冲动,可是那样做就意味着违背真昼的心意。

因此,周给自己设下了限制,只给予真昼所希望得到的疼爱。不知道真昼有没有发现这一点。\\

总之,似乎周每多说一句话,真昼的羞耻槽就会不断累积,现在她满脸通红,浑身颤抖,一副泫然欲泣的样子。但愿她这会儿快哭出来的模样不会让他没法过关。\\

「啊,我并不是因为真昼可爱才宠她的。正因为她总是努力不懈、严以律己,赢得了我的尊敬和尊重,我才更希望自己能成为她安心的避风港。我不会在她不希望的时候那么做的!」\\

就算再怎么喜欢真昼,若是违反本人的意愿,过度的宠爱对双方反而都没有好处。周也不可能主动卸下自己的枷锁。

周最优先的还是让真昼幸福安稳地度过每一天,所以他打算适当控制一下。\\

「那个,我发誓我绝对没有做出久慈川阿姨担心的那种事情,像是夺走真昼的事物、单方面地要求她,或是伤害她之类的。光是用说的可能没什么分量,但我绝不会违背承诺」\\

小雪听到这里,稍微睁大了眼睛,接着佩服似的轻轻叹了口气,看来周的回答符合她问题的意思,而她想要的答案应该就是这个了。\\

「对我来说,真昼是个惹人怜爱、份量举足轻重的女孩,我想带给她幸福,但同时我们也是互为平等的。我们不会单方面地强加负担给其中一个人,也不会无视彼此的意见,而是会好好沟通,努力让双方都过得更顺心。我希望我们都能成为彼此幸福的归宿。如果是我和她,就一定做得到」\\

周喜欢疼爱真昼,也想为她尽心尽力,但真昼并不希望自己成为只需要享受这一切的存在。\\

她希望的是彼此都能接纳对方的优点和缺点,转化成双方都能接受的样子,互相关怀,平静地生活下去。

不能让其中一方背负太多,而是要两人分担负担,互相支持,一起生活下去。\\

周也完全抱持着同样的想法。\\

「所以,请不用担心。我会让真昼幸福的。我们会一起得到幸福」\\

虽然周觉得这番话会让旁人听了觉得很老套,但这无疑是他的真心话,也是他不会改变的信念,更是今后仍会继续努力的宣誓。\\

对彼此抱持敬意、信赖与尊重,接纳彼此的差异,分担彼此的辛劳,互相扶持,一起走下去,这样才是幸福的妙方。\\

周认为,如果是和真昼在一起,他就能够付出努力去得到这些幸福。\\

虽然并不是完全不害羞,但唯有这点,周一定要诚恳地、正确地传达出来,于是他真挚地注视着小雪的眼睛,如此说道。对此,小雪则是缓缓地做了个深呼吸。\\

虽然他对于自己莫名加快的心跳有种事不关己的感觉,不过他还是屏息等待着小雪的回应,接着对方脸上便浮现如花朵绽放般柔和的笑容。\\

原先让人不由得正襟危坐的,又不同于威压的气息烟消云散,就好像一下子卸下了力气,无比柔和的微笑由内而外流露出来。\\

『我再次感觉到,真昼的选择没有错』\\

小雪这句话是说给周听的,还是说给她自己听的呢?周不知道,但至少小雪无疑是认可他了。\\

『我自然是相信真昼的眼光,但为了以防万一……我还是试探了一下,不好意思啊。虽说我本来还打算,要是人品有问题,我不惜驱策这把老骨头也要把你们拆散』\\

周现在才意识到,要是走错一步,事情很可能会演变成一场大骚动。他暗自松了口气,不动声色地庆幸自己入了小雪的法眼。\\

本来还打算花费一生去让真昼幸福的,被拆散那可是一点都不好笑。但若是没能得到小雪的认可,那他也会为自己没达到小雪的合格线而难受。\\

「不、不用做到那种地步也没关系!周君是个好人,父母也很好……!」

『哇,已经和他父母打过招呼了嘛』\\

周不禁心想,这个人其实和志保子一样,听话只听对自己有利的那一半。虽然小雪的个性理所当然和志保子不同,但她想必也是个很不好对付的人。\\

『这样很好啊,重要的就是先确保一个珍惜自己的人,再扫清周围的障碍。毕竟他可是现在这个时代很宝贵的人才呢』

「呜……呜呜,小雪阿姨,你这样说太直白了,或者说说法不太好。我并没有这个……」

「抱歉,真要说的话,可能是我先开始清理的障碍才是」

「周君!?」

「应该说,有一半是都是妈妈在使劲清理……她好像很有干劲,说什么都得让这么可爱又懂礼貌的好孩子变成自己的女儿」\\

回想起来,周总觉得志保子早在自己迷上真昼之前,就开始干劲十足地为他扫平周围的障碍了。该说她嗅觉和洞察力惊人,还是她横冲直撞呢?\\

对于志保子这番强硬的作为,周也不是不觉得她多管闲事,但就结果而言,父母也是周能和真昼在一起的原因之一,所以周无法断言志保子很烦人。话是这么说,就想要靠自己的周来看,他倒是也有点想告诉志保子别多管闲事。\\

「……照这么说,我感觉后半部分是就像是我借出去了一个挖土机一样」

「咦?」

「不,没什么」\\

真昼小声地以补充的口吻说了些什么,但刚才在心里埋怨志保子的周并没有听清楚,于是又问了回去。\\

不过真昼似乎已经不想再说下去,她把头扭向一边。\\

这是真昼想隐瞒什么的时候会有的动作,不过周并不打算强行追问,只能等她哪天愿意主动开口了。

看来笔电的麦克风有接收到真昼刚才说的话,刚好听到的小雪因此露出十分愉快,却又藏不住的高雅笑容,随口应了一声『哎呀哎呀』,让这件事情到此为止。\\

『看来是我多虑了。我也真是老糊涂……居然这么紧张兮兮的,还多管闲事』\\

小雪反省似的垂下眼眸,只有视线停留在慌了手脚的真昼身上。\\

『这么一来,我心中的疙瘩就消失了。因为我早已不是能够随便插手的立场,所以我真的很担心真昼的未来』\\

周听见身旁传来一声「啊」的轻呼。\\

『不过,已经没问题了吧。看你们的样子,我觉得可以放心交给你了。你可能会觉得,我这种曾经离开过的局外人凭什么说这些,但身为一个关注过真昼的大人,我的想法就是那样』\\

小雪完全是为了真昼着想,才会试探周的。周也明白这一点。\\

为了不让真昼为孤独所苦,小雪自幼陪伴在她的身边;为了不让真昼遭受他人的伤害,小雪给予了正确的教导;为了不让真昼不为将来发愁,小雪让真昼磨砺自己;为了让真昼能够对他人敞开心扉,不让她对人彻底失望,小雪怀抱满满的爱来对待真昼。\\

而她也认为,可以将如此珍爱的真昼托付给周。\\

『请你们两位下次一定要来我家坐坐,我想把你们介绍给我儿子和媳妇认识,让他们看看我的另一个可爱的孩子和她的男朋友。对了,我儿子不会因为我多了一两个孩子就吃醋的,尽管放心好了』\\

听到孩子这个词,真昼似乎再也忍不住了,一度止住的泪水又开始不断涌出。

真昼就像是要把至今为止的人生中该流的眼泪全都在这里消耗完毕,又像是她脆弱的部分逐渐剥落一般,她发出微弱的呜咽声,再也忍不住泪水,开始哭泣。\\

小雪看着她那副模样,脸上始终只是挂着充满慈爱、仿佛要将她温柔包裹起来的笑容,和周一起静静地等待真昼自己克服情绪的波动。\\

『呵呵,我还没同意你们结婚呢。我不能只靠通话,而是得亲眼确认他是个什么样的人才行』\\

小雪看准真昼冷静下来的时候,故意用开玩笑的语气这么说道,害周差点猛咳起来。\\

他嘴唇直打颤,想要反驳小雪,对方却用颇有深意的眼神看着他,仿佛在问他「就是这样不是吗?」让周完全无法反驳,只能嘟着嘴巴。\\

(她的本性果真和妈妈很像!)\\

小雪没准要成为跟志保子「危险请勿混合」的二号组合(一号是千岁),周对此感到颤栗,不过就没有穷追不舍这点来看,小雪要比志保子和千岁要好上一些。

小雪似乎察觉到周无法摆出强硬的态度,她轻笑了一声后,便挺直腰杆,重新转向真昼,脸上的表情就像是在看着心爱的孩子一般,任谁都会从小雪的态度中感受到母性的光辉。\\

『所以,你们两个就别客气,尽管来我们家玩。我很欢迎你们』

「……好的!」

「谢谢」\\

周感觉就像约好要回老家打招呼一样,一股缓缓涌上的、伴随着酥痒的喜悦与安心,让嘴角不由得露出笑容;至于真昼,不知道是不是因为她太高兴了,原本以为已经干涸的眼眶又落下了一滴泪水。小雪则是以美丽的微笑面对两人。\\

『啊,要是把真昼弄哭,我可饶不了你』

「……刚才那不是我害的吧?」

『哎呀,这个就……呵呵,请放过我吧』\\

小雪俏皮地笑着说道,随后周和真昼对视一眼,忍不住笑了起来。\\

『如果是喜极而泣,就让她尽情地哭吧。看起来她还不习惯幸福的滋味,麻烦你连同至今为止的份也一起补偿给她』

「那我就不客气了。今后我也会努力让她喜极而泣」

「周君,你……」\\

真昼慌张地阻止周说下去,但他完全不打算收回那句话。\\

如果让真昼因为悲伤或愤怒而哭泣,自然是岂有此理,但若是因为喜悦而哭,则又另当别论了。泪水是发自内心的感情,如果这个感情是正面的,起因于喜悦,那便没什么好忌讳的。\\

她至今为止都没有得到过这样的机会,所以就算周带她去体验各种各样的快乐、独占了她的泪水,也不会有人有意见吧。\\

『那就交给你了……请让他给你很多幸福,等下次见面的时候再跟我分享吧。我很期待哦』\\

周的回答大概足够让小雪满意了,只见她露出开朗的笑容,就像是要疼爱他们一般,用柔和至深的目光轻抚着两人。\\

那就像是过去志保子看他的眼神。\\

『那么,再见了。希望真昼今后也能过得健康幸福』\\

小雪用毫无阴霾的清澈声音为真昼的未来祈祷,有些依依不舍地看了感动得发抖的真昼一眼,接着画面就黑了。\\

颜色无声无息地消失,画面只反射出周他们的身影和房间的装饰。虽然告别得很干脆,但周的心中确实充满了温暖的余韵。\\

真昼肯定也一样,她暂时沉浸在余韵中,有些出神地继续看着刚刚还显示着幸福的屏幕……最后,她慢慢地把身体靠向周。

真昼依偎着周的上臂和肩膀,撒娇似的靠在他身上,然后静静地深呼吸一口。\\

她亮丽的秀发随着胸口的上下起伏而从肩上滑落,周看着这幅光景,等待她整理好自己的心里话。\\

「……周君」

「嗯」\\

真昼小声地呼唤。\\

「……我不知道该说什么才好,只是真的很高兴,高兴到都快烧坏了……我从来没想过会有这么一天」\\

她一定在心底深处渴望着,渴望能像家人一样对待小雪。\\

可是,她没有足够的意志力去下决心执行。

真昼总是以他人为优先,准确一点的说法是她很胆小。\\

她应该想到了许多和小雪取得联系、听到小雪说话、和小雪见面的方法——之所以没有、没能付诸行动,多半是因为害怕被小雪拒绝,所以下意识地自我克制。\\

周至今仍在反省自己制造了她的恐惧和不安,但他丝毫不后悔与小雪取得联系。

因为真昼此时露出的表情是如此地满足。\\

「……你觉得幸福一点了吗?」\\

周也知道明知故问的自己性格很差劲,但他无论如何都想听到这个答案。

即使是自我满足也好,周就是想要知道,自己是否为心爱的女友带来了幸福。\\

「当然。那个,我高兴得不得了,感觉好幸福,脑袋轻飘飘的,心跳也很快……可是,一想到这一切都要结束,就觉得好难过。我自己也知道我的情绪很奇怪」

「嗯,毕竟发生了好多事嘛,你就慢慢消化吧」\\

真昼用比平时更稚气的声音断断续续地低语着,这种口吻与其说是在回答周,更像是在整理自己心中涌起的感情。周没有催促她,而是应和了一句。

真昼似乎还无法驾驭涌上心头的感情波涛,她从原本靠在周肩上的姿势,改为搂住周的手臂,脸靠着周的上臂,就像是紧紧黏着周的手臂一般。\\

真昼额头抵在周的手臂上,喉咙发出闷哼声,似乎在发泄积蓄了不少的冲动,周见状便轻笑出声,伸出空着的另一只手,用指尖梳理好真昼亚麻色的川流。\\

「……别担心,这份幸福不会消失,只要你慢慢体会就够了。我们要一起牢牢记住,今天的你非常开心」

「……嗯」

「希望总有一天回想起今天的事情时,能笑着说今天真是个幸福的日子」\\

但愿今天能成为众多幸福回忆之中的一个。\\

周希望今后能和真昼一起感受更多的幸福,也打算让她幸福,所以如果真昼能将这样的幸福当成每天的日常之一,而非今天的专利,在回想起来的时候觉得幸福,他就会很高兴。\\

「……喏,生日还没结束哦?」

「我已经很满足了,感觉肚子都饱了」

「这样啊,真伤脑筋。蛋糕还没吃完呢……」\\

周知道真昼说她饱了是什么意思,但还是故意开玩笑似的装出遗憾的表情,真昼则一副忸忸怩怩的样子,撒娇地把额头靠在周的手臂上。\\

「……如果你愿意喂我,我可以再吃一点」

「嗯,只要你想吃,要我喂多少都行」\\

尽管有些克制,但真昼还是抬眼看着周。她应该是在用自己的方式向周撒娇吧。

周的器量没有小到不让她撒娇,于是他温柔地摸了摸真昼的头,像是在说只要她希望,自己什么都愿意做。真昼则是一脸难为情地、舒服地瞇起眼睛。\\

「这么多也吃不完,我分你一些吧」

「嗯,谢谢……明年我会做个小一点的。那样我们俩就能轻松吃完了」

「明年……」\\

真昼听到「明年」这个词,就像是在遥想未来一般,用几乎听不见的声音复述着。她一定是在想像自己和周待在一起的景象吧。\\

真昼的脸颊染上淡淡的红晕,仿佛一道从黑暗中浮现的灯光,既黯淡又清晰。她抬头看向周,像是在窥探他的反应。\\

她的眼中蕴含着无法完全隐藏的期待。\\

从她的表情中,周能感受到她对未来的期待,不再讨厌过生日,而是真心期待生日到来的想法。周让发自心底涌出的喜悦表现在脸上。\\

「没错,明年也会有的。期待不期待?」

「嗯」

「太好了。我也很期待明年」\\

无论是今后要与真昼共度的时光,还是亲手让真昼获得幸福的喜悦,或是真昼信赖并期待着自己的兴奋感,对周来说都是期待、喜悦与幸福。

现在的周可以肯定,真昼一定也是这样想的。\\

「……我真的很谢谢你,出生在这个世界上,还愿意喜欢我。我一定会让你幸福的」\\

周并不是说给真昼听,只是自己不小心说出来罢了,但这句话似乎完整地传进了真昼的耳里。\\

真昼睁大了那双宛如琥珀般晶莹的眼眸,脸上浮现了仿佛要融化般的甜美笑容,放松身体依偎在了周的身上。
