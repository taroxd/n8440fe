\subsection{三方面谈后的忧郁}

三方面谈过去后,来临的就是从月底到月初的,定期考查的准备时期。\\

由于准备真昼生日和打工,现在每天都挺忙碌,再加上考前的针对性复习,要做的事情就更多了,周也就没多少能放松下来的时间。\\

但这样的日子说充实也充实,丝毫不引人反感。\\

「三方面谈才结束,就是备考期了,真愁啊真愁」\\

树看着亲切的教师给的一沓考试资料,叹了口气。

根据科目不同,有的任课教师会将出题范围整理和发放下来作为勉励,虽说肯定会善加运用,但分量摆在那里,也有不少学生很头大。从那沓纸的数量就看得出来,出题范围相当大,要记忆的点也是数不胜数。\\

「跟高一相比,考试的压力更大了啊,毕竟对成绩也开始在意了。我这的负担相当重哎……话说这范围真够离谱的,虽然是因为讲课速度快,也没办法就是」

「但这也太多了吧~」\\

千岁也不例外地抱着一沓纸走来,脸上是花季少女往往会顾忌做出的一张苦瓜脸。旁边的真昼见千岁这副面孔,露出了苦味相当浓的苦笑,她肯定也是看出了千岁正烦闷无比。\\

「说真的,不行的吧,根本做不到啦」

「我也有点烦这个」

「话是这么说,周分数都能好好地拿到,成绩也不错哎」

「那是因为我上课还挺认真的」

「说话就是从容……呜呜……」\\

就算是在这儿哭丧着脸,周能做的也充其量就是教她学习。分数本身都是源于平日里的努力,只有靠千岁自己加油了。\\

「千岁最好再拿出点干劲啦……就数学你是明摆着一点干劲都没有……」

「完全不知道怎么才能喜欢数学,我该怎么办才好」

「这个看人吧,我还挺喜欢数学的,起码在我们做的范围里基本都是能推出答案的。像解谜一样,用记住的公式去解开谜题,还是很有意思的」

「要说的话我也算是差不多吧」

「问题就是推不出啊!」

「要不先把公式好好记下再说」

「气哦!」

「千岁首先是太畏难导致拿不出干劲,这才是最头疼的吧。背书的学科没那么不擅长,为什么公式就是记不住呢……」

「因为看到数字就头痛」

「那、那我好像也没什么办法……」\\

见千岁内心的拒绝说是过敏都不为过,要讲课的真昼也只能困扰地向周投去求助的视线。\\

周也觉得,若是她自己不拿出努力和干劲,那做不到的事情依然是做不到,唯一的办法就是引导她的干劲了。\\

「总之把绝对要用到的那些公式先记下来吧,哪怕只记住基础的,也起码能混过及格线了。我可不想看到朋友考个不及格,掉到补习地狱里去」

「不要!」

「这个不能不要。必须要」

「呜哇昼儿妈妈,爸爸欺负我~」\\

千岁直接搂上了旁边的千岁,不过明显是千岁更高,一点都没有小孩的样子。\\

「谁有你这么大一个孩子了。还有别跟真昼黏着」

「吃醋了吗」

「对的对的吃醋了」

「……都承认吃醋的话那我就算了吧」

「你是不是在拿人打趣?」

「不存在的不存在的」\\

那毫无疑问就是打趣的态度,而千岁则一转方才任性小孩的模样,带着轻飘飘的态度扭头朝向一旁,实在让人不由得头疼。\\

她还小声地说出了一句多余的话「倒是不否否认是夫妻呢」。周瞪了她一眼让她闭嘴,然后一边将那一大叠纸收到文件夹里,一边叹了口气。\\

「说起来周考试前的打工怎么说?」\\

今天没有打工,能够稍微清闲一些,正当周拿手机看着日程表,在脑海中整理今后安排的时候,千岁抛来了一个问题。\\

「嗯,和平时一样都有安排打工。我平常学习都没落下,考前两天和考试期间也请好了假,打算在那会儿解决」

「你倒是很有信心来得及嘛」

「那得谢谢真昼,在家她教了我很多,而且真昼还特别会教人」\\

自己学习好,也不见得就一定擅长教别人,而真昼就属于非常擅长教人的。\\

由于她已经提前就完美地掌握了课程内容,因而也理解问题的要点,会先询问是哪里遇到了困难,然后利用范例和提示,帮人解出题目。\\

背书科目只能靠自己坚持不懈的努力,但除此之外真昼都会细致耐心地讲解不懂的部分,于是周也就不再有哪里特别弄不懂了。\\

「我也那么觉得,不过能迅速理解也是因为扎实地打下了基础吧」

「毕竟是基础,是要从积累得来的」

「快停下扎心的攻击~」\\

会觉得这是攻击,说明自己很有问题——这种过于刺耳的话实在是说不出口,不过千岁从视线中察觉出了那意思,露出蔫了似的表情。\\

「以及我打工的地方还有学习很好的前辈,没顾客的闲暇时间也总会教我。到底是要靠朋友、真昼和前辈啊」

「啧……我家哥哥在学习方面作用就……他学不来,靠不住」

「小千你哥哥听到了要哭的哦」

「我都被他惹哭多少次了,就这,小意思小意思」\\

异性的兄弟姐妹之间会有许多故事,千岁无奈地耸肩招手,心中看上去也是颇有微词。\\

听说千岁家里关系非常良好,这方面倒是没什么可担心的,至于成绩倒是要担心担心,但愿她能够拿出干劲来吧。

\vspace{2\baselineskip}

「日程排得很紧,能行吧?」\\

此后男女分成了两头,跟树一起留在教室里的周对树的关心点了点头。\\

顺带一提,真昼是被千岁带去了附近的杂货店。

周想跟树当面讨论计划,便拜托千岁自然地支开真昼,不过听到她考试相关的这些,周又十分烦恼是不是不该再占用她的时间为好。\\

「嗯,算是能行。像是这样的忙碌程度,之后两头开花应该也没问题,而且也能成为很好的经验」

「爱情的力量啊」

「你好吵」

「好好好」\\

像这样的对话也已成习惯,两人都将其轻描淡写地带过,在确认周围没有会给真昼走漏消息的人后,周说起了正题。\\

「说起来,那件事情安排得过来吗?」\\

在给真昼庆祝生日这件事上,有件必须要人帮忙的事,周将此事委托给以树为首的朋友们,这次问的便是事情的进展。\\

「我跟小千都可以。优太还没问,多半没问题。木户的话你直接去问应该更快,你俩比我俩关系更好」

「嗯,我知道了……要是能空出时间来就好了」

「感觉为了椎名还是会来的」

「实在不行的话就有多少人就多少人吧,也不能给人家添麻烦」

「大概不会觉得麻烦的吧。既是朋友,也是不怎么会拜托事情的人,能卖你个人情,她应该会很乐意帮忙的吧?」

「……希望吧」\\

周也知道树是故意说得好像玩笑一样,因而有些难为情。他笑着想隐藏这种心痒之感,树便无奈地长叹一口气「说的就是你这种地方」,轻轻将拳头打在周的肩膀上。\\

「不过没问题吗?千岁她们也想当天庆祝吧?」\\

周姑且是事先跟真昼确认过,可以把生日告知千岁等人,因此请求协助的时候他也做了解释。但周所拜托的愿望,换言之其目的是独占真昼的一天,这同时也就延后了她们给真昼庆祝的权利。\\

大家对此不会介意吗?周心里感到不安,树便朝他撂下了一句「你好笨」。\\

「至少真昼心目中的优先级……这么说可能不太合适吧,就是高兴的标准是你。小千也说『重要的是昼儿要最开心』。我也是这么觉得的,而且——」

「而且?」

「她说『第一位就让给男朋友吧』」

「她当她谁啊」\\

那仿佛真昼是她的东西一样的口吻,让真昼不由得笑出声,但千岁心里真昼占了如此大的比重,周自然是高兴都来不及。\\

起初不跟朋友深交,喜好独自一人的真昼,有了能够交心的朋友。\\

那对真昼来说,一定是非常幸福的。\\

对周而言,也是一样。\\

「那第一个我就收下咯」\\

感受到她们十足的关怀和善解人意,周便领下了她们的好意,树也带着温和的眼神点点头,俨然是在表示认可。\\

\vspace{2\baselineskip}

「接下来,就只剩下做我能做到的事了」
