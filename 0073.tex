\subsection{天使大人的布丁}

就布丁来说,虽然流行的那种放了大量鲜奶油的、入口即化的款式也很美味,但周最喜欢的还是较硬的、放大量鸡蛋的、用勺子挖也不会破坏形状的类型。\\

布丁保留住了鸡蛋原本的味道,同时还蕴含着浓郁的鲜奶油味。尽管味道偏甜,但多亏了微苦的焦糖,这种甜味并不会让人发腻。

相反,味道的余韵很爽口,诱惑人一口又一口送到嘴里。\\

周并不特别爱吃甜的东西,但真昼亲手制作的布丁他却吃得入迷。转眼间,盘子上的布丁就无影无踪了。\\

「呼啊,好吃」

「承蒙夸奖,不胜荣幸」\\

布丁是作为晚饭后的甜点拿出来的。周一下子就吃完了,他只吃一个还覺得不夠,所以又吃了第二个。

作为男高中生,周不算胃口特别好的,但真昼亲手制作的布丁他即使是吃饱了也能继续吃得下去。\\

周感受到超过吃下的布丁分量的满足感,摸过自己的肚子,愉悦的心情显露无疑。\\

「你什么都能做啊」

「因为大致都被灌输了一遍」\\

尽管真昼说得没什么得意,但实际上她制作的料理种类丰富,偶尔还会蹦出周不知道的料理。%真晝這麼說道。儘管真晝一點也不覺得有什麼好驕傲的(參考

当然,这些料理既美味又吃不腻。真昼这样的人待在身边为了自己做料理,是件非常幸福的事。\\

「不愧是你啊,多亏了你我倒挺幸福的」

「……幸福?」

「是啊。每天都能吃到这么美味的食物,怎么会不幸福呢。这可是我每天的乐趣」\\

真昼的料理占了每天乐趣的一半。用真昼的料理给一天收尾,大部分不开心的事情都可以忘却。

真昼每天都會帮忙做饭,这本身就已经是幸福的事情了。周每次都一边吃一边品尝着幸福,但真昼应该是不太了解自己料理的价值吧。\\

周以前也说过真昼的料理是幸福的味道,但真昼却对此没有多少认识。要是周不使劲夸奖,她大概就不会理解到自己料理的价值。

而且,称赞美味的东西是一种礼仪,应该坦率地传达出去。\\

「……这、这样啊」\\

面对正面的赞美,真昼脸稍微有点红,缩了缩身子。\\

「……周君表扬我,我很开心」

「如果我就可以的话,表扬要多少都没问题啦。光是每天说饭菜好吃还不够吗?如果想听更详细的感想,我说就是」\\

据说,世上夫妻的不和都是因为忘记了互相感谢。\\

虽然周和真昼并不是夫妻什么的,但周站在每天得到料理的立场上,不能忘掉感谢的心情。而且,味道的感想也会带来动力,所以只要真昼想听,周很愿意详细说说。\\

只不过,真昼摇摇头,表示出拒绝的意思。\\

「不、不用了……我会死的」

「这么夸张」

「沒有夸张。现在这样已经够了」

「是吗?不过,今后还得每天都靠你為我做飯,还是得好好道个谢。一直以来谢谢你啦」\\

周的伙食全都靠真昼支持,所以他始终怀着感谢之情,也不可能忘恩负义。一切都是托了真昼的福。

要是没有真昼,周就直奔废人去了。所以,他希望今后真昼也能在自己的身边;如果再贪心点的话,直到永远。\\

周心怀感激地露出笑脸后,真昼就好像来电振动的手机一样身体发着颤,然后站了起来。\\

「……周君大笨蛋」\\

不知为何,真昼用可爱的声音骂了句笨蛋,然后拿着餐具去洗碗了。于是,周也跟在后头,把自己用的餐具搬到了水槽里。\\

由于事发突然,周感到了一阵不解,心想着「饭后的家务都是自己的任务,不需要真昼来做」,便轻轻抓住了真昼的胳膊。接着,真昼猛地转向了周这边。\\

真昼比起刚刚更加涨红的脸,在看到周之后红得更深了。因此,周总觉得非常坐立不安。\\

「……我、我来做这些,你在沙发上等我。好嗎?」\\

周摸了摸真昼的头,然后赶她出了厨房,真昼便小声念叨着冲到沙发上陷了进去。\\

周看到真昼不像平时那么冷静的行动,眨了眨眼。\\

接着,周想起了先前真昼满是害羞的脸。为了让头脑冷静下来,他决定使用冷水洗碗了。
