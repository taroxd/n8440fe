% 48 天使様のお迎え
\subsection{迎接天使大人}

%  真冬と言えるこの季節、それも日が差さなくなった夜は気温が低い。\\
现在是隆冬时节,没有阳光的晚上气温很低。\\

%  防寒面と装飾性を考えてライトグレーのセーターにネイビーのピーコート、裏起毛の黒スキニーを合わせているが、それでも地味に寒いので制服にコートの真昼はどれだけ寒い事か。\\
考虑到防寒和装饰性,周穿的是浅灰色的毛衣、藏青色的水手外衣、内层绒毛的黑色卡其裤,然而即使这样周依然觉得冷。穿着校服加外套的真昼该有多冷啊。\\

%  真昼は冬場に厚手のタイツをはいているとはいえ、女子高生らしく校則違反や下品にならない程度の丈にしてあるスカートは、見ていて非常に寒そうなのだ。下にジャージを着せたくなる。
尽管真昼在冬天会穿厚实的紧身裤,然而那显出女高中生范儿的,高度勉强不至于违反校规或者不雅的裙子,看着就让周觉得非常冷,想让她下面穿上运动裤。

%  たまにすれ違う女子高生も無駄に短いスカートを揺らしているので、美に対する女子高生の努力は恐ろしい、と痛感した。\\
偶尔擦肩而过的女高中生也摇晃着短得没意义的裙子。周痛切体会到,女高中生对于美的努力有多么可怕。\\

%  そんな事を考えながら、真昼からもらったマフラーに口許をうずめつつ足早に最寄り駅に向かう。\\
想着这些事情,周用真昼送的围巾捂住嘴角,快步前往最近的车站。\\

%  どうやら大型商業施設に出かけたらしく、電車を使ったようだ。最寄り駅から自宅は徒歩圏内であり、千歳情報ではもうすぐ電車が到着する筈なのでちょうどいい頃合いだろう。\\
似乎真昼是去了大型商业设施,路上乘坐了电车。最近的车站是自家能步行走到的距离,听千岁说电车马上就要到了,现在时候差不多正好吧。\\

%  歩けば風にセットした髪が軽く揺られるものの、崩れるまでは行かない。
行走的时候,风轻轻吹动了周弄好发型的头发,但并不至于把头发吹乱。

%  ぐしゃぐしゃになったら流石に直さないといけないので億劫だ。日頃からおしゃれしている人間は尊敬ものである。\\
如果头发乱成一团就不得不修了,很是麻烦。平日里就打扮的人类真是值得尊敬。\\

%  そんな事を考えながら黙々と歩くと、駅が見えてくる。
周想着这些事情默默走着,接着就看到了车站。

%  マンションの方向から考えてこの出入り口に姿を現す筈なので、出入り口付近で待っていれば確実に真昼と出会えるだろう。\\
考虑到公寓的方向,真昼应该会在这个口子出来。在车站口附近等着的话,应该就能确保遇上真昼了吧。\\

%  出入り口の壁に背を預け、時間を確認しつつ真昼を待つと、ほどなくすれば見慣れた亜麻色のストレートヘアの少女が駅から出てきた。\\
周靠在车站口的墙上,一边看了看时间一边等着真昼。没过多久,车站里就走出了眼熟的亚麻色直发少女。\\

% 「真昼」\\
「真昼」\\

%  声をかけると、聞き慣れた声だからか警戒なく振り返って――そして、周を視界に収めたであろう瞬間に固まった。\\
周一搭话,真昼听到熟悉的声音就没有警戒地回过头——接着,视野里看到周的瞬间就愣住了。\\

% 「え、……はい? な、なんで」\\
「诶……嗯?为,为什么」\\

%  なんで、というのはこの格好の事だろう。
为什么,指的应该是这身打扮的事情吧。

%  迎えに来る事自体は恐らく千歳から伝わっていただろうが、初詣以来の姿で来るとは想定してなかったらしい。\\
周会来迎接这件事恐怕千岁已经告知她了,但她似乎没想到周会以新年参拜那时的样子过来。\\

%  流石に、周も普段の適当な格好にいつもの髪形のままこようとは思わない。
再怎么说,周也不会想用平时随意的样子和发型就直接过来。

%  もし周りに見られて謎の男と周をイコールで結ばれても困るし、真昼の隣を歩くならそれなりに格好をつけないと真昼まで軽んじられる。
如果被周围人看到,将谜之男人和周等同起来的话,周会很困扰。而且,要在真昼旁边走路的话,一点样子都没有的话连真昼都会被轻视的。

%  変装目的ではあるが、真昼の隣に並べる程度にはやはり着飾るべきだろう。\\
尽管周的目的是变装,但要达到与真昼相称的程度,果然整理仪容还是必要的。\\

% 「自分で出来ないと思ったか。流石に普段着で迎えに来る訳にいかないだろ」
「是想着我自己办不到么。再怎么说平常那副样子来接你也太不像话了啊」

% 「……そうですけど」
「……这倒是啦」

% 「似合ってないか? 一応鏡で確認したけど、変かな」\\
「不合适吗?我对着镜子确认过来着,还是很奇怪吗」\\

%  普通に無難な組み合わせで髪型は先日の初詣とそう変わらないものなので、おかしくはないと思っているが、美的センスに優れた人間からしてみたら駄目なのかもしれない。
周一身平凡而朴素的穿搭,发型则是和前几天新年参拜的时候用的一样,因而自认为应该不算奇怪,不过也从许美感优秀的人看来说不定还是不行。

%  たまにちらちらと視線を感じたのは、もしかしたらおかしかった、という可能性もある。\\
偶尔感到的往这边看的视线,或许也有可能是因为自己的样子很奇怪吧。\\

%  それなりに格好つけたけどダサかったのか、とちょっとショックを受けかけたが、真昼が慌てて首を振って「似合ってますっ」と肯定してくれたのでほっと息をつく。\\
自认为样子已经整的不错了,却似乎还是很邋遢,这让周稍稍有点惊讶,但真昼却连忙摇着头「很合适的」这么肯定到,让周总算松了一口气。\\

% 「それならいいんだ。ほら、冬だしすぐ日が暮れるだろ。一人で帰るのは危ないし」
「那就好。行吧,现在大冬天的一会就要天黑了。而且一个人回家还很危险」

% 「……そ、それは分かってます、けど」
「……虽,虽然这我还是知道,了啦」

% 「それとも、迎えにこられるのが嫌だったか? 並んで歩くのが嫌なら後ろについてきてくれたらいいよ。俺少し前歩くから」
「还是说,不想我来接你?不想一起走的话那在后面跟着也行。我走前面」

% 「い、嫌とかは、言ってませんっ。あの……ありがとう、ございます」
「我,我没有,讨厌的意思。那个……谢谢你了」

% 「ん」\\
「嗯」\\

%  嫌がられてはいないようなので安心しつつポケットから手を出して差し出せば、おずおずと重ねられる。
看起来没有被讨厌,安心下来的周从口袋里掏出了手伸向真昼,真昼则慢慢地把手放了上去。

%  寒さからか、想定より随分とひんやりした感触が伝わってきた。\\
或许是天气太冷,手上传来的感触比预想的还要冰凉。\\

% 「つめたい。手袋どうした」
「手这么冷啊。手套呢」

% 「今日は洗っていたのです。周くんこそどうしたんですか」
「今天拿去洗掉了。倒是周君你的手怎么这么热啊」

% 「俺はポケットに手を突っ込んできたから」\\
「我是把手抄在口袋里了啦」\\

%  ポケットに入れてやってきたなんて良い子は真似しないでほしい方法でここまでやって来たので、あまり偉そうな事は言えない。\\
周是以两手抄着口袋这种不希望好孩子照做的姿势过来的,因而也没什么好称道的。\\

%  それ以上は何も言わず、ただ冷たくなった華奢な手を包み込むように握った。
除此以外也没有什么好说的,周便如同要包住般握紧了那只纤细的手。

%  真昼の手は、本当にか細くて、繊細で、頼りない。
真昼的手,真的是纤细而弱不禁风。

%  簡単に周の手で覆えてしまう。\\
轻易地就被周的手包了起来。\\

% 「……あたたかい」\\
「……好暖和」\\

%  小さく呟いて、真昼は笑うように瞳をへにゃりと細めた。\\
真昼轻轻感叹道,如同在微笑般眯细了双眼。\\

%  その無邪気な表情にどきりと心臓が跳ねたものの、表には出さずにただ握った手を意識するだけに済ませる。
那纯真的表情令周不禁心脏加速,但周还是把意识集中在握着的手上,没有把这悸动表现出来。

%  手を握ったついでに彼女が千歳とのお出かけで買ったらしい何かの袋と鞄をさりげなく真昼の手からするっと手に取り、そのまま歩き出した。\\
周把手握了上去,顺带便把似乎是装着她和千岁去买的包包袋袋接了过来,然后迈开了步子。\\

%  ちら、と見上げられたので「何だよ」と返す。
真昼突然抬头望向周,周回道「咋了啊」。

%  真昼はじーっと周を見ていたものの、やがてふいっと視線がそれた。
真昼盯了周一会,最后总算是移开了视线。

%  ほんのりと耳と頬が赤いのは、寒さのせいなのか、視線を合わせすぎたせいなのか。\\
那微微泛红的耳朵和脸颊,不知是寒冷,还是看太久了的羞耻感所致。\\

% 「ほら、帰るぞ。途中でコンビニ寄るか? 今の季節にくまんうまいぞ」
「好啦,回去了。要顺便去趟便利店么?这天气肉包子很好吃哦」

% 「……あんまんがいいです」
「……我喜欢豆沙馅的」

% 「甘いの好きだなお前。……晩ごはんどうする?」
「你还真喜欢甜食啊。……晚饭吃啥?」

% 「味付け卵とチャーシューとメンマ用意してますからラーメンです」
「备好了溏心蛋、叉烧和笋干那就吃拉面吧」

% 「寒い中ラーメンも乙なもんだなあ」
「大冷天吃拉面听着就很有胃口呢」

% 「そうですね」\\
「确实呢」\\

%  冷蔵庫を覗いていないので分からなかったが、どうやら用意していたらしい。\\
虽然周没在意过冰箱不大清楚,不过听上去这些都是以前就买好了的。\\

%  流石にスープと麺は市販品だろうが、具材はしっかりと手作りで、肉厚チャーシューとしっかり味の染みた半熟卵を想像すると、思わず喉が鳴る。
虽说汤料和面确实只能去买,但配菜都是精心手工制作的,仅仅是想象那厚厚的叉烧和入了味的溏心蛋,周便不住地咽起了口水。

%  きっと、冷えた体に染み渡るだろう。\\
那味道,一定能沁入这寒冷的身体吧。\\

% 「……あんまん食べた後に食べられますかね」
「……吃完豆沙包不知道还吃不吃得下呢」

% 「ならあんまん半分こしとくか。それなら入るだろ」
「那豆沙包就一人一半好了。这样就吃得下了吧」

% 「……はい」\\
「……嗯」\\

%  提案に淡いはにかみを返されたので、周も小さく笑って握った掌に少しだけ力を込めた。\\
真昼微带害羞地接受了周的提议,令周微微笑了起来,牵着真昼的手握得更紧了一点。\\

\vspace{2\baselineskip}

% 「椎名さん、また例の男と歩いてるの目撃されてるんだが」\\
「椎名同学她啊,好像又被看见和那个男人走在一起咯」\\

%  翌日、樹に噂も消えない内に燃料足してどうすんだ、といった眼差しを向けられて、周は知るかよとそっぽを向いた。
第二天,被树以「流言还没散呢你咋又添了一把火啊」这样的眼神看着的周,「你问我我问谁」地嘟哝着别开了头。