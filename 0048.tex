% 48 天使様のお迎え


%  真冬と言えるこの季節、それも日が差さなくなった夜は気温が低い。\\


%  防寒面と装飾性を考えてライトグレーのセーターにネイビーのピーコート、裏起毛の黒スキニーを合わせているが、それでも地味に寒いので制服にコートの真昼はどれだけ寒い事か。\\


%  真昼は冬場に厚手のタイツをはいているとはいえ、女子高生らしく校則違反や下品にならない程度の丈にしてあるスカートは、見ていて非常に寒そうなのだ。下にジャージを着せたくなる。


%  たまにすれ違う女子高生も無駄に短いスカートを揺らしているので、美に対する女子高生の努力は恐ろしい、と痛感した。\\


%  そんな事を考えながら、真昼からもらったマフラーに口許をうずめつつ足早に最寄り駅に向かう。\\


%  どうやら大型商業施設に出かけたらしく、電車を使ったようだ。最寄り駅から自宅は徒歩圏内であり、千歳情報ではもうすぐ電車が到着する筈なのでちょうどいい頃合いだろう。\\


%  歩けば風にセットした髪が軽く揺られるものの、崩れるまでは行かない。


%  ぐしゃぐしゃになったら流石に直さないといけないので億劫だ。日頃からおしゃれしている人間は尊敬ものである。\\


%  そんな事を考えながら黙々と歩くと、駅が見えてくる。


%  マンションの方向から考えてこの出入り口に姿を現す筈なので、出入り口付近で待っていれば確実に真昼と出会えるだろう。\\


%  出入り口の壁に背を預け、時間を確認しつつ真昼を待つと、ほどなくすれば見慣れた亜麻色のストレートヘアの少女が駅から出てきた。\\


% 「真昼」\\


%  声をかけると、聞き慣れた声だからか警戒なく振り返って――そして、周を視界に収めたであろう瞬間に固まった。\\


% 「え、……はい? な、なんで」\\


%  なんで、というのはこの格好の事だろう。


%  迎えに来る事自体は恐らく千歳から伝わっていただろうが、初詣以来の姿で来るとは想定してなかったらしい。\\


%  流石に、周も普段の適当な格好にいつもの髪形のままこようとは思わない。


%  もし周りに見られて謎の男と周をイコールで結ばれても困るし、真昼の隣を歩くならそれなりに格好をつけないと真昼まで軽んじられる。


%  変装目的ではあるが、真昼の隣に並べる程度にはやはり着飾るべきだろう。\\


% 「自分で出来ないと思ったか。流石に普段着で迎えに来る訳にいかないだろ」


% 「……そうですけど」


% 「似合ってないか? 一応鏡で確認したけど、変かな」\\


%  普通に無難な組み合わせで髪型は先日の初詣とそう変わらないものなので、おかしくはないと思っているが、美的センスに優れた人間からしてみたら駄目なのかもしれない。


%  たまにちらちらと視線を感じたのは、もしかしたらおかしかった、という可能性もある。\\


%  それなりに格好つけたけどダサかったのか、とちょっとショックを受けかけたが、真昼が慌てて首を振って「似合ってますっ」と肯定してくれたのでほっと息をつく。\\


% 「それならいいんだ。ほら、冬だしすぐ日が暮れるだろ。一人で帰るのは危ないし」


% 「……そ、それは分かってます、けど」


% 「それとも、迎えにこられるのが嫌だったか? 並んで歩くのが嫌なら後ろについてきてくれたらいいよ。俺少し前歩くから」


% 「い、嫌とかは、言ってませんっ。あの……ありがとう、ございます」


% 「ん」\\


%  嫌がられてはいないようなので安心しつつポケットから手を出して差し出せば、おずおずと重ねられる。


%  寒さからか、想定より随分とひんやりした感触が伝わってきた。\\


% 「つめたい。手袋どうした」


% 「今日は洗っていたのです。周くんこそどうしたんですか」


% 「俺はポケットに手を突っ込んできたから」\\


%  ポケットに入れてやってきたなんて良い子は真似しないでほしい方法でここまでやって来たので、あまり偉そうな事は言えない。\\


%  それ以上は何も言わず、ただ冷たくなった華奢な手を包み込むように握った。


%  真昼の手は、本当にか細くて、繊細で、頼りない。


%  簡単に周の手で覆えてしまう。\\


% 「……あたたかい」\\


%  小さく呟いて、真昼は笑うように瞳をへにゃりと細めた。\\


%  その無邪気な表情にどきりと心臓が跳ねたものの、表には出さずにただ握った手を意識するだけに済ませる。


%  手を握ったついでに彼女が千歳とのお出かけで買ったらしい何かの袋と鞄をさりげなく真昼の手からするっと手に取り、そのまま歩き出した。\\


%  ちら、と見上げられたので「何だよ」と返す。


%  真昼はじーっと周を見ていたものの、やがてふいっと視線がそれた。


%  ほんのりと耳と頬が赤いのは、寒さのせいなのか、視線を合わせすぎたせいなのか。\\


% 「ほら、帰るぞ。途中でコンビニ寄るか? 今の季節にくまんうまいぞ」


% 「……あんまんがいいです」


% 「甘いの好きだなお前。……晩ごはんどうする?」


% 「味付け卵とチャーシューとメンマ用意してますからラーメンです」


% 「寒い中ラーメンも乙なもんだなあ」


% 「そうですね」\\


%  冷蔵庫を覗いていないので分からなかったが、どうやら用意していたらしい。\\


%  流石にスープと麺は市販品だろうが、具材はしっかりと手作りで、肉厚チャーシューとしっかり味の染みた半熟卵を想像すると、思わず喉が鳴る。


%  きっと、冷えた体に染み渡るだろう。\\


% 「……あんまん食べた後に食べられますかね」


% 「ならあんまん半分こしとくか。それなら入るだろ」


% 「……はい」\\


%  提案に淡いはにかみを返されたので、周も小さく笑って握った掌に少しだけ力を込めた。\\


\vspace{2\baselineskip}

% 「椎名さん、また例の男と歩いてるの目撃されてるんだが」\\


%  翌日、樹に噂も消えない内に燃料足してどうすんだ、といった眼差しを向けられて、周は知るかよとそっぽを向いた。
