% 48 天使様のお迎え
\subsection{迎接天使大人}

%  真冬と言えるこの季節、それも日が差さなくなった夜は気温が低い。\\
现在是隆冬时节,没有阳光的晚上气温很低。\\

%  防寒面と装飾性を考えてライトグレーのセーターにネイビーのピーコート、裏起毛の黒スキニーを合わせているが、それでも地味に寒いので制服にコートの真昼はどれだけ寒い事か。\\
考虑到防寒和装饰性,周穿的是浅灰色的毛衣、藏青色的水手外衣、内层绒毛的黑色卡其裤,然而即使这样周依然觉得冷。穿着校服加外套的真昼该有多冷啊。\\

%  真昼は冬場に厚手のタイツをはいているとはいえ、女子高生らしく校則違反や下品にならない程度の丈にしてあるスカートは、見ていて非常に寒そうなのだ。下にジャージを着せたくなる。
尽管真昼在冬天会穿厚实的紧身裤,然而那显出女高中生范儿的,高度勉强不至于违反校规或者不雅的裙子,看着就让周觉得非常冷,想让她下面穿上运动裤。

%  たまにすれ違う女子高生も無駄に短いスカートを揺らしているので、美に対する女子高生の努力は恐ろしい、と痛感した。\\
偶尔擦肩而过的女高中生也摇晃着短得没意义的裙子。周痛切体会到,女高中生对于美的努力有多么可怕。\\

%  そんな事を考えながら、真昼からもらったマフラーに口許をうずめつつ足早に最寄り駅に向かう。\\
想着这些事情,周用真昼送的围巾捂住嘴角,快步前往最近的车站。\\

%  どうやら大型商業施設に出かけたらしく、電車を使ったようだ。最寄り駅から自宅は徒歩圏内であり、千歳情報ではもうすぐ電車が到着する筈なのでちょうどいい頃合いだろう。\\
似乎真昼是去了大型商业设施,路上乘坐了电车。最近的车站是自家能步行走到的距离,听千岁说电车马上就要到了,现在时候差不多正好吧。\\

%  歩けば風にセットした髪が軽く揺られるものの、崩れるまでは行かない。
行走的时候,风轻轻吹动了周弄好发型的头发,但并不至于把头发吹乱。

%  ぐしゃぐしゃになったら流石に直さないといけないので億劫だ。日頃からおしゃれしている人間は尊敬ものである。\\
如果头发乱成一团就不得不修了,很是麻烦。平日里就打扮的人类真是值得尊敬。\\

%  そんな事を考えながら黙々と歩くと、駅が見えてくる。
周想着这些事情默默走着,接着就看到了车站。

%  マンションの方向から考えてこの出入り口に姿を現す筈なので、出入り口付近で待っていれば確実に真昼と出会えるだろう。\\
考虑到公寓的方向,真昼应该会在这个口子出来。在车站口附近等着的话,应该就能确保遇上真昼了吧。\\

%  出入り口の壁に背を預け、時間を確認しつつ真昼を待つと、ほどなくすれば見慣れた亜麻色のストレートヘアの少女が駅から出てきた。\\
周靠在车站口的墙上,一边看了看时间一边等着真昼。没过多久,车站里就走出了眼熟的亚麻色直发少女。\\

% 「真昼」\\
「真昼」\\

%  声をかけると、聞き慣れた声だからか警戒なく振り返って――そして、周を視界に収めたであろう瞬間に固まった。\\
周一搭话,真昼听到熟悉的声音就没有警戒地回过头——接着,视野里看到周的瞬间就愣住了。\\

% 「え、……はい? な、なんで」\\
「诶……嗯?为,为什么」\\

%  なんで、というのはこの格好の事だろう。
为什么,指的应该是这身打扮的事情吧。

%  迎えに来る事自体は恐らく千歳から伝わっていただろうが、初詣以来の姿で来るとは想定してなかったらしい。\\
周会来迎接这件事恐怕千岁已经告知她了,但她似乎没想到周会以新年参拜那时的样子过来。\\

%  流石に、周も普段の適当な格好にいつもの髪形のままこようとは思わない。
再怎么说,周也不会想用平时随意的样子和发型就直接过来。

%  もし周りに見られて謎の男と周をイコールで結ばれても困るし、真昼の隣を歩くならそれなりに格好をつけないと真昼まで軽んじられる。
如果被周围人看到,将谜之男人和周等同起来的话,周会很困扰。而且,要在真昼旁边走路的话,一点样子都没有的话连真昼都会被轻视的。

%  変装目的ではあるが、真昼の隣に並べる程度にはやはり着飾るべきだろう。\\
尽管周的目的是变装,但要想达到

% 「自分で出来ないと思ったか。流石に普段着で迎えに来る訳にいかないだろ」


% 「……そうですけど」


% 「似合ってないか? 一応鏡で確認したけど、変かな」\\


%  普通に無難な組み合わせで髪型は先日の初詣とそう変わらないものなので、おかしくはないと思っているが、美的センスに優れた人間からしてみたら駄目なのかもしれない。


%  たまにちらちらと視線を感じたのは、もしかしたらおかしかった、という可能性もある。\\


%  それなりに格好つけたけどダサかったのか、とちょっとショックを受けかけたが、真昼が慌てて首を振って「似合ってますっ」と肯定してくれたのでほっと息をつく。\\


% 「それならいいんだ。ほら、冬だしすぐ日が暮れるだろ。一人で帰るのは危ないし」


% 「……そ、それは分かってます、けど」


% 「それとも、迎えにこられるのが嫌だったか? 並んで歩くのが嫌なら後ろについてきてくれたらいいよ。俺少し前歩くから」


% 「い、嫌とかは、言ってませんっ。あの……ありがとう、ございます」


% 「ん」\\


%  嫌がられてはいないようなので安心しつつポケットから手を出して差し出せば、おずおずと重ねられる。


%  寒さからか、想定より随分とひんやりした感触が伝わってきた。\\


% 「つめたい。手袋どうした」


% 「今日は洗っていたのです。周くんこそどうしたんですか」


% 「俺はポケットに手を突っ込んできたから」\\


%  ポケットに入れてやってきたなんて良い子は真似しないでほしい方法でここまでやって来たので、あまり偉そうな事は言えない。\\


%  それ以上は何も言わず、ただ冷たくなった華奢な手を包み込むように握った。


%  真昼の手は、本当にか細くて、繊細で、頼りない。


%  簡単に周の手で覆えてしまう。\\


% 「……あたたかい」\\


%  小さく呟いて、真昼は笑うように瞳をへにゃりと細めた。\\


%  その無邪気な表情にどきりと心臓が跳ねたものの、表には出さずにただ握った手を意識するだけに済ませる。


%  手を握ったついでに彼女が千歳とのお出かけで買ったらしい何かの袋と鞄をさりげなく真昼の手からするっと手に取り、そのまま歩き出した。\\


%  ちら、と見上げられたので「何だよ」と返す。


%  真昼はじーっと周を見ていたものの、やがてふいっと視線がそれた。


%  ほんのりと耳と頬が赤いのは、寒さのせいなのか、視線を合わせすぎたせいなのか。\\


% 「ほら、帰るぞ。途中でコンビニ寄るか? 今の季節にくまんうまいぞ」


% 「……あんまんがいいです」


% 「甘いの好きだなお前。……晩ごはんどうする?」


% 「味付け卵とチャーシューとメンマ用意してますからラーメンです」


% 「寒い中ラーメンも乙なもんだなあ」


% 「そうですね」\\


%  冷蔵庫を覗いていないので分からなかったが、どうやら用意していたらしい。\\


%  流石にスープと麺は市販品だろうが、具材はしっかりと手作りで、肉厚チャーシューとしっかり味の染みた半熟卵を想像すると、思わず喉が鳴る。


%  きっと、冷えた体に染み渡るだろう。\\


% 「……あんまん食べた後に食べられますかね」


% 「ならあんまん半分こしとくか。それなら入るだろ」


% 「……はい」\\


%  提案に淡いはにかみを返されたので、周も小さく笑って握った掌に少しだけ力を込めた。\\


\vspace{2\baselineskip}

% 「椎名さん、また例の男と歩いてるの目撃されてるんだが」\\


%  翌日、樹に噂も消えない内に燃料足してどうすんだ、といった眼差しを向けられて、周は知るかよとそっぽを向いた。
