\subsection{忙碌的排班}

一如所料,从开店起,始终有顾客——大部分是学生——来到周的班级。\\

「天使大人效应真可怕」\\

发出这声自言自语的,是安排在一起待客的同班男生山崎。

开始才几十分钟,就已经座无虚席,这在学生的活动里并不多见。他恐怕是对此感到震撼,或者准确来说,可能是对顾客的热情感到震撼吧。\\

在餐饮店里,实在不容易一口气迎接这么多客人,所以对进店的顾客数也有所限制。然而,这般盛况,让人发怵也是没办法的。\\

每当真昼走过通道,男性的视线都会被吸引过去,周觉得又无奈、又佩服、又不快,脸都差点塌下来了。\\

尽管周早就料想到这种事情,死了心,但让人不开心的事情终究是让人不开心的。不过,从真昼的角度来看这话对周也适用,这件事上应该算是彼此彼此。\\

「也算是预料之中吧。不说这个,有客人来了」\\

周告诫了一声山崎,把进店的新顾客带领到座位上。\\

一般来说,顾客是由空闲的工作人员去接待的,但有些顾客会想去指名某个人去接待他,让人很为难。店里不提供这种服务,如有需要,真希望这些人去专门的店里。\\

他刚刚接待的女学生大概是奔着门胁去的,但门胁正在接待其他客人,只能抱歉她用周凑合了。\\

「请您在这里就坐」\\

周拉开椅子,露出木户传授的微笑,接着,原本因为不是门胁接待而略有遗憾的女学生好像吓了一跳似的看向周这边。

「果然是奔着别人去的,真是抱歉了啊」怀着这样的心情,周告诉她放行李的篮子的位置,然后把菜单摆在她面前。\\

「本店今天推荐的菜单是这里的A套餐,您看怎么样」

「那、那就来一份……」\\

顺带一提,尽管说是推荐,但菜单实际上只有ABC三类烤点心和饮料的组合。因为不想让人只买饮料赖在店里,所以是成套出售的。

门口的接待人员已经提醒过,如有过敏反应需要提前告知,应该不会发生问题。\\

女学生带着点犹豫下单后,周郑重行礼说道「好的,请稍等,我们马上上餐」,然后前去通知后台的人员。\\

「一份A,点单越积越多了,你们加油」\\

在后台,有些同学在把点心盛进盘子,有些同学在来往于烹饪室的一块地方和教室之间;有个正好空闲着的同学缓缓抬起头。\\

「哦……我看了看门口,太夸张了」

「你可别死了啊」

「一开始就准备好了不少东西,应付是能应付过去」

「但是?」

「……看那样子,你们一会儿会很辛苦吧」

「是吗?不过门胁那么抢手,之后应该会更忙就是」

「我不是说这个啦」\\

对方叹了口气,但没有讲出详情。周摸不着头脑,但这应该也不至于引发太大的麻烦吧。

周摆出「莫名其妙」的眼神,对方反倒可怜起他了。\\

「……还有,从刚刚开始啊,椎名每次到这里来,表情都有点不满」

「又怎么了」

「我觉得是你的原因吧」

「我要待客的,实在没办法吧」

「也有这个因素,但大概不是这么回事啦」

「你到底想表达什么啊,我搞不懂」\\

周感觉自己受到了委婉的责备,但他没能弄清楚,皱起了眉头。

大概是说真昼在吃醋,然而根据对方的说法,听上去就好像真昼还因为别的什么事情在闹着别扭。\\

周决心过后问问真昼,并适时结束了对话,把准备好的菜品端上了桌子。
