\subsection{天使大人成绩也很完美}

「周~考得怎么样?」\\

期末考试终于结束,总算熬过了考试地狱的学生们,比平常更加兴奋地在教室里聚成了几团。

周和树也是一样因为考试结束而松了一口气,评判着自己这次的发挥。\\

「嗯?一般吧。差不多还行」\\

听到树的发问,周虽然做出了回答,但其实并没有什么可说的。题目都在考试范围内,只要平时做好复习的话这场考试并不算难。

这次写起题来的手感跟以前并没有什么不同,所以周也没有什么特别的感想。\\

周虽然是个怕麻烦的人,但复习还是基本不会落下。

上课学的内容他大致都懂了,考试也发挥正常。虽然满分还是有些难,但考个八九十分还是没问题的。\\

「然后你年级前三十稳了是吧……你个学霸」

「靠平时习惯啦」

「就你那平时习惯你还有脸吹?」

「再怎么样也轮不到你这个天天秀恩爱不读书的家伙讲」\\

论起造成周和树成绩差距的原因,与其说是头脑的差距,还不如说是树在女朋友千岁身上花了太多时间的关系。

树脑袋也不笨,要是认真起来的话应该也能拿个挺不错的名次。只可惜树把时间都优先花在了千岁身上,结果成绩就比不上周了。\\

「……女朋友可是个好东西哦?」

「对对对对」

「我说啊周,你也去找个咯」

「想有就能有那这世上男儿们也不会流下血泪了啊」\\

这世上想要女朋友而求之不得的人比比皆是,对某些人来说树这句无心之语听上去想必是十分扎心。\\

不过周倒是并没打算对树生气,说到底周现在根本也没有想要个恋人的欲望,于是只管听过便算了。\\

「再说,女朋友咋找啊」

「来个双重约会——」

「然后我和那个幻想的女朋友就会被你俩秀到闪瞎吧」

「那你们也秀啊!」

「你觉得我这性格能干出那种事吗」

「……看样子不行」

「嗯哼」\\

周也对自己这淡泊的性格有所认识。

周的性格怕麻烦而且说话直来直去,有些人可能会觉得冷淡,因而给人的印象不算很好。这种性格根本没法找到女朋友。\\

万一真的有了女朋友,关系想必也会很平淡,至少不可能像树那样大庭广众狂撒狗粮。\\

「不是我说,周你至少该找个喜欢的人咯。话说啊,周你要是剪掉点前发,弄清爽点,整整发型,背挺直了,女生们绝对会刮目相看的」\\%我改的地方语气怪怪的

周自认为对自己有正确的评价,即便达不到门胁那种帅哥等级或是树那种稍显轻薄的端整外表,周也觉得自己的外表绝对谈不上丑。

要是周好好打理打理自己的仪表和形象的话,也是有不输同龄高中男生的水准的。\\

不过,周即使好好打扮,他也没有能耐对接近他的人和颜悦色。\\ 

「光凭外表就来套近乎的可都不是什么好货色哦」

「说是这么说,可要是对方对你没兴趣,你也没法了解对方的性格吧?」

「……就算是那样,我现在也没找女朋友的心思」\\

就算找到了女朋友,看见周平常的样子肯定也会幻想破灭吧。

周这人生活不能自立,日子过得邋遢,而且对人还不友好。「不如说要是有女孩子对自己感兴趣我倒还真想看看」甚至周自己也如此苦笑道。\\%苦笑道感觉怪怪的,苦笑著想?

毕竟周那不适合和人交往的性格也觉得与人相处麻烦,因而并没有想要女朋友的想法。\\

而且,现在真昼在自家做着晚饭,万一交了女朋友说不定会酿成惨剧。虽然周完全没有找女朋友的打算,所以并不会对此感到不安,但是单从这个理由上来说,周也不会想去找一个。\\

周心目中的优先级是真昼的料理>还没找着的女朋友,而且这个优先级恐怕没法轻易改变吧。\\

「真是个没欲求的家伙……要不让小千给你介绍几个朋友也行哦?」

「你可别瞎操这闲心。千岁她朋友都是群吵闹的家伙吧,光是当朋友怕就够让我头疼的了」

「毕竟周你是个阴暗角色嘛」

「是啦咋地」

「嘛,你要这么说那暂且就算了吧。不过啊,美妙的高中生活,连女朋友都没有,一个人空虚度日,不难受么?」

「不需要,而且感觉很麻烦」\\

虽然周并没有「你把学校生活当什么了」这种较真的思考,但反正女朋友这东西不是非要不可,所以周也没有想着去找一个。

再说了,喜欢的人既不那么好找,也不容易产生结果。\\

「……可惜了啊」

「是是是」

「不过啊,周你要是有了喜欢的人一定会变的哦?」

「你哪来的自信啊」

「就是你这样的家伙,宠起女朋友才会不要不要的」

「好好好你说啥都对」\\

周不但认为自己绝无可能变成那种甜的发腻的人,也想像不出自己变成那样的情况,于是把树的话当成耳旁风就这么吹过去了。\\

树一脸无奈地看着周……接着,他忽然移开了视线,表情也舒缓了下来。\\

「\ruby{阿树~}{\jpfont いっくーん},回家吧?」

「哦,小千啊」\\

正好,树的女朋友千岁过来了,两人似乎是约好了一起回家。刚刚周和树聊了这么久,都是在陪他等着千岁。\\

周回过头,便看见一位一头亮茶色短发,带着男孩子气的少女,正满脸笑容地朝着这边——准确来说,是朝着树招手。

那活泼的气氛和明快的笑容,甚至让看着的周感到有些耀眼。她的性格也正如外表,为人友善、活泼明快,好也好坏也好,她都负责着炒热气氛,是个与真昼风格不同的美少女。\\

她跑到这边来之后,露出了笑嘻嘻的表情。

周希望她能就那样别说话,因为,千岁一说起来基本上周都会被欺负。\\

「小千你说是不,周这样的家伙,肯定会宠女朋友的」

「别多嘴」

「唉?什么?周有女朋友!?」

「有个毛啊」

「哎,什么嘛。有的话我还想打好关系呢~」\\

「切」的一声,千岁瘪着嘴一脸失望。\\

「我那幻想的女朋友要遭一波你那美其名曰打好关系的过分身体接触的罪还真挺可怜」

「唉,原来你有虚拟女友吗?」

「我是说假如有的话好吧!?」

「玩笑啦玩笑」

「应付你可真够累人的……」

「只是周你体力不足吧」

「是体力连着精神力全被你消耗掉了啦……」\\

比起体力,感觉累的还是精神。\\

本来周平常过着的就是除了熟悉的人以外基本上不说话不起眼没精神的学生生活,要被迫跟千岁这种全天精神高涨的生物对话,实在是艰难。\\%不说话跟不起眼接在一起中文看感觉有些奇怪?

即便周的回应有些刻薄,千岁也毫不在意,对着一脸疲劳的周说着「真是不像样呢」,十分愉快地笑着。\\

树也同样笑着给出了「你赶紧习惯啦」这样随意的建议,因此周除了累得叹口长气以外毫无办法。\\

\vspace{2\baselineskip}

「……在干什么呢?」\\

周回到家吃完真昼亲手做的晚饭之后,洗碗回来就看到真昼在客厅摊开了试卷。\\

洗碗这事是轮班,但周为了不给真昼添负担抢先去洗了,因而这段时间真昼便在客厅里待着。她说是因为如果就这样把事情全部扔给周自己回去,会有些过意不去。\\

「给卷子算分」

「嗯,看得出来」\\

大概是在检查答案,真昼似乎正对着课本确认有没有写错。\\

「话说结果怎么样」

「如果答题纸上我没有写错的话就是满分了呢」

「只能说不愧是你啊」\\

真昼满分的回答太过平淡,让周也没有什么太大的反应。\\

毕竟已经好多次在月考排名上看见真昼那雷打不动的年级第一,周也不吃惊了。\\

本来周就觉得真昼的话说不定能做到,因此他听见满分也只有果不其然之类的想法。\\

「学习我不讨厌啊。再说我已经提前一年把要学的东西全部学过一遍,所以只要复习就足够了」

「呜哇,太可怕了。不愧是学神……」%看你要不要改啰

「藤宫你学习不也挺上心的么」

「你还知道我成绩啊」

「名次能上榜的话,我都有点印象」\\

看来在搭话之前她就已经在一定程度上知道周这么一个人了。

本以为排不到个位数的人根本就进不了她的眼,不过真昼却不假思索地说出了周上次的排名,看来她还挺关心成绩表的。\\

周会花上一定的功夫学习,其原因,并不是因为学习是学生的本分……这种较真的脑回路,而只不过是家里给出的条件罢了。\\

「毕竟是让我独居的条件嘛,保持成绩这事」\\

家里同意周一个人住的时候,提出了要保证成绩不下滑的要求。

另外还有半年回家一次这个条件,不过关于这一条在放长假时回一趟就行,所以基本上只要保持住成绩家里就不会多指手画脚。\\

「我的成绩也就保持不会造成自己的麻烦的程度而已,比不上你。你是超努力的吧」

「……我的话,是因为不努力不行」\\

真昼轻声嘟哝了一句,低下了头。

虽然她的表情被前发遮住而看不太清,但肯定不怎么开心吧。\\

不过,真昼很快便抬起头恢复了平常的表情,所以周就错过了指出这事的机会。

就算有机会,周也不会去问吧。毕竟那氛围,就像是在忍耐着痛苦一般。\\%把机会换成来得及来不及会好点吗?

时不时地,真昼就会露出这样的表情。

虽然真昼从来不会说自己正因为什么而感到痛苦和厌恶,但她给人的印象便是被一些事物所束缚,挣扎于其中的样子。\\

不难想象,变成这样的原因是家庭环境。\\

因此,周来插嘴干预是不合适的。

周十分明白这是自己这个局外人不应踏入的区域,因而一直保持着作为邻居的适度距离感。\\

周同样有不想被他人提及的东西。\\

他也常常切身体会到,干涉他人私事十分失礼,反而装做浑然不知别人会比较感激。\\%这句感觉有点不好改

真昼隐藏起刚才的情绪,以平日里清爽的声音说道「我差不多也该告辞了」,接着开始把课本和试卷收进包里。\\

周也不打算挽留,「噢」地简单应着,望向收拾着东西的真昼。

正当真昼把拿出来的东西全部收拾好,从座位上站起来的时候,周突然注意到,在空杯子的阴影处,放着一件不属于周的东西。\\

周伸手拿起来,发现这是每个学生都有的装着学生证的塑料套。

估计是她连着课本一起拿了出来,整理的时候却忘记了吧。\\

周看着这印着正面照、姓名、学号、出生日期和血型这些简单信息的学生证,喊住了正在门口穿着鞋子打算回去的真昼。\\

「落下了哦」

「啊,抱歉让你特意送过来。那么,晚安」

「晚安」\\

真昼礼貌地弯腰行礼之后离开了周的屋子。周目送着她,轻轻地叹了一口气。\\

回忆起刚才看见的学生证上写着的出生年月日——特别是月和日的部分,周扶住了额。\\

「……这不就在四天后嘛」\\

要是周没看到学生证的话,恐怕他永远都不会知道真昼的生日。想着要是早些知道的话就好了,周再次深深地叹了一口气。
