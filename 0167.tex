% \subsection{167 天使様と食いしん坊}
\subsection{天使大人与贪吃鬼}

% 「ふいー、食べた食べたー」
「呼~,吃饱啦吃饱啦~」

% 「どこにあの量が入ったんだ……」\\
「这么大的量都给你装哪去了……」\\

% 屋台を粗方回り終え、千歳はお腹をさすりながら満足そうに頬を緩めていた。
摊子差不多都逛完了一遍后,千岁一遍摸着肚子一边露出了一脸满足的放松神情。

% 腹部は屋台を回る前よりやや膨らんでいるように見えるがそれでも細く、よくあの量が入ったなと感心すればいいのか呆れればいいかが悩み所だった。\\
尽管那肚子比逛摊子之前稍稍鼓起了那么一些,但依旧很苗条。居然能塞进这么多东西,真是让人不知是该感叹还是该傻眼了。\\

% 「んふー、こういうお祭りのご飯は格別ですなあ」
「嗯呼~,这样祭典上的饭还真是别有风味呢」

% 「まあお前が満足してるならいいんだけどさ……食べ過ぎには気を付けろよ」
「嘛虽然要是你满足了那也没差……小心别吃太多了哦」

% 「普段はこんなに食べませーん。ちゃんと調整してますー」\\
「平常不会吃这么多的啦~。有好好注意量啦~」\\

% スレンダーな体型を維持している千歳の言う事なので信じるしかないが、それにしても食べすぎな気がしなくもない。ただ本人は納得しているようなので、周がとやかくいう事でもないだろう。\\
毕竟千岁也正保持着纤细的体型,她这么说倒也不得不信,但周还是不禁担心她是不是吃太多了。不过千岁本人似乎是觉得没什么大问题,周便也不再多嘴,就此作罢。\\

% 「そういう周は足りるの? 私からすれば全然食べてないけど」
「说起来周你吃饱了没?我好像没怎么看见你吃东西耶」

% 「ん……俺は家でちょっと食べるつもりだったし。真昼が出汁冷やしてるからレトルトの飯で冷やし出汁茶漬けでもしようかと」
「嗯……我打算回家里再吃点东西。真昼正凉着高汤,我打算等回去之后倒在速食米饭做泡饭吃」

% 「なにそれ美味しそう」
「听上去挺好吃耶」

% 「まだ食う気力あるのかよ……」\\
「还能塞啊你肚子……」\\

% 屋台の品もいいが一日の〆は真昼の料理がよかったので家で真昼の作り置きの出汁を使って茶漬けにしようと思ってあまり食べずにいたのだが、まさか千歳がまだ食欲を余らせているとは思わなかった。\\
虽说摊上的吃的也不错,但周还是想以真昼做的饭作为一天的收尾,便打算着回去用真昼做好了放着的高汤泡点饭来吃,便没有吃多少东西,但周实在没有料到千岁居然还有食欲。\\

% 千歳の食欲に苦笑している真昼は「また今度にしてください」と窘めている。今日見ただけでも焼きそばや唐揚げ、フランクフルトに真昼の買ったたこ焼き一粒やチョコバナナ、かき氷と男子でもお腹が満たされる程度に食べているので、胃の心配をしているのだろう。\\
对千岁的食欲,真昼只得苦笑说「你还是下次再吃吧」让千岁打消这一念头。今天光是在眼皮底下,千岁就已经吃了炒面,炸鸡块,法兰克福香肠,一颗真昼买来的章鱼烧,巧克力香蕉再加上刨冰这就算是男生,也该饱了的量的食物,实在是让人担心她的胃。\\

% どこに入ったんだろうか、と細い腰を見ながら考えていたら、視線に気付いたらしい千歳が「やんえっち」と体をくねらせたので、白けた眼差しを返しておいた。\\
周一边望着千岁那柳腰,一边想着东西到底装哪去了。注意到周目光的千岁喊着「呀~好色」扭着身子,被周白了一眼。\\

% 「まあ千歳の胃の容量は今後要観察でいいとして」
「嘛总之千岁的胃容量之后再观察现在先放一边」

% 「わおつれない」
「哇哦好冷淡」

% 「どうする? もう帰るか?」\\
「你们什么打算?差不多回去了?」\\

% ある程度遊び回ったし、夏場で日が暮れるのが遅いとはいえ既に空は闇色。もうすぐ二十時半になるので、家のある区域から離れた周と真昼の移動時間も考えてそろそろ解散にするのが無難だろう。
已经差不多都逛了一遍了,即便夏天天黑的晚,现在天色也已经暗下来了。时间差不多已经晚上八点半,考虑到路上的时间,家里离得远的周和真昼是该考虑回家了。

% 千歳も樹が居るとはいえあまり遅い時間に出歩かせるのもよくない。\\
尽管有千岁和树在,但太晚了在外面走总归还是不好。\\

% 「んー、帰るのは構わないけど私まひるんち泊まるよ?」
「嗯~,回是无所谓啦,不过我今晚在昼儿家睡哦?」

% 「は?」
「啥?」

% 「事前にまひるんちに荷物運んでおいたしー、ちゃんとこっちは前から許可取ってたよ?」\\
「我可早就把行李放在昼儿家里了~,也好好提前说好了哦?」\\

% ねー、と真昼に笑いかけた千歳に、真昼も苦笑しつつ頷く。
千岁满脸笑容像真昼示意,真昼则苦笑着点了点头。

% ちなみに嫌がるような表情ではないので、周も心配はしていないが、出来れば先に言ってほしいところでいる。周が食材の買い出しをするので、三人なら三人分の食材の用意が必要なのだ。\\
顺便因为真昼脸上并没有露出不悦,所以周也不担心,但还是想事前先跟自己说一声。毕竟负责出门买菜的是周,三个人的话就得去买三人份的食材。\\

% にまーっと笑う千歳に「オレも周に頼んどけばよかった」と樹は惜しそうにしている。彼だけ一人で帰るのは可哀想だと思いつつ、着替えがないのでどうしようもない。\\
看着一脸嘚瑟的千岁,树一脸惋惜地说「早知道我也去拜托周了」。虽然树一个人回去很是可怜,但没带换洗衣物确实没得办法。\\

% 「……まあ、真昼がいいってんならいいけど」
「……嘛,反正只有真昼同意的话就行」

% 「おやぁ周くん、まひるんを取られてご機嫌ななめー?」
「哦呀周,昼儿被抢走了心情不快~?」

% 「女に妬いてどうするんだよ。真昼は俺のものって分かってるから別にいい」\\
「嫉妒个女人是什么鬼哦。反正记着真昼是我的,剩下的随便」\\

% どちらかといえば真昼が千歳にべたべたされるのが嫌というより、同性だから気軽に家に出入り出来るのが羨ましいといった感じだ。
要论的话,周想着的不是不喜欢真昼被千岁上下其手,反倒是羡慕同性间可以随意进对方家门。

% 今度真昼の家にあがらせてもらう約束はしているが、こちらにも覚悟が要るのであっさりと入れる千歳が羨ましかった。\\
虽说约好了什么时候去真昼家里一趟,但周要去是得抱着觉悟的,因而周心里有些羡慕能够自如进门的千岁。\\

% なので今更千歳に妬きはしない、と肩を竦めた周だったが、真昼が頬を赤らめて千歳の方にすすすと逃げていく。\\
因此,周早已不再羡慕的千岁,淡然地耸了耸肩。可真昼这时则红起了脸,簌簌地逃到千岁身边。\\

% 「……千歳さん、これですよ。周くん最近こういう風になってきたんですよ……」
「……千岁,就是这样子耶。周君最近变成这个样子了耶……」

% 「あやー、これはまひるんも大変ですなあ」
「哎呀~,看来昼儿你也不好过呀」

% 「なんだよその顔」
「什么啊你那表情」

% 「別にー?」\\
「怎么啦~?」\\

% ねーまひるん、と先程真昼に同意を求めたのとはまた別の悪戯っぽい千歳の笑みに、真昼はこくこくと無言で頷いて千歳にくっついて恥ずかしそうにこちらを窺うのだった。
千岁一边一脸与刚才征求真昼同意时风格不同的坏笑,一边在真昼耳边叨叨着什么,而真昼则是一边听着一边默默点头,然后贴着千岁一脸害羞地偷看着这边。
