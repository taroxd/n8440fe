% 63 天使様の異変


%  春休みと言うのは、特にこれといった趣味を持ち合わせていない人間にとってはかなり暇な期間である。


%  別に周も趣味がない訳ではないのだが、本を読んだり散歩に行ったりというもので、クラスメイトには渋い趣味だなと苦笑された事がある。\\


%  そういった趣味のため、アウトドアに出かけたり何かレジャー施設に行ったりという事を進んではしない。誘われない限り出かけてもランニングか散歩、食材の買い出しといった事くらいなのだ。\\


%  樹には高校生なのに青春謳歌しなくてもいいのかと呆れられたが、別にある程度健康に気を使って運動しているのだからいいだろうと思っていた。\\


%  真昼も基本的にはあまりどこかに出かける様子はない。


%  もちろん運動しているのは見かけるし、必要なものを買い出しているのは見るが、どこか遊びに行くというのはあまりなかった。\\


% 「どこか遊びに行きたいとかないのか?」\\


%  自分も人の事が言えないが、華の女子高生がそれでいいのだろうか……と夕食後の真昼に聞いてみると、しばし悩んだあと苦笑される。\\


% 「遊びに行きたい……とかは、今のところないですね。私インドア派ですし」


% 「まあ俺もそうなんだよなあ。別に出掛けたところでって気分だ」


% 「……志保子さん達の所に帰ったりとかは?」


% 「正月会ったしいいだろ、と。夏には帰るし。あと、真昼の料理食えないってのは味気ないからな」


% 「……そ、そうですか」\\


%  最早真昼の料理を食べないとしっくりこないくらいには馴染んでいるし、毎日食べたいという気持ちの方が強い。何だかんだ真昼が隣に居る事にも慣れてきて居るのが当たり前になってしまった、というのもある。\\


%  やはり可愛らしさやいじらしさ、健気さに意識してしまう事は多々あるが、側に居て落ち着くのだ。真昼の醸し出す空気が周の性にあっていた、という事だろう。\\


% 「ま、帰ったとしてもどこかしら連れ回されて疲れそうだしなあ」


% 「……連れ回される?」


% 「行楽地とかショッピングとか。予定がなかったらどこかしら連れていかれる。中学時代は冬期休暇に温泉旅行とかもあったかな」\\


%  志保子はインドア派でもありアウトドア派でもある、というより全部に精力的で何でも楽しくこなすタイプだ。


%  それに家族との時間を大切にする人間でもあり、先約があったり周が嫌がったりしない限りはどこかに連れていこうとする。選ばせてくれるのは良心的だが、承諾してしまえば振り回される。\\


%  遊園地やショッピングモールなどは可愛いものだが、沢下りやらサバゲーやら、ものはチャレンジだと同伴で参加させているので、大変だった。あの細い体のどこにあんな力が宿ってるのか不思議でならなかった。


%  お陰で色々と学べたり体もそれなりに鍛えられたりはしたのだが、その反動で自分でする分には大人しい趣味になったのは否めない。\\


% 「……楽しそうですね」


% 「それが連日になると疲れるぞ。あのテンションに付き合わされて疲弊して新学期を迎えるんだ」


% 「ふふ、想像できます」


% 「……お前もうちに来たら分かるぞ。むしろお前が居たら関心がお前に行く」


% 「そ、それはまあ……」\\


%  仮に真昼が来たのなら、喜んで彼女と出かけるだろう。


%  流石に危ないような事はさせないだろうが、確実に買い物やレジャー施設に連れ回す。娘が欲しかったらしい母親は、年頃の女の子、それも真昼が滞在するなら嬉々として構う筈だ。\\


% 「夏にでも来てみれば分かるから。多分めちゃくちゃ連れ回されたり着せ替え人形にされる」


% 「……夏」


% 「どうせ真昼連れてこいって言われそうだしなあ」\\


%  夏期休暇の時は恐らく志保子直々に誘いが彼女の元に届くのではないだろうか。\\


% 「……あ、嫌なら全然断ってくれていいぞ」


% 「い、いやだなんて! 嬉しいです」\\


%  ぶんぶん、と首を振っているので、髪が波打ちシャンプーの香りが鼻をくすぐる。\\


% 「ん。まあ母さんに聞いとくよ、一応。多分喜んで迎え入れるけど」


% 「……ありがとうございます」


% 「むしろ被害分散でこっちがお礼言いたいくらいだよ」


% 「もう」\\


%  ぺし、と二の腕の辺りを掌で軽く叩かれた。


%  もちろん痛みなど全くなく、押された程度のものだったが、少し心臓に悪い。


%  小さなスキンシップを真昼から取ってくるようになった事についどきどきしてしまう。\\


% 「……周くん?」


% 「い、いや、別に、なんでも」


% 「何でもという割には視線泳いでますけど……」


% 「なんでもない。ああほら、スマホ何か受信してるぞ」\\


%  動揺した事を気取られたくなくて、話をそらすためにも震えて通知のランプが光っているスマホを示す。


%  それに思考が切り替わったのか「なんでしょう」と不思議そうにスマホを手にとってアプリを開いた。\\


%  流石に内容を読むのは失礼というのと、今はあまり目を合わせたくないというのがあり、目をそらしていたが……ぽす、と音がして、視線が真昼に戻ってしまう。\\


%  どうかしたのかと真昼の顔を見て、それから固まった。\\


%  真昼は、スマホを膝の上に置いたクッションに落として、泣きそうな、迷子のような表情になっていた。


%  目に涙が溜まっているとか口許がゆがんでいるとか、そんな訳ではないのに……触れたら壊れてしまいそうな、そんな印象を抱かせる。\\


%  この表情を見たのは、いつだっただろうか。


%  そう、初めて話した時の表情によく似ていて――。\\


% 「……真昼?」


% 「いえ、何でもありません。気にしないでください」\\


%  周が何事かと聞く前に、強張った声が返って来た。\\


% 「すみません、私そろそろ帰ります。明日は用事が出来たので、夕ご飯は無理そうです。ごめんなさい」\\


%  周に何か口を挟ませる隙もなく真昼は告げて、手早く荷物をまとめて去っていった。


%  手を伸ばしても、彼女はそれに気付かなかったのか、わざと無視したのか。伸ばした掌は、空気だけを掴む。\\


% (……何で、急に)\\


%  確実に、トリガーは届いたメッセージだろう。


%  真昼にあんな表情をさせるなんて、周の知る限り一つしかない。\\


% 「……真昼の、両親」\\


%  真昼はあまり人に連絡先を教えていないらしく、極限られた人間のみしかメッセージアプリのIDを知らない。


%  周や志保子、千歳に樹、口の堅いクラスの数人の女子までは聞いた事がある。それ以外で知っているとなると、親くらいなものなのではないだろうか。\\


%  親から連絡が来たのだとしたら。


%  昨日までは何も言っていなかったのに、急に用事が出来たと言って逃げたのは、もしかしたら親と会うからではなかろうか。


%  両親と確執があるのは知っているからこそ、あんな表情になったのは両親に原因があるのではないかと推測出来る。\\


%  推測出来たところで、何も出来ないが。\\


% 「……真昼」\\


%  去り際に、くしゃりと顔を歪めたのが見えていた。見えていたのに、何も言ってあげられなかった。\\


%  どうしようも出来ずに、小さく今はここにいない少女の名を呼んで、先程まで彼女の膝に載せられていたクッションに拳を落とした。

