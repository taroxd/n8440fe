\subsection{相似的是}

停留了一小时的志保子称明天还有工作,便优雅地匆匆离去了,原本热闹的空间一下子恢复了宁静。\\

周一边为回到了以前安宁祥和的氛围而安心,却又同时由于志保子在的时候真昼很开心,想让志保子再多待会儿。

只不过,她常常会说出狠狠磨削周的精神的发言,感觉早点让她退场也是正确的。这属于志保子的问题,若是她不捉弄人,周甚至都希望她跟在真昼身边了。\\

「嘻嘻。志保子也生龙活虎的,真好啊」\\

真昼悠悠地笑着靠在沙发上,周则苦笑着坐到她身边,啜饮了一口已经冷掉的红茶。\\

「啊,妈妈倒是一直活蹦乱跳的,不过人精神就再好不过了。要是再沉稳点就更好了,说真的」

「我觉得那很有志保子阿姨的特色,挺好的」

「特色倒确实是很特色」

「嘻嘻,周君很不擅长应付阿姨的生龙活虎呢」

「准确来说是不擅长应付被波及受害的这部分吧?」\\

而且其中有大概一半是真昼提供的强化效果,但真昼许是没有那方面的自觉,仅仅是欢快地笑着。

周倒是觉得真昼觉得开心是最重要的,没有半点责备的意思。是不是还是锻炼一下与志保子周旋的方式比较好呢?他心想着这17年来都无能为力的事,叹了口气。\\

「志保子阿姨,看上去很忙呢」\\

或许是想起了谈着谈着就匆匆启程的志保子,真昼一声细语喃喃道。\\

「是一些工作排期要到了。妈妈能来我就已经很感谢了。爸爸本来也想来的,就是现在实在是忙得没边,没办法」

「嘻嘻,他们真爱你呢」\\

欣慰、艳羡、感慨的声音组成的这句话让周咬住嘴唇听到心里,真昼则是目光柔和下来,凝视着周。\\

「周君真是好懂,你很在乎我三方面谈的事情吧」\\

趁着疏忽不备,忽然一道锋利的小刀刺来,使周绷紧了身子。「当然啊」真昼见到周的态度,依旧是柔声说道。\\

心思被发觉,这本身就可能成为真昼的负担,原本其实该表现得不让人察觉出任何异常才对。然而看到真昼的模样,周却做不到单纯地笑着,将一切都深埋心底。\\

「虽然这种温柔的地方才是周君的特色,不过我也不想增加周君的负担,所以不用放在心上的」\\

真昼似乎完全看穿了周的意图,面对他称得上是尴尬的表情,轻轻笑着。

那副模样,并非受伤或是类似的什么,而是理解并接受了目前状况和事实的淡然之状。\\

「周君不用放在心上的啦?毫无疑问过错在我父母,不担好产品责任怎么能行」

「……嗯」

「而且,是我确信父母不可能会来,就没告诉他们三方面谈的事情,他们自然是没可能来的。是我这边切断了所有的可能性,也就不会再有任何的期待了」\\

——所以这次是自己招致的结果。\\

看到真昼露出昙花一现的笑容,周怎么也无法稳住自己的表情。\\

「要把他们会关注我的可能性,这根极其纤细又脆弱的丝线拉到自己身边而不破坏它,简直是不可能。我不想为期待那点可能性,这十有八九都无用的期待而劳心。所以,现在这样就好」

「真昼……」

「三方面谈也不需要父母的理解,我自己能做决定」\\

不扭曲、不停顿,真昼坚决地说完,便露出了伶俐的眼神和平静的微笑。平时传递给周的温暖,现在已然不再。\\

「我知道就算不用跟父母商量,学习成绩和综合评价方面都不会有问题,而且还投保了教育险,钱也不需要担心。而且还另外准备了升学就职用的资金。幸好他们让我起码不用为钱发愁……只要不扯上关系,至少在金钱方面他们就会给我最大程度的关照,这一点我很感谢他们」\\

相应地,除此之外的东西就都没有亲手给予过——如此暗示的真昼带着称得上是自嘲的笑容,发出叹息。\\

本应温暖的吐息,却反而让人感觉冰冷。\\

「我反倒是很幸运的了。他们把小雪这么好的人安排到了我身边,而且可能他们内心里还有一点点愧疚这种东西,没让我的生活有任何不便之处。也是多亏了这样,我才能好好地像正常人一样长大」\\

反过来讲,这也就代表了若是没有小雪的存在,真昼的成长必然会扭曲,周对此不可能坦率地感到高兴。\\

「换个角度想,这也让我能够不被父母指指点点,完全靠我的意志决定……所以没事的,周君不用摆出那样的表情」

「抱歉」

「为什么周君要道歉啦,真是的」\\

正因知道不过脑子的同情、安慰和同意会让真昼受伤更深,所以周只能接纳真昼的说法,接纳真昼看不见的眼泪。\\

他握住真昼纤细的手,一股较之平时更低的温度和周的温暖一点一点融合。

哪怕只有一点点,周也希望自己的温暖能够成为真昼的东西。他紧紧包覆住那稍微颤抖的手心,然后悄悄缩短了跟真昼的距离。\\

由周主动且毫不犹豫地紧贴真昼,这对她来说似乎很稀奇,她轻轻瞪大眼睛,然后好像很痒一样地眯起了眼睛。\\

「没事的。要我说,对父母的看法已经是推翻不了的了,而且这也不是一天两天的事。要说完全不痛不痒那也是骗人的,但我并没有特别悲伤,这都是理所当然的日常」

「哪怕你知道这样理所当然是不对的吗」

「嗯。毕竟木已成舟,再视而不见也没有意义了,反正总会在某个地方痛切地感受到的。我已经做好割舍了,这样就好。而且就因为我一直一个人过,最后才能遇到周君,这一点上我也很感谢」

「……是吗」\\

真昼以坚决有力的态度断言。她实在是太耀眼,太惹人怜爱,这次周不止包覆了她寒冷的手心,而是连着整个身体都拥抱过来,她身体惊颤了一下,而后很快就失去力气。\\

凭借小小的身体,却没有歪曲地率直生活至今的真昼,仅仅是周将她拥在怀中接纳她,便仿佛一下子安心了似的,将身体全都依靠上去。看得出她是如此地接受周、依赖周。\\

真昼把身子扭到了恰好的地方,把脸探出到周方便看的位置,而后她看着周的脸,似有些困扰地笑着,\\

「周君真爱操心啊,我还没软弱到这样就会折断。要是每次都失魂落魄的,可怎么生活呀」

「不是说坚强、软弱的问题……是说喜欢的女孩子受伤,这件事情变得理所当然,是很让人讨厌……或者说烦心的。这让我不得不面对一个事实——明明心里想要保护,实际上却凭我根本改变不了什么」\\

真昼生长的环境,以及她现在所处的家庭环境,都是周无从下手的。

过去不可改变,现在亦无法触及。\\

即便心爱、珍重,也强烈地希望保护她,但只要有名为「别人」的阻隔,周就没办法去做什么。硬是去越过那道阻隔,也就代表会践踏真昼柔软的地方。\\

所以,目前周能做到的,就只有包覆住真昼柔软的部分,不使其受伤,再将不必要的杂音和刺激弹开。\\

「毕竟这是我的问题……不是我要拒绝,而是这问题应该由我亲自处理,而且也只有当事者才能解决」

真昼也清楚这不是能靠周搞定的事情,她也并不希望去拜托周来解决。

按周的理解,她所希望的,就是周不要放开手,成为她的支柱。\\

周朝着怀中静静凝视他的真昼,轻轻颔首。\\

「我再怎么样也不可能完全理解你的心情,因为我终究是从跟你不同的环境生活至今的」

「是啊,毕竟说到底我是我,你是你,哪怕想象得出来,也无法完全掌握」

「嗯」\\

这一事实是无论谁如何挣扎都改变不了的。\\

周是周,真昼是真昼,即便人生交汇,相互依靠,也不会成为一体。真昼这个个体,无论任何人做什么事情都不会变成其他人,也不可能让人无一错漏地详悉其内在。\\

她的感情只属于她自己,她的想法只有她自己明白。\\

正因为理解这一点,周才不打算硬是问出真昼的心情,也不打算硬是去做点什么。\\

「不过,我喜欢愿意来理解我的周君,喜欢不把自己的解释强加于我,在旁边守望着我的周君」

「……嗯」

「我很清楚周君是为我着想,非常非常珍惜我。我一直都觉得自己是个很幸福的人」\\

想必这是从心中流露出的真心话,真昼露出绵软天真的笑容,将脸颊贴在周的胸膛,然后惬意地享受着周的体温,靠在周的身上。\\

面对对她来说最大限度的撒娇动作,周向着画出平缓的波纹洒落的亚麻色亲吻一口,再把额头顶到她的脑袋上。\\

「……我会让你更加幸福的,所以,如果真的难受,要好好讲出来。你总是忍着说自己不要紧」

「不要紧的。啊,这次是真的不要紧」

「……意思是还有假的不要紧咯」

「以后我会注意的。我也知道我受伤也会让周君受伤。我被周君满满地爱着呢」\\

真昼比以前更为清晰地,带着毫不动摇的自信宣言道,足见其的确接纳了周的爱情。

如今,真昼已经能在这种地方自信满满地说出那番话了,那样的真昼实在是让人情不自禁地想疼爱她。于是周跟她贴得更紧,欲与她的温暖融合在一起,而真昼则只是笑着接受。\\

「嘻嘻,这样要是我再消沉一点,幼稚到自暴自弃的话,接下来就是说『在被爱中长大的周君懂什么!』然后演变成吵架了呢」

「这么一说我也没法反驳啊」\\

周深知自己是倍受父母珍爱,在大量爱情的滋养下长大的,因此,如果真昼那么说,他便无从反驳,届时显然是除了道歉什么都做不了。\\

甚至,就连道歉都可能更加触怒她。\\

无论饱汉子对饿汉子说什么,都可能引动后者内心的波澜,还可能由此产生隔阂,这是周生活至今深刻理解的道理。\\

「这话既会伤到周君也会伤到自己,所以我不会那么说的」

「……就算嘴上不说,心里会不会有这么想过?」

「要说完全没有倒也是假的。不过,没法应对或改善的东西,就算带着拒绝的意思大喊大叫,也解决不了问题。何况这是就算周君也没辙的环境因素,从这一点去责备也是无济于事的吧?话刚说出口就会后悔,也是显而易见的了」\\

「我既不是想吵架也不是想伤人」真昼努力用理性去组织语言,表情依旧温和。\\

「说到底,有差异是理所当然的……我的家庭缺少了那种通常的爱情,跟大多数的家庭处在相反的位置,有很多机会感受这一点。这种嫉妒一样的东西,在小学初中的时候就体会了一遍,都咽了下去」\\

也亏她能没有扭曲,周心想,应该还是小雪的存在占了很大因素。\\

「我也是类似,不了解因被爱而生的纠葛,也不知道遭受干涉的烦心。所以对于这件事,我并不能说这说那的。所以,我可能会有一点点的,真的只有一点点的嫉妒……不过应该都健全地咽下去了哦?」\\

如此总结的真昼担心地探头看着周这边,这让周为自己的丢脸不禁苦笑:到底谁才是被担心的一边啊。\\

「我了解你不怎么会感情用事,也了解你很理解自己的感情,找到了妥协点将其接受……啊,这时候我说了解没问题吧」

「嘻嘻,没问题……我能感受到周君是好好看着我的」

「那是自然。喜欢的人,当然会,好好看着」\\

因为是喜欢的人,所以想要了解她;因为是喜欢的人,所以希望能够理解她。正因为是喜欢的人,所以才希望能够为她提供关怀,让她可以惬意地生活,才希望为她做让她高兴的事,才希望助她远离糟心之事。

尽管理由众多,但都可以概括为因为一心一意地喜欢真昼,所以想要好好看着、守望着她自身。\\

由于想要不止于表面,而是好好看清她的内在后对待她,周将这些不加掩饰地说完,怀中的真昼便闹腾地接连把脑袋捶向他的胸脯。\\

「……怎么说呢,能坦然说出这种话的周君感觉越来越像修斗了」

「是怎么才能从刚才的话题说到这个的」

「没什么——」\\

话题突然扯到修斗,周头顶上冒出一个大问号,而真昼却俨然是无意解释,一边扭头一边又送来一记头槌。\\

周差不多看出来这就是在遮羞,便以抚摸后背的方式安抚她,然后就碰上了微微鼓起脸的真昼。那神态也是颇为可爱,周笑着继续抚摸,而后真昼的闹别扭模式也就解除了,她以一句「真是的」收尾,不再抵抗。\\

「不过我倒是挺愿意像爸爸的,不得不说他真的是个很优秀的人」

「不是那个意思,不过也有点那个意思,也行吧。你可以尽情为此感到骄傲,志保子阿姨估计也会那么说的」

「妈妈对爸爸爱得痴迷,评价标准倒是可能更严格」

「嘻嘻,这可不好说」\\

真昼不知为何开心地笑着。一看向她的脸,她便带着淘气的笑容,愉快地眯起眼睛靠到周的胸前。周边有些摸不着头脑,边对坦率地来撒娇的真昼翘起嘴角,分享彼此的温度。\\

如小猫蹭上来一样的可爱让周翘起嘴角,周这时忽地说起了一件自己在意的事。\\

「顺带一提啊」

「嗯」

「你是觉得我跟我爸爸很像对吧?」

「嗯,是啊。那个,不止是长相,言行更是一模一样」\\

亲生父子,长相相似也是自然,这一点先不说,周那么问其实是为了接下来的话题。\\

「那,真昼你跟小雪阿姨像不像呢?」

「咦?我,我吗?」

「嗯。听你之前说的,总感觉你也会跟小雪阿姨很像」\\

性格和言行有时是遗传决定,有时则是受到周围亲近者的影响而决定的。

不知道真昼是前者还是后者,但至少真昼似乎不太像她的母亲,从之前的交谈来看,她跟父亲也不一样。\\

既然这样,那觉得她会类似对形成人格颇具影响的小雪也没什么奇怪的。\\

「怎、怎么说呢……那个,小雪阿姨教给了我很多东西是事实,那方面应该算是很像吧……不过周君不亲眼看看也是没法判断的」

「也是啊。不过感觉会很像」

「依据是什么……」

「我的直觉?我觉得应该像,虽然只是猜的」

「你啊」\\

尽管这样或许显得有些随便,但周心中却有种奇妙的确信。\\

真昼曾评价小雪是一个非常优雅、心善又沉稳的人,真昼自己也说她很憧憬小雪,同时周眼中的真昼也一样具备那些品质。

不管这是不是潜意识的影响,很难不认为如此行为举止的两人不相像。\\

由于没有机会确认那一点,目前的周还不能下判断,但他觉得小雪一定是个不逊色于真昼的优秀女性。\\

「说着我又觉得,希望能日后见上一面啊。她是你非常珍视的人吧?」

「嗯。她是我最承蒙关照的,非常珍视的人。那个,我也想和她再见上一面,都已经好久没见过了。因为对方也有自己的安排,而且身体也有问题,勉强她也不合适。虽然偶尔会有些书信往来……但真想见见啊」

「这样……信是定期写吗?」

「嗯。不过突然联系也给人添麻烦,差不多也就一个季节写一次吧。我都好好全部留着,那些都是我的宝物」

「嗯」\\

真昼发自内心地感到喜悦,脸颊泛起了红晕,在讲述的时候,她两眼放光,显然是真的非常倾慕小雪。

真昼会如此倾慕,让周越发想与小雪这名女性见上一面了。\\

「……啊,对了。小雪阿姨的信里附着照片,请稍等一下,我去从家里取来」\\

周表达出的兴趣似乎让真昼挺上心,她用手轻轻解开周轻柔的束缚,站起身向周露出甜美的笑容。\\

「可以吗?总感觉很对不起你……」

「看你很想了解小雪阿姨是怎样的人嘛。我也希望你能了解小雪阿姨」

「毕竟是你的养母……而且喜欢的女孩子珍视的人,我肯定想要了解一下」

「……老说这种话,真是的」\\

明明是实打实的真心话,真昼却嘟嘟地鼓起脸颊,同时又明显很高兴地眯起眼,啪嗒啪嗒踩着拖鞋就出了房间。\\

毕竟是珍视的人,信肯定也是精心管理着的。她从保管地点翻出信,很快回到了这间屋子里。\\

似乎是连着保管地点一起带来了,真昼慎之又慎地抱着一个可爱的盒子,就好像抱着的是婴儿一般。她简单打了声招呼说自己回来了,便去沙发坐下,将盒子置于大腿之上,轻轻取下封盖。\\

盒子选用的是和信匹配的尺寸,信封整齐地成一沓,在那上面还放着一张笔记模样的纸条。\\

这管理的状态也看得出真昼的细致,周在心里为她鼓掌。这时,真昼雪白的指尖避开那张纸条,找出某一只信封。\\

或许是用裁纸刀打开的,花边的信封封口被平整地剪开,处于可以取出内容的状态。真昼从中拿出一张照片,就这么递了过来。\\

泛着光泽的纸上,映着一名女性抱着襁褓中的婴儿,露出和蔼笑容的模样。

周隐隐约约感觉,那面容沉稳的女性恐怕比自己的父母更加年长,她的视线落在怀中的婴儿,脸上则是挂着洋溢出幸福,同时又优雅端庄的微笑。\\

「这位就是小雪阿姨。这是她在儿子和媳妇的家里抱孙子呢,照片好像是让她儿子拍的」

「所以才抱着个小孩啊……果然感觉跟真昼有点像」

「应该是错觉吧,我们一点血缘关系都没有的」\\

仿佛真昼的心声接着补了一句「要是有的话该有多好」,周心中有些苦涩,却有意识地用清爽的口吻继续往下说,以免让她察觉自己的心思。\\

「唔,我觉得这种就算没有血缘关系,在一起生活也会让人变得相似,比如说话方式、想法、动作之类」\\

周并不觉得构成一个人的因素只有血缘。\\

形成真昼的东西确实包括基因没错,但周看过照片更加确信:支撑真昼的过往,创造出她的未来的,是小雪。\\

「至少我觉得,你笑的方式和这上面小雪阿姨的微笑方式一模一样」\\

真昼恐怕还没有从客观角度确认过自己笑着的模样。

她不太喜欢别人为她拍照,也不自拍,再加上当她知道会被拍的时候往往是装出的假笑,和周独处时展露出的微笑是她所不知道的。\\

多少有些踌躇地,周拿起自己的手机,滑动照片图库寻找要找的东西,然后侧过手机向真昼展示。\\

先前周曾拍下过真昼跟他一起过的时候微笑得十分幸福的瞬间,允许他拍的真昼仅仅是害羞,没有想要确认其中的内容。\\

真昼很尊重隐私,便没有来确认,这或许是她的失策。\\

但凡看过也会知道,她和小雪重合的部分究竟是何其之多。\\

「你看,笑得多好看啊。这边嘴角翘起的地方、眼神、眉毛下垂的地方,都一模一样。虽然只是照片,但整体有种很接近的气质」\\

显示的画面上,是真昼展露着同样柔和的美丽微笑,其中的满足感,就仿佛集合了全世界所有的幸福一样。\\

首次见到自己由衷笑容的真昼凝视着手机屏幕,然后像是难以置信似的,用手指触碰着自己的脸颊,视线在小雪的照片和屏幕之间徘徊。\\

「……还没人跟我说过这些」

「那是因为没有人看到吧?只有彼此相互看着对方才能看得到。应该还有一些自己发觉不了的相似之处,等到见面了,就会更加觉得像了」\\

尽管不是只靠照片就可以判断,但周的预料想必是不会错的。\\

「……像」\\

真昼喃喃着,像是在体会着周的话。她含着一点泣声,夹在吐息中发出一声微颤而又高兴的呢喃,然后靠在周的胳膊上。

将头顶过来让周触碰的真昼垂着头,从周的视角看不到她的表情,但即便不看也知道,那绝对不会是消极的表情。

周朝着以不让照片起皱的姿势贴到他胸口的真昼微笑,直到她舒心为止都静静地陪着她。\\

\vspace{2\baselineskip}

「真昼,东西掉了」\\

抬起头的真昼已经恢复到了原来的模样,不过放松的神情中带着一点自豪。她小心翼翼地把照片收回原来的地方,此时那沓信上的纸条滑到地上,周也没多想便捡了起来。

捡起来的时候,碰巧有字的面朝上,周的视线不禁扫过点缀在纸条上的笔墨。

上面不是真昼的字,而是另一种细腻、精致的字:一行英文和符号、一行数字,再加一行汉字和数字的组合。\\

简短的几行字让周目不转睛地看了一会儿,旋即他便意识到了这串文字是什么意思——这是不该看的东西,他慌忙将视线从纸条上挪开,将其放到了真昼的宝箱中。\\

「谢谢」\\

真昼露出纯真的笑容,似乎没有注意到周的模样,她坦率地道谢后,便合上盖子,珍惜地抱在怀里。\\

那是特别让周看到的,真昼所珍视之物、重要的存在。\\

周默默感受着从她身上传来的对小雪满满的尊敬之意,摸了摸正品尝着喜悦的真昼的头,然后将逐渐涌起的负罪感避而不见。\\

\vspace{2\baselineskip}

「……该怎么办呢」
