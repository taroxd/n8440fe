\subsection{天使大人与布偶}

今天,真昼身穿围裙,绑着丸子头迎接了周。

做菜的时候真昼会扎起头发。她有时会扎成辫子,有时会像这样搞成丸子头,到底是女孩子,在实用性之中也追求着可爱。\\

真昼似乎已经提前做好了饭,在周回到家的同时她也来到了门口,然後露出稍显安心的微笑。\\

周姑且有联系说会迟一些,不过似乎还是让真昼担心了。在那之后,周和门胁简单喝了会儿咖啡,听门胁发了一阵子牢骚所以有些迟。真昼會担心可能就是这个原因吧。\\

「欢迎回来,周君……那个袋子是?」

「去了趟游戏厅,这是战利品」\\

除了兔子,周还拿了其他东西,所以大袋子塞得满满的。真昼也看得出里面塞了很多东西吧。\\

「……有相当多呢?」%有點怪怪的 還真多?

「只花了学校食堂两份每日套餐的钱」

「啊,里面都是些什么?」%這邊是はあ 應該是類似於嘆氣的感覺 用啊會不會有點奇怪?

「肚子饿了,稍后再说」\\

虽说现在就给真昼也不是不行,但周想慢慢看真昼的反应,所以就把这事搁到了后头。

而且,肚子饿了也是事实,周想早点吃到真昼做的饭。\\

「那你先去洗手换衣服吧。也别忘了漱口。我趁这时间去盛饭」

「了解」\\

就算不用她说,周平常也一直会这么做。但这样的惦记和體貼还是让周很开心。\\

周虽然心里想着真昼像老妈一样,但没有说出来,而是照着真昼的吩咐前往了洗手间。\\

\vspace{2\baselineskip}

「……那么,这么多都是些什么?」\\

晚饭后,真昼似乎很在意,她瞥了一眼靠在沙发侧面的战利品袋子,向周询问道。\\

「嗯?是布偶」\\

周并不打算隐瞒,于是就提起袋子放在膝盖上,一边撕下贴着的胶带一边做出了回答。\\

「布偶?」

「真昼喜欢不是吗?」

「是、是喜欢啦」

「因为有挺多玩偶,我感觉真昼会喜欢,所以拿来了。给」\\

今天最大的收获,应该是和之前送的熊差不多大小的兔子布偶吧。

虽然布偶还挺大,但因为只花了一枚硬币,所以周还是有些自豪的。\\

周拿出白毛圆眼的兔子,放到真昼的膝盖上。

周不是很清楚这兔子是什么角色,但觉得真昼应该会喜欢,所以就抓来了。然而真昼卻仅仅只是凝视着膝上的兔子。\\

「你不喜欢兔子吗?」

「……很可爱」

「那就好」\\

真昼就像抱着平时的坐垫一样,用双手紧紧抱着兔子蹭在脸上。周一瞬间产生了掏出手机的想法,不过还是作罢了。

看到真昼柔软的笑容,周一边用脑子拍摄着,一边从仍旧满满当当的袋子里取出了其他的布偶。\\%腦子拍攝著 感覺也很奇怪

「还有哦。猫和狗什么的」\\

多亏了那个游戏厅的抓手跟其他遊戲廳比起來力道相对强大,大部分的东西都能以很少的预算弄到手,所以周就抓了一堆真昼可能会喜欢的东西。

周再额外放上了一只米色和白色毛的,与真昼微妙地相似的猫布偶,还有一只柴犬形的吉祥物布偶。接着,真昼则露出了显而易见的困惑神情。\\

「那、那个,这么多……?」

「会添乱吗」%礙事?感覺是太多了可能會不知道要放哪 所以有點礙事 添亂感覺有點奇怪

「才没有那种事!正好房间里没有装饰品,而且都很可爱,我很开心」

「那就好」\\

真昼被各种布偶围着的样子,和周想象的一样可爱。

现在真昼还没有把兔子放下,不过她兴奋地比照着猫和狗,好像不知道要选择哪边一样。\\接下來要選擇哪邊(來抱著)?

由于那副样子很让人欣慰,周不由得就露出笑容凝视著她。而真昼似乎是注意到了周的视线,脸红了起来,然后用兔子遮住了半张脸。

因为兔子是白色的,所以真昼的脸红状态一目了然。

从兔子耳朵的缝隙中露出了她湿润的眼睛。由于这副模样透出的奇妙的妖艳和可爱,结果周还是凝视着真昼。\\

或许终于是撑不住了,真昼把头贴在旁边的周的上胳膊那里,开始藏住了自己的脸。准确来说,她是像撒气一样拿头撞着周。

不过,与其说是拿头撞,其实她也只是砰砰地顶着,所以周一点都不觉得痛。\\

「……请不要笑」

「我没有」

「你有,就是有,在笑我孩子气」

「没在笑这个,就是觉得你挺可爱的」

「……这不就是在笑我嘛」

「啊」\\

「露馅了吗」周像是要蒙混过关一样地笑着。接着,真昼啪地拍打起了周的大腿,于是周为了安抚她而摸了摸她的头。

这样一来,真昼就变得老实了。周这次注意着不要露馅,笑了出来。\\

「……总觉得你在糊弄我」

「你想太多了」

「……今天就放你一马」\\

真昼还是不满地嘟哝着。周没有指出她表情和台词之间的不一致。\\

周看着真昼膝上的猫和怀里的兔子,心想「这是兔和猫的混血吧」。同时,周又摸了一会儿她的头,然后真昼仰起了脸。

尽管她红润的脸蛋没有什么变化,但她的眼神里露出了和方才不同的不满之色。\\

「……我总是从周君那里拿到东西」\\

她似乎是在意起自己得到太多东西了。\\

「是我自作主张给的,你不用在意」

「但是,我一直在从周君那里得到。礼物、关心、温暖的气氛等等,全都是」%氣氛改成環境?

「只是我想给才给的,所以你不需要在意這些」\\

周并不是希望得到回报,只是因为真昼会开心才给了她这些。

虽然这种说法听上去就好像真昼的开心是回报一样,但说到底,周的给予还是来源于周的自我满足和自身的愿望,其中没有任何真昼需要在意的事情。\\

即使如此,真昼似乎还是因为得到太多而感到介意。

周反倒觉得,自己受了她太多照顾,这点东西就连恩情都还不尽。\\

「我也想回礼點東西」

「你好固执啊……不过,要是你这么在意,我就收下一个吧」

「只要是我能给的,什么都可以」\\

周感觉真的说什么她都会做,所以觉得不太妙,不过他自然也不可能拜托她做一些给她添担子的事情。

然而,什么都不拜托的话,真昼又会沮丧。\\

「要不做个布丁吧」\\

于是,周就高兴地拜托了她不会造成负担的事情。\\

「……布丁,是吗?」

「放很多鸡蛋的布丁。我想吃真昼亲手做的」

「……不是为了省钱吧?」

「怎么可能。因为是真昼做的才有意义啊」\\

周并不是喜欢吃甜食,但奶蛋类的甜点是例外。

他喜欢布丁以及只放糕点奶油的泡芙。如果是真昼亲手制作,肯定能做出美味的东西吧。

喜欢的女孩擅长料理,周当然会想吃她亲手制作的东西。\\

周认真请求之后,真昼直直仰视了周一阵子,然后点了点头。\\

「……那么,下次休息日我會做。多加鸡蛋,做硬一点对吧」

「嗯」

「我一定会做出好吃的布丁」

「不用那么鼓足干劲啦」

「是我想做才这么做的」

「这样」\\

不知为何,真昼显露出了没意义的满满干劲和决心。周虽然觉得她不必那么努力也没问题,不过既然能吃到美味的布丁,他也没什么好抱怨的。\\

周带着為真昼加油的心意,又一次摸了摸她的头,然后真昼有些腼腆地把嘴角埋到了兔子的后脑勺后面。%埋入了兔子的後腦勺?
