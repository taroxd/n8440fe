\subsection{高二最后的到校日}

白色情人节过去后,接下来也没有什么特别的活动,就这样迎来了结业式。\\

准确来说,在新学年开始前都还是高二,但感觉上更像是高二已经结束了。\\

「为什么只有我……」\\

树在亲身实践了在校长讲话时无聊打瞌睡的学生常见行为后,被班主任发现,挨了一顿唠叨才回来。\\

「我们班主任对这种事还挺严格的,刚才也因为要升上新年级,被唠叨了好一阵身为备考生的注意事项,看走廊都知道其他班已经解散了」\\

现在班会已经结束,学生们表情各异地踏上归途,有充满解放感的、有纯粹觉得麻烦的,还有即将成为考生感到不安的表情。\\

至于周则是属于充满解放感的那一边。虽然对于成为考生这件事感到不安,但他怀着不乱步调、不颓废堕落、天天向上的志向,既要做好准备,也要好好休息。\\

「可恶,其他人没被发现,真可惜」

「放弃吧。其他人只是稍微低着头,你是明显在睡觉」\\

自己班里只有树一个打瞌睡的,周还看到其他班有几个人的动作估摸着也是在打瞌睡,他们恰好处在老师看不到的座位上,逃过了一劫。\\

树不甘心地发出呜的声音,周冷眼看着他,一边把笔盒和装文件的袋子收进书包里。树见状嘀咕道「你在这方面还真冷淡」周则以眼神表示「现在还说这个干嘛」。\\

「我对没有过错的人很温柔」

「你是说我有过错吗!」

「怎么想都是你自作自受……」\\

听别人用缺乏抑扬顿挫的声音慢悠悠地说话,难免会招来困意。不过,人类就是要靠理性与毅力来抵抗。即使乍听之下是毫无意义的对话,有时也会谈到重要的事情,所以还是应该听一听。\\

话说回来,周听到最后的感想是觉得在聊闲话,这件事就没必要告诉树了。\\

周一边装作闹别扭的样子,一边收拾东西准备回家,然后瞥了一眼树的行李。

树的行李比去年少。他理所当然地提前把课本带回家,大概是意识到要准备考试了吧。\\

「对了,你去年突然来我家过夜呢」

「对啊对啊,那时候和老爸吵架了」

「我是真没想到有人会到最后一步才说这事的,就算是早上吵的架,正常来说突然闹这么一出也吓人。得亏我家收拾到能住人的水平了,你得谢谢才是」

「谢谢椎名」

「喂」

「叫我吗?」\\

周正想吐槽树干嘛在这时候提到真昼的名字,她本人就突然探出头来。

周在教室里没什么事情,单纯是在等真昼。她出现的时机太刚好,树也发出哇的感叹声。顺带一提,千岁也跟在真昼后面,一手拿着文件夹从后面走了过来。\\

「和老师的事情办完了吗?」

「是的。刚才你们在叫我,发生什么事了吗?」

「啊,没有啦,去年的这个时候我不是去周家过夜吗?他说是因为家里收拾好了,所以即便这么突然也能让我住下。所以我是在向你道谢,谢谢你帮忙整理和传授打扫技巧」

「嘻嘻,原来如此。那我就心怀感激地收下了」

「连真昼都这样」\\

的确,多亏了真昼的大力协助和传授打扫技巧,周的家在认识真昼前后有了翻天覆地的变化。\\

东西不再乱摆在地上,窗框上不再积灰,地板也变得亮晶晶的。多亏如此,周不再发生脚被东西绊到而差点摔倒的情况,也不再会有袜子少了一只而遍地找的事件,树睡觉时也有空余的地方铺被子了。

周的家能够维持在能让别人放心来访的状态,真昼的功劳很大。\\

树知道周的家以前不只是杂乱,根本就是一团混沌的状态,或许他才是对变化最感到惊讶的。\\

「因为刚认识周君的时候,他的房间真的很脏嘛」

「是、是这样没错啦」

「对啊对啊」

「被真昼这么说还能接受,被树这么说就无法接受了」

「偏心啦偏心」

「啊,偏心又怎样?」\\

周嗤之以鼻地表示,要是给树特殊对待才更不好吧。接着,教室里轻轻响起树的低吟。\\

「这家伙竟然直接承认了」

「周老早就偏心昼儿了,你在说什么啊?」\\

千岁无奈地说道,这话让周感到心情复杂,但他也无法否认,只能在另一种意义上和树一起发出低吟。\\

「小千,你是站在我这边的吧?」

「当然是站在你这边的,可是阿树坐在那么显眼的位置还打瞌睡,实在太过分了。按你的座位号,肯定是坐在一排最边上,根本就藏不住,摆明了会被骂」

「不要讲那么现实的话,小千」\\

树的姓氏是赤泽。

周不知道其他学校是怎样,不过这所学校是按照姓氏的五十音顺序来决定座号的。而在这个班上,没有比赤泽这个姓更靠前的学生。\\

树的座位号一般都是1号,在这个班里也不例外,在典礼上往往会被当作标志。指定座位时,通常都会把他安排在边缘,这次树的座位也在一排最靠边的位置。\\
】
在非常引人注目的位置打瞌睡,自然会被老师盯上。\\

「要好好睡觉才行。你忘了之前熬夜到搞坏身体的事吗?」

「我会注意的」

「哦哦……千岁竟然讲道理了……」

「你以为我是什么人?」

「好、好了好了……」\\

被千岁用严厉的眼神盯着看,周佯装不知地移开视线,环视教室里还剩下的不少同学。

也有些同学已经回家了,但大家好像都还依依不舍,也有很多学生正热烈地聊着回忆。

再怎么说,周对这个度过了一年的班级也有了感情。\\

「大家还不回家吗?」

「嗯?要回去啊,只是觉得要和这个班级告别了……有点感慨」

「是啊。今年的班级真的超棒的」\\

应该说,这个班级的凝聚力远超高一的班级,彼此的关系也还算融洽。

虽然每个人之间多少有些合不合得来的问题,但需要合作的时候还是会互相帮忙。这可能是因为大家都是聪明人,能够理性地做事。

同时也是因为,班里没有脾气暴躁的,或是不认真到需要让老师特别关照的学生。\\

「大家都很善良,关系也很好」

「老师们也说我们班是最认真、最乖的班级,上课的时候很轻松」

「毕竟聚集的都是比较乖的人嘛」

「乖……?」

「周,你为什么要看我?」

「不为什么?」

「你好烦」

「好了好了,你们两个别吵架。大家确实都很认真,也很安静。不过优太倒是会被女生们的尖叫包围」\\

「大家都有分寸,不会在上课的时候做那种事,所以没问题」\\

虽然班上也有不少女生对门胁抱有好感,但她们根上都很认真,只要门胁开口制止,就不会再犯同样的错误。所以门胁今年应该过着相当安稳的校园生活。\\

周记得高一的时候,班上的吵闹还更加连绵不断一些。\\

「那优太呢?」

「他说今天没有社团活动,中午还有事,所以很快就回去了。反正春假还会见面」\\

周、树和门胁三人约好在春假时一起闲聊放松,借此转换一下学习的心情,根本没必要舍不得分开。

女生们对门胁干脆的态度感到有些遗憾,但也没有强行挽留,只是目送他离开。\\

树看着门胁的桌子,上面已经完全看不到有人使用过的痕迹,他笑着说这真有门胁的风格。\\

「喂——那边的四个——接下来有空吗?」\\

当有些感伤的气氛笼罩着四人时,一道开朗的声音传了过来。

往声音的方向看去,只见太刀川似乎原本在教室里和朋友们聊天,现在正带着不逊于声音的开朗笑容朝这边挥手。\\

这一年来,周他们和太刀川不同组,所以很少有机会交谈。这么说来,自从借了笔记本之后,他也开始会找周说话了。周一边想着,一边点头。\\

「我今天没有打工,也没什么事」

「我也是」

「我晚上要和家人吃饭,不过在那之前没什么事」

「我也没有。怎么了吗?」\\

周他们并不是所有人都有空,但今天也没有什么特别的计划,所以便坦率地回答了。太刀川见状,露出高兴的笑容。\\

「哎,今天是班上的解散聚会,我想借这个机会找一波能参加的人一起去唱卡拉OK。所以想问你们四个要不要一起去」

「刀刀,这种事情你可以先在班级的群组里说啊」

「因为是刚刚才决定的!」

「哇哦,真是行动力十足」\\

周打开手机,确认了一下文化节时加入的群组里的消息,发现刚才太刀川确实发了消息说『今天要举办班级解散卡拉OK聚会,要参加的人请回复,并在下午一点半之前到车站前的卡拉OK店集合!』,还有几个人已读并回复要参加。\\

因为实在太突然了,除了回复要参加之外,还有一连串『早点说啊,我跟女朋友约好要出去了』『那还是女朋友重要』『渡边好差劲』『都呆了一年了连同学名字都写不对,你也好差劲』之类的对话。\\

看着群组信息如此自由奔放,周忍不住自然地露出笑容,太刀川则对他们投以期待的目光。\\

「所以呢?你们怎么样?我不会强迫你们啦」

「我是无所谓」

「那么,机会难得,就让我参加吧」

「我也是、我也是」

「如果不会拖到晚上的话,我也去」\\

既然周要参加,真昼当然也会跟来。

只不过,太刀川应该没有那种意图,只是露出大型犬般亲近的笑容握拳喊着「好耶」。\\

「这次就让周好好唱歌吧」

「这是在整我吗?」

「咦?藤宫很会唱歌吗?」

「很遗憾,只是普通水平」

「还想听你和椎名合唱来着」

「拜托不要把真昼也卷进来」\\

真昼不太擅长在众人面前大方唱歌,而且在这种场合也会感到害羞,要是被别人催着唱歌,她一定会退缩的。\\

如果她本人希望的话,周也会一起唱,但他不想给她增加负担。\\

「啊,藤宫的份我请客!」

「怎么突然请客?」

「因为上个月欠你的人情还没还啊。错过这次机会,可能就没有还的机会了吧?」\\

这么说来,借笔记本的人情并没有多大,周也是每次都坚决表示不需要,让太刀川每次都发出苦恼的声音。看他现在想还上人情的样子,没准是一直在虎视眈眈地等待机会。\\

「其他科目的笔记也一在里面,我也顺便复印了一份,不好意思,所以让我请客当作谢礼吧」

「哦!那我就不客气了」

「可惜赤泽就没什么理由请了」

「好过分,我被排挤了,想哭」

「被你敲诈的我才想哭」

「树真差劲」\\

「别若无其事地混进来」周轻轻戳了戳树,然后对窥探着这边的太刀川笑着说:\\

「嗯,那我就恭敬不如从命了」

「嗯,就这么办吧」\\

呼呼,太刀川有些喘着粗气,不知为何给人一种大型犬的感觉。周背起背包。

消息说是要在下午一点半前集合,所以必须早点吃完午餐才能过去。\\

周看了看真昼,询问她要先回家还是在外面吃,得到「今天不用准备晚餐,所以都可以」的答复。既然要准备晚餐,那午餐在外面吃就好。于是周向真昼提议去快餐店。\\

「我想吃照烧鸡蛋汉堡,你吃汉堡可以吗?」

「我没关系。每次有鸡蛋商品限时推出时,你都会去吃呢」

「毕竟鸡蛋是天下第一嘛……」

「啊,我跟小千也可以一起去吗?小千家里现在好像没人,我又不想每次都回家」

「好好好」

「啊,我也可以去吗?反正我本来就打算在那里吃」

「可以啊。啊,不过他们没关系吗?」\\

周用眼神示意太刀川刚才聊天的对象——几个跟他关系很好的男生正在谈笑风生,太刀川却满不在乎地笑着说「他们说在去卡拉OK之前要先去家庭餐厅,好像要吃限时的巨大芭菲。我又不怎么喜欢吃甜食,看了就觉得胃痛,所以就不去了」。\\

这么说来,周记得好像有看到网络新闻报道某间家庭餐厅推出了奶油超级加量的草莓芭菲,他们大概是想去用高中男生的胃袋赢得胜利吧。\\

周一边回想新闻报道的图片,一边嘀咕着「希望他们能凯旋归来卡拉OK」,然后和真昼等人一起走出教室。\\

「好久没去卡拉OK了」\\

对没有参加社团活动的人来说,校门是暂时见不到了。他一边毫不怀恋地向校门道别,一边回想。

本来不太爱去卡拉OK的周上次前去,应该还是文化节的庆功宴吧。\\

「你平时不和椎名一起去吗?」

「昼儿不太想在别人面前唱歌」

「咦!那我是不是不该邀请椎名?对了,我记得那里有出租沙槌和铃鼓!如果不喜欢唱歌,也可以专心负责伴奏哦!」

「不,没关系……原来还有那种东西啊」

「借来带进包厢的话,感觉会很吵闹呢」

「就是要吵闹一点才好玩啊」

「也要有个限度」\\

虽然有隔音措施,但要是太吵的话,还是有可能被隔壁包厢抱怨,或是被店员警告,甚至隔壁包厢的人会闯进来,说到底还是要适度。\\

「不过,真怀念文化节的庆功宴啊」

「对的对的」

「还发生了优太拉着周一起唱双人合唱的事件」

「你以为是谁的错?万恶的根源就是你吧」

「为什么在我们不知道的时候发生这么有趣的事情?你们几个感情很好,总是聚在一起,所以我们都不知道你们在包厢里发生了什么事」\\

太刀川好像也有参加庆功宴,但因为人数的关系,他是在另一个包厢,所以不清楚周他们那边的情况。\\

「气氛超级和谐的」

「门胁不在的包厢里,女生们的情绪有点低落。老实说,我觉得我们不够格,觉得很抱歉」

「这次门胁呢?」

「我先问过了,他要参加。不过他白天有时,时间比较紧,所以约好在卡拉OK碰面,吃饭就分开吃」\\

周看了一下手机,因为没有收到门胁的回复,所以不确定他是否会参加,不过他好像还是打算参加的样子。\\

(要是知道门胁会来,女生应该会很激动吧)\\

这次没有直接在群组里提到,所以很多人不知道门胁会来。他一来包厢,场面应该会变得很热闹吧。周事不关己地想着,一边对露出爽朗笑容说「真期待啊~」的太刀川苦笑点头。
