\subsection{圣诞早晨}

真要说起来,一般真昼比周起床更早,但今天却是周更早清醒。\\

或许是因为想看真昼的睡脸,以及想看看她的反应,身体自然而然地就醒过来了。\\

如果真昼先醒来,周就会错过她起床时的反应,那样可就太遗憾了。不过,看来是不用担心这一点,他的恋人就在旁边,毫无防备地露出安详的睡脸。\\

那张可爱又天真无邪的睡脸,或许是因为待在周身边的安心感,看起来纯洁无瑕,没有一丝担忧与不安,十分平静。\\

即使在只有从窗帘缝隙间透进来的朝阳作为光源的昏暗房间里,真昼的睡脸看起来依然耀眼,这一定是因为周深深迷恋着她吧。\\

(……幸福的睡脸)\\

周望着那张百看不厌的睡脸,等待她从睡梦中醒来。几分钟后,由于周起来多少有些动静,真昼的意识也因刺激而醒来。她颤动着长长的睫毛,缓缓地将眼皮抬起。\\

或许是还没彻底清醒,那双焦距对不准的茫然焦糖色眼眸,眼看着又要被垂下的眼皮盖住。\\

即使如此,真昼似乎还是察觉到周的存在,她又眯了一会儿后,同样以缓慢的动作坐起身,揉着眼睛缓缓环视四周——然后僵住了。\\

直到刚才还带着困意的双眼,一下子睁得老大。\\

「咦?」

「早」\\

周温柔地打了声招呼,真昼却依然僵在原地。\\

周很清楚,她并不是因为自己的存在而感到惊讶。\\

证据就是,真昼的视线不是对着躺在旁边的周,而是对着枕边。\\

「……怎么了?」\\

明知原因,周还是装作不知情的样子反问。他明白自己的性格很差劲,但无论如何都想从真昼那里得到反应,于是他一边起身,一边凑过去看举止可疑的真昼的神情。\\

「咦?盒、盒子」

「嗯,有盒子」\\

真昼惊慌失措,说出来的话支离破碎,但周很清楚她想说什么。

真昼那边的枕边,放着两个包装精美的小盒子。\\

「为、为什么?」

「一个身材发福、留着白胡子的大叔……不对,是眼神凶巴巴的男朋友送的。见谅」

「你、你不是说没有礼物吗……」\\

真昼用软绵绵的丢脸声音表达不满,同时捶了周的胸口一下。她的眼中满是困惑、不满与喜悦,就连她本人似乎也不清楚哪种感情更加强烈。\\

几周前,真昼过生日时,周因为不想让她花太多时间挑选礼物,所以约好这次圣诞节不交换礼物,而是两人一起度过大把的时间。只不过——\\

(反正我早就看出来真昼很期待圣诞老人了)\\

真昼本人说她从来没有遇到圣诞老人,也不相信有圣诞老人。所以周偷偷策划,这次趁她睡觉时让她体验一下礼物放在枕边的冲击。\\

「嗯,抱歉」

「好狡猾!」\\

周非常抱歉自己打破了约定,只能低头道歉,但他一点也不后悔做了这件事。\\

而且,这也不是周一个人的主意。\\

「不过,不过不都怪我一个人」

「咦?」

「这里面也有妈妈他们的份」\\

没错,包装精美的盒子有两个。\\

其中一个是由周准备的。\\

另一个则是随圣诞树一起寄来的。\\

周本来还在想是什么,结果志保子传了信息说『给小真昼的圣诞礼物!难得有这个机会,你就当一次圣诞老人吧!』于是周就一起放在床头了。\\

志保子为什么能预料到自己会在半夜把礼物放在床头,这让周有些害怕,不过如果真昼会更高兴,那他也没有理由拒绝。\\

「……志保子阿姨、修斗叔叔……」

「顺带一提,我不知道里面是什么。我只是被合作方圣诞老人藤宫分部委托,负责送货而已」

「嘻嘻」\\

真昼像是想起了昨晚的对话,开心地笑了起来。从她身上已经感觉不到愤怒和困惑,而是被柔和的喜悦所占据。

真昼将为她准备的两个盒子轻轻抱在胸前,珍惜地低头看着。周则是摸了摸她的头,整理好她有些乱掉的头发。\\

「我不觉得我爸和我妈会送你奇怪的东西,可是完全想象不出来他们准备了什么」\\

按志保子的溺爱程度,弄不好她什么东西都想送,好在有修斗会帮忙踩刹车,结果应该还是在常识范围里的。\\

盒子是比手掌稍微大一点的薄长方体,看真昼拿着很轻巧,里面也没有发出什么声响,估计是什么小东西。

就算再怎么疼爱,周也不认为他们会送学生会感到惶恐的那种东西,但毕竟是志保子……周心里有些怀疑,而真昼则是用眼神问他能不能打开。\\

当然,那是真昼收到的礼物,所以周点头表示随她高兴。真昼带着有些紧张的表情,解开绑得很整齐的缎带,然后以不让包装纸破损的方式将其拆开。\\

她一定会保管在家里吧。周回想起真昼至今的举动,不禁莞尔。这时,盒子从包装纸中出现了。\\

周对盒子有印象,所以猜到了里面的东西,但他觉得不该夺走真昼的期待,于是抿着嘴唇,只是静静地看着她慎重的动作。\\

真昼小心喜喜地掀开盒盖。

里面装着两支文具。\\

「是圆珠笔和自动铅笔呢」\\

木轴的自动铅笔和圆珠笔精心准备好的空间内,笔上强调的一处黄铜部分反射着早晨的阳光,闪闪发亮。

木制的文具色调温暖,想必很能陪衬真昼白皙的手指。\\

「应该是商量后决定的吧。顺带一提,这支笔非常好用。写起字来很顺畅,握起来也舒服」

「我就想说怎么这么眼熟,原来和你放在笔袋里的笔款式一样」\\

周之所以能在打开之前就知道里面的东西,是因为父母在周高中入学时送了他手表和这些文具。\\

他们俩虽然衣装华美,但其实挑选物品时会以实用性为优先。这是只有他们才会挑出来的的入学贺礼,而周也确实爱用这支笔,可见他们确实有眼光。

送给真昼的和周自己的虽然木头部分的素材不同,但品牌和系列都一样,这套笔看重的是好用和耐用,用起来的舒适度方面周可以担保。\\

「看来是犹豫了很多呢,他们不想送用不上的东西。如果送钢笔,上学期间就没什么机会用了」

「光是能收到这份礼物,我就很感激了……」

「他们大概是出于父母心吧。既然要送,就送你能够一直用下去,不会用完就扔的实用礼物」\\

在志保子和修斗看来,真昼等于是女儿,他们大概想代替她的父母尽责吧。选择的礼物不管怎么看都像是给亲女儿的。\\

周一方面佩服父母很了解真昼喜欢什么,一方面也感到有些吃醋。不过,看到真昼腼腆地表示「我好高兴,之后得向他们道谢才行」用全身表达喜悦的样子,周复杂的心情也烟消云散了。\\

最重要的是真昼高兴,周无聊的吃醋根本无关紧要、不足挂齿,放在心上才叫愚蠢。\\

倒不如说,周对让真昼高兴的父母充满了感谢。

真昼慎重地盖上笔盖,仿佛在收起宝物一样。她能那么高兴,志保子他们身为代行父母职责的人想必也会心满意足的。\\

如果她现在不是穿睡衣,周就能录下她拆礼物的样子发过去了。虽然有点可惜,但周更想让真昼能有醒来后发现有礼物的首次体验,便忍住了。\\
%

真昼盖上盒子,把包装纸和缎带整齐叠好,将打开过一次的志保子他们的礼物放在床头柜上,然后把视线转回周这边。

她手上还留着周准备的礼物。\\

「……我可以打开吗?」

「不打开收起来我反而更心情复杂」

「我、我才不会那样!只是有些人不喜欢别人在面前打开礼物」

「我是为了你选的礼物,所以希望你打开看看。虽然不知道你会不会喜欢」\\

周对自己的品味并没有那么怀疑,但他不确定能不能入真昼的法眼。挑选生日礼物的时候也是,正因为周确信真昼收到什么都会高兴,所以才会烦恼那么久。\\

这次周没有找任何人商量就决定了,不知道真昼会不会喜欢。\\

周没有表现在脸上,只是静静地看着真昼纤细的指尖小心翼翼地解开缎带。\\

里面装着一对款式相同的耳环和项链。

真昼不喜欢太华丽的饰品,饰品的设计很简洁,但上面到处镶着闪闪发光的宝石,模仿着花朵的造型,和真昼美丽的容貌相比也并不逊色。

送她饰品并不是什么稀奇的事情,不过在圣诞节送饰品是经典中的经典,以周的性格来说,他觉得有点不好意思。\\

「生日的时候我不是送过你收纳盒吗?那个,我想送给你一个我没有送过的,又适合放在里面的东西」\\

白色情人节送了手链,夏祭送了发夹和饰品,所以周选了不和那些重复,又适合真昼的饰品。\\

一般能戴饰品的部位只剩下耳朵、脖子和手指——手指的话,周已经提前预定了,不是今天这个日子要送的礼物。\\

既然如此,剩下的耳朵和脖子,两边都用周选的饰品来装饰如何?\\

周心想,自己这种想法还真是随便又自私,不知道真昼会怎么看待。\\

他忍不住想自嘲,心想自己平常隐藏的占有欲都表露无遗了,但还是没有收回礼物,而是观察着真昼的反应。\\

「你觉得……怎么样?」\\

到头来,真昼会不会高兴才是最重要的,所以周战战兢兢地看向真昼。只见真昼愣愣地盯着盒子里闪闪发光的花朵看了好一会儿,然后才注意到周的视线,缓缓抬起头来。

看到她的表情,周放心地吐了口气。\\

「……很可爱。非常漂亮」

「太好了,我还在想如果不喜欢的话该怎么办」

「你送我的东西,我当然都很喜欢,不过不考虑这一点,这个也很符合我的喜好。非常可爱」

「你喜欢的话,那作为圣诞老人,啊不,作为男朋友就没白忙活」\\

根据真昼的衣服和平时的饰品倾向,周觉得真昼应该会喜欢这一款式,所以选了它。看来是猜中了,原本紧张得加速跳动的心脏,现在终于有机会静下来了。\\

「周君经常选花卉主题的饰品呢」

「不喜欢吗?」

「不,只是觉得你有理由吧」

「没什么特别的理由……只是觉得你可能会喜欢,而且也容易搭配你平常穿的衣服。太简单的款式不符合你的品味」\\

真昼喜欢简单的东西,但太简单、轮廓太单调的款式她不太喜欢。

她喜欢利用曲线的优美设计和可爱款式,所以排除掉这些要素,选了她喜欢的花卉款式,最后就选到了这个。\\

真昼在这种时候不会说客套话,可见她的确喜欢。周松了口气,真昼则是轻轻盖上盖子,小心翼翼地捧着,嘴角微微摇动。\\

这么说不太好,但她的表情像是在忍耐着不笑出来。\\

「……我会珍惜的」

「谢谢。下次戴的时候要给我看,我会好好夸奖你的」

「请、请不要预告,而且也不用勉强夸奖」

「为什么?我是觉得适合你才买的,绝对很适合。适合的话当然要夸奖啊。夸女朋友可爱不行吗?」\\

女朋友愿意戴上自己送的礼物,既幸福又能大饱眼福,而且真昼被夸奖了,至少不会觉得不愉快。\\

周认为这都是好事,而且夸奖优点很重要,对建立彼此圆满的关系来说也很重要。不说出来的话,心意就无法正确传达,最好还是将这些都说出口。\\

这又不是什么奇怪的事。在周疑惑的时候,真昼却嘀咕着「修斗叔叔灌输的……」说的好像他被灌进了面包还是什么东西一样,让周不禁苦笑起来。\\

修斗确实说过,把觉得好的地方说出来,彼此都会觉得开心,所以要说灌输的话,确实可以这么说。看着父母的情况,周便确信了此言不虚,所以就拿来作为自己的行动准则了。\\

周老实地对嘴巴一张一合的真昼说「你这样也很可爱」结果她大概是害羞得受不了,发出了呻吟,所以周决定到此为止,不然就只是延长真昼的重启时间罢了。\\

周轻笑着等待真昼冷静下来,只见她反复深呼吸,好不容易才消退了脸上的红晕,然后看向这边。\\

不知为何,她的表情有些不满。\\

「……我也想给你,只有周君太狡猾了」\\

真昼在奇怪的地方吃醋,这次轮到周露出笑容了。

真昼似乎无法容忍单方面地获得,气呼呼地微微鼓起脸颊,那副模样真的很可爱。不过要是对本人说出口,周可以想象到自己的被褥会被抢去作遮掩之用,所以就把涌上心头的话咽了回去。\\

「我已经收到礼物了」

「……该不会指的是我吧」

「很遗憾,因为我已经预约了你的未来」

「好狡猾!明明我也预约了!」

「哈哈哈」\\

尽管被褥没有被抢走,但真昼却用小拳头捶着周的大腿。明明是受到攻击,却莫名地勾起周的保护欲。他一边说着「好痛好痛」,一边却觉得不痛不痒,甚至觉得挺舒服的。从某种意义上来说,这或许算是重伤。\\

刚睡醒的真昼比平时更好懂的感情让周翘起嘴角。随后真昼似乎察觉到他在想什么,送出一句「周君你个笨蛋」的可爱骂声,让周的脸更加绷不住了。
