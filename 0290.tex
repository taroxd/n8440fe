\subsection{两人的决意}

「他们两个都变了呢」\\

新年开学第一天,虽然还是一样从第一天就有课,但树和千岁都没有像平常那样抱怨。他们甚至积极地投入课堂,让不了解情况的优太都感到困惑。

看来他们两人在那之后好好谈过,决定要展望未来活下去。上课态度的转变也是其中一环吧。\\

午休时间,两人吃完饭后就盯着课本看,就连知道情况的真昼也是一半困惑一半佩服。\\

「是啊……这样不是很好吗?总比闷闷不乐要好得多」\\

至少周可以断言,现在的两人的状态比起郁闷要好得多。\\

「是啊。而且,这对他们两人的将来也是件好事。站在教他们的立场,学生的心态积极一点,学习效果也会更好」

「看你好像很开心」

「与其说开心,不如说是为千岁找到了目标而高兴……如果是勉为其难地学习,本来能学会的东西也学不会了。而且,身为他们的朋友,能帮上他们增进感情,我也很高兴」\\

真昼在学习方面是最受他们依赖的,但她似乎完全不觉得这是负担,反而还高兴地让微笑变得更加柔和。周笑着心想:站在他们的角度,能交到这样的朋友也真是幸运。\\

周和真昼的视线前方,是一如既往开朗的两人,正认真地面对着书桌的身影。\\

千岁有时会盯着课本露出一脸不解的表情,周虽然不能笑出来,但不免觉得有点有趣。\\

「不过,要是不稍微采取斯巴达式教育,感觉会很辛苦」

「我可不打算宠着她」

「……嗯,这一点我不担心。我担心的是千岁的承受能力」\\

真昼说不会宠着她,那就真的不会宠着她。\\

一旦决定要教,她就会带着笑容严格指导。虽然会配合当时的精神状态和身体状况做出一定程度的调整,但在达成目标之前,真昼会把笑容的施压当成鞭子和糖果,灵活运用。因为她知道这样才是为了对方好。\\

千岁现在的基础学力是否有余力接受所有的指导?周身为了解她成绩的人,对此有点担心,但真昼似乎并不怎么担心,而是用信赖的眼神看着千岁。\\

「我觉得没问题。千岁本身很会抓要领,之前只是缺乏干劲,或者说只是卡住了而已,现在已经不用担心了」

「也就是说,之后她不会昼儿昼儿地找你哭诉了?」

「不,这个……」

「昼儿~!救救我!」\\

正当真昼支支吾吾的时候,千岁的座位那边传来哭声般的惨叫和求救声。「这是两码事」真昼一边苦笑,一边从座位上起身,似乎也为被依赖而感到高兴。\\

「嗯,也不是说马上就能学会的」

「是啊,只能持之以恒了」\\

周感慨地心想,要是稍微认真一点就能学会的话,也就不会有什么辛苦了。不过她的努力很了不起,周也愿意帮忙,于是他也一起走向了两人。
