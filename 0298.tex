\subsection{甜蜜之日后的苦涩}

第二天上学时,周也是一脸平常的表情。

为了不让一起吃早餐的真昼察觉,周极力装作平静的样子,而真昼不知道有没有注意到他的异样,也没有指出他的不对劲。

无论如何,她没有提起这件事让周松了口气。他走进教室,发现千岁难得先到了,正在和提早到校的木户有说有笑地聊着天。

千岁很快就注意到周和真昼的到来,脸上顿时绽放出格外开朗的笑容,用力挥了挥手。周和真昼对视一眼,轻笑起来。\\

「早安!昨天怎么样?」

「早安,藤宫君和椎名。你们今天也很要好呢」\\

千岁和木户都对两人露出爽朗的笑容,但不知为何,周总觉得两人的笑容性质不太一样。\\

「你们早。你这么问是在期待什么」

「那还用说,当然是涂满巧克力诱惑你的昼儿」

「你白痴吗?到底在想什么才会得出这种结论?」

「哎~可是这种桥段很常见嘛」

「小千你的想法太跳跃了啦」

「我觉得可以直接说她很奇怪」

「小千从各方面来说都很有趣呢」\\

木户没有肯定也没有否定,只是笑咪咪地带过这个话题,放弃了吐槽千岁。她可能是觉得就算吐槽也没用吧。\\

「我觉得是千岁看太多奇怪的漫画或杂志了。到底要怎么想才会得出那种结论?真昼怎么可能那么做」

「为什么是巧克力……?周君又不怎么喜欢吃甜食,我想是不会高兴的。而且浪费食物的行为不值得鼓励,还不卫生,我不会做的」

「你们两个都认真回答,然后指责我的想法很邪恶,我好难过」

「你这不是知道自己很邪恶吗?」

「讨厌啦」\\

千岁害羞地把手贴在脸颊上扭动身体,周则是对她投以鄙夷的眼神。\\

「顺便问一下,你吃我的了吗?」

「还没。昨天只吃了真昼的」

「我想也是。就算你吃了,应该也是先吃了昼儿的。落差恐怕会很大,可能也算是件好事吧」

「我说你啊……」\\

造成落差的原因是什么,身为制作者的千岁明明最清楚,本人却一副若无其事的样子,这让今后要吃巧克力的周有很多话想说。\\

不过,千岁虽然半开玩笑,却没有恶意。如果本人吃不了,周当然会生气,但他也知道千岁在制作的时候就已经好好品尝过了,所以作为收礼的一方,周也不好意思把话说重。\\

「……我姑且问一下,真的能吃吧?」\\

去年的破坏力至今仍记忆犹新,为了慎重起见,周还是观察了一下千岁的反应,只见她鼓起脸颊,露出非常不满的表情。\\

「从昨天开始就太怀疑我了啦。我可是好好和昼儿商量,经过多次试吃才做出来的。用量都有控制过,可以吃的……要是刺激性强到吃不下去的话,你打算怎么办?」

「加进热牛奶里总能勉强中和了吧……」\\

周知道真昼有帮忙监制,但真昼的容许量和周的容许量不同,假如真昼误判了周的极限值,可以想见后果会相当悲惨。到时候就算要改变形式,周也打算吞进胃里,希望她能允许这点。\\

周明确表示自己不打算丢掉,千岁也一脸佩服地盯着板起脸的他。\\

「你不会剩下来呢」

「毕竟是收的礼物,而且姑且、姑且是为我做的吧。那当然要吃掉」

「因为很重要,所以要说两次嘛。我当然是为周你好好~地考虑过的哦」

「绝对不是为我,而是为我的胃和鼻子的伤害着想吧」

「嘿嘿!」

「这家伙竟然用笑来蒙混过去」

「小千就是喜欢提供惊奇嘛……要是捉弄藤宫君过头的话,椎名会生气哦?」

「这是昼儿监制的~」

「幸好有人踩刹车呢」

「就是说啊。以后我会慢慢吃掉的。如果是人类不能食用的东西,我会考虑丢掉并通报相关部门」

「没那么夸张啦~!」

「你们两个一大早就很有精神呢」\\

就在千岁噘起嘴来气呼呼地否认时,树也刚好来到教室,厚厚的围巾下露出一抹苦笑。\\

千岁因为男朋友的到来而眼睛一亮,不过从树的态度来看,周觉得期待他能帮忙制止纯粹是想多了。\\

「阿树,周好过分,超怀疑我的」

「在说什么?」

「昨天的巧克力」

「那是你的问题」\\

树毫不留情地否定了心爱的女朋友,然后对周投以同情的目光说「你也真辛苦啊」。周倒是希望他在千岁完成之前就阻止她。\\

「你到底站在哪一边啊~真是的」

「这次是站在周这边。我可是有试吃过哦」\\

树似乎有收到女朋友的本命巧克力,不过他作为试吃员也加入了周的巧克力制作,所以算是站在周这边的。

周事先听他说过「好厉害」「太扯了」等因为太辣而使得语言能力降低的感想,自然会感到战战兢兢,甚至怀疑。\\

「好过分」

「过分的是谁啊……」

「小千」

「是小千」

「是千岁呢」

「怎么连昼儿也这样」\\

这次好像没有人站在千岁那边。\\

「在一旁看着的我都知道她一直挑战极限,就算我阻止她也没用。周君会怀疑也是没办法的事」

「呜呜~」

「……放心吧,至少还是能吃的」\\

真昼完全无视千岁假装受伤的哭腔,对周露出微笑。既然真昼都这么说了,周也想相信巧克力不会对肠胃造成影响,但结果如何还是要看千岁加辣的分寸。\\

「昼儿之前可是吃得很淡定呢」

「我算是比较能忍受辣味的人,而且也挺喜欢吃辣的。周君不太能吃辣,可能会相当不好受」

「让人不敢下嘴的情人节礼物算个什么事……真昼,我吃的时候你在旁边看着吧」

「我会先准备好牛奶。另外,先吃点酸奶保护胃粘膜吧」

「这个建议好吓人哎」\\

从现在开始就必须担心肠胃受到伤害了,周对此感到恐惧,但他不打算逃避,所以决定在回家路上买些奶制品来保护肠胃。\\

他在脑中牢牢记住,打工回家路上要买牛奶和酸奶。同时他摸着肚子,把千岁的「不用那么警惕……」这句话当作耳边风,走向自己的衣柜去放外套。\\

光是想象就让胃里开始发烫,实在太可怕了。

周叹了口气,脱下外套,正好和刚到校的日比谷等人对上视线。\\

「早安」

「早安,藤宫君」

「早、早安」\\

周像平常一样向他们打招呼。

他知道自己比较怕生,但遇到班上同学还是会正常打招呼,也会闲聊几句。不管对方是谁都一样。\\

周留意着保持平常心,向日比谷和小西打招呼,她们也一如往常地回应。\\

小西讲话有点结巴,眼睛好像也有点红红的,但周不能提起这件事,只好把注意力从基于另一种原因而绞痛的胃上移开,挥挥手后看向衣柜。\\

小西那边似乎也不打算提起昨天的事,只是那困扰的眉梢比平时更倾斜了一点。旁边的日比谷好像也注意到了,却没有提起,而是轻轻推了推她的肩膀,催促她走进教室。\\

回想起来,从日比谷昨天那副心知肚明,或者说欲言又止的语气来看,她应该也知道小西的心意。毕竟是亲密的朋友,或许也商量过一两次。\\

既然如此,她应该会责怪周才对,日比谷却什么都没说,只是在一旁看着,完全没有责备周的意思。

这样反而让周感到过意不去,但日比谷和小西都没有对他说什么。她们本人以爽朗的态度从他身旁走过,周只是抿了抿嘴,然后就把外套折好,收进衣柜里。\\

\vspace{2\baselineskip}

「年轻人,你有没有度过一个美好的情人节?」\\

尽管感到尴尬,但周还是撑到了放学后的打工时间。一进到店里,宫本就带着不怀好意的笑容这么问道。\\

周昨天没有排班,为了弥补,他在第二天排了满满的工作班次。他隐约猜到宫本会这样调侃自己,所以脸颊不至于抽搐起来。\\

有伴的人大多都请了假,因此宫本才排了班补上,多亏有他,店里才能顺利运作。周很感谢他,但不想被他这样调侃。\\

「你那是什么角色设定啊?宫本明明也很年轻吧」

「我已经敌不过高中生的青春活力了」

「你才二十几岁而已,说什么呢……不过,情人节本身我算是度过了一个满足的时光」\\

周一边确认有没有客人点餐,一边小声回答,眼角余光瞥见宫本放心的表情。

幸好今天客人比较少,店员这边也能稍微放松一点。听说昨天因为是情人节,客人比平时还要多。刚才交班下班的前辈水濑在更衣室里对周投来带有些许怨气的目光,还说「昨天真是累死我了」。\\

「那就好。因为情人节经常发生争执……」

「我从朋友那里听说过,所以很清楚。照这样看来,宫本也有过实际体验?」\\

宫本那像是亲身体验过的感慨语气,不知为何让人感到一股哀愁。

到底怎么了……周凝视着比刚才更无精打采的宫本,这时大桥把客人送到店外后回来,做作地耸了耸肩。\\

「大地就算是这种人,也还算受欢迎的」

「什么叫『这种人』啊?」

「长得好看但内在空空如也,一点都不贴心」

「我可不记得除了你还对谁做过那种事」

「真是差劲透顶」

「是你先不贴心的吧?」

「才没有呢,真没礼貌。对吧,小藤宫?」

「我不予置评」

「为什么!?」\\

其实他们只是在互相攻击,或者说迁怒于彼此,要说也是两人都有各自的问题,但要是说出来的话,可以想见肯定会闹得他们不愉快。\\

周很喜欢宫本和大桥这两位前辈,可是现在的情况可能会演变成情侣吵架,他实在不敢开口。

他反而想叫他们别把自己卷进去,只是说不出口。\\

周像贝壳一样紧闭着嘴,保持沉默,以免遭殃。宫本似乎擅自解读了他的意思,不知为何露出得意的笑容。\\

「你看,连后辈都是这么看你的」

「宫本也别乱解释,我没有话要对你们说。还有,声音太大了,请安静一点」\\

虽然他们只是在偷偷争论,但要是再大声下去,就会传到客人的耳中,于是周竖起食指放在嘴前,示意他们安静。

两人一听就明显安静下来,周庆幸着他们还保有冷静,同时把视线转向员工通道的方向。\\

「我知道你们关系好到不会客气,但请在这里稍微克制一下。要吵的话,请到休息室去吵。虽然店长会偷偷观察就是了」

「啊,知道了,是我们不好,我们会反省的」\\

在丝卷的注视下,以争论为名的嬉闹似乎难度太高了,宫本率先道歉。周微微一笑,心想他果然不是真的生气。

大桥也垂下肩膀,冷静地说了声对不起。周松了口气,耸了耸肩。\\

\vspace{2\baselineskip}

「顺便问一下,宫本收到了大桥送的什么?」\\

营业时间结束,完成打烊工作后,周打开更衣室的门,准备换衣服回家,这时他忽然想起这件事。\\

周在前往更衣室时,从大桥那里收到了迟了一天的情人节巧克力。因为是在宫本面前收到的,他一瞬间吓得心惊胆跳,不过大桥送的是两枚十元硬币就能买到的长销巧克力,还说「抱歉,我现在缺钱!」这让周明显地松了口气。\\

昨天宫本和大桥都有上班,刚才宫本还抱怨不公平,大桥便吐槽说「我昨天不是给过你了吗?」所以宫本肯定是收到了巧克力的。\\

那么令人在意的就是经常捉弄自己的宫本了。关键的宫本又是如何呢?

周把巧克力收进包包,同时向宫本抛出疑问。宫本骂了一句「害我想起多余的事」但看起来不像真心感到厌恶,反而有些难以启齿的样子。\\

「那家伙扔给我了几个独立包装的雷神巧克力」

「挺好啊,又好吃,而且还有点豪华」

「哪有人会扔到别人脸上啊」\\

宫本一脸不情愿的样子,不过大桥扔巧克力给他的情景非常容易想象,周忍不住笑了出来,结果被宫本狠狠瞪了一眼。\\

「别事不关己似的啊」

「但你不讨厌吧?」

「我可不想接在脸上」

「那收到巧克力呢?」

「……无可奉告」

「这样啊」\\

打工才刚开始,周也是像这样敷衍过去的,所以没资格说什么。反正问不出来也不要紧,他没有不识趣地追问,而是开始折起围裙。\\

「真不可爱」

「现在才说这个也太晚了」\\

周笑着心想,你到底什么时候觉得我可爱了?宫本则咂了咂嘴,音量不小,让周又笑了出来。

从反应来看,宫本果然也很高兴收到大桥的巧克力。周心里暖洋洋地穿上西装外套,而宫本一边发牢骚,一边粗鲁地解开领带,扔进衣柜里。

不过,从他的动作中感觉不到怒气,周心想差不多可以了,于是重新转向宫本。\\

「我有件事想问你」\\

今天他有件事想问宫本。\\

「怎么了?这么郑重其事的」\\

或许是注意到周的表情不像平常闲聊时那样,宫本也整理好衣服,站在他的正对面。\\

「听大桥的说法,你好像经常被女性喜欢上吧」

「怎么?你觉得看着不像?」

「为什么是这种反应?我觉得你应该是很受欢迎的。只要不被别人看到你对大桥的态度,就是个爽朗又会照顾人的好人」

「你这是在夸我吗?」

「对大桥的态度那段不算是在夸」

「啰嗦」\\

周再次体会到这个人唯独在跟大桥有关的事情上特别不坦率,但周想说的并不是这个,于是他向宫本投以怀疑的目光,然后轻笑一声。\\

「所以,怎么突然问这个?」

「……那个,我可以先确定一个大前提吗?」

「什么?」

「就是宫本对大桥是怎么想的」\\

宫本又咂了第二次嘴,但周毫不畏惧地继续说:\\

「别咂嘴啦。我也知道被人这么说很讨厌」

「……所以呢?」

「如果觉得难以回答的话,不说也没关系。那个,有别人想你示好的时候,你是怎么做的?」\\

如果只论拒绝的经验,身边最丰富的人就是优太了,但如果可以的话,周不想找优太和树商量。

周无意间听总司说过,他虽然嘴上不饶人,语气也很强硬,但他从很久以前就一直喜欢着大桥,所以周认为他的立场最接近自己。\\

因此,周才想问他。\\

宫本没有把这突如其来的问题当作玩笑,只是眨了几下眼睛,然后缓缓吐了口气。\\

「我只说了抱歉,然后就正常拒绝了……难道我被误会成了会跟不喜欢的对象交往的类型?」

「哪里哪里。只是、那个,宫本在我眼中也很受欢迎,感觉这种事发生过很多次,所以我想知道你会不会很烦恼」

「怎么,你是想说,你是被表白了,虽然已经拒绝但是内心有愧?」\\

宫本大概察觉到周想说什么了。

他露出一丝苦笑,垂下眉梢看着因为被他说中而僵住的周,说「你真是正经又体贴啊」。\\

「不过,这算是你的优点吧。应该说,再往前都没发生过那样的事,反而还比较意外」

「因为我……在遇到女朋友之前,是个不想跟人扯上关系的阴沉家伙,所以也没人喜欢过我。自从改变自己和女朋友交往之后,周围的人都知道我对女朋友很专情……好像也因为这样,没有人来打扰我们」

「哇,真想看看藤宫你被迷得神魂颠倒的样子」

「还、还是别了吧……」

「所以,看你的样子,应该是有人知道你的情况还来告白,然后你就拒绝了,但心里还是觉得痛」

「……是啊」

「哎,拒绝本来就是板上钉钉的事情,你也没办法啊」

「关于这一点,你说得没错,我一开始就没打算答应对方。那样会背叛女朋友,而且我的眼中只能容下一个人,所以不管发生什么事,我都不打算接受对方」\\

这一点无论何时被人问起,都不会改变。

周不会选择真昼以外的人是众所周知的事实,他对真昼的感情也已经坚定到足以让他本人如此断言。即使有人在他面前哭泣,他也绝无可能选择别人。\\

只不过,周对于拒绝并伤害对方这件事还是感到自责。这是他做出的选择,所以即使无法接受告白,他也做好了会伤害到对方的心理准备。

周不知道该如何消化这股想吐却不能吐的郁闷和痛苦,于是来请教宫本这个前辈,而宫本则是眯起眼睛,短促地呼出一口气。\\

「那你就该让思考停在那里。当时会感到愧疚也是没办法的事,但最好别一直放在心上」\\

他的话语流畅而尖锐,震动了空气。\\

「我算是看过周围很多八卦的人。根据我的经验,喜欢上你这种类型的人,多半是那种会注意到细节的纤细乖孩子,不是随便交往,而是认真想要在一起的类型。虽然这只是我擅自的假设,不过有猜对吗?」

「……没错」

「这种类型的人——虽然这只是我的想象——如果你心里有罪恶感,她会不会觉得是自己的错?就像你因为拒绝而感到内疚一样,对方也会觉得是因为自己的表白」

「这……」

「你没有意愿交往是无可动摇的事实。一直放在心上的话,对那个女生和你的女朋友都不好」

「……是」

「虽然这么说不太好,但甩人的那一方如果一直耿耿于怀,对方就永远无法向前看了。为了让她对自己的感情做个了断,在你的温柔变成自私之前,最好还是果断一点。温柔在某些情况下也会变成利刃」\\

正如宫本本人所说,他出于温柔的话语,以某种言语利刃切开了周内心淤积的沉淀物。不过,他也明白他是为了让自己吐出心中的疙瘩。

这和周那种无意间伤害他人的温柔完全不同。

周咬紧嘴唇,铁锈味在口中扩散开来。他将其咽下,心想这和他带给别人的痛苦相比根本不算什么。\\

「……我可没叫你自责,只是说你现在回头想想也行。你对自己这方面很不温柔呢」

「我只是又一次深切体会到自己伤害了对方而已」

「啊!你就是这种地方!」\\

宫本不顾自己梳理整齐的头发被弄乱,用手指挠了挠头,然后夸张地深深叹了口气。\\

「恋爱能顺利才稀奇,所以大家都会看开,觉得这也是没办法的事。毕竟某人的幸福往往意味着某人的不幸。大部分人在成长过程中都会学会一定程度的容忍和观念,知道事情不会总是如自己所愿」

「……宫本在这方面好像很看得开呢」

「因为我没有藤宫你那么温柔,对他人也没什么兴趣。我没办法那么关心别人,而且既然不打算为对方的人生负责,拒绝之后最好就别再放在心上了。我不想让对方抱有奇怪的期待,说到底只是陌生人,应该要妥协才对」\\

宫本转换想法的方式是周所没有的,这或许是周应该学习的地方。

周也不是博爱主义者,比起不熟的陌生人,他会优先照顾亲近的人,也能区分对待。\\

只是,在自己没有余力的时候,束手无策的时候,要冷酷地把自己和亲近的人与路人分开——也就是将一些人不再当作自己重要的人,周不免产生犹豫。他难以将这些人断定成生活在与自己无关的地方的人。\\

周现在明白了,这种半吊子的天真反而会伤害到别人。\\

虽说理解了,但周也没有果断到能够马上转换想法,不过他强烈意识到最终这样对双方都没有好处,于是点了点头。\\

「还有,我不得不习惯看开。说起来,那家伙的对象一直在变,我根本没办法一一在意」\\

宫本最后补充了一句像是抱怨的话,周瞪大眼睛凝视着他,只见宫本露出了尴尬的表情。\\

「我只希望她最后能待在我身边」

「这就是爱啊」

「吵死了」\\

宫本一定也以自己的方式一直守护着大桥,有时也会感到心如刀割吧。

尽管如此,他依然对大桥一心一意的样子十分耀眼,让周眯起眼睛,嘴角柔和地翘起。

