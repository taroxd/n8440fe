\subsection{天使大人与生日}

周向树和千岁寻求完建议后,总算选好了礼物,在真昼的生日当天以一副紧张的表情看着她的背影。\\

以车站前的可丽饼屋卖的特制可丽饼(冬日限定非常莓果特辑)为报酬,周说动了千岁帮自己忙买了个东西,并把这东西也加进了礼物……可现在周却苦恼着该在什么时候把这礼物送出去。\\

而那过生日的本人,正和往常一样做着晚饭。\\

虽然周不清楚菜单,不过真昼似乎是在做和食的样子,但怎么看她都没有什么特别的感觉,表现得跟平时一样自然。

从当事人身上完全感受不到生日的氛围。不如说那淡定程度,简直让人觉得她是不是根本就不记得这回事。\\

甚至到了晚饭端出来后也没有发生变化。两人在餐桌上虽有对话,但进餐还是一如往常。\\

周真的拿不定主意该什么时候把东西交给真昼,于是看向藏在沙发后面那放着礼物的纸袋,皱起了眉。\\

总之周先收拾好了餐桌。等他回到客厅的时候,发现真昼正坐在那刚好两人位的沙发上看着似乎是自己带来的书。

就连看书的模样也美如画作,到底是不虚天使之名。\\

虽说不知为何周对要不要坐在真昼旁边有些微妙的犹豫……但一直退缩也不是个办法,于是周提起放在那里的纸袋,坐到真昼身旁。\\

真昼突然抬起了头。

大概是注意到了周的气息和纸袋摩擦的声音,真昼那焦糖色的双眼看向了周,然后又移向了周拿着的纸袋。

真昼的表情似乎有些不解。看来,到了这个地步,她还没有注意到自己生日的事情。\\

「嗯,给你的」\\

周把纸袋推出去放在了真昼膝上,使真昼脸上更加茫然。\\

「这是什么」

「今天不是你的生日吗」

「是倒是……话说为什么你会知道。我可不记得自己有跟谁说过这回事」\\

真昼的眼里微微露出警戒的态度,但听到周说「你上次把学生证落屋子里了吧」之后,或许是接受了这个解释,便恢复了平时的表情。\\

「其实,没必要在意的。反正我也不过生日」\\

那冷淡而透出排斥感的声音,应该不是周听错了吧。

真昼那眼神,如同对生日这词汇本身抱着不知从何而来的忌讳感。\\%不知从何而来不通顺可以删

周明白了,原来如此。\\

明明是生日,她的态度却毫无变化,其原因,并非是不记得生日的事情。

因为生日很烦人所以故意忘掉的——应该说是这么回事。

若非如此,她也不会用那种语调吧。\\

「啊这样啊。那这个就当作是平常受你照顾的回礼吧。权当我一厢情愿想要报恩」\\

「你不过生日也没关系,但作为感谢平日照顾的回礼是另一回事,所以这就当作我表达感激的心意而不是生日礼物」周这么解释,然后把礼物塞了过去。

每天都吃着这么好吃的饭,偶尔还来帮忙打扫屋子,虽然都是小事,但也实在是受照顾了。即便只是一点一点的,周也想要慢慢的回报真昼。\\

周虽然很轻易地就退让了,但却执意要把礼物送过来,这让真昼有些混乱,尽管她有些困扰的皱着眉,不过还是接过了礼物。

真昼的视线,集中在纸袋里面用另一层袋子包装的东西上面。\\

「我可以现在打开吗?」

「嗯」\\

看见周点头,真昼紧张地从纸袋里把盒子拿出来,小心地打开包装纸解开缎带。

看着赠与别人的礼物在自己面前慢慢被打开,让周格外的紧张。\\

里面放着的是树推荐的护手霜。因为是套装一起卖的,所以这个大盒子里还附带着一点小点心。\\

顺带一提这并不是那种带有香味的时尚品,而是以没有香味、适合家务、亲和肌肤、滋润保湿为卖点的东西。

周也确认过网上的评价,效果应该是不用担心的。\\

「抱歉,不是什么值钱东西。看你干家务手应该会干吧。虽然也有带香味的,不过那种你估计有了所以没买。听说这东西对皮肤好而且挺有效的」

「实用品呢」

「你的话更看重实用性吧」

「是的。谢谢你了」\\%语气词 主要是不想重复上上句

看着真昼微微露出笑容,似是在说「你还挺了解我的嘛」,周也稍稍放松了嘴角。

看来印象不坏。\\

之后虽然还有一件东西……但要当面打开周还是觉得有些害羞,如果可以的话周还是想真昼回到家再发现那个东西。\\

可事不如愿,在把护手霜放回纸袋里的时候,真昼似乎注意到了纸袋里还有一件东西,于是眨了眨眼。\\

「……是还有一件东西吗?」

「啊。呃,那个,怎么说。就是个来自于独断和偏见的附赠」

「附赠?」

「……附赠」\\

周撇开视线,只回答了这么一句。真昼歪着头搞不明白周的意思,但她觉得不如直接打开来得快,便从纸袋里把那东西拿了出来。\\

为了让那东西尽可能不起眼,周用了跟纸袋一个颜色的包装,还将其塞在最底下,但果然这个大小还是很显眼。不如说真亏能在打开护手霜的盒子之前都没被发现。\\

那东西的包装并非盒子,而是聚酯塑料袋。其大小,正好够真昼双手抱住。

看着她把那深蓝色的丝带小心地解开,周想着「我要不要先离开一下」的时候——真昼正好把里面的东西取了出来。\\

她用两只手小心地把里面的东西提了起来,相当意外地眨巴着她那两颗大眼珠子。\\

「……熊?」\\

真昼说着的,便是那东西的原型。\\

那是一个不算太大,大概小学生抱着大小正适合的布偶。

布偶那近于真昼头发的淡色的软毛是其特征,而它透出一股天真气息的脸上缝着的乌黑圆润的眼珠里正映着真昼的身姿。\\

「都高中生了还玩偶啊」她说不定会这么想。\\

尽管如此,听了千岁「女孩子不论长到多大都会喜欢可爱的东西」这样的建议后,周就选择了这个。\\

再怎么说男的一个人跑去买这东西实在是非常害羞,周便以车站前的可丽饼为报酬让千岁陪着自己去买了。

结果从挑选到打包周一直在被千岁笑嘻嘻地看着,说不定其实一个人去买羞耻感还会少一点。\\

「……觉得女孩子会喜欢这个吧所以」\\

周挠着头,不知是在跟谁解释般嘟哝了一句。\\

这种事周实在是不擅长。

不如说给异性送礼这件事,除了小时候送妈妈的以外就没有干过,周甚至没有想到自己会去做这种事情。\\

从男的那里收到这么可爱的玩偶会不会让她受不了啊……周偷偷瞄了眼真昼,看到她正紧紧盯着熊的脸不放。

也不知是高兴还是不高兴,真昼只是呆呆地望着布偶熊。\\

「嗯,不喜欢的话扔了也行」\\

「如果不喜欢的话那也没办法」周想着,玩笑般地说了一句,结果真昼却皱着眉刷地把头扭了过来。\\

「那种事不会做的!」

「嗯、嗯。看椎名的性格我想应该也不会的」\\

真昼的否定比预想要强烈,令周一边退缩一边点了点头。而真昼则再次看向手中的熊布偶。\\

「……我不会做,那么过分的事情的。会好好珍惜」\\

真昼纤细的手腕,像是要将其拥入怀中般,紧紧抱着熊布偶。

那姿态看上去,既像是孩子不愿喜欢的玩具被拿走,又像是母亲慈爱的拥抱。

一句话来形容,便是极为珍重地抱着布偶。\\

仿佛能配以啾的音效一样,真昼紧紧把布偶抱在怀中,并稍稍垂下眼帘看着它。\\

那脸上的表情,既不是平常的那种冷淡的表情,也不是被周的脱线惊呆时的表情,而是心安、柔和、泛着慈爱的、爱惜的表情。

还有她那天真无邪的纯洁微笑,美丽而又惹人怜爱,让周不禁屏息。\\

(——不该看的)

望见这样的表情,周不由自主地有了反应。\\%这边感觉有偏义了,以中文来说

让数一数二的美少女露出了这样的表情,还被自己看见了,这一事实,就算没有恋爱上的喜欢,也足以让周心跳加速了。\\%用数一数二感觉要补词

真昼那珍惜地抱着布偶,露出淡淡微笑的姿态,已经可爱到不论谁看见了大概都会着迷的程度吧。就算是自认为无欲无求的周也差点入了迷。\\

为了确认自己脸上积蓄了多少热度,周伸手捂了下自己的脸。手上传来了比平时更加明显的热感。

由于自己害羞得太过明显,所以周用真昼听不到的声音「……靠」地骂了一句。\\

幸好,真昼正紧紧抱着熊布偶,把半张脸埋进里面,因此并未注意到周。

那副模样也是一样地可爱,让周好不容易才忍住了发出怪声的冲动。\\

「……这么喜欢的话,我也就心满意足了」\\

周想着说些什么于是挤出了这么一句,真昼则瞄了这边几眼。\\

「……我是第一次,收到这种东西」

「咦,以你的人气这算是日常贡品吧……」

「你把我当成什么了……」\\

这带有稍许无奈的声音与表情,反而让周安心了下来。这大概是因为不再需要直视那样的表情了吧。\\

「……我没有告诉过别人过我的生日。因为不喜欢生日,所以我从来都不说」\\

「不喜欢」真昼在断言之后将视线移向了布偶熊。

真昼看着布偶的眼神很安详,与嘴中的话语截然相反,不知为何却让周觉得不大自在。\\

「一般,不认识的,或者没什么关系的人送我礼物我也觉得可怕所以不会收」

「我送的倒是收了啊」

「……藤宫同学又不是不认识的人」\\

真昼小声地回答着,然后把脸埋进布偶里仰头看向了周。周则开始后悔自己直视了她这件事。\\

她那无意中向上看的眼神,还有放松下来的,与年龄相应的天真感流露出来的表情,实话说,相当令人怜爱。

那可爱令人不自觉地产生了想要摸摸头的冲动,于是周在不经意间把手伸向了真昼的头,然后慌忙用力收回来。\\

「……怎么了吗」

「没、没什么」\\

不知是注意到了周一瞬间动了的手,还是察觉到了周那几近爆发的心痒感,真昼咚地歪了歪头。

仅是这样,周的视线便差点被夺走了。美少女这种生物还真是可怕。\\

但是再怎么说,直接回答因为可爱所以看呆了,周还是会感觉羞耻,而且就算说了周也确信真昼只可能回答「啊?」。\\

而且,如果那样说的话周在各种意义上都会死亡,所以还是决定把这个冲动深藏于心。\\

「……谢谢你,藤宫」\\

周撇开了脸,而真昼纤细的声音再一次传进了他的耳中。

%************************************************
% 后记:※娇化输出 20%
