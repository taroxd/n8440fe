\subsection{不安和安心}

「我声明一下,我可没有怀疑你哦?」\\

树他们回去之后,真昼用干巴巴的语气干脆地说道,让周吓了一跳,不知道是在说什么事。

慢了一拍,周才理解到这是当时那件事的后续,他连眨了几次眼,旁边坐在沙发上的真昼则是微微垂下眼眸,轻轻捏住他的袖子,看起来有些无助的样子,或许是因为真昼也感到有些不安吧。\\

「我很清楚周君,那个,爱着我。而且我也知道周君不是会违背约定的人」

「可是这也说明我的所作所为让你担心了,我会多加注意的」

「我认为你受到赞赏本身是件好事,而且这也不是我能限制的」

「可我不希望你有不好的感受」

「虽然我觉得周君最好更加警惕一下靠近你的女生,但老实说,被喜欢这件事也不是能预防的。这说明大家认可你,而且我也明白,比起被大家讨厌,还是多少受到些喜爱比较好」

「嗯,是这样没错。那个……虽然说得有点远了,但我有个单纯的疑问」

「请讲」

「刚才我也说过类似的话。假如有个女生喜欢我,那她为什么会来接近我?」\\

周无法理解这一点。

向喜欢的人表达好感,这本身倒也合理。

可是,如果补充一个前提,也就是喜欢的人已经有了明确的伴侣,那事情就另当别论了。

虽说人类的感情无法完全控制,但明知对方有恋人或伴侣却依旧采取行动,这种想法超出了周的理解范围。\\

「采取主动行动,就表示想要我做出回应,对吧?」

「嗯。我认为这是希望你对她抱有好感,所以才接近你,期待你能够回头看看她」

「说得难听一点,就是想要从你身边抢走我,对吧?」

「是这样没错」

「我明明这么喜欢你,却会被人认为有机可乘,真让人伤心。不知道是看的哪儿,才会觉得我会离开你」\\

先不论两人是否算是在众目睽睽之下秀恩爱,但可以肯定的是,周无论何时何地都很疼爱真昼,也很珍惜她。\\

他没有和真昼吵过架,也从未对她冷淡,两人一直互相尊重,过着平静安稳的生活。周对此有自信,真昼也认可,而且周围的人也认为他们不会分手。

周从未关心过其他女生——不如说过于没有兴趣,甚至让树都感到无语了。如今再说周跟其他女生有一腿,认识他的人肯定会一笑置之,觉得根本不可能。\\

假如真的有女性想要攻陷周,周又被当作是稍微示好就会轻易改变心意的人,那实在是太遗憾了。而且,如果他对真昼的心意被如此蔑视,那他会不爽到直接将对方拒之门外。\\

「说到底,要是有人明知真昼的存在还来接近我,我当场就会提高警惕。看我平常的样子,还能看不出来不成?肯定得坚决拒绝」\\

周是公认的交友范围极窄的人,而且私人空间的范围也比较大,众所周知,一旦跨越了周划出的界限,在那一刻就会成为他警戒的对象。如果对方从结果上还会伤害真昼,那就更不用说了。\\

往本来圆满的关系横插一脚,使其中产生裂隙,能做出这种事的人究竟是怎样的脑回路?周无法理解、也不想去理解,光凭这一点,周就不会正眼看对方了。\\

「周君在这方面比我还洁癖呢」

「与其说是洁癖,不如说很普通吧?会自然地觉得讨厌」

「我明白你的意思。我也会在心里贴上标签加以区别,觉得对方就是会做出那种事的人」

「……真昼,你这不也有你刚自己说的洁癖吗?」

「是那样没错。不过,周君会比我更明显地用一条线来拦住别人。我是筑起高墙,周君就是拒绝了」\\

这么说来,周的拒绝反应或许比真昼更加激烈。\\

「没办法,毕竟不想被你误会是最大的理由。而且,我不会想和那种在异性关系上优柔寡断、随波逐流的人有个人上的往来。所以我会注意的,能够在产生奇怪的误会之前预防,那还是做好预防为妙」\\

周认为,将来可能导致误会或吵架的因素,最好还是从源头上排除。

周和真昼基本上都是冷静地倾听对方的话,寻找妥协点的类型,因而几乎不会吵架。如果要说怎么做才会产生摩擦,那就是其中一方明确地做了坏事或违反约定的时候。\\

无论是事实还是误会,一旦让对方产生怀疑,便理所当然地会产生不信任感,所以重要的是尽量让对方没有地方可以怀疑。\\

不做出可疑的行动,做好报告、联系、商量,有时还要请公平的第三方作证。这些注意事项应该随时记在心里。\\

「回到正题,所以就算有女性对我做出那种行为,我也会感到困扰,只能拒绝。虽然我怀疑到底有没有那种人就是」

「你怀疑这点的话,那说来说去都是在兜圈子」

「可是……现在的确没人对我有那个意思」

「那是因为周君对恶意很敏感,对好意却很迟钝」

「……真昼小姐,你是不是在生气?」

「与其说生气……应该说,我回想起了之前费了多大劲才让你察觉到……」\\

真昼把手轻轻放在额头上,有些疲惫似的叹了口气。而周正是导致她费心的那个人,他没法轻易安慰,只感觉脸上连带着笑容一起僵在那里。\\

周和真昼从大约半年前开始交往,至于在那之前的过程,还从一直关注着他们恋爱进展的树和千岁那里,得到了令人不太愉快的背书「基本是因为周又迟钝还没有自信,磨磨蹭蹭的,所以才花了那么久」。\\

周也认识到自己因为自卑和疑虑,没能完全相信真昼的好意,所以非常能理解真昼在这方面费了多少功夫,他甚至还感到过意不去。\\

「啊不是,现在我肯定知道当时你在很拼命地表示了!只是我没有自信而已!」

「要是你说这样都还没注意到的话,我想到那会儿的自己,简直就像捂住脸了……周君,就算明明白白把好意摆在面前也很迟钝呢」

「对不起」

「不、不过,我知道周君现在都能察觉到,而且也很会表达!总、总之,你现在对我的好感非常敏感,能够接下好意并回应,但我感觉你会不会对别人的,尤其是异性的好意,依然很迟钝呢?嗯,就是迟钝」

「说、说得那么坚决……?」

「因为的确是到了毋庸置疑的程度……这也证明周君的视野里只有我一个,我也没法生这个气」

「视野里只有你不是理所当然的吗?」

「……你真的可以脸不红气不喘地说出这种话,这样不好」

「咦咦……?」

「我、我知道只有我……因为我被爱着」

「嗯,的确如此」\\

真昼扭扭捏捏地说着,声音到最后仿佛害羞似的越来越小,让周觉得她可爱极了。他一边接住她撒娇般倾斜过来的身体,一边把手轻轻绕到她背后。\\

他们已经能够正面相互表达爱意,也能正面撒娇了。\\

周用全身接纳着真昼的感情,万分珍重地抱住她柔软的身体,以免她的身体受伤,同时将那小小的身体中所蕴含的不安和焦躁分担、承受,然后消除。\\

剩下的一丝丝僵硬,也随着周用手掌缓缓抚摸而消解。她把身体靠在周的身上,仿佛把全身都交给了他似的。周想要进一步感受她的重量,便慢慢改变姿势,保持着真昼在身上的状态,平躺到沙发上。\\

真昼像是压在周的身上一样——不如说是周使得姿势变成了这样。压在上面的真昼惊讶又害羞地眨了眨眼,想要起身,周没有放开环在她背后的手,用行动而非语言表达出了他的拒绝。\\

「……很、很重吧?」

「不重不重……而且,我还希望你靠得更近一点」\\

周对摇晃着头发、有些慌张的真昼细语着,既是想让她撒娇,也像是自己撒撒娇。旋即真昼白皙的脸颊泛起淡淡的红晕。\\

「既然你这么说,我也希望你更靠过来一点」

「反过来比较好吗?」

「笨蛋!」\\

真昼猛地把脸埋进周的胸口,发动直接攻击。周忍不住笑了出来,而真昼察觉到周是因为胸部的动作而笑,便将头抬起一半,露出泛红的脸颊和尖锐得很可爱的眼神。\\

周虽然装作从容的样子,却也明白自己做出了还算大胆的发言和行动,于是他为了掩饰,把手掌轻轻放在真昼抬起的头上,以不缠到发丝的动作轻轻抚摸。光是这样,真昼抬起来的头又没入了周的胸口。\\

咕咚咕咚,迁怒一般摇头晃脑的直接攻击都十分可爱。周笑着承受她的头槌,同时尽情享受着她特有的甜蜜、温暖和柔软,将手指缓缓划过那保养得当的亚麻色河流。\\

「……周君」

「嗯?」\\

两人互相接触了一会儿,真昼忽然用平静的语气叫了一声,周不解地微微偏头,只见她静静地望着自己。

她的眼中没有刚才的动摇和羞耻,只是真挚地凝视着周。\\

「我并不认为所有表达好感的都是那样……我想,也会有女孩子把忍耐不住、压抑不住的心情向你展现。也许有一天,会有人面对面说喜欢你。到时候,你——」

「我会拒绝……我会慎重地听她把自己的心意说完,但不打算接受」\\

不用听到最后,周也知道她想说什么。\\

并非所有向有对象的人表白的人全都怀有恶意,即便是对真昼一心一意的周也心知肚明这一点。把憋在心里的心意在决堤之前宣泄出来,自然也是会有的。\\

周完全不打算断定所有这类行为都是不好的,那种感情本身也不是别人可以否定的。感情本身的产生,终究无人可以控制。产生的感情无法承受,等到别无他法而决堤之时涌向对方,也是时有的事。\\

只不过周不打算接受这种好感罢了。\\

「我喜欢的是真昼,考虑将来时也只有真昼。所以我不会让其他人进入我身边的空间。这里只有真昼一个人」\\

即使拒绝会伤害到对方,周也无法退让。

周只爱真昼一人,不可能选择真昼以外的人。

而真昼也不可能把周让给别人。

正因为彼此都理解这一点,两人的言语和心意都重叠在一起,将不安消除。\\

「所以你不用担心。我只需要真昼」

「……嗯」\\

周感到真昼仿佛放下了心,加重了压在自己身上的重量。这令人舒心的重量让他眯起眼睛,轻轻拥住了怀中的温暖。
