\subsection{白色情人节的安排}

「咦?藤宫你排了白色情人节的班?」\\

今天的打工也结束了,宫本在回家前看了看写着员工联络事项的白板,一脸意外地看向周。\\

刚才文华说,三月份的班表已经排出来了,但还没有发布到群组里。

班表会在之后发送,周打算先在白板上确认,以便管理自己的工作日程,结果宫本就对他这么问了一句。

从宫本的话中可以听出,周拜托的事情已经顺利实现了。\\

「啊……关于那件事」

「咦?你们该不会分手了吧?」

「能不能不要说这种不吉利的话,太晦气了。再说要是分手了,表情上想看不出来都难。那样的话我都没自信还能来上班,而且也就失去打工的意义了。我会一直消沉下去的」

「抱歉抱歉。那你怎么偏偏要在白色情人节排班?」

「啊,那是因为……那个,我女朋友太喜欢我了」

「为什么突然跟我秀恩爱?」

「不是在秀恩爱」\\

周瞪了眼一瞬间露出傻眼表情的宫本,用眼神示意他把话听完。\\

「是先有了这个前提,然后我女朋友一直想来我打工的地方看看。她想看我穿制服工作的样子」

「啊。然后你不好意思让她看,所以一直劝她不要来。这么说来,你好像说过这件事」

「……就算不是女朋友,如果被亲近的朋友看到自己不习惯、慌慌张张的样子,或是犯错被骂的样子,难道不会觉得难为情吗?」

「我懂你的心情。我也被莉乃笑过」

「我能想象她笑嘻嘻的样子」

「那家伙是真的会狠狠地笑。先不说这个,我不会在别人面前骂人,再说藤宫也没什么需要被骂的地方吧。和某人不一样,会细心地使用器具,也没出过岔子」

「之前连续弄坏虹吸壶的事件有点……」

「那次连店长都板起脸了,虽然没有发火,但还是狠狠地警告了一番。莉乃也意识到情况不妙,严肃地接受了训话」\\

说是优点也好缺点也罢,大桥是个干脆——不,粗枝大叶的人。虽然现在会好好地使用器具,但一开始打工的时候,她似乎有好几次都因为没控制好力道而弄坏虹吸壶。

一次就算了,但她弄坏了好几次,结果被丝卷狠狠地警告了一番。因为没有被训斥,她自己反倒心里过意不去,从此以后就开始小心对待器具了。\\

宫本知道当时的情况,他露出追忆的眼神,而周则是继续看着贴出来的班表。\\

「先不说这个,也就是说你已经习惯了工作,所以觉得可以让她你女朋友看到你工作的样子了」

「差不多就是那样。还有,她好像很喜欢白色情人节的菜单,所以我想说机会难得,希望可以让她吃吃看」

「原来如此」

「啊,当然,回礼不是只有这一点,下班后我们打算一起去买东西,所以才拜托店长帮我排了超短的班。已经得到店长的许可了」

「店长肯定会笑咪咪兴冲冲地爽快答应吧……」

「嗯,算是吧……」\\

白色情人节的出勤时间之所以排得那么短,是周和丝卷商量之后的结果。\\

就算是打工,如果只是稍微工作一下就走人,没有事先申请也是不行的,所以周详细地说明情况后提出了请求。

虽然对本人肯定是说不出口,但周在心里打着小算盘,认为丝卷应该会理解他的苦衷,甚至会积极地提供协助,所以才提出申请的。结果不出所料,丝卷比想象中还要有干劲地答应了,周在松了口气的同时也涌起了一丝愧疚。\\

周对那副调侃地笑着的「你也是个坏蛋」的表情皱了皱眉,然后叹了口气,低下头。\\

「宫本你那天也有排班,所以我想先跟你道歉。那天我会提早下班,真的很抱歉」

「嗯?啊,没关系啦。要是你不以女朋友为优先的话,我反而会有点担心呢。我知道你就是这样的人,节日活动是很重要的」\\

「因为这种活动会成为美好的回忆啊」宫本爽朗地笑着,划过班表上的白色情人节排班行程,耸了耸肩。\\

「再说,店长是在你提前申请的情况下调整的排班吧?所以把节日活动考虑在内,排班的人比较多。店长是判断这样可行才会安排这个班表,就算忙不过来也是管理方的责任。明白了吗?」

「……谢谢」

「还有,我和店长接下来都不太打算把你们算进战力里」

「咦?」

「啊。这么说可能会引起误会。不只是你,茅野也是」

「茅野也是?」

「你们从春天开始就是考生了吧,一定会有很多事情导致无法脱身,像是模拟考和补习之类的,不能太勉强你们吧」\\

宫本这番话显然是考虑到周的情况。

周和茅野从下个学年开始就会成为高三考生,会比高二的时候更忙,日程安排也会将考试纳入考量之中。

明知会变得忙碌,周还是选择继续打工,在存到自己决定的金额之前,他不打算辞掉这份打工。\\

即使如此,应该也会有无论如何都不能推掉的事情,所以周认为自己必须在学校更加集中精神,这样才能只在有那些要紧事的时候再请假。\\

「啊,我不是叫你别打工。我知道你要以考试为优先,只是不能把你算进一定能够调动的战力里而已。以能力来说,你确实可以算进战力,这点你放心」\\

宫本像是看穿了周心中的不安,笑着说道。他的手指滑过现在员工姓名栏下方的空白部分。\\

「我和店长都认为,藤宫和茅野都是能顺利完成工作、不会偷懒、准时出勤的认真员工。所以,我们才觉得如果店里真的特别需要人手,你们怕是会过来……考虑到这一点,我打算增加人手来应对。而且,藤宫你也不打算长期打工吧?」

「……我打算夏天的时候辞职。那会儿应该存够目标了,而且也要开始认真准备考试了」\\

周打算在高三退出社团活动的时期辞掉打工。关于这一点,他一开始就和丝卷商量过,并且得到了她的理解。

尽管还没有明确的目标,但周想买的东西大多是几十万日元的市售品,如果是订制的话,要花的钱大概还要翻一番。\\

他排班时就是以买得起那些东西的金额为基准,目前也顺利地存着钱。照这个速度,像刚才宫本说的那样,等到夏天的时候即便辞职,存款也不成问题了。\\

「对吧?我和店长都知道极限差不多就在那会儿,所以会考虑到这一点来安排班表。所以你不用担心」

「谢谢……宫本,你讲的都是店长那边的意见呢」

「因为店长找我商量过。我在这里待得比较久,所以店长会找我商量这些事」\\

宫本嘴上嘀咕着「店长太会使唤打工人了」,看起来却并不排斥,周对他微微一笑。\\

「我之后也会忙着找工作和写毕业论文,没时间管别人的闲事就是了。店长也说过,这种事情就像雇用学生会产生的新陈代谢一样,你也不用放在心上」

「好」

「……顺便问一下,你要给女朋友的这个那个,目标金额很高吗?」\\

宫本似乎很在意这件事,压低声音问道。他看起来有些难以启齿,但眼中流露出好奇的目光。周挠挠脸颊,斟酌着该怎么回答。\\

「说高吧,嗯,确实高。以学生的身份做到那种程度,不知道会不会被觉得太沉重」\\

周要送给真昼的东西,是明确的契约和将其化作实体的物品。

他一开始对贵金属没什么概念,所以在工作前调查过,发现那价格对普通学生来说几乎负担不起,需要相当大的一笔钱。难怪常有人说要花上三个月的薪水。\\

以前的标准薪水对一个学生来说实在负担不起,所以周才会选择自己买得起的东西,但他想送的不是玩具,而是能担得起发誓的东西。\\

是装饰在纤细手指上的,重要的约定。\\

一般来说,学生大概不会做出这种拿走对方将来的约定,周也知道自己很沉重,宫本却平静地表示:「这样不是很好吗?」\\

「我觉得你花这么多时间和精力也要送给她,已经很了不起了。她会不喜欢吗?」

「不会……我想她会很高兴。不过,要是把话说死的话就太自恋了,所以我就不说了」

「那不就好了吗?我很佩服你年纪轻轻就能下这么大的决心」

「因为我就是这么喜欢她」\\

人们常说学生之间的交往只是玩玩或暂时性的,但周不认为自己的感情是暂时的。\\

他敢肯定,今后恐怕——不,是绝对不会出现比真昼更让他想陪伴、支持、保护、依偎——想让她幸福的人了。\\

一般人要是知道周的心意,多半会想逃走,真昼却接受了,还将其全部纳入。她接纳了周的心意,也让周感受到了她的热情。\\

「真是火热啊」

「我觉得宫本也没资格说别人」

「啰嗦」

「那宫本也别多嘴吧。不想被戳到痛处的话」

「真、真不可爱……」

「我本来就不可爱」\\

「你是从什么时候开始产生这种错觉的?」周笑了一声。宫本用力挠挠头,明显地重重叹了口气。

周没有指出他的脸颊微微泛红。
