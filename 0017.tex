\subsection{天使大人与至高的料理}

大约过了一个小时,饭桌上开始排起了一盘盘料理。

由于是真昼定下的菜单,故而桌上的都是符合真昼健康追求的和食。\\

「这边的厨具和调料也算挺够用,看来是不用我回家取了。明天开始还能做更精致一些的菜」

「你肯为我做饭就让我感激不尽啦」\\

或许是因为真昼不清楚有多少厨具和调料能用,所以比起精致的菜肴更多是简单的东西。但即便如此,色彩和摆盘也堪称完美。\\

青菜煮鱼、味噌煎蛋等等,各种对周来说连想都不敢想要去做的和式菜色并排摆在桌上。\\

尽管周不怎么挑食,但他基本上还是喜欢和食。看到真昼稍稍抱有歉意的样子,周甚至都想告诉她说自己想吃的就是这个了。\\

「……看上去超好吃的」

「这么夸我让我很高兴。但还是赶快趁热开动吧」\\

真昼边说着边坐上了椅子,于是周也坐在了正对方向的椅子上。\\

单人生活准备的餐桌尺寸偏小,不论怎么坐两人靠得都很近。

幸运的是家里姑且有准备两把给客人用的椅子,但面前坐着一位美少女还是让周产生了一种难以名状的感觉。\\

不过,一旦开始品尝料理,真昼的美貌什么的也都无所谓了。\\

例行示意开动之后,周首先尝了一口味噌汁。

在嘴唇碰上碗沿那一刻,周一边享受着味噌与高汤的香气,一边慢慢地将其含进嘴里,然后与那香气相称的味噌与高汤的风味便在舌尖散开。

这种与速食味噌汤完全不同的柔和口味,肯定是经过了精心计算和调整才得到的吧。\\

味噌不太浓,咸淡上也保留住了高汤的风味。\\

第一口略显清淡,应该是因为真昼考虑到了味噌要和其他料理一同食用,这样的味道在喝完的时候恰好会觉得浓淡适中吧。\\

与其说是有什么不足,不如说是让人安心的、会引起品尝米饭和其他菜品欲望的味道。\\

「好吃」

「谢谢夸奖」\\

周坦率地表达出自己的感想,而真昼则放下了心,微微眯细了眼。

尽管周平常一直在夸她做的菜好吃,但是当面说出感想还是会让她紧张的吧。\\

看着刚刚一直在关心这边反应的真昼开始吃了起来,周也向着菜品伸出了筷子。\\

把桌上的菜全部尝了一遍后,周觉得真昼的料理果然非常美味。\\

煮鱼非常入味,同时还保持了肉中的水分。

为了做到入味而长时间加热的话,水分就会流失,使得肉的口感变得干巴巴的。但真昼做的煮鱼肉质却十分鲜嫩,口感很好。\\

煎蛋卷的调味则是正中周的喜好。

在表面鲜艳的金黄色引诱下,周尝了一口,舌尖传来的果然是高汤那柔和的风味。

煎蛋卷有加糖或者除了盐什么都不放等等的各种各样的派系,真昼做的则是加入了高汤略带甜味的蛋卷。\\

隐约而柔和的甜味,或许是蜂蜜吧?

放的量应该并不是很多,但留有余韵的甜味增加了味道的深度。\\

当然不论是甜味的还是咸味的煎蛋卷,周都不讨厌。

不过,周最喜欢的还是这种加入了高汤略带甜味的调味精致的煎蛋卷。如今吃到这理想中的蛋卷,周甚至有些感动。\\

「好吃」周自言自语地感叹了一句,然后又吃了一口。

火候的调整也是绝佳。周咀嚼着这饱含高汤、口感鲜嫩的煎蛋卷,静静地享受着这美味。\\

周一边默默想着「确实比我妈做的还要好吃啊」这种对不在现场的母亲有些失礼的事情,一边幸福地大快朵颐。接着,周注意到真昼正盯着自己在看。\\

「……看起来吃得很香呢」

「实际上也很好吃嘛。面对美味应该要抱有敬意不是么」

「嗯,这倒是」

「而且,比起板着个脸吃,还是这样坦率地表达好吃,我们两边都开心吧?」\\

就算料理十分好吃,不从表情上表达出来的话,制作者也会感到不安和在意。板着个脸的话就算说好吃也会让人怀疑到底是不是真话。\\

比起那样,不如坦率地把自己的感受表现在脸上,对双方都有好处。毕竟不管是感谢还是被感谢的人,都喜欢有个好心情。\\

「……是呢」\\

真昼似乎是接受了周的解释,微微露出了笑容。

如同松了口气般的、表达着安心的柔和笑容,其可爱程度,甚至让周的大脑有一瞬间变得一片空白。\\

「……藤宫?」

「啊,……呃没什么」\\

看得入迷了——这话自然是说不出口。周压抑住渐渐涌起的羞耻感,为了不被真昼发现而继续吃起了晚饭。\\

\vspace{2\baselineskip}

「……我吃饱了」

「喜欢吃就好」\\

周将摆在桌上的饭菜一扫而空,满足地表示自己吃饱了,而真昼则淡淡地回应了他。

不过,真昼表情很柔和,应该是因为看到周这样将饭一点不剩的吃完而感到喜悦吧。\\

「很好吃啊」

「看你的样子就知道了哦」

「比我妈做的还好吃呢」

「把女孩子亲手做的料理跟妈妈做的比较好像是禁忌哦」

「那不是贬低的时候的说法么?话说你很在意?」

「我倒是不在意呢」

「那不就得了。反正好吃的事实也不会变」\\

真昼的厨艺可不是光靠一点点下厨的经验便可以达到的程度。\\

周的母亲虽然和真昼相比有着更丰富的下厨经验,但她调味的喜好不同,而且还很随便,自然比不过真昼那精心计算调整的调味了。\\

不如说在做饭上连父亲都比母亲更加擅长,更不用说和真昼比了。\\

「……哎呀感觉我是不是太幸福了啊。毕竟每天都能吃到啊」

「我们都没事的时候是这样吧」

「……话说,每天一起吃饭真的好吗」

「不好的话我也不会这么提议了」

「话是这么说啦」\\

周也十分清楚像真昼这种直率的人,要是不喜欢的话一开始就不会这么提议,但即使如此他还是会烦恼这样到底好不好。

虽说周付了一半的材料费加上人工费,但还是不禁担心真昼的负担会不会太大。\\

「……我说,一般来讲,你会给谈不上喜欢的男的做饭吗?」

「还不是因为你生活太不健康了吗。再说,我很享受做饭这件事本身,也并不讨厌看你吃得津津有味的样子」

「但是啊」

「……要是你这么在意的话,我其实不给你做也无所谓的哦?」

「别别别还请你务必做上我的份」\\

周立刻反射性地回答,这也代表真昼的料理对周来说就是如此必要和符合喜好。\\

事到如今,要是真昼真的不再做料理,对周来说那可就真的算得上是性命攸关的问题。

虽说周对自己的胃已经被抓住一事早有认识,但现在的问题是真昼的料理实在太过美味。这样下去,一旦回到小菜就饭的日子,生活就会变得无滋无味,想想就可怕。\\

听到周那好懂的回答,真昼那有些无奈的脸上露出了似是苦笑的表情。\\

「那就请你老实收下吧」

「……哦」\\

想到与这大慈大悲的天使大人共进晚餐的日子还将继续,周带着喜悦、期待和罪恶感,不得不叹了一口气。
