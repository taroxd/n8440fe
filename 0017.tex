\subsection{天使大人与至高的料理}

大约过了一个小时,饭桌上开始排起了一盘盘料理。

由于是真昼定下的菜单,故而桌上的都是符合真昼健康追求的和食。\\

「这边的厨具和调料也算挺够用,看来是不用我回家取了。明天开始就能做精致一些的菜了」

「哎呀只要有东西吃我就感激不尽啦」\\

因为不清楚有多少厨具和调料能用,比起精致的菜肴更多是简单的东西,但即便如此,色彩和摆盘也可称得上是完美了。\\

青菜煮鱼,味噌煎蛋等等,各种对周来说连想都不敢想要去做的和式菜色并排摆在桌上。\\

在不怎么挑食,但基本上还是喜欢和食的周心里,甚至都想对稍稍抱有歉意的真昼说自己想要的就是这个了。\\

「……看上去超好吃的」

「这么夸我我是很高兴的啦。趁热开动吧」\\

真昼这么说着坐上了椅子,而周则坐在了正对方向的椅子上。\\

单人生活准备的餐桌尺寸偏小,不论怎么坐两人靠得都很近。

还好备了两把给客人用的椅子,但面前便坐着一位美少女还是给人一种难以名状的感觉。\\

不过,一旦开动起来,真昼的美貌什么的也都无所谓了。\\

例行示意开动之后,周首先尝了一口味噌汁。

周享受着在嘴唇碰上碗沿那一刻,味噌与高汤的香气。慢慢地将其含在嘴里,与那香气相称的味噌与高汤的风味便在舌尖散开。

这种与速食味噌汤完全不同的柔和口味,肯定是经过了反复计算调整的吧。\\

味噌不太浓,咸淡上也保留住了高汤的风味。\\

略显清淡的第一口,应该是考虑到了要和其他料理一同食用,这样的味道在喝完的时候恰好会觉得浓淡适中吧。\\

与其说是有什么不足,不如说是让人安心的,引起品尝米饭和其他菜品欲望的味道。\\

「好吃」

「谢谢夸奖」\\

周直白地表达自己的感想,而真昼则放下了心,微微眯细了眼。

尽管平常一直在夸她做的菜好吃,但是当面说还是会紧张的吧。\\

看着刚刚一直在关心着这边反应的真昼也开始动起了筷子,周也向着菜品伸出了筷子。\\

周甚至想要把摆着的吃的一齐包进嘴里,可见真昼的料理实在是美味。\\

煮鱼好好地入了味以及保持了肉中的水分。

为了做到入味而长时间加热的话,水分就会流失,肉质就会变干巴巴的,但真昼做的煮鱼肉质却十分鲜嫩,口感很好。\\

煎蛋卷的调味则是正中周的喜好。

被表面鲜艳的金黄色引诱,周尝了一口,舌尖传来的果然是高汤那柔和的风味。

煎蛋卷有加糖和仅盐这些各种各样的派系,真昼做的则是加入了高汤略带甜味的蛋卷。\\

隐约而柔和的甜味应该是蜂蜜吧。

放的量应该并不是很多,留有余韵的甜味增加了味道的深度。\\

当然不论是甜味的还是咸味的煎蛋卷周都不讨厌。

不过,周最喜欢的还是这种加入了高汤略带甜味的调味精致的煎蛋卷,如今吃到这理想中的蛋卷,周甚至有些感动。\\

好吃啊——周以谁都听不见的音量轻轻感叹了一句,然后继续吃着菜。

火候的调整也是绝佳。周咀嚼着这饱含汤汁的口感鲜嫩的煎蛋卷,静静地享受着这美味。\\

周一边默默想着「确实比我妈做的还要好吃啊」这对不在现场的母亲有些失礼的事情,一边幸福地大快朵颐。接着,周突然注意到真昼正盯着自己在看。\\

「……看起来吃得很香呢」

「实际上也很好吃嘛。面对美味应该要抱有敬意不是么」

「嗯,这倒是」

「而且,比起板着个脸吃,还是这样坦率地表达好吃,两边都开心吧?」\\

就算料理十分好吃,不从表情上表达出来的话,制作者也会感到不安和在意。板着个脸的话就算说好吃也会让人怀疑到底是不是真话。\\

比起那样,不如坦率地把自己的感受表现在脸上,对双方都有好处。毕竟不管是感谢还是被感谢的人,都喜欢有个好心情。\\

「……是呢」\\

真昼似乎是接受了周的解释,微微露出了笑容。

如同松了口气般的,表达着安心的柔和笑容,其可爱程度,甚至让周的大脑有一瞬间一片空白的程度。\\

「……藤宫?」

「啊,……呃没什么」\\

看到入迷了——不可能如此坦白的周,为了掩饰自己渐渐涌起的羞耻感,继续吃起了晚饭。\\

\vspace{2\baselineskip}

「……我吃饱了」

「招待不周」\\

将摆在桌上的饭菜一扫而空的周满足地表示自己吃饱了,而真昼则淡淡地回应道。

不过,真昼温和的表情,似乎是在表达看到周这样将饭一点不剩的吃完感到的喜悦吧。\\

「很好吃啊」

「看你的样子就知道了哦」

「比我妈做的还好吃耶」

「把女孩子亲手做的料理跟老妈做的比较好像是禁忌哦」

「那不是贬低的时候的说法么?话说你很在意?」

「我倒是不在意呢」

「那不就得了。反正好吃的事实也不会变」\\

真昼的厨艺可不是光靠一点点下厨的经验便可以达到的程度。\\

周母亲虽然有常年的下厨经验,但在调味上和周的喜好不同,而且还很随便,自然比不过经过仔细地计算调整的真昼的调味了。\\

不如说在做饭上连父亲都比母亲更加擅长,更不用论跟真昼比了。\\

「……哎呀感觉我是不是太幸福了啊。每天都能吃到耶」

「虽然只有在我们都没事的时候才行呢」

「……话说,每天一起吃饭真的好吗」

「不好的话我也不会这么提议了」

「嘛虽然话是这么说啦」\\

周也十分清楚像真昼这种直率的人要是不喜欢的话一开始就不会这么提议,但还是不禁烦恼这样到底好不好。

虽说周付了一半的材料费加上人工费,但还是不禁担心真昼的负担会不会太大。\\

「……我说,一般来讲,你会给谈不上喜欢的男的做饭吗?」

「还不是因为你生活太不健康了吗。再说,我很享受做饭这件事本身,看着你吃得津津有味我看着也不会不高兴」

「但是啊」

「……你要是实在在意,我其实不给你做也无所谓的哦?」

「别别别还请你务必做上我的份」\\

周反射性的回答,也表现出对周来说真昼的料理是如此地必要和享受。\\

事到如今,要是把真昼的料理给停了,对周来说那可就真的算得上是死活问题了。

周对自己的胃已经被抓住一事早有自觉,不过问题是真昼的料理实在美味。照这样下去眼看就要回到每天小菜就饭的无味日子,想想都可怕。\\

听到周那易懂的回答,真昼一张实在无奈的脸上露出了似是苦笑的表情。\\

「那就请你好好收下吧」

「……哦」\\

想到与这大慈大悲的天使大人共进晚餐的日子还将继续,喜悦、罪恶和期待感在心中交织的周不住地叹了一口气。
