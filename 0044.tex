% \subsection{44 天使様の手をとって}
\subsection{牵起天使大人的手}

% 周達が住む地域から車で小一時間ほど離れた地域にある有名な神社に到着すると、やはりというか人数はテレビで見た時よりもかなり減っていたものの、人が途絶える事はなさそうだった。\\
从住的地方出发,开车将近一个小时,周一行到达了这座在这一片有名的神社。虽然如他们所料,比起在电视里人已经少了不少,但看来还是不至于到没人了的地步。\\

% 「大分人が減ってはいるけど、それでもやっぱり結構居るわねえ」


% 「そうだね」


% 「真昼ちゃん、はぐれないようにしてね。私達も気を付けるしスマホがあるから集まるのは簡単だと思うけど、それでもやっぱり一緒に参拝したいものね」


% 「はい」\\


% 着物に身を包んでいる真昼が一番動きにくいし、足も遅い。靴はブーツとはいえ、着物は歩幅が制限されるので歩みは他の人と比べれば遅いものだろう。


% 人混みをかき分ける、ほどではないものの、やはり肩がぶつかりやすい程度には居るので、こちらも気を配ってやらなければならない。\\


% 「じゃあ、行きましょうか」\\


% 志保子の先導で人混みの中まずは手水舎に向かって手と口を清める事となったのだが、やはり真昼に視線が吸い寄せられている人間の多い事。\\


% 着物を着ている人間も少なからず居るし、着物を着てきた真昼がそう目立つ訳でもない、という事はなかった。


% そもそも何も着飾っていない制服姿ですら人目を引くのだ。清楚系な正統派美少女が和装していて目立たない筈がない。\\


% 口を清めている仕草すら美しいので、視線を集めている。\\


% 「……どうかしましたか?」


% 「何でも」\\


% 何か面白くない、とは思ったものの、それを口にする事はせず、周も両親達と同じように手と口を清めて前を歩く両親の後をついていく。\\


% 一応真昼に歩みを合わせてはいるのだが、やはり和装は普段着にでもしていないと裾さばきが難しいらしく、人混みのせいもあっていつもより遅々とした進みとなっていた。\\


% 「真昼、大丈夫か」


% 「はい、これくらい……ひゃっ」\\


% 他の参拝者に肩がぶつかって体勢を崩して転びかけているので、周が腕で押さえた。\\


% 「大丈夫じゃなさそうだな」


% 「……すみません」


% 「ほら、手を貸せ」\\


% 流石に慣れない格好で歩かせているので、気遣わない訳にはいかない。


% 袖から覗く小さな掌に手を伸ばせば、真昼がこちらを見上げてくる。\\


% もしかして嫌だったのかと手を引っ込めようとすれば、慌てて掌を重ねてまたじっと周を見上げてくるので、周は訳が分からずに見つめ返してしまった。


% じ、と見ていたら、先に真昼が視線を逸らして周の掌をきゅっと握る。\\


% 何なんだと首を捻る間もなく流れに乗って賽銭箱の前までたどり着きそうだったので、周は繋いだ手の感触を確かに感じながら、小さな疑問を胸にしまいこんだ。\\


% \\


% 「結構長く願ってたけど、何願ってたんだ」\\


% 参拝を終えて少し列から離れたところで、静かに祈っていた真昼に問いかけてみる。


% これぞ見本といった風な美しい所作で参拝した真昼は、周の倍くらい瞳を閉じて手を合わせていた。その後の礼の優美さに気をとられかけていたが、彼女が何か願い事をしていると思い出して聞いたのだ。\\


% 「無病息災ですかね」


% 「すごい無難なやつ」\\


% 真昼らしいといえば、真昼らしい。


% あまり本人は物欲やら金銭欲やら名誉欲等はないので何を願うかと思っていたので、予想の範疇内のもので拍子抜けしたと言えばよいのか。\\


% 「それと」


% 「それと?」


% 「……このまま、穏やかな日々をすごせますように、と」\\


% これもまた、真昼らしい願いだった。


% 刺激や変化をあまり好まない真昼が願いそうな事であり、平和と静穏を好む真昼ならではの願いだろう。\\


% 「うちの母さんいたら穏やかでないけどな」


% 「それはそれで楽しめるものですよ」\\


% そういうものなのか……とは思ったものの、本人が楽しそうなので口は挟まず、柔らかい表情の真昼の手を取る。


% まだ人混みから完全に抜けきった訳でもないし、先にお参りを終え少し離れた位置で待っている両親のところにたどり着くまでに転ばれても困る。\\


% そういった意味で手を繋いだのだが、真昼は小さく瞬きをして、少しだけ恥ずかしそうに瞳を伏せてから周の手を握り返した。\\


% 「二人とも、こっちよー」\\


% 志保子の声は明るくハキハキとしていて分かりやすい。


% 促されるように二人で両親のもとに向かえば、志保子が目を丸くして、それから口元に手を添えて微笑ましそうにこちらを眺めてくる。\\


% 「あらあら」


% 「何だよ」


% 「ナチュラルに手を繋いでいるのねえ、と」\\


% 言われて、志保子の前で手を繋ぐのは失策だと今更に気付く。


% これでは真昼が周の特別だ、と言っているようなものではないか。志保子に勘ぐられて常ににやにや笑いをされるなんて冗談ではない。\\


% 「……はぐれないようにするためだろ。それに着物だと転びやすいし」


% 「そうだね。着物だと歩きにくいし、エスコートしてあげるべきだろう。私も志保子さんにするし」\\


% 修斗は理解あるので、真昼の手を取っている事に違和感はないらしい。同じようにするりと志保子の手を握っている。


% あのように父親ほどスマートに手を差しのべて繋ぐなんて出来れば苦労しないが、性格上無理だとも思っていたので、真昼が素直に手を繋いでくれたのはありがたかった。\\


% 志保子の意識が修斗の方に移った事にほっとしつつ、そっと手を離そうとしたら、真昼の手から力が抜けてくれない。


% きゅっ、と控えめながら離す気がないのは分かるので、どうしたと小声で聞いても返事はない。ただ、か細い指が周を捕らえているだけだ。\\


% 「真昼ちゃん真昼ちゃん、温かい飲み物買おうと思うんだけど、おしること甘酒どっちがいい?」


% 「じゃあ、おしるこでお願いします」\\


% 志保子に遮られて問いかけるタイミングも離すタイミングも失って、そのまま華奢な手を握ったまま。\\


% 「あなたはどうする?」


% 「……じゃあ俺甘酒」


% 「はいはい」\\


% ただ、真昼が嫌がっていないならそれでいいか、と胸の中で起こったかすかなざわつきを抑えて飲み込み、志保子に希望を伝えて真昼の手を握り直した。\\


% \\


% ほどなくして出店から帰って来た志保子がそれぞれの注文した品を手渡してくるので、流石にこれは手を離さないとどうしようもないので一度離して一息つく事になった。\\


% 両親は共に甘酒を飲みながら穏やかに笑い合っている。


% 二人の世界というほどではないもののいちゃいちゃしているので、声をかける気にもならず周も渡された甘酒に口をつけた。\\


% 飲む点滴と言われるほどに栄養のあるものだが、米の甘味やコクはほっと染み渡る味で、つい感嘆と安堵の混じった吐息がこぼれる。


% おしるこも捨てがたかったのだが、やはり新年という事で気分的にこちらを選んだのだが、個人的には正解だった。\\


% ちらりと真昼を見てみると、穏やかな顔で紙コップに入ったおしるこを少しずつ飲んでいる。\\


% 「おしるこはうまいか?」


% 「美味しいですよ」


% 「一口頂戴」


% 「どうぞ。私ももらっていいですか」


% 「ん」\\


% 折角なので一口味見で交換する事になったので、コップを交換してとろみのついたいかにも小豆といった色合いのそれに口をつける。\\


% 柔らかく漂う小豆独特の香りをかぎながら口に含むと、やはり甘くて濃厚な味が広がる。いささか甘さが強いように思えるのは、周があまり甘党ではないからだろう。


% 真昼は甘いものはそれなりに好きらしいので、ちょうどよいのかもしれない。\\


% 「おいしい」\\


% 真昼の方は甘酒も気に入ったらしく、かすかに目尻が下がった笑みが浮かんでいる。\\


% 「……ナチュラルねえほんと」\\


% 二人の様子を見守っていた志保子が、小さくこぼした。\\


% 「何がだよ」


% 「気にしなくてもいいのよ。……今日が寒くてよかったわぁ」


% 「暖かい方がいいだろ」


% 「二人はそうかもしれないけど、私達は……ねぇ?」\\


% 志保子が同じく見守っていた修斗に同意を求めると、修斗は穏やかな笑み……微妙に苦笑混じりではあるが、柔らかく微笑んで「そうだね」と返す。\\


% こちらを見る眼差しが妙に生暖かくて居心地悪げに肩を揺らした周を、真昼は不思議そうに眺めていた。

