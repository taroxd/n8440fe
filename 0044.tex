\subsection{牵起天使大人的手}

从住的地方出发,开了将近一个小时车,周一行人便到达了坐落在这一片地区的有名的神社。虽然如他们所料,比起在电视上看到的时候人已经少了不少,但看来还是不至于到没人的地步。\\

「虽然人已经少了挺多了,不过看来还是剩下不少啊」

「是呢」

「小真昼,不要走丢了哦。虽然我们会留意,你也有手机,要汇合也不算麻烦,但即使如此还是一起去参拜比较好吧」

「好的」\\

身着和服的真昼在一行里行动最不方便,步速也是最慢的。虽说她脚上穿的是长靴,但穿着和服还是会限制步幅,因而走得比较慢也是自然。

虽然到不了得挤着人才能前进的程度,但人还是多到经常撞到肩,所以周这边仍必须留个心眼。\\

「那就出发吧」\\

志保子领着众人扎进人群里,打算首先去趟洗手处\footnote{洗手处:原文为 {\jpfont 手水舎},是指给参拜者洗手漱口以清净身体的地方。}洗手漱口,但果不其然,还是有很多人的眼球都被真昼所吸引。\\%主要觉得原句怪怪的

穿着和服来的人也不算少,按理说穿着和服的真昼不会太过显眼,然而事实并非如此。

说到底,真昼就算没有装饰只是穿着校服的样子就已经够吸引人了。清纯系正统美少女的和服姿态,要说不显眼那是不可能的。\\

就连她漱口的动作都是那么美丽,吸引着旁人的视线。\\

「……怎么了吗?」

「没什么」\\

尽管周觉得旁人看着真昼让他不是滋味,但周并没有说出口,而是跟着父母一样洗完手漱完口,然后跟了上去。\\

虽然周也在放慢步子等着真昼,但毕竟真昼不是日常穿着和服,下摆的处理似乎对她来说还挺难,加之人也多,导致真昼的步伐变得比平常还要慢。\\

「真昼,还好吗」

「嗯,就这点程度……哎呀」\\

被其他的参拜者撞到了肩,真昼身体失去平衡,眼看就要摔倒,于是周赶紧用手扶住。\\

「看上去不太好啊」

「……抱歉」

「好啦,手伸过来」\\

毕竟现在是让真昼穿着不习惯的衣服走着,很有必要照顾下她。

周把手伸向衣袖中露出的小小的手掌,而真昼则仰起头看向了周。\\

看到真昼这副模样,周觉得她或许不乐意,正打算把手收回来,真昼却慌忙把自己的手放在周的手中,再次仰头凝视着周。这么一搞,周也迷糊了起来,凝视了回去。

两人这么盯了一会之后,真昼先移开了视线,紧紧握住了周的手。\\

连让周表示疑惑的空隙都没有,两人眼看便要顺势走到赛钱箱面前了,于是周一边清楚地感受着牵着的那只手传来的触感,一边把这小小的疑问埋在了心里。\\

\vspace{2\baselineskip}

「花了挺长的时间啊,许了个什么愿?」\\

趁参拜完稍稍离开队伍的时候,周向刚刚静静地许着愿的真昼问道。

真昼以称得上是示范的美丽动作进行了参拜,闭眼合掌的时间有周的两倍长。看到她结束合掌后那优雅的行礼,周差点入了迷,现在回过神来才想起来问真昼许了什么愿。\\

「只是无病无灾啦」

「真是个平淡的愿望」\\

要说的话,这倒也是真昼的风格。

周想着真昼这人既没物欲又没钱欲还没名欲还能许什么愿,结果一如所料,甚至让人感觉有点扫兴。\\

「还有」

「什么?」

「……想一直过着这样,平稳的日子」\\

这个愿望同样很有她的风格。

确实,这个像是不大喜欢刺激和变化的真昼会许下的愿望,而且也只有喜欢平稳和安宁的真昼才会有这种愿望吧。\\

「要我妈在那可就不平稳咯」

「那样也有那样的乐趣啦」\\

是这么回事么……周虽然怀疑,但看着真昼本人很高兴的样子,便闭上了嘴,以一副温柔的表情牵起了她的手。

毕竟现在还没有完全穿过人多的区域,而且父母已经先参拜完在稍远处等着了,要是走去那里的这段路上摔着了也麻烦。\\

虽然周是抱着这样的想法牵起了手,但真昼却微微眨了眨眼,略带害羞地垂下眼帘回握住了周的手。\\

「你们两个,这边这边~」\\

志保子的声音十分明亮而富有活力,很容易分辨出来。

像是被催促着一样,两人走向父母所在的地方,这时志保子则瞪大了眼,然后用手捂着嘴,似是在微笑般地望着这边。\\

「哎呀哎呀」

「咋了啊」

「想着你俩怎么就自然地牵起手来了呢」\\

听到志保子这么讲,周才反应过来自己牵着真昼的手走到她面前这一失策。

这岂不是在说,真昼对周来说是特别的存在了么。如果被志保子误会然后整天被这么嘿嘿笑着,这可一点也不好笑。\\

「……是为了不让她走丢啦。而且穿着和服还很容易摔着」

「说的也是。穿着和服很难走路,确实需要一个护花使者吧。我可是保护着志保子呢」\\

修斗是个明白人,没有对周牵着真昼的手一事感到奇怪。他也和周一样,轻轻地牵着志保子的手。

要是能像父亲那样灵巧地伸出手牵起对方的话,那就没这么多累人事了,但周从性格上便做不到,因而他很感激真昼坦率地把手牵了过来。\\

看着志保子的注意力转向了修斗,周松了口气,正想悄悄松开手,可真昼却没有放松手上的力气。

虽然动作很轻,但周还是理解了真昼不愿松开手的意思,轻声问她「怎么了」,却也没有得到回答。她仅仅是用细细的手指抓着周。\\

「小真昼小真昼,我打算去买些热饮,汁粉\footnote{汁粉:原文为 {\jpfont おしるこ},一种日本的红豆沙甜品,一般放入麻糬等食用。}和甘酒\footnote{甘酒:又称醴,是一种甘甜的日本传统浊酒,以白米发酵酿成。}你要哪种?」

「那我就要汁粉吧」\\

因为志保子的打断,向真昼提问和放开她手的时机都错过了,周便只好继续握着那娇嫩的手。\\

「你呢?」

「……那就甘酒」

「好好」\\

不过,要是真昼不讨厌的话那这样也不错——周抚平那心中泛起的微微瘙痒感让自己接受了,告诉志保子自己要什么后,便重新握紧了真昼的手。\\

\vspace{2\baselineskip}

没多久,志保子就从店里回来了,并把买来的各种东西分了下来。再怎么说这时候不放手也没法喝,于是两人便暂时松开了手稍做休息。\\

父母则一起喝着甘酒放松地笑着。

虽然不至于进入二人世界,但两人还是亲亲热热了起来,所以周也不没甚么兴致搭话,喝起了刚刚到手的甘酒。\\

虽然甘酒很有营养,被誉为能喝的点滴水,但令周享受的还是米中沁人心脾的甘甜与回韵。一口下去,周不禁叹出一口夹带感叹和安心的吐息。

虽然汁粉也难以舍弃,但既然是新年,考虑到气氛周便选择了甘酒。从个人喜好上来看是选对了。\\

周瞄了一眼真昼,发现她神情安稳的一点点地喝着纸杯里的汁粉。\\

「汁粉好喝吗?」

「很好喝哦」

「让我尝一口」

「给。我也能尝一口吗」

「嗯」\\

难得有这个机会,两人便决定交换热饮各尝一口。周换过了杯子,将那微粘的红豆色的汁粉送到了嘴边。\\

嗅着空气中飘着的那红豆独特的香味,周将汁粉含在嘴里,一如所料有种甘甜而浓厚的风味扩散开来。觉得略微有些过甜,大概是因为周不是甜派的人吧。

而真昼似乎是挺喜欢甜味的东西,这个甜度对她来说或许正好。\\

「好喝」\\

真昼似乎也挺中意甘酒,微微弯起眼角露出了笑容。\\

「……还真是自然呢」\\

守望着两人的志保子小声地感叹道。\\

「咋了啊」

「不用在意哦……今天是个冷天挺幸运呢」

「明显是天气暖和更好吧」

「你们俩说不定是那样,我们的话……是吧?」\\

志保子向同样守望着两人的修斗寻求同意,修斗则以平和……而微妙地混有苦笑的笑容,微笑着回答道「确实呢」。\\

在那微妙的温暖视线中,周略感不适地抖了抖肩膀,而真昼则以不可思议的眼神望着那样的周。
