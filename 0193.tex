\subsection{关于文化节的项目}

考试结束,接下来就是一年一度的重大活动——文化节了。

周他们的学校很重视这种全体学生一起举办的活动,因此每个班预算很多,年年都会有精彩的项目。\\

「所以说,来决定我们班的项目吧!」\\

当然,班级要做什么是由班里所有同学决定的。到了这会儿,气氛自然热烈起来。\\

兴致勃勃站在讲台上的是树。

喜欢祭典的树显然是会报名文化节委员的,不过报名后他还真的漂亮地当选了,让人只得一笑。\\

「嗯嗯,关于文化节的项目啊,首先有一点,每个年级有多少食品店都是已经确定好的。基本上每个班都会考虑食品店,这种情况下竞争会很激烈,这一点要做好准备」\\

当然,能摆出来的食品店数量是决定好的。

选择食品店能够进行经营方面的实践,有一做的价值,所以这个选择很有人气,弄不好几乎每个班都会想弄。这么一来,到时候除了食品店就没别的了,因此才会设下限制。

此外,考虑到其他料理实习室的空闲情况和卫生指导的因素,也无法满足所有的需求。\\

「然后,你们先看看发下来的文件,里面列出了一些看上去要花预算但实际上已经有了的东西。哪怕上面没写,一旦有什么能让大家凑出来的,也都要确认一遍。总之,大家说说这个预算下面能做到的事情……嗯,有什么想办的项目就举手发言吧」\\

树说完,同学们争先恐后地举起了手。

大家都目光炯炯,这项活动的重要性可见一斑。

毕竟对学生而言,文化节是一场饱受期待的大规模活动。\\

(不过我去年敷衍过去了就是)\\

当时,周毫无学生那般小鲜肉的模样,文化节也是得过且过的。那次班里的项目是贩卖自制的产品,他便按照要求做了些东西,轮到他的时候来看店,仅此而已。\\

因此,周在远处看着热情洋溢的同学们。\\

「我来我来!我推荐文化节少不了的咖啡厅!」

「嗯嗯,意料之中吧,只是普通的咖啡厅吗?」

「女仆咖啡厅怎么样」

「你想,椎名在这个班里……绝对很合适的」\\

同学讲着讲着声音渐弱,并往真昼那里瞧过去。周心里有点不是滋味,但也不至于要说些什么。\\

「哈哈哈,我怀疑你压根没考虑预算,不过要的就是你这气势。总之先加到候补里咯」\\

听到真昼会穿女仆装,男生们就活跃了起来。周傻眼地看向这群人,然后与树合上目光。

树以视线问道「没问题吗」,周便露出一脸不情愿的表情。\\

要说有没有问题,那还是有的。\\

真昼平日里本就显眼,常引得众人围观。

最近,真昼的可爱更上一层楼,要是给她穿上女仆装,必然会人山人海,让她不好应对吧。\\

反过来说,好处则是销售额有了保证。真昼的存在是绝对的广告,男生毫无疑问会蜂拥而至,以求见上一眼。\\

至于当事人真昼,被拿出来当话题后,她露出着难以言喻的、为难的笑容。

这也是自然的。把自己拿出来受众人围观,绝不是什么舒服事。\\

不过,这终究只是个提议,不好刚说出来就批评什么。要是真昼真的不乐意,也只能由周来拒绝了。\\

「女仆咖啡厅确实是男人向往的,但是提出来之前要考虑预算。好啦,下一位——」\\

在树的引导下,大家又提出了鬼屋、卖咖喱和乌冬面这种文化节必备的店之类的点子。黑板上逐渐写满了白色的字。

然而,大家的……主要是男生的兴趣似乎都放在了女仆咖啡厅上面,能听到交头接耳的声音。\\

「果然还是椎名的女仆装……」

「可是有藤宫那家伙……」

「没事,藤宫也是男的,肯定会想看女朋友穿女仆装吧」\\

尽管周听到了,但很可惜他并不赞成。

要说完全不想看是假的,然而周并不希望拿出来炫耀。知道这样真昼会累到,周便完全不愿主动去让真昼穿。\\

周向那边瞪了瞪。他们注意到视线后,猛地移开了眼睛。

而真昼似乎目睹了这副场景,轻轻笑着,于是周把眼神放得缓和了些。\\

「话说周啊,你有什么建议?」\\

树突然这么一问,周满脸不情愿地看向了他。\\

「为什么问我啊」

「因为看你好像有话想说?」\\

周有话想说也不是想说给树听,但树这么一点名,周围同学都看了过来,什么都不说也会把气氛搞僵。

想了想该怎么办之后,周说出了最为轻松的提议。\\

「……硬要说的话,调查整理乡土史什么的,然后展示出来吧」\\

没想到,周说完自己的意见,班里一下子安静了下来。

这就好像往大家的热情上浇了盆水一样,让人非常尴尬。\\

「这有什么意义啊」

「……别说,其实还挺好的。能调查的都调查之后,展示时留一点点人看着,接下来就能自由行动,享受文化节本身了。这样的话,我们不用在意时间,想去哪个班看就去哪个班」\\

周换了个说法,教室里便四处响起了「原来如此」的声音。\\

周也不觉得展示乡土史会让学生觉得有趣,他真正的目的是能在之后空出自由行动的时间。\\

虽然食品店有人气,但无论怎样都需要人手和大量劳动力,也要占用很长时间。既然涉及金钱,对待店铺就不得不慎重,显然会非常辛苦。\\

而展示一样东西的话,只要在准备期间完成一切,之后只要留一两个人来看着就行。

文化节有两天,均摊到每个人身上,一个小时都不会有,劳力和时间效率很高。

还有很重要的一点是,这样的项目不会产生金钱,只要随意站着就好了。\\

除此之外,至于那些对接客、外貌和厨艺没有自信的人,没什么能比展示更轻松了。周也是这样的人,所以对此一清二楚。\\

「咋说呢,真有你的作风」\\

树毫不掩饰自己的傻眼,而周只是讲出了自己的意见而已,于是他转向旁边,闭上了嘴。

真昼也看了过来,眼神好像在说很有周君的作风。周感到尴尬,但说出去的话也收不回来了,他便只是轻轻叹了口气。\\

\vspace{2\baselineskip}

「呃,女仆咖啡厅得票最多,就决定是女仆咖啡厅了行不?」\\

最后,暂定就是获得众多男生票的女仆咖啡厅了。\\

「接下来要把定下来的项目报告给学生会,然后抽签决定;如果没抽中的话,我们就弄得票第二多的鬼屋。还有,衣服肯定是预算解决不了的,得靠我们找人来弄。你们谁认识这方面的人的话,可以先去问问看,记着,要是这个解决不了的话,我们就只能开普通的咖啡厅了」\\

负责主持的树拿出了他天生的开朗性格和能干的本领,干脆利落地讲出需要的东西和注意事项,然后离开了教室,大概是去向学生会转达了。

教室的气氛明显舒缓、喧闹起来。周轻叹一口气托着腮,这时他注意到真昼走了过来。\\

「你怎么办」

「就算说怎么办……都已经定下来了,没有办法」\\

真昼露出苦笑,而周则觉得,虽然是没有办法,但心里总有点焦躁。\\

「要是你不乐意的话,得要好好讲出来啊?」

「我也不是不乐意……那个,周君,你不喜欢女仆装吗?」

「既不喜欢也不讨厌,倒是觉得你穿着应该挺合适的。毕竟围裙就很合适」

「这、这样啊……我会努力」

「你没必要勉强自己的」

「要是能让周君开心,我就会穿」\\

说完,真昼绽放出美丽的微笑。看到背后男生暗自摆出胜利的姿势,周只得忍着让自己的笑容不要抽搐了。
