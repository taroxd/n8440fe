\subsection{为了生日的重要拼图}

目前是三方面谈和定期考查的准备期间,前不久还有民间企业发起的模拟考,高中生最近往往比较忙,周也不例外。\\

只不过,12月上半有即将来临的定期考查,再加上真昼的生日,周清了许多假,没多少时间去打工,都是见缝插针抽空去的。\\

「啊,都这时候了啊,年轻人真不容易」\\

快到夜晚的时间段,工作闲下了许多,便有了聊天的功夫,于是周闲聊着道出了自己的近况。

排班在相同时间段的宫本,则是一边细心地洗着不能用洗碗机洗的虹吸壶,一边嗯嗯地点头,好像很怀念的样子。\\

「我觉得宫本也挺年轻的」

「我已经是过了备考阶段的老油条了」

「你之后还得去过求职关吧」

「你也是一样啦」\\

刚去送走大约是今天最后一位客人然后回来的大桥来了一句吐槽之后,又被宫本以吐槽回敬,变成一副有话想说却又说不出的表情。

大桥总体上是个友善开朗的人,跟宫本却是彼此都很随便,周心中的感想是两人关系真好,不过他憋在了心里,不然一旦说出来,恐怕就会被异口同声地否认。\\

「话说你俩备考的时候是怎样的」

「哪壶不开提哪壶啊你」

「啊」

「我算是普通地学习然后普通地考上了,比起刚进高中的时候倒是花了好几倍的功夫去学习」

「藤宫,这家伙看着轻飘飘的,其实脑子很灵光,最好别太相信他。会突然背叛的」\\

成绩方面周没怎么问过,也没机会去问,现在听大桥说宫本成绩优秀,大桥还朝宫本呸了几声。\\

「成绩还是比你好的,我学习老认真了」

「可恶的时髦轻浮男」

「哈哈哈,随你怎么说。平时的努力可是不会背叛的」

「闭嘴」\\

两人是老熟人了,语气十分干脆,弄得周都有些担惊受怕,不过看宫本并不介意,类似的交谈恐怕是常有的事。\\

再想到自己和树也会有类似的对话,周便又觉得倒也不必那么担心,不过他又有些怕宫本对这种冷淡的对待会不会已经习以为常了。

虽然不适合由外人插足,但周也隐隐察觉到了宫本对大桥的感情。\\

周希望这份感情能有所回报,不过他也不好掺和,只好跟了解情况的茂野一起默默关注。\\

「讨论去向啊定期考查啊圣诞啊都在这个时期,学业和生活都很忙。顺带一提,考查之前的打工安排最好能早做调整,让店长也省点心」

「啊,这块肯定是提早说好的,我打算在定期考查前稍微分点时间去学习」\\

考虑真昼的生日,也需要有一些准备期,却也不能完全请假不来。周打算精心安排时间为此做准备,他相信自己可以做到。\\

「OKOK,我们也算是先听你打了个预防针了」

「圣诞节的排班……你肯定想跟你的小女友见面,不想把她抛在一边吧」

「这个嘛……」\\

听说圣诞节的时候单身容易被安排工作,周哪想得到自己成为了把工作推给别人的一方。

圣诞有安排的话,事情自然会轮到有空的人头上,周现在切身感受到了这一事实。他饱含内疚想要低头道歉,宫本见状则是哈哈大笑道:\\

「没事的没事的,跟你的小女友好好玩」

「不过就我一个人请假的话」

「没事的,有个你还没见过的上午上班的同事已经确定能来了,他说从圣诞前到年底的时薪会涨不少,想借此挣一笔。我也有这个打算」\\

宫本又笑着说,到了繁忙期,餐饮业无论如何都会变得忙碌,丝卷会相应地提升时薪和发放补助,对这会儿有空的人来说算得上是个正中下怀的活动了。\\

「没对象是这样的」

「烦不烦,你不也是」

「你怎么知道我分手了」

「还不是你自己全给说了一遍,都不记得了吗」

「好、好啦好啦冷静点」\\

不知怎么的,让这两人单独对话就容易吵架,周便插进了他们的交谈,重新转向两人,改变了话题。\\

「是说我有件事想问问」

「嗯?」

「大学生活是什么样子的?我虽然参加过开放日,但也就是感受一下氛围,不知道实际上学的人有什么感受」

「啊,高中生会在意这种的呢」\\

看来是成功改变了话题,两人一同放下了脸上的不快,一边沉吟着一边看向空中,好像在苦恼的模样。

周心说,两个人真是太像了。\\

「还挺难描述的……至少不算是高中的延续吧,不会像高中那样很有规律地上学,不过这也看院系和专业。我算是选了很多课了,日程也没有高中那么满满当当的。倒是考试季是真的没一点闲工夫,从早学到晚,脑袋都要炸了」

「为什么会那么忙呢,感觉都要成人生最充实的时期了」

「大概是因为缺了平时的积累吧?」

「闭嘴吧你」\\

为什么这些人稍微讲两句就会吵起来呢?周一边想着,一边将它解释为这两人之间特有的社交。\\

已经习惯的茅野说过,这两个人吵架才是正常状态。周现在感慨,说的可真是太对了。\\

「感兴趣的领域上课特有意思,不感兴趣的必修课老实说还挺无聊的,这个也没办法」

「没人觉得那玩意儿有意思吧……虽然在就拿学分来说很要紧就是了。要是能没有的话该多好,老想翘课了」

「都说了那么多遍基础很重要——」

「你说啥」

「好啦好啦」

「总之跟高中不一样,需要自己主动,或者说探究学术才是主题,听课、拿学分都必须自己管理好,更需要为自己负责;而且在要学的课程上有一定的选择空间,做好取舍是很重要的。还有,早上起不来的话得留意第一节课,会死人的,适应大学生活之后一旦放松就是最危险的时刻」

「好几次差点睡过头的大地发出了过来人的忠告」

「那时是我太蠢了」

「哇哦笨——蛋笨——蛋」

「你也没资格说别人吧,高中睡过头那么多次」\\

周逐渐觉得他们是在说夫妻相声,大概是因为逐渐适应了这两位前辈吧。\\

「反正,虽然也看大学,但大学生活比预想中,怎么说呢,更轻松,还挺棒的。不过倒也不至于做梦都想去。社团的话,进不进都可以,进去的话经常能从中得到自己不知道的方向上的知识和人脉……交换信息确实有用,怕的就是偶尔会有些社团破坏者,弄得人际关系一团糟,甚至危及到人身安全」

「你别吓人啦」

「不是,真的很可怕哦?」

「呃,那个,嗯」

「我也不是很想听,就到此为止吧」\\

两人似乎是窥见了周所不了解的深渊,光是听他们讲都感到了凉意,那两人则是似乎见过十分出格的场面,安分地嗯嗯点头。\\

「痴情的纠缠,那是真可怕」

「我记下了」\\

结果最大的障碍不是上课、作业,而是人际关系。周在心中铭记下这一点,决定保护好自己,免得被卷入乱七八糟的事件中。\\

「不如说,我也不打算接近除了自己女朋友之外的女生,就算进了社团也会记得跟异性保持恰当的距离。被怀疑出轨就不好了」

「藤宫有种不会去乱勾引人的安心感呢,倒是感觉会被一些奇怪的家伙自发地勾引上」

「能不能别说这些不吉利的!?」\\

周身子发颤地求饶,宫本便哈哈大笑地说是开玩笑,只不过那玩笑不太有玩笑的样子,周的心里仍旧留有恐惧。\\

「那就不说那些了,总之有什么事可以来找我商量,到那时候我求职希望是能结束得差不多了」

「没什么事才是最好的啊」

「……听你们说的这些破事,感觉对大学生活要没有幻想了」

「你本来有?」

「完全没有。感觉就是踏入社会前的节点,或者说是为了就业的必要过程……这样子真的好吗,去大学不是为了做学术,而是为了确保未来的职业」\\

在周看来,他并不怎么觉得大学是去享受的,而认为那是为了未来求职的一份追加的准备期。\\

当然,周对想学的领域还是有志气和热情去学习的,但那也不是打算将其作为职业,更像是为了拓宽今后的人生选择而上大学。\\

虽然周知道大学原本是研究学术的机构,但要问他愿不愿意付出一切去探求学问,他也只能摇头。\\

这样真的就好吗?周感到不安,而身为在读大学生的宫本则是很傻眼地说「烦恼这个?也太认真了吧」。\\

「也没什么不好的吧?不如说,我感觉没多少高中生会坚定地跟打了鸡血一样喊着说就是要做某某职业,所以去上某某大学,做某某方面的学问的吧?大家要么是单纯对某个学科感兴趣,要么是因为没有学历的话就业会遇到困难,然后还有一些人的理由是随大流,或是能延缓就业、多一点自由时间之类的」

「呜哇扎心了」

「莉乃你最好谢谢高中时把你弄哭的阿姨」

「有好好谢谢妈妈的。而且现在我也有个正经目标,没问题的啦」

「嚯」

「你好烦」

「……总之,人各自有各自的理由,由外人来评判也终究不妥。我觉得比起理由,更重要的是就读期间有什么收获,如何将其应用到人生中。到毕业之后,都是要靠自己的脚来行走的,一切都能在自己的人生中看到结果,所以只要自己对此满意就好了,不用在乎别人怎么说」\\

背后被轻轻推了一把,周原本还有些沉重的后背稍稍变轻了一些。\\

正因为不是父母那样的亲子关系、不是利害关系,也不是亲密的朋友关系,仅仅是打工的,以及人生的前辈说的话,所以落在心中更有种清风拂过的感觉。

若是从志保子或是树口中听到这些话,肯定又会有不一样的感受,但周觉得,能从宫本这里听到刚才那番话,属实是颇为幸运。\\

「哎呀,是在说大学的事吗?」\\

正在周静静体会宫本的话时,丝卷从后台缓步现出身影。

平时丝卷都在服务员不会进去的后台工作,今天还是第一次见到她。她一如既往地展现着随和甜美的微笑,还仿佛是氛围里渗出来一般,飘荡着烤制甜点的香气。\\

「啊,店长。辛苦了」

「好香啊」

「嘻嘻,我在里面试验了一下限时出售的蛋糕,打算着要不要在圣诞节上个新」

「啊,我说怎么不在这边的厨房」\\

看来丝卷是在后台的个人厨房工作,也怪不得带着令人垂涎的香气。

在制作点心方面,丝卷有自己独到的见解,店里的蛋糕都是只有改良到丝卷满意后才会向顾客出售。店里每年都会推出不同的圣诞蛋糕,都是像这样在幕后不断摸索做出来的。\\

「完蛋,我没吃零食,能量不够了……明明快到晚饭了……呜呜好香」

「啊,正好还想找一个人来试吃,你去偷偷尝一个吧。可别告诉其他来兼职的哦?」

「神啊……」\\

大桥夸张地开始合掌拜了起来,丝卷乐呵地笑着,然后注意到周的目光便笑吟吟地做出过来的手势。\\

「两位也来吧,如果不讨厌甜食的话」

「太好了。店长的蛋糕可好吃了」

「哎呀,小嘴真甜」

「就是就是」\\

宫本和大桥走过呵呵笑着的丝卷面前。\\

而丝卷看向周,好像在问他要不要来。周沉默片刻,然后直勾勾地注视丝卷的眼睛:\\

「那个,店长,能稍微占用点时间吗?」

「嗯?」\\

丝卷愣了愣,她完全预判不到周打算说什么。周先是闭上了一次眼,为这话该不该说而烦恼。\\

唐突的介绍后二话不说就录用、在排班上行了很大的方便、提供了咖啡豆(虽然是给真昼那边)、像这样表示关怀,周在很多方面都深受她的照顾,若是没有她,也就不会有现在舒适的工作环境。\\

因此,周才不知道再继续拜托她是否合适——但觉得她很胜任,这一点是没错的。\\

对疑惑地打量他的丝卷感到抱歉的同时,周为了让即将来临的圣诞节过得成功,为了取得那块重要的拼图,踌躇地开口道:\\

「……像是新手也能做的蛋糕,这方面的食谱您是否了解?」
