\subsection{学习会与休息}

吃完饭后,众人重新开始了学习。然而,千岁终究还是集中不了注意力,在吃零食的时候又边喊「我累了~」边打起了滚。\\

「周~我可以玩游戏吗?」

「要玩随便你,成绩怎么样我就不管了」

「哎哟好严格」

「只是放松一下的话倒没问题,但你会玩过头。要是觉得自己能调整好的话就去玩吧」\\

周一边做着教辅上的题目一边说道「反正我要继续学习」。这时,在他的眼角余光里出现了微微鼓着脸蛋的千岁。\\

周本来就预料到讨厌学习的千岁差不多该腻了,所以他在电视柜里的游戏机旁边备齐了四人份的游戏和手柄。\\

毕竟人类的注意力是无法一直集中的,如果只是玩一小会儿休息一下,周觉得没有问题。\\

至于周自己,由于每隔一小时就会稍事休息,即使没有长时间的休息也没有问题,况且他不讨厌学习,能坚持很长时间。\\

「周好冷淡~」

「今天是来开学习会的吧。算了,玩一会儿也行,手柄有四人的份,你就玩一会儿权当休息吧」

「那就听你的啦~不过周你一直紧绷着神经学习也不行哦?」

「我休息过了」

「这么认真吗,啊不周你就是这么认真。那我要玩~阿树来吗?」

「来吧。不过我不会一直玩就是了」\\

树连续学习两三个小时确实是学累了,他对游戏也起了兴致。\\

「优太你也来吗?」

「可以吧。藤宫,行吗?」

「嗯」\\

比树和千岁要认真的门胁也对玩游戏来休息表示出了兴趣。周表明完「随你们喜欢」的态度后,再次将视线投向教辅书。

顺带一提,真昼在旁边静静地做着习题集,并没有分心的迹象。\\

「真昼不玩吗」

「我再学一会儿」

「嗯」\\

周之所以会学个不停只是因为他发过誓,这次要努力学习,而真昼向来都是这样。因此,周对她的勤勉佩服之至。

正是不懈的努力,才造就了她始终第一的名次,而这不懈的努力恰恰是真昼厉害、了不起的地方。\\

周看着他们三人急忙离开书桌,占领起了电视机前的阵地。接着,周把他们从脑海中赶走,拿自动铅笔写了起来。\\

笔芯划过纸张的声音、橡皮擦的声音,还有旁边真昼的呼吸声都清晰可闻。\\

周隐约听着稍远处传来他们的大呼小叫声,一边回忆着每位老师的出题倾向,一边重点解答着可能会考的题目。\\

有个老师从高一开始就一直在出题,那个老师负责的考试还挺轻松的。在去年一年里,根据他的性格和课程范围,周牢牢把握住了他会考哪些东西。

至于今年开始教他们班的老师,周计划在这次考试和今后的小测中掌握出题的倾向。\\

周姑且是预测了出题的范围,教着千岁这些范围里的内容。虽然这属于投机,但是这些预测差不了多远,只要重点学习这些部分,挂科还是不至于的。\\

「周君,请用」\\

周默默解答着问题,不知不觉中,原本在旁边的真昼已经站了起来。在周手边放好了一杯咖啡,里面大概放有奶精和一小块方糖。

看到这杯咖啡,周舒缓下了脸颊。\\

「和平时的一样,可以吧?」

「嗯。Thank you」\\

两人在彼此身边已经有半年,都非常清楚对方喜欢吃什么。

真昼正好在他想要喝的时候把咖啡拿了过来。周一边道谢一边捏住杯把,这时他发现咖啡之外还放着一个小盘子。\\

「这是什么?」

「用小型模具烤的费南雪。我觉得学习需要点糖份,所以昨天就烤好了」\\

小盘子里放着一口大小的费南雪,呈现出火候适中的褐色。\\

而且她还周到地考虑到要在学习间隙时享用,为了不弄脏手,上面插着小叉子,费南雪的大小也恰好合适。\\

玩着游戏的树他们似乎也有份。这边的托盘上有三人份的费南雪盛放在盘子里,分量不少,小叉子附在旁边。

咖啡也是准备了三个人的。托盘上放着糖包和奶精,可以自由加糖或加奶。\\

「千岁你们也请用」\\

真昼微笑着悄悄走近他们,把托盘放到了那边的一张小桌上。\\

「哇~!谢谢昼儿!」

「哇,有点心,时间也正好。谢谢了,椎名」

「不用谢」\\

真昼高兴地望着为点心而欢喜的三人,回到了周这边。看到她,周也自然地翘起了嘴角。\\

「……总觉得,害你花了不少功夫准备啊」

「不,这是我自己想做的。我做这些是利用了学习的空闲时间,所以正好当作休息」

「你真是那种尽心尽力的类型啊」

「……我只是为想尽力的人尽心尽力而已」\\

听到这句小声的嘀咕,周感觉喉咙涌上了一阵热量。

趁着还没吐出来,周慌忙喝下咖啡将其咽回去。他只觉得咖啡很甜。按说砂糖的量和平常没什么区别,他却觉得相当甘甜。\\

周不知道对这种并不讨厌的甘甜、对真昼的话做出怎样的反应才好,只能自欺欺人似的把视线落向了教辅书。
