\subsection{春日风暴来临的预感}

太刀川临时企划的卡拉OK活动虽然几经波折,但还是在所有参加者都心满意足的情况下结束了。\\

像是预约的时候名字出了差错,差点导致活动无法顺利进行(似乎是打电话预约的时候对面把太刀川听成内川了);门胁的到来让几个人之间擦出了火花;还有人搞错了卡拉OK的地点,需要去接他等等。尽管发生了很多事情,结果还是玩得很开心。\\

真昼一开始也很拘谨,不过习惯之后,她本人可能也没意识到自己变得很兴奋,只见她用比平时更红润的脸颊,笑咪咪地凝视着唱歌的周,其他男生也被那笑容的余波击中,或许该让她学会控制一下。\\

真昼本人也怯生生地用清脆可爱的歌声唱了几首流行歌曲,让所有人都听得非常开心。\\

就这样,在卡拉OK度过约三个小时后,一行人便解散了。真昼看起来有些疲惫,但更多的是满足感,今天的卡拉OK活动对她来说应该也是很好的刺激。\\

「玩得很开心呢」\\

和所有人道别后,周和真昼踏上回家的路。两人手牵着手,悠闲地走在平常回家的路上。\\

四月的脚步将近,太阳落山也变晚了。现在还不到下午5点,天空却很明亮,呈现出傍晚的氛围,橙色的光芒照耀着两人。

沐浴在夕阳下的真昼心情似乎平静了一些,脚步却比平时更轻快。\\

「是啊,大家都唱得很起劲」

「这种多人一起唱歌的时候,与其一本正经,还不如放轻松一点,这样才比较开心」

「……难道我违反了什么礼仪?」

「不是不是,不是那个礼仪的问题,而是那样做会比较开心。还有,你今天也比平常更兴奋哦」

「咦」

「你做这种不常经历的事情时,和朋友或男朋友……嗯,就是我啦,一起行动的时候,情绪都会比平常高亢个一两级。与其说是积极地参与,更像是比平常更开心地在一旁看着我们玩得很开心的样子」

「有、有那么夸张……」

「我、千岁还有树都注意到了」

「为什么不告诉我!」\\

真昼发现自己被周围的人关注着,顿时满脸通红地瞪向周。不过,即使她用湿润的眼眸和不悦的表情瞪着自己,周也完全不觉得可怕,反而涌起一股想微笑的冲动,觉得她很可爱。\\

「因为看你笑咪咪地玩得很开心的样子,大家都很高兴,而且也很可爱」

「讨厌」

「不过,你很开心吧?」\\

真昼闹起了别扭,但她生气的原因是周他们观望着自己为这种符合年龄的事情感到高兴,而不是一起玩得很开心。\\

所以周用柔和的语气询问真昼对今天有什么感想,她一下子支支吾吾地闭上嘴,然后露出害羞的表情垂下视线。\\

「……很开心。我希望能再像这样和变得亲近的朋友们开心地交流」

「是啊」\\

虽然周自己也没什么资格说别人,但真昼现在能卸下天使的面具,表现出椎名真昼这个少女的一面,简直是刚刚遇到的时候无法想象的。\\

尽管她还没有改掉掩饰的习惯,但也越来越常表现出真实的自己,交到了不是表面关系的『朋友』。

只有周才知道的真昼的部分逐渐减少,他感到有些寂寞,而真昼能够过上充满期待的开朗生活,却是让他成倍地高兴。\\

(而且,我还知道很多其他人不知道的真昼的样子)\\

周不打算让别人看到那些,现在这样就足够了。\\

「唱了那么多歌,肚子饿了」\\

真昼害羞地用眼睑遮住眼睛,同时戳着周的手背以掩饰害羞。周用一如往常的语气对她说道。

真昼似乎渐渐恢复了平静,她深吸一口气后笑了起来。\\

「是啊,回去以后我马上准备做饭」

「今天晚餐主要是由你负责吧。今天晚餐吃什么来着?」\\

有打工的日子基本上是由真昼负责做饭,其他日子则是由周负责,或者由真昼主动要求做晚餐。虽然分配得并不公平,但两人基本上是轮流做饭的。

话虽如此,不管谁是主要做饭的人,另一方若是在场也会帮忙,所以值班制已经流于形式了。\\

昨天是真昼去采购食材并决定菜单,所以晚餐的内容只有她才知道。\\

「昨天去采购的时候猪肉嫩绞肉很便宜,所以就决定做姜汁烧肉、卷心菜丝和凉拌西红柿还有味噌汤。啊,我想把冰箱里的东西吃掉,所以也会把预先做好的炒青椒加进去。味噌汤要放什么配料?」

「给我点选项吧。今天放学后不会绕去其他地方了吧?」

「豆腐、海带和紫菜是常备菜,金针菇和舞菇是前几天冷冻的,所以也可以用。还有胡萝卜和洋葱……如果切碎的话,也有冷冻的大葱」

「那就放紫菜、豆腐和金针菇吧。我喜欢紫菜」

「了解,紫菜要多一点对吧……今天没什么需要帮忙的,你可以慢慢来哦?」

「不要」

「真是的」\\

自从周开始打工以后,真昼就认为他会很累,所以不想让他下厨。但周并不想把所有事情都交给真昼,更重要的是,和真昼一起做饭的话,无论是和她共度的时间,还是做完饭后悠闲放松的时间都会增加,所以力所能及的事他都想去做。\\

「你不觉得我最近切菜切得越来越好了吗?」

「确实切得很细呢。之前……」

「切得大概有铅笔那么粗吧」

「为什么要在这时候得意洋洋的?」

「因为我觉得这是很显著的进步」

「嗯,你进步了很多,很棒哦」

「对吧」\\

以水平来说,算是获得了真昼的认可……倒也不至于,但周已经进步到可以独自做出像样的料理,而且就算让真昼吃,也不会被挑太多毛病的程度。以一个人生活来说,这样的厨艺已经足够了。\\

希望真昼能让进步到这种程度的他做点什么,而不是让他闲着。\\

「所以,我要让你见识一下我的进步」

「……真是的」\\

根据至今为止的相处,周很清楚真昼说的「真是的」并不是在拒绝他。那是伴随着妥协和喜悦。\\

证据就是真昼的嘴唇柔和地弯起,眼神也充满了怜爱。\\

光是这样,就足以让周察觉到自己是如此地被爱着,他感到胸口一阵发痒,同时又充满了温暖,于是重新握紧了真昼的手。\\

「那么,回家后就准备晚餐吧」

「在那之前要先洗手漱口哦」

「知道的啦」\\

听到真昼像监护人一样唠叨,周忍不住背脊发抖,强忍着笑意。「你刚才在想我像妈妈对吧?」真昼撅起小嘴。

周没有吐槽她「你这不是自己承认了」,只是笑着与真昼十指交扣,心情平静地望着两人长长的影子,悠闲地走在回家的路上。\\

\vspace{2\baselineskip}

两人以稍微缓慢的步伐欣赏着景色回到公寓,发现大厅的大门前站着一名少年。\\

他的年纪大概不到十五岁,身高比真昼稍微高一点。明亮的发色,稚气未脱的端正五官,以及充满活力的眼神是他最大的特征。\\

在这栋公寓住了将近两年,周也记住了不少住户的长相。

虽然不是全部,不能把话说死,但他不记得有看过这个孩子。至少像这个少年这样显眼的孩子,只要看过一次就不会忘记。\\

不知道他是被关在门外,还是有事要去某户人家。\\

周不知道他的情况,但他的表情看起来像是遇到了什么麻烦。\\

「我们公寓没有那个孩子吧」

「在我的记忆中也不记得有见过。他看起来很困扰的样子,是来见某个人的吗?」\\

既然真昼也不认识,那他恐怕不是这里的住户。

无论如何,不通过自动门就无法回家,小孩子这个时间在外面走动也很危险,所以周还是决定去跟他搭话。周走向对讲机,而少年的表情随着周的接近而变得越来越僵硬。\\

「那边的小朋友,你怎么了?来这栋公寓有事吗?想找人的话,你知道房间号码吗?」\\

周尽量用柔和的声音询问,还蹲下来避免让少年感到压迫感,可是少年的视线却避开了周。\\

少年并不是移开了视线。\\

他的视线带着明确的意志,捕捉到了站在旁边的真昼的身影。\\

「……姐姐?」\\

尚未变声,还带着点稚嫩的高音断断续续地编织出话语。\\

那声音并不大也不尖锐,却清晰地回荡在空无一人的大厅里,仿佛不允许周和真昼漏听。
