% \subsection{40 天使様の照れと不機嫌}
\subsection{天使大人的害羞与不悦}

% 羞恥から立ち直った真昼は一度家に帰り、着替えて戻ってきた。\\
从羞耻中回过神来的真昼先回了趟家,换好了衣服回来了。\\

% ただ、まだ恥ずかしいのか周と視線が合うと微妙に視線が逸れるため、周も気まずさを覚えてしまう。
不过,似乎还在害羞着的真昼每每与周对上眼便会微妙地偏开视线,搞得周也开始尴尬了起来。

% ソファの隣に座ったはいいものの、非常に居たたまれない。\\
虽说万幸真昼还愿意一起坐在沙发上,可周却觉得如坐针毡。\\

% 「……許してくれ」\\
「……原谅我吧」\\

% なんというか居心地が悪く思わず謝罪をすると、真昼がちらりと周を見てそっとため息をつく。
不知为何总觉得不安稳的周下意识地向真昼道歉,而真昼则呼得看向周,然后叹了一口气。

% 大分表情から照れが除去出来たのか、一応いつも通りの表情に戻っている。\\
大概是为了隐去脸上的害羞吧,叹完气后,真昼恢复了一如往常的表情。\\

% 「怒ってる訳ではありません。周くんが謝る必要性はありませんから」
「我没有生你的气。周君并没有跟我道歉的必要」

% 「いやでもなあ」
「不过啊」

% 「私はただ、自分の迂闊さを後悔しているだけです。あんなだらしなくて見るにたえない顔を見せてしまったので」
「我只是对被看见了那样见不得人的脸的不像样子的自己感到后悔罢了」

% 「見るにたえないって……普通に可愛かったけどな」\\
「见不得人……虽然其实只是普通的很可爱啊」\\

% 天使というあだ名に恥じない、まさに天使のような寝顔だったし、起きてからの寝ぼけ眼も油断して緩みきったあどけない顔も非常に可愛らしかった。\\
那不负天使这一外号的,实在如同天使般的睡脸、醒来之后的惺忪睡眼、还有那毫无戒备放松下来的天真表情,全都十分可爱。\\

% 寝ぼけると普段の冷静で落ち着いた表情が一変して幼さの強い表情になるというのは新発見である。
与平时那冷静而平稳的表情截然不同,在睡迷糊的时候真昼会露出十分幼气的表情,算是一个新发现吧。

% むしろもっと見ていたいくらいにはよいものだったのだが、真昼的にはやはり油断しきった表情は見られたくないのだろう。\\
周倒是想多看看,从这个角度来说反而是个好事,不过真昼应该是不想自己粗心大意的表情被看到吧。\\

% だらしないとも見るにたえないとも思わなかったのでそこだけは否定させてもらうと、真昼はきゅっと唇を噛んで、何故か抱き抱えていたクッションで周をぽすぽすとはたいてきた。\\
周并不觉得那表情很不像样子或者见不得人所以想要否定那一部分,结果不知为何却让真昼咬着嘴唇用抱在怀里的靠枕嘭嘭地拍起了周。\\

% 痛くはないし真昼も本気ではないのだろうが、いきなりはたかれて訳が分からない。\\
周也不痛,真昼感觉也不是认真的,但周还是搞不明白真昼怎么突然就拍起自己来了。\\

% 「何だよ」
「干嘛啊」

% 「……周くんのそういうところがだめです」
「……周君这种地方真的是不行呢」

% 「何がだよ……どう直せと」
「什么啊……那你要我咋样」

% 「そういう事を軽々しく言うものではありません」
「这种事情是不能这么轻描淡写的说的」

% 「別に他の誰に言う訳でもないし……」\\
「反正我也没跟别的人这么说的机会……」\\

% 周の周りに居る女性など、真昼か千歳しか居ない。
数起周身边的女性,除了真昼和千岁就没了。

% 千歳は確かに可愛い分類に入るものの、周にとってはめんどくさいという気持ちの方が先に来るし面と向かって誉める必要もないので、真昼くらいしか称賛する相手は居ないだろう。\\
虽说千岁可爱倒也名副其实,但一提到她周下意识就觉得是个麻烦,因而也没必要当面称赞她,因此除了真昼周也没有谁能夸了。\\

% 真昼が固まっているので不審に思いつつ、肩を竦める。\\
想到真昼性格认真可能会觉得这很可疑,周耸了耸肩。\\

% 「お前、そういうの言われ慣れてるだろ? 別に今更だろ」\\
「唉,你的话早该习惯被这么说了吧?咋还这么介意」\\

% そもそも真昼には何度も可愛いと認識している事は伝えているので、今更そこに突っ込まれるなど思いもしなかった。\\
再说周向真昼表达自己觉得她可爱早就不是一次两次了,事到如今真昼还在介意这里让周匪夷所思。\\

% 真昼は真昼で自分がどれだけ見目麗しいか正確に把握しているだろうし、褒められるのも慣れている筈なのだ。
真昼的话也应该对自己长得有多漂亮心知肚明,被夸奖也应该早就习惯了。\\

% 周一人にどうこう言われたところで、そう照れるものではないだろう。\\
照理不论周怎么夸,真昼也不至于害羞成这样啊。\\

% そう思っていたのだが、真昼は何故だか渋い顔をしている。\\
周这么想着,真昼却不知为何变得一脸不快。\\

% 「ほんとにさっきからどうしたんだ」
「所以说你到底咋了从刚才开始」

% 「……何でもないです」\\
「……什么也没有啦」\\

% 最後にもう一度ぽすんとクッションで物理攻撃を加えた真昼は、ぷいとそっぽを向いて「お雑煮作ります」と言い残してエプロンをつけてキッチンに向かってしまう。\\
最后又嘭地用靠枕补了一发物理攻击的真昼,哼地扭过了头,丢下一句「我做年糕汤去了」后,便穿起围裙去了厨房。\\

% 押し付けられたクッションを手にしながら、周はいきなりほんのりと不機嫌になった真昼の背を眺めるしか出来なかった。\\
手上拿着强塞过来的靠枕的周,一时间只得呆呆地望着心情有点不好的真昼的背影。\\

% \\


% お雑煮を食べ終わる頃には、真昼は平常通りの表情に戻っていた。
吃完年糕汤之后,真昼恢复了一如往常的表情。

% お雑煮を食べ始めた時点では微妙に違和感を抱かせるような強張りがあったものの、お雑煮もおせちも美味しかったので夢中になっていたら、いつの間にか真昼の機嫌は戻っていたようだった。\\
刚开始吃年糕汤的时候真昼还板着个脸让周微微有些违合感,但这年糕汤太过美味让周吃着吃着就入了迷,等周回过神来,就发现不知什么时候真昼的心情已经恢复了。\\

% ダイニングからお互いにソファに座り直した時には、すっかり元通りだ。\\
一起离开餐厅坐回沙发上的时候,一切都恢复了往常。\\

% 「そういえば、真昼は初詣行くのか?」
「说起来啊真昼,新年参拜你去不?」

% 「初詣ですか? あまり行くつもりはないですけど……人混み好きじゃないんですよね。なんか、じろじろ見られるし」
「新年参拜吗?去倒是不大想去……毕竟我不喜欢人挤人的地方。总有种被盯着的感觉」

% 「それはお前が……」\\
「那还不是因为你……」\\

% とんでもない美人だから、と言おうと思ったが、先程真昼の機嫌を損ねたばかりなので言葉を飲み込み「まあ仕方ないな」と返す。\\
是个不得了的没人啊——周正想这么说,突然想起这话跟刚才的一样只会坏了真昼心情,便把这话咽了回去「嘛这也是没办法啊」回道。\\

% 「周くんは初詣行くのですか?」
「周君打算去新年参拜吗?」

% 「実家に居た頃は両親と行ってたけど、どうしようかなとは思う。少なくともわざわざ元日からは行かなくてもいいなとは思ってるよ」
「在老家的话倒是会跟着爸妈一起去,现在倒是拿不定注意呢。至少我是想着没必要挤着年初一去」

% 「同感です」
「同意」

% 「千歳達は千歳の家で仲睦まじくするらしいし、まあ今時の子供なんてそんな初詣行かないんだよなあ。別に後回しでいいな」\\
「千岁他们的话估计是在千岁家里培养感情吧,嘛要说的话现在孩子们倒也不怎么会去新年参拜的样子。反正以后搞也没差」\\

% なんでも昔に比べれば……特に十代二十代の子供は初詣をする割合が減っているらしいし、周達がおかしいという訳ではない。
要非得跟老一辈们比的话……特别是十几二十多岁的这些年轻人们去做新年参拜的比例似乎少了不少,周他们的这种想法倒也不奇怪。

% 別に行きたくないという訳ではないが、人が多すぎて身動きとれなくて疲弊するだけだと分かっているので、人が落ち着いた頃に行けばいいだろう思っている。\\
虽然也不是不想去,但周也明白人多的动都动不了实在是让人筋疲力尽,因而想着等人少下来了再去也不迟。\\

% 「それにまあ、三が日はゆっくり過ごしたいからなあ。俺は福袋とかどうでもいいし」
「再说了,前三天还是想悠闲点过啊。我的话福袋什么的倒也不在意」

% 「私としては福袋はちょっと気になりますけどね」
「我的话倒是对福袋有点兴趣呢」

% 「ショッピングモールにでも行って来るのか?」
「购物中心你也敢去?」

% 「……あの人だかりに突撃する勇気はないんですよねえ」
「……我是没有朝着那人堆突击的勇气呢」

% 「同感だ」\\
「同意」

% 先程真昼が周にしたような返事を周も返し、ソファに体を預ける。\\
周作出了跟刚才真昼说的很像的回复,把身子靠在了沙发上。

% 別に、正月だからといって、どこかに行く必要もないだろう。
反正,也不是说正月就非得去哪里不可。

% 基本的に面倒くさい事は避けたい周は、こうしてゆったりとするだけで結構に満足だった。どうやら食事の都合上正月中は周の家で過ごすらしいので、会話の相手にもご飯にも困らない。\\
大致上想着要避开麻烦事的周,只要能这样悠悠闲闲地过着日子便十分满足了。而且看上去考虑到方便做饭整个正月真昼都打算在周家里过的样子,这下不管是聊天对象还是伙食都不用愁了。\\

% とても贅沢な正月だな、と思いながら、隣の真昼をひっそりと眺めて小さく笑った。
这可真是个豪华的正月啊——周这么想着,偷偷瞄了一眼坐在一旁的真昼,微微地笑了起来。
