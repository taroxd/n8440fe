\subsection{40 天使様の照れと不機嫌}

% 羞恥から立ち直った真昼は一度家に帰り、着替えて戻ってきた。\\


% ただ、まだ恥ずかしいのか周と視線が合うと微妙に視線が逸れるため、周も気まずさを覚えてしまう。


% ソファの隣に座ったはいいものの、非常に居たたまれない。\\


% 「……許してくれ」\\


% なんというか居心地が悪く思わず謝罪をすると、真昼がちらりと周を見てそっとため息をつく。


% 大分表情から照れが除去出来たのか、一応いつも通りの表情に戻っている。\\


% 「怒ってる訳ではありません。周くんが謝る必要性はありませんから」


% 「いやでもなあ」


% 「私はただ、自分の迂闊さを後悔しているだけです。あんなだらしなくて見るにたえない顔を見せてしまったので」


% 「見るにたえないって……普通に可愛かったけどな」\\


% 天使というあだ名に恥じない、まさに天使のような寝顔だったし、起きてからの寝ぼけ眼も油断して緩みきったあどけない顔も非常に可愛らしかった。\\


% 寝ぼけると普段の冷静で落ち着いた表情が一変して幼さの強い表情になるというのは新発見である。


% むしろもっと見ていたいくらいにはよいものだったのだが、真昼的にはやはり油断しきった表情は見られたくないのだろう。\\


% だらしないとも見るにたえないとも思わなかったのでそこだけは否定させてもらうと、真昼はきゅっと唇を噛んで、何故か抱き抱えていたクッションで周をぽすぽすとはたいてきた。\\


% 痛くはないし真昼も本気ではないのだろうが、いきなりはたかれて訳が分からない。\\


% 「何だよ」


% 「……周くんのそういうところがだめです」


% 「何がだよ……どう直せと」


% 「そういう事を軽々しく言うものではありません」


% 「別に他の誰に言う訳でもないし……」\\


% 周の周りに居る女性など、真昼か千歳しか居ない。


% 千歳は確かに可愛い分類に入るものの、周にとってはめんどくさいという気持ちの方が先に来るし面と向かって誉める必要もないので、真昼くらいしか称賛する相手は居ないだろう。\\


% 真昼が固まっているので不審に思いつつ、肩を竦める。\\


% 「お前、そういうの言われ慣れてるだろ? 別に今更だろ」\\


% そもそも真昼には何度も可愛いと認識している事は伝えているので、今更そこに突っ込まれるなど思いもしなかった。\\


% 真昼は真昼で自分がどれだけ見目麗しいか正確に把握しているだろうし、褒められるのも慣れている筈なのだ。


% 周一人にどうこう言われたところで、そう照れるものではないだろう。\\


% そう思っていたのだが、真昼は何故だか渋い顔をしている。\\


% 「ほんとにさっきからどうしたんだ」


% 「……何でもないです」\\


% 最後にもう一度ぽすんとクッションで物理攻撃を加えた真昼は、ぷいとそっぽを向いて「お雑煮作ります」と言い残してエプロンをつけてキッチンに向かってしまう。\\


% 押し付けられたクッションを手にしながら、周はいきなりほんのりと不機嫌になった真昼の背を眺めるしか出来なかった。\\


% \\


% お雑煮を食べ終わる頃には、真昼は平常通りの表情に戻っていた。


% お雑煮を食べ始めた時点では微妙に違和感を抱かせるような強張りがあったものの、お雑煮もおせちも美味しかったので夢中になっていたら、いつの間にか真昼の機嫌は戻っていたようだった。\\


% ダイニングからお互いにソファに座り直した時には、すっかり元通りだ。\\


% 「そういえば、真昼は初詣行くのか?」


% 「初詣ですか? あまり行くつもりはないですけど……人混み好きじゃないんですよね。なんか、じろじろ見られるし」


% 「それはお前が……」\\


% とんでもない美人だから、と言おうと思ったが、先程真昼の機嫌を損ねたばかりなので言葉を飲み込み「まあ仕方ないな」と返す。\\


% 「周くんは初詣行くのですか?」


% 「実家に居た頃は両親と行ってたけど、どうしようかなとは思う。少なくともわざわざ元日からは行かなくてもいいなとは思ってるよ」


% 「同感です」


% 「千歳達は千歳の家で仲睦まじくするらしいし、まあ今時の子供なんてそんな初詣行かないんだよなあ。別に後回しでいいな」\\


% なんでも昔に比べれば……特に十代二十代の子供は初詣をする割合が減っているらしいし、周達がおかしいという訳ではない。


% 別に行きたくないという訳ではないが、人が多すぎて身動きとれなくて疲弊するだけだと分かっているので、人が落ち着いた頃に行けばいいだろう思っている。\\


% 「それにまあ、三が日はゆっくり過ごしたいからなあ。俺は福袋とかどうでもいいし」


% 「私としては福袋はちょっと気になりますけどね」


% 「ショッピングモールにでも行って来るのか?」


% 「……あの人だかりに突撃する勇気はないんですよねえ」


% 「同感だ」\\


% 先程真昼が周にしたような返事を周も返し、ソファに体を預ける。\\


% 別に、正月だからといって、どこかに行く必要もないだろう。


% 基本的に面倒くさい事は避けたい周は、こうしてゆったりとするだけで結構に満足だった。どうやら食事の都合上正月中は周の家で過ごすらしいので、会話の相手にもご飯にも困らない。\\


% とても贅沢な正月だな、と思いながら、隣の真昼をひっそりと眺めて小さく笑った。

