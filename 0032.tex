% 32 天使様の幸せの味


% 「……はー、うまかった」\\


%  相変わらず、真昼の料理はうまかった。\\


%  クリスマスという事でいつもより手の込んだ料理が出ていた。


%  真昼によりじっくり煮込まれたビーフシチューはポットパイにされて、割りながら食べ進める形となっていた。\\


%  パイを割る楽しみを味わってからのサクサクの食感にビーフシチューのコクのあるソースを絡めて食べるのは、至福のひとときとしか言えない。\\


%  パイ生地をわざわざ仕込んだらしい真昼の謎に高い技術力に感服しつつ、本日二つ目のケーキを平らげたところで、一息つく。\\


%  ちなみにケーキまで真昼の手作りである。\\


%  ポットパイ用のパイ生地を仕込んだついででもう一つ菓子用の生地を同時に仕込んでいたらしく、ミルフィーユを作ってくれた。最早職人レベルである。\\


% 「お粗末様でした。……よく食べましたね」


% 「ん。美味しかったからな」


% 「それはありがとうございます」\\


%  微かな笑みも、見慣れてきた。\\


%  彼女は美味しいというと安堵したような笑みを浮かべるので、それを見るのが日課のようなものだった。


%  普段の表情よりもずいぶんと柔らかい顔が浮かぶのを見るのは周の特権のようで、なんだかくすぐったさがある。\\


% 「……明日はオムライスか……すげえ楽しみ」


% 「オムライス好きなのですか」


% 「卵料理全般好き」


% 「ああなるほど……だし巻き玉子とかすごい勢いで食べてましたからね」


% 「うまいんだから仕方ないだろ」\\


%  いくら好物の卵料理でも、不味ければ食べない。あんなに食が進んだのは、真昼の料理が美味しいからだろう。\\


%  独り占めしているのはすごく贅沢なんだろうな、と思うものの、誰かに譲る気はない。真昼が作るのをやめるまでいただき続ける所存である。\\


% 「……周くんって、ご飯食べてる時はすごく幸せそうですよね」


% 「事実幸せというか、真昼の料理がうまいからな」


% 「それはありがたい限りですけど、お安い幸せですね」


% 「いや割と高いぞ……お前自分の価値を把握しろよ……」\\


%  なにせあの天使様の手料理である。一部の男子には喉から手が出るほど食べる権利がほしいだろう。\\


% 「私にとっては毎日作ってるものですからねえ」


% 「幸せもんですなあ俺も」


% 「……そんなにです?」


% 「そりゃ、うまい料理毎日食えてる訳だし」\\


%  基本的には物欲があまりない周にとっては食欲の方が強く、毎日美味しい料理を出来立てで食べられるというのが一番の幸せだ。\\


% 「どうやってこんなに料理を作れるようになったんだ」


% 「お世話してくれていた人が教えてくれました。『必ず幸せにしてくれる人の胃袋掴むのよ』って」


% 「すまんな俺なんかの胃袋掴ませて」


% 「予行練習という事で」\\


%  くすりと小さく笑んだ真昼に、不覚にもどきりとした。\\


% 「……しかしまあその世話してくれた人もすげえな」


% 「そうですね、あの人はすごく料理がうまかったですから。まだまだ私はあの人には敵いません、あの人の料理は、幸せの味がするんですよ」\\


%  微かながらに柔らかな笑みを浮かべて少し遠い目をした真昼に、周はひっそりと安堵した。\\


%  この言い方であれば、真昼はその世話役の人間に可愛がられていたであろう。真昼からもその人物を慕っているのがよく分かる。


%  親に見向きもされなかった代わりに、その人が真昼に色々と大切なものを教えてくれたようだ。\\


%  そんな人が真昼の側に居てくれたのは、本当に僥倖だろう。


%  恐らく話ぶりから女性だと思うが、彼女が居てくれたからこそ、真昼はこうしてまっとうに生きているのだと思う。\\


% 「余程うまかったんだろうな。まあ俺にとっちゃお前のが幸せの味なんだが」\\


%  母親はさておき、父親の料理もうまいが真昼の方が周の好みの味付けだ。\\


%  真昼の料理は、毎日食べても飽きないようなほっとする味がする。心休まる癖に心踊る料理で、全然飽きる気配がないしもっと食べたいとすら思うのだ。


%  まあ流石に真昼の負担が大きすぎるのでそんな事は言わないが。\\


%  うんうんと頷いていたら、真昼が固まっていた。\\


%  虚をつかれた、と言えばいいのか。


%  どこかぽかんとした、幼さを隠さない表情で、こちらを見ている。\\


% 「……真昼?」


% 「え、……なんでもないです」\\


%  声をかけられて我に返ったらしい真昼が慌てて首を振って、俯く。


%  お気に入りのクッションをぎゅっと抱き締め、そっと吐息をこぼしている真昼は、先程とは打ってかわって妙に色っぽさを感じた。\\


% 「どうかしたのか」


% 「……単に、私なんかが幸せの味作れていたのかな、って」


% 「何で卑下してるのか分からんが毎日食いたいくらいにはうまいぞ」


% 「……ありがとうございます」\\


%  ちら、とこちらを見上げて少し照れ臭そうに目尻を下げて小さく微笑んだ真昼に、今度は周が俯いて顔を隠したくなった。\\


%  本当に極たまに見せるこんな表情は、異性として好きという訳でなくても心臓を否応なしに跳ねさせる。\\


%  いつもの仮面を剥がして無防備といってもいい笑顔を見せてくれる真昼に、周は今すぐ顔を冷やしたい気持ちで一杯だった。


%  じわじわとせり上がってくる熱がばれてしまうのは、嫌だ。お互いに照れていたら、確実に気まずくなる。\\


% 「あー、その、……そうだ真昼」


% 「はい」


% 「明日は、昼からでいいんだよな?」\\


%  この空気に耐えきれず強引に話題を変えてしまったのだが、真昼はさほど気にした様子はなく周の言葉に思案している。\\


% 「はい、そういう約束ですよね? お昼ご飯作って、それから約束のげーむ、する……んでしたよね」


% 「おう」


% 「いや……でしたか?」


% 「違う、確認しただけだ。……本当に、イブを過ぎたとはいえクリスマスにそういう過ごし方でいいんだな?」


% 「嫌なら言いません。……たのしみに、してます」\\


%  また小さくふわっと緩んだ笑みが彼女に浮かんで、周は直視出来ずに「おう」とおざなりに返事して真昼と反対側の肘置きにもたれて羞恥を隠すしかなかった。\\


