%\subsection{54 贈り物の選び方}
\subsection{选择礼物的方法}

% 元々勉強面では勤勉であり授業態度は真面目そのものの周は、特に苦労する事なく学年末考査を終えた。\\


% 真昼と共にテストの確認をしてもいつも通りの点数は取れそうだったし、まず学校での普段の態度はよろしいので留年なんて事はほぼないだろう。


% 樹もそれなりの点数を取っているし、千歳も赤点は免れていそうなくらいの出来だったらしいので、周が親しくしている人間では留年の危機はまずなかった。\\


% 後は特に関わりのない三年生を送る卒業式があり、その後修了式が待ち構えているのだが……その間にある一つのイベントが問題だった。\\


% 「……何を返そう」\\


% そう、バレンタインデーの勝者に訪れるお返しの日である。


% 周が勝者かどうかはさておき、真昼と千歳からもらったのだから、当然お返しはするつもりである。\\


% ただ、困った事に、何が良いのかと悩んでしまう。


% 千歳は無難にクリスマスにケーキを買った店のホワイトデー用に用意された詰め合わせと、彼女がコレクションしているキャラクターのグッズを用意するつもりだ。\\


% 問題は真昼だ。


% 真昼は、おそらく何でも喜んで受け取ってくれそうな気がする。


% 周からの贈り物は普通に受け取ってくれるし、気持ちを重視しているようなので特にものには拘っていなさそうなのだ。正直一番困る。\\


% 好みから選ぼうにも甘いものと可愛いものが好き、といった女子なら割と共通していそうな嗜好しか知らないので、どんなものを選ぼうかとずっと悩んでいた。\\


% さすがに前言っていた砥石は色気もへったくれもない上に予算的に厳しいものがあるので除外するとしても、何にしようか悩ましい。出来る事なら、今回は実用品より嗜好品をあげたい。\\


% とりあえずで雑貨屋でホワイトデー特集のコーナーを眺めているのだが、彼女が本当に喜んでいる姿をうまく想像出来ない。


% 出来れば、くまのぬいぐるみをあげた時のような、あんな反応をしてもらえるようなものがよい。\\


% (さすがにぬいぐるみ二回目だと芸がないしなあ)\\


% 可愛らしいぬいぐるみなら棚に沢山陳列されているが、同じ贈り物をするのは新鮮味に欠けるだろう。


% かといって、女子が喜びそうなものなんて周の貧困な想像力ではアクセサリーとかくらいしか思い付かない。\\


% しかしアクセサリーを贈る間柄なのか、と言われるとすぐに頷く事は出来ないのだ。\\


% 多分普通に受け取ってもらえるだろうが、向こうが喜ぶかどうか。


% 一応、男女にしては仲がよいとは思うが……果たしてアクセサリーを贈って喜ばれるのだろうか。\\


% これが樹で千歳に贈るなら間違いのないチョイスだが、周が真昼に贈っていいものなのか。\\


% 悶々と悩んではうろうろと特集コーナー付近をうろついているため、恐らく不審者に見られている事だろう。


% 一応外行きの格好をしているものの、男が可愛い雑貨の前でさまよっていれば怪しいに違いない。\\


% ああでもないこうでもないと唸っていると、後ろから「何かお探しですか?」という声がかかる。\\


% 振り返ると、店のエプロンをきた妙齢の女性がにこやかに立っている。


% あまりに悩んでいる周をみかねて声をかけてくれたのだろう。でなければ不審者のようにうろうろおろおろしている周にわざわざ話しかけたりはしない。\\


% 「あー、その……ホワイトデーのお返しに悩んでいて」


% 「こちらのコーナーに目ぼしいものはなかったのですか? 他のコーナーにもホワイトデーのお返しによく選ばれるものもありますので、ご案内しますよ」


% 「あ、いえそういう訳ではなくて……何とも言いがたい間柄で、贈っても嫌がられないものは何か悩んでいて」


% 「というと?」


% 「彼女ではないけど親しいといった感じなので……たとえばですけど、アクセサリーとかは好きでもない相手にもらって嬉しいのかな、と」\\


% 相談するのは気恥ずかしく、ぼかして説明していると、店員の女性はくすりと笑みを浮かべる。恐らく、微笑ましいといった意味合いで。\\


% 「男性の方がそう悩んでいるのもよくお見かけしますよ」


% 「因みに先人はどんな決断を?」


% 「悩んでいましたが、購入を決意する方が多いですね。親しいのであれば、贈っても嫌がられるという事は多分ありませんよ」\\


% 嫌ではない、と言われて少し安堵してしまったが、それでもあの真昼にアクセサリーを贈るのはやはり少し気後れしてしまう。


% 彼女は身なりはきっちり整えているが、あまりアクセサリーは着けない。たまに着けている事もあるが、どれも品のよいものばかりだ。\\


% センスのよい彼女の審美眼に認められるような品物を選べる自信がない。\\


% 「よろしければ、あちらのコーナーで女性に人気の品を幾つかご紹介しましょうか?」


% 「……お願いします」\\


% ありがたい申し出に、周は思わず姿勢を正してうなずいた。\\


% \\


% 「んで買ってしまったと」\\


% 事の顛末を樹に話すと、先日の店員と同じような眼差しで笑われた。\\


% 食堂の端で日替わり定食を二人で食べていたのだが、ホワイトデーの話題になってつい言ってしまったのだ。\\


% 「……うるせえよ。でも、やっぱ交際してないのにアクセサリー贈るって引かれそうでさあ」


% 「女々しいぞ、男は度胸と勢いだ。あの人なら周相手だったら何でも喜ぶ気がするぞ?」


% 「……そうだけどさ」\\


% 真昼の性格的に、何でも普通に喜んで受け取ってはくれるだろう。


% 周としては本当に喜んで使ってもらえるものを贈りたいので、これでいいのかと悩んでいるのだ。\\


% 「結局どんなの買ったんだ?」


% 「……ピンクゴールドカラーの、花モチーフのブレスレット」\\


% 真昼はクールな雰囲気のシルバーや華美な印象を抱かせるゴールドより、華やかさはありつつ柔らかく可愛らしい色合いのピンクゴールドが似合うと思ったのだ。\\


% 流石に学生の身で高価な貴金属は買えないのであくまで見た目だけなのだが、その色のアクセサリーの中から真昼に似合いそうな繊細なデザインを選んだつもりである。\\


% 「何だ、聞く限りには普通に喜ばれそうなやつじゃん」


% 「……引かれないか?」


% 「いや心配しすぎだろ。何でそこは後ろ向きなんだよ……」


% 「女にプレゼントなんてまともに渡したのあいつだけだぞ」\\


% 母親はまずそういう対象ではないし、千歳はノーカウントだ。そもそも彼女に渡すのは本人たっての希望でスイーツになるので、あまり贈り物という意識すらない。\\


% 「お前そういうところ自信ないよなあ……」


% 「むしろなんで自信が持てるんだよ……あいつだぞ?」


% 「くまのぬいぐるみは喜ばれたんだろ」


% 「そりゃそうだけどさ」


% 「周、気持ちだ気持ち。ある程度既にコストをかけて選んだんだからあとは気持ちを込めるだけだ」\\


% 軽く言ってくれる樹に「そう割り切れたら苦労しない」とぼやいて、周は額を押さえた。\\


% ホワイトデーまで、しばらくこの決断がよかったのか悩まされそうである。

