%\subsection{54 贈り物の選び方}
\subsection{选择礼物的方法}

% 元々勉強面では勤勉であり授業態度は真面目そのものの周は、特に苦労する事なく学年末考査を終えた。\\
在学习上本就很勤奋,上课态度也很认真的周,并没有太费劲便通过了期末考试。\\

% 真昼と共にテストの確認をしてもいつも通りの点数は取れそうだったし、まず学校での普段の態度はよろしいので留年なんて事はほぼないだろう。
和真昼一起复查试卷,算出来的分数也跟平时差不多,况且平常在学校里的态度也挺好,留级的事情是可以不用担心的了。

% 樹もそれなりの点数を取っているし、千歳も赤点は免れていそうなくらいの出来だったらしいので、周が親しくしている人間では留年の危機はまずなかった。\\
树考的分也还行,看起来千岁似乎也避免了挂科,这么一来周熟悉的人就都不用担心留级的问题了。\\

% 後は特に関わりのない三年生を送る卒業式があり、その後修了式が待ち構えているのだが……その間にある一つのイベントが問題だった。\\
然后就是没什么关系的送别高三学生的毕业典礼,接着就是休业式了……不过在这之间还有一个问题很大的节日。\\

% 「……何を返そう」\\
「……该回什么礼呢」\\

% そう、バレンタインデーの勝者に訪れるお返しの日である。
这正是,朝着情人节的胜者们逼近的回礼之日。

% 周が勝者かどうかはさておき、真昼と千歳からもらったのだから、当然お返しはするつもりである。\\
且先不论周这到底算不算是胜者,但既然从真昼和千岁那收了巧克力,那么自然要有回礼。\\

% ただ、困った事に、何が良いのかと悩んでしまう。
只不过,周还在头疼该送什么好。

% 千歳は無難にクリスマスにケーキを買った店のホワイトデー用に用意された詰め合わせと、彼女がコレクションしているキャラクターのグッズを用意するつもりだ。\\
千岁的话去那家买了圣诞蛋糕的店定一个白色情人节款的,然后再送个她正在收集的角色周边,那就应该问题不大了。\\

% 問題は真昼だ。
问题是真昼。

% 真昼は、おそらく何でも喜んで受け取ってくれそうな気がする。
真昼的话,总觉得无论送什么她估计都会欣然收下。

% 周からの贈り物は普通に受け取ってくれるし、気持ちを重視しているようなので特にものには拘っていなさそうなのだ。正直一番困る。\\
周送的东西她都平常地收下了,而且看起来她比较在意心意而对送的东西并不是很关心。讲真这样反而很头疼。

% 好みから選ぼうにも甘いものと可愛いものが好き、といった女子なら割と共通していそうな嗜好しか知らないので、どんなものを選ぼうかとずっと悩んでいた。\\
就算想要选点她喜欢的东西,可周清楚的她的喜好也就只有甜的东西和可爱的东西这种女生差不多都喜欢的东西,因而周一直在头疼到底该送啥好。\\

% さすがに前言っていた砥石は色気もへったくれもない上に予算的に厳しいものがあるので除外するとしても、何にしようか悩ましい。出来る事なら、今回は実用品より嗜好品をあげたい。\\
再怎么说,上回说过的磨刀石这种不但一点意思也没有而且还爆预算的东西肯定是免谈,但说到要选啥,周还是没主意。倒是有如果可以的话,这回比起实用品更想送享受品的想法。\\

% とりあえずで雑貨屋でホワイトデー特集のコーナーを眺めているのだが、彼女が本当に喜んでいる姿をうまく想像出来ない。
抱着这样的考虑,周打算总之先去杂货店的白色情人节商品角看看,但一圈下来也没找到什么感觉真昼会真的喜欢的东西。

% 出来れば、くまのぬいぐるみをあげた時のような、あんな反応をしてもらえるようなものがよい。\\
可以的话,周想要送一件能让真昼露出上次收到熊布偶那时的那种反应的礼物。\\

% (さすがにぬいぐるみ二回目だと芸がないしなあ)\\
(不过送两次熊布偶就显得有点应付了啊)\\

% 可愛らしいぬいぐるみなら棚に沢山陳列されているが、同じ贈り物をするのは新鮮味に欠けるだろう。
虽然可爱的熊布偶货架上倒是摆了一堆,但送两次一样的东西还是欠缺新鲜感。

% かといって、女子が喜びそうなものなんて周の貧困な想像力ではアクセサリーとかくらいしか思い付かない。\\
但话说回来,以周那贫乏的想象力,能想到的女生喜欢的东西,除了小饰品以外也没别的了。\\

% しかしアクセサリーを贈る間柄なのか、と言われるとすぐに頷く事は出来ないのだ。\\
可两人的关系到底到没到送小饰品的程度,周仍然摸不大准。\\

% 多分普通に受け取ってもらえるだろうが、向こうが喜ぶかどうか。
估计真昼还是会好好地收下,但问题是她到底会不会高兴。

% 一応、男女にしては仲がよいとは思うが……果たしてアクセサリーを贈って喜ばれるのだろうか。\\
虽然,以男女之间来说应该算是关系不错……但绕来绕去问题还是在她会不会高兴上。\\

% これが樹で千歳に贈るなら間違いのないチョイスだが、周が真昼に贈っていいものなのか。\\
这要是树给千岁的那肯定没有问题,但周给真昼送就得画一个问号了。\\

% 悶々と悩んではうろうろと特集コーナー付近をうろついているため、恐らく不審者に見られている事だろう。
恐怕周这样一脸烦恼地在特卖角旁边晃来晃去,被当成是怪人了吧。

% 一応外行きの格好をしているものの、男が可愛い雑貨の前でさまよっていれば怪しいに違いない。\\
虽然周也换上了外出用装束,但还是改变不了男性在可爱的杂货前晃来晃去很奇怪这一事实。\\

% ああでもないこうでもないと唸っていると、後ろから「何かお探しですか?」という声がかかる。\\
周正叨念着这也不行那也不行,突然被从身后「在找什么吗?」搭话了。\\

% 振り返ると、店のエプロンをきた妙齢の女性がにこやかに立っている。
周一回头,便看见一位身着店里围裙的妙龄女性温柔地站在身后。

% あまりに悩んでいる周をみかねて声をかけてくれたのだろう。でなければ不審者のようにうろうろおろおろしている周にわざわざ話しかけたりはしない。\\
大概是看不下去周这实在是苦恼的样子,过来帮忙的吧。要不然的话也不会特意向这跟个怪人一样晃来晃去的周搭话吧。\\

% 「あー、その……ホワイトデーのお返しに悩んでいて」
「啊——,那个……白色情人节的回礼,我拿不定主意」

% 「こちらのコーナーに目ぼしいものはなかったのですか? 他のコーナーにもホワイトデーのお返しによく選ばれるものもありますので、ご案内しますよ」
「在这边没有看上的东西吗?别的地方也有常被选去做回礼的商品,我带你去看看吧」

% 「あ、いえそういう訳ではなくて……何とも言いがたい間柄で、贈っても嫌がられないものは何か悩んでいて」
「啊,不是这个意思……只是说和她的关系有些难以形容,所以不知道该送什么不会被讨厌」

% 「というと?」
「怎么说?」

% 「彼女ではないけど親しいといった感じなので……たとえばですけど、アクセサリーとかは好きでもない相手にもらって嬉しいのかな、と」\\
「并不算是女朋友但关系感觉挺熟的……打个比方吧,小饰品这种东西,从称不上是喜欢的人那儿收到,会不会感到高兴啊,这样」

% 相談するのは気恥ずかしく、ぼかして説明していると、店員の女性はくすりと笑みを浮かべる。恐らく、微笑ましいといった意味合いで。\\
因为解释起来很害羞,周的说明便有些含混,但女店员听完后却笑了起来。恐怕是周的烦恼确实引人发笑吧。\\

% 「男性の方がそう悩んでいるのもよくお見かけしますよ」
「男性烦恼这种东西是很常见的哦」

% 「因みに先人はどんな決断を?」
「那他们是怎么决定的?」

% 「悩んでいましたが、購入を決意する方が多いですね。親しいのであれば、贈っても嫌がられるという事は多分ありませんよ」\\
「大部分人虽然犹豫,但最后还是决定购买呢。如果关系亲近的话,就算送了一般也不会被讨厌的哦」\\

% 嫌ではない、と言われて少し安堵してしまったが、それでもあの真昼にアクセサリーを贈るのはやはり少し気後れしてしまう。
不会讨厌——听了这话,周稍稍安心了一点,但对送真昼饰品还是有些犹豫。

% 彼女は身なりはきっちり整えているが、あまりアクセサリーは着けない。たまに着けている事もあるが、どれも品のよいものばかりだ。\\
真昼她虽然身上打理的很整洁,但并没有什么装饰。虽然偶尔会戴一戴,但那些看起来都是好东西。\\

% センスのよい彼女の審美眼に認められるような品物を選べる自信がない。\\
周并没有自信自己选出的东西能达到美感优异的真昼的审美标准。\\

% 「よろしければ、あちらのコーナーで女性に人気の品を幾つかご紹介しましょうか?」
「有需要的话,我给你介绍几个那边在女性间很有人气的饰品吧」

% 「……お願いします」\\
「……拜托了」\\

% ありがたい申し出に、周は思わず姿勢を正してうなずいた。\\
听见这求之不得的提议,周不自觉地摆正了姿势点了点头。\\

% \\


% 「んで買ってしまったと」\\
「然后我就买了」\\

% 事の顛末を樹に話すと、先日の店員と同じような眼差しで笑われた。\\
周跟树讲了整件事的经过,结果树却露出了和前几天那个店员一样的笑容。\\

% 食堂の端で日替わり定食を二人で食べていたのだが、ホワイトデーの話題になってつい言ってしまったのだ。\\
本来两人是在食堂一角吃着每日套餐的,结果讲到白色情人节的话题,周不小心便把这事讲出来了。\\

% 「……うるせえよ。でも、やっぱ交際してないのにアクセサリー贈るって引かれそうでさあ」
「……别多嘴。不过啊,果然明明没有交往却送对方饰品显得有些恶心不是么」

% 「女々しいぞ、男は度胸と勢いだ。あの人なら周相手だったら何でも喜ぶ気がするぞ?」
「你怎么这么矫情啊,男子汉做事得靠勇敢和气势懂不。她的话反正只要是周送的什么都会欣然收下哦?」

% 「……そうだけどさ」\\
「……虽然是这么回事啦」\\

% 真昼の性格的に、何でも普通に喜んで受け取ってはくれるだろう。
以真昼的性格,不论送什么她都会平常地高兴收下吧。

% 周としては本当に喜んで使ってもらえるものを贈りたいので、これでいいのかと悩んでいるのだ。\\
但周的希望是送一件能让她真的高兴而且能用上的东西,因而还在担心这到底达不达得到要求。\\

% 「結局どんなの買ったんだ?」
「结果你买了个啥?」

% 「……ピンクゴールドカラーの、花モチーフのブレスレット」\\
「……粉金色的、以花为主题的手镯」\\

% 真昼はクールな雰囲気のシルバーや華美な印象を抱かせるゴールドより、華やかさはありつつ柔らかく可愛らしい色合いのピンクゴールドが似合うと思ったのだ。\\
感觉对真昼来说,比起给人以冷淡感的银色和华贵感的金色,还是这华丽中泛着些许可爱感的粉金色比较适合。\\

% 流石に学生の身で高価な貴金属は買えないのであくまで見た目だけなのだが、その色のアクセサリーの中から真昼に似合いそうな繊細なデザインを選んだつもりである。\\
虽说身为学生,高价的贵金属是肯定买不起的,所以这里只是谈的颜色,但周还是打算从这种颜色的饰品中选择一款设计纤细而符合真昼风格的。\\

% 「何だ、聞く限りには普通に喜ばれそうなやつじゃん」
「咋了,听上去不是挺能让她高兴的嘛」

% 「……引かれないか?」
「……不会被觉得恶心?」

% 「いや心配しすぎだろ。何でそこは後ろ向きなんだよ……」
「哎呀我说你怎么这么瞻前顾后的。都到这了还畏畏缩缩的……」

% 「女にプレゼントなんてまともに渡したのあいつだけだぞ」\\
「给女性送礼还当面给对方的我可是只对她干过啊」\\

% 母親はまずそういう対象ではないし、千歳はノーカウントだ。そもそも彼女に渡すのは本人たっての希望でスイーツになるので、あまり贈り物という意識すらない。\\
首先母亲肯定是成不了这样的对象,而千岁也不算数。不如说给她的东西是她自己闹着要的甜品,周甚至不怎么觉得那算是礼物。\\

% 「お前そういうところ自信ないよなあ……」
「你在这种事情上面还真是缺乏自信啊……」

% 「むしろなんで自信が持てるんだよ……あいつだぞ?」
「不如说怎么可能会有自信啊……那可是那家伙哦?」

% 「くまのぬいぐるみは喜ばれたんだろ」
「明明熊布偶那时候她挺高兴来着」

% 「そりゃそうだけどさ」
「虽然是这么回事啦」

% 「周、気持ちだ気持ち。ある程度既にコストをかけて選んだんだからあとは気持ちを込めるだけだ」\\
「我说啊周,重要的是心意啊心意。既然你钱也花了东西也买了,那剩下的就只有加入你的心意了啊」\\

% 軽く言ってくれる樹に「そう割り切れたら苦労しない」とぼやいて、周は額を押さえた。\\
对以轻松的口气说着的树,周嘟哝着「要是真能那么干脆就好了啊」用手扶住了额。\\

% ホワイトデーまで、しばらくこの決断がよかったのか悩まされそうである。
看来直到白色情人节当天,周都得要在这个决定到底好不好的纠结中度过了。
