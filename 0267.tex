\subsection{庆祝的约定}

「啊,椎名她说出去了?」\\

午休,周去找木户谢谢她帮了真昼的忙,换来的是她淘气的笑容。\\

她之前在暗中帮真昼准备了那份惊喜,还一副佯装不知的模样,周既有感激,也稍微有点受骗的感觉。木户没跟他说,就代表总司也牵扯在内,周围所有人都在瞒着他。\\

能获得如此多的协助,自是真昼人缘所致,这方面周纯粹地感到佩服,也不禁会想,何必要瞒得那么彻底呢。\\

「还是说你是能自己吃出味道区别的那种?」

「算是吧。虽然我也就是隐约有种味道一样的感觉」

「果真是吃得出啊。可能是婶婶那边的咖啡太好喝了吧」

「……顺带一提,当初是怎么想到给真昼提供咖啡的?」

「这个啊,我当时看椎名为蛋糕发愁来着,就跟她一起看食谱看杂志,那会儿我说要不要用咖啡试试。然后椎名就兴冲冲地答应了」\\

「我这主意还真是不错」木户笑道。周苦笑着点点头,确实还是挺好吃的。\\

「看你样子很满意吧,那就好那就好,文华婶婶肯定也高兴」

「……虽然很谢谢你帮忙,不过这事要跟店长详细汇报吗?」\\

受人恩惠,道谢是应该的,也自然需要告知一些事情的发展,丝卷的热情让周不由得感到担忧。说句心里话,有那样一场印象深刻的初次见面,真不知道她又陷入那种兴奋状态的话该如何应对。\\

或许是知道周想表达的意思,木户露出一抹苦笑,小声道『说、说个大概就行了吧。婶婶也没有太爱刨根问底的……大概」。\\

最后那个「大概」更是放大了内心的不安。不过文华不是什么坏人,能在那混点饭吃也还不错——只要别太过分的话。\\

「不管怎么说,真的谢谢了,就为了我……啊,这么说要惹真昼生气的。总之谢谢了」

「没事没事~朋友的生日总归帮帮忙的嘛。然后我这边也有一份礼物」\\

木户从手里的包中拿出了一个包装好的盒子,一只手都不太拿得住。

周哪料到木户也会给他东西,愣了愣神。很快木户一句高兴的「我也成功送了个惊喜呢」让周回过神。\\

「这个是我跟小总一起出资的,请收下吧」

「倒是用不着那么费心……谢谢。另外我能问一句里面是什么吗?」

「蛋白粉!」

「还得是你」\\

被这么精神地来上一句,周不由得露出笑容,心里也有了分晓。木户又得意地打起了广告「这个既好吃,吸收率又好!小总亲测!」让周笑容更深了一分。\\

「总之,真的谢谢了。什么事都有你帮忙」

「没事啦没事啦,是我自己爱插手的,小总还叫我别管那么多闲事呢」

「帮了这么多忙,哪会是什么多管闲事呢」

「嗯。不过是我因为自己想做才做的,藤宫君不需要放在心上啦。况且对我也有好处」

「好处?」

「嘿嘿,藤宫君和小总关系好了,小总的心情就会变好,小总心情好了,摸肌肉的时间就会增加哦」

「……这样」\\

背后调皮且利己的目的让周不禁苦笑,但看木户平素的样子便知道那并非全部理由。由于她非常爱帮别人的忙,即便是稍微有点交情的周也能对她有所了解,而且还隐约知道她以此为乐。\\

而且她还会打着趣让周别太放在心上,所以周很感谢她的这份细腻,便耸耸肩说「你跟茅野觉得行那就行吧」。\\

\vspace{2\baselineskip}

「今天好厉害啊」\\

回家吃完饭歇息的时候,沙发上坐在旁边的真昼带着柔和的笑容和口吻感慨地喃喃。\\

很快,周就明白了这厉害是指的什么,应和了一声「是啊」。真昼则是感同身受般高兴地翘起了嘴。

在她的表情中,安心、满足、欢喜和谐地交织在一起。被她笑盈盈地看着,周十分难为情,便不禁把视线从真昼身上挪开,转向电视那边。\\

沙发和电视之间的小矮桌上,还有不少今天朋友们送来的礼物。\\

以树、千岁和木户的礼物为始,后来其他一些比起他们来说关系没有那么亲近的同学们,也在那样的气氛下送了礼。

这些礼物大都是点心、果汁这类的东西,但同学开心地送上了祝福,这叫周尽可能做出一副平淡表情的同时,心里满是难为情和高兴。\\

要说去年,周也就把生日告诉过树和千岁,班里没什么风浪,跟那会儿比起来,这次收到的祝福可谓是满满当当。

周本身自认性情淡泊,倒也没什么想要别人祝贺他的欲望。只不过这次他真切觉得,为自己庆生的那些恭喜的话语着实是令人高兴。\\

「……没想到能收到那么多祝贺」

「说明你跟班里的各位都拉近了距离」

「是这样就好了……收到这么多祝贺真的好吗」\\

不经意间脱口而出的这番话让真昼迅速有了反应,她一瞬间就变了脸,从原先的柔和变成有些闹别扭一般,夹杂着担心和傻眼的表情。\\

「你不放心个什么呀?大家之所以祝福你,是因为你跟班里的大家有了交情,结下了友谊。这些都是你的人缘,明白了吗?」

「抱歉抱歉。我也不是觉得自卑,就是还有点没反应过来。毕竟我基本上不跟别人说我生日在什么时候的」\\

跟关系一般的人聊天,要是在没什么关系的话题下硬是说起自己的生日,那就像是非要对方为自己庆生一样。再者,只要至交好友能跟他提上一嘴,周就已经十分幸福了。

周想说的就是,这些祝贺一下子多了太多,多到让他不禁困惑了。\\

「嘻嘻,那说明周君得到了周围的认可和祝福啦,高兴都来不及呢」

「是这样就好了」

「周君」\\

听到责备般的声音,周笑了出来。

「自卑可不行哦」真昼无语的眼神看了过来,似乎在这么说。这样的表情被摆在眼前,那也就没有畏缩消极的余地了。\\

「抱歉抱歉,我知道的。……真好啊」\\

「嗯……就请坦率地收下祝福吧」\\

在周坦率地接受后,真昼也恢复了平日的笑容,朝他胳膊靠了过来。感受到那微微压过来的触感和重量,周嘴角翘起,往下瞥了眼真昼。

周在生日收到祝贺,她也高兴得感同身受,想必那也是她的真情流露。\\

(……真昼应该是觉得,生日是件值得庆祝的事吧)\\

若是心爱的和亲近的人,就更是如此。\\

对于有来往的人,就算是交情不算太深,真昼应该也会由衷地道出她的祝贺。

而生日祝福,并不适用于她自己的生日。想到去年的事情,今天一整天在心里积聚的温暖又柔和的心情中,便扎进了一根冷冷的棘刺。\\

但这棘刺并非讨厌之物。\\

那是让他认识到现实的忠告,同时也是一副起爆药,鼓动着周说出接下来的一番话。\\

「……那个啊」

「嗯」\\

周本打算尽可能顺畅地唤一下真昼,锉掉其中的棘刺和激烈的起伏,而真昼则是察觉出声音中细微的变化,挺直了身子,不再靠在周身上。

这番动作不是戒备,而是真昼料想到之后的话会比较重要,将姿势端正起来。周干咳一声道:\\

「那个,有件事怕你不开心得先跟你说一声。我不是那么擅长藏着掖着的,不说出来也会让你看出端倪,惹得你怀疑」

「嗯」

「你的生日就在下个月吧?」

「啊,这么说还真是」\\

周的话让真昼有所反应,她似乎是真的刚刚才想起来似的,眨了眨眼睛,又在空中画了个漩涡一样眼珠乱转,然后点头。\\

从那态度来看,她大概是完全把这事抛到脑后了,因为内心提不起一点兴趣,所以半天脑子才转过弯来。

看得出来,她对自己实在是没有一点关心。周口中不由得泛起苦涩。\\

「对你来说,有人给你庆祝生日,你心里其实不太好受的吧」

「与其说是不太好受……不如说是无所谓吧」\\

她对自己的生日,正是如同字面意思的无所谓。

从去年的生日其实就已经看出来了,但像现在这样交往后,还听到她这么坚决地说出来,哪怕不是自己的事,也让周心里感觉悲哀。\\

「我本来觉得,生日不过是涨了一岁的分界线,不是什么值得庆祝的日子。事实上也没怎么庆祝过。啊,去年周君给我庆祝生日我还是很开心的!不是有人为我庆祝这件事情无所谓,而是我自己的事情比较无所谓吧」\\

去年那小小的庆祝似乎留在了真昼的记忆里,真昼慌忙摆手,肯定了去年的事情。\\

那副模样一看就是在为他打圆场。「我不是想要你说这些的意思,抱歉」于是周心怀一些内疚,然后继续说了下去,\\

「对你来说,那不是什么特别的日子。我知道的」\\

正因理解真昼的出身和环境,周才更为知晓,对她而言的生日是一个找不出什么意义的日子。

真昼似乎不觉得这样有什么难受的,但只是如此的话,周并不满意。

即便只是周的自我满足,他也希望真昼能切身感受到,有人希望她得到爱并变得幸福,为她的诞生而心怀感恩。\\

「那个,虽然这是我自说自话这么想啦,对我来说你的生日是个特别的日子」

「……特别」

「就像你觉得我的生日很特别一样,我也觉得你的生日是最特别的」\\

周从很多人那里听闻,知道真昼为了他的生日拼命做了准备。

他也知道对方打从心底爱着自己。

周不想成为得到这些爱还只图享受的人,他也想一比一地,不,是包括至今为止的部分,送上他全身心的祝福。\\

「我喜欢你喜欢得无法自拔,我很感谢你能诞生到这个世界上,或者说这让我很开心吧。真昼能诞生到世上,我真的很高兴,也很感激。我一直都很谢谢你诞生下来,遇见我,还喜欢上了我……对我来说,你出生的那天,是非常特别的一天」\\

他希望真昼知道他毫无虚假的内心,对他来说,真昼的存在就是最特别的,她诞生到这世界上的那一天,就是最特别的一天。\\

「所以——如果你不会觉得不开心的话,可不可以就像你给我庆祝一样,我也来给你庆祝一次?我发自内心地感谢真昼能诞生下来,可以不可以?」\\

如果真昼会觉得不开心,那么周就打算和往常一样度过那一天。他并非无视真昼的心情也要给她庆祝。

如果她希望静静地度过那一天,周也便不会再提及这一话题,过上如往常一般的日子。

只不过,如果能获得允许的话,周希望能用上自己全部的资源,来庆祝真昼的生日。\\

他想要传达给真昼,会感谢她诞生的人就在这里。\\

周直勾勾地盯着真昼询问,等待她的回应。他很快发现,真昼脸上变成了和刚才的惊讶不同的另一种惊讶,似乎交杂着期待和不安。\\

「……可以吗?」

「你不讨厌?」

「哪会讨厌呢。那个,我很开心,就是担心,我可能不值得……」

「真昼,你刚才还一副不准我自卑的态度吧?」\\

「指出别人的问题,对自己可也不能马虎」周看真昼那困惑、不安、踌躇交织在一起的软弱模样,便毫不客气地抓住了她的脸颊。\\

周玩弄般地拉扯那绝妙地介于绵绵软软和嫩嫩弹弹之间的脸蛋,将即将落入真昼心底的负面感情温柔地拉起。真昼合不拢的嘴巴中流露出傻乎乎的声音「哇,兹、兹到了」。\\

即便力道有所控制,不至于弄疼,但这一下让真昼略受冲击,放开手后她也是魂不守舍地看着周。周又问了一句「你知道我有多在乎你了吗?」之后,真昼的脸色便变得红润起来,不像只是刚才的捏脸导致的。\\

轻轻的「啊呜」「呜呜」这种凑不成单词的哼哼声从真昼的口中传出,然后她怯生生地抬起头向周看了过来。\\

那副表情之中,已经不再有不安的影子。\\

「……谢谢。光是能有周君的祝福,就已经很幸福了……或者说,感觉有点怪怪的,明明自己的生日怎么样都无所谓的」

「那从今年开始,我会让你再也说不出无所谓这种话的」\\

真昼的无所谓,要追根究底,源头恐怕还是父母的事情。

周无法将之根除,何况那正是形成现在的真昼的因素之一。至少,那显然是一处不想让人触及的敏感部分。\\

因此,周想要覆写掉那种「无所谓」——直让人觉得是破罐破摔一般的漠不关心。他希望真昼感受到,在乎真昼,想要感谢她的诞生的人就在这里。\\

「大家一起大张旗鼓……也不符合你的兴趣呢。那就安安稳稳地庆祝一场吧」

「……嗯」\\

虽说是庆生,真昼也擅长社交,但她原本就属于怕生,或是提防别人的性格,而且喜欢安静的环境。再加上她并不太想让外人知道她的生日,似乎还是小范围地为她庆祝更好一些。\\

她现在似乎觉得,让亲近的人知道生日也没什么问题。这方面还得再跟真昼和一定很想给她庆祝的千岁等人商量。

周在脑中渐渐梳理出今后的计划。他发现真昼正直直盯着他,有些害羞,或者说有些呆不住的样子,却又高兴地将身子缩成一团,便小声笑道:\\

「你还没习惯别人为你庆祝吧,真昼。我可是才预告了下而已」

「这、这不是」

「嗯,看样子已经能好好接受了,太好了……所以我为了给你庆祝会暗地里有点小动作,就原谅我吧」

「嘻嘻,好」\\

都已经预告说要庆祝生日了,那就该光明正大地说自己要在背地里行动。

真昼的话,在预告的那会儿应该就会理解的。不过为了庆祝反而引起她的不安也不好,于是周申请允许自己有些可疑的行为,然后真昼就开怀地笑了出来。

那轻快的笑容和明朗的声音让周放下了心,他摸了摸有点撒娇意味地凑过来的真昼的头。\\

「我会尽量让你开心的。也会努力从各个方面调研」

「这也要在人家面前说吗?」

「啊」

「嘻嘻,最后关头掉了链子啊,真是的」

「您说的是」\\

周抿住嘴唇,表示无言以对,然后就柔和地响起了银铃般的笑声。\\

「……我很期待」

「嗯,我会努力不辜负你的期待」

「好,那我就期待着了」\\

真昼对说了这么多次无所谓的生日开始有所期待,这让周欣喜至极。他用力点头,决定在剩下一个月左右的时间限制中为真昼奔波。
