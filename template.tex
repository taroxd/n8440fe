\documentclass{article}

\usepackage{luatexja}
\usepackage{luatexja-ruby}
\usepackage[no-math]{luatexja-fontspec}
\usepackage[a5paper]{geometry}
\usepackage{luacode}
\usepackage{float}
\usepackage{titlesec}
\usepackage{tocloft}
\usepackage{setspace}
\usepackage[yyyymmdd]{datetime}
\usepackage{indentfirst}
% \usepackage{tipa}  % for \textsubdot
\usepackage{enumitem}
\usepackage[perpage]{footmisc}
\usepackage[unicode,hidelinks]{hyperref}
\usepackage[numbered]{bookmark}

% 思源黑体
\defaultjfontfeatures{
  YokoFeatures={JFM=prop},
  Script=CJK
}
\newjfontfamily{\jpfont}{SourceHanSansJP}[
  UprightFont=*-Normal,
  BoldFont=*-Medium,
  Language=Japanese
]
\setmainjfont{SourceHanSansCN}[
  UprightFont=*-Normal,
  BoldFont=*-Medium,
  Language=Chinese Simplified
]
\setsansfont{SourceHanSansCN}[
  UprightFont=*-Normal,
  BoldFont=*-Medium
]
\renewcommand{\familydefault}{\sfdefault}

\ltjsetparameter{jcharwidowpenalty=0}
\ltjsetparameter{prebreakpenalty={`—,10000}}
\ltjsetparameter{prebreakpenalty={`⸺,10000}}
\ltjsetparameter{prebreakpenalty={`…,10000}}
\ltjsetparameter{prebreakpenalty={`~,10000}}
\ltjsetparameter{jaxspmode={`,,inhibit}}
\ltjsetparameter{jaxspmode={`。,inhibit}}

\onehalfspacing
\titlespacing*{\subsection}{0pt}{8.9ex}{3.4ex}
\newcommand{\subsectionbreak}{\ifnum\value{subsection}>1\clearpage\fi}
\newcommand{\sectionbreak}{\clearpage}
\setlength{\parindent}{2\zw}
\renewcommand{\cftpartnumwidth}{3em}
\setlist{noitemsep,topsep=\baselineskip,left=\zw .. \parindent}

\tolerance=500
\setlength{\emergencystretch}{1.5em}

\counterwithout{subsection}{section}
\newcommand*{\twodigits}[1]{\ifnum#1<10 0\fi\number#1}
\renewcommand{\thesubsection}{\twodigits{\value{subsection}}}
\renewcommand{\contentsname}{目录}
\renewcommand{\abstractname}{简介}
\renewcommand{\dateseparator}{.}

\newcommand*{\sectotoc}[1]{
  \section*{#1}
  \phantomsection
  \addcontentsline{toc}{section}{#1}
}
\newcommand*{\addchaps}[2]{
  \luaexec{
    for i=#1,#2,1 do
      tex.sprint(string.format("\\setcounter{subsection}{\%d}", i-1))
      tex.sprint(string.format("\\input{\%04d}", i))
    end
  }
}
\newcommand{\psline}{
  \par
  \kern20pt
  \hrule
  \kern1pt
  \hrule
  \kern10pt
}
