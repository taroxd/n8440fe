\subsection{考试结果和圣诞安排}

真昼生日的几天后,便到了十二月上旬举行的定期考查出成绩的日子。\\

高二的后半,也就是正式备考的阶段,气氛开始变得紧张起来。当考试结果发到学生手上,班里有人欢喜有人愁,顿时吵闹起来。\\

周担心的是开始打工导致成绩下滑,不过将实际到手的评分过目一遍后,他松了一口气。\\

如果成绩明显下降,周就会对自己的无能感到失望,也会辜负教他学习的真昼和宫本的期待,不好意思面对他们。\\

「顺便问一下,周你考得怎么样?」

「别偷看别人的考试成绩。还有,只想知道排名的话,都贴出来了,不用问我」

「总分和排名都有啊。你好像比上次高了一名?」

「嗯,还好还好」

「真亏你能这么努力……你是同时在打工的吧?」

「正因为有打工,为了兼顾两边,我才会在平日里下足了功夫。读书时间减少了,所以要提高效率」\\

既然无法保证量,那就只有提高效率。周围的学生们也差不多要开始认真学习了,容不得浑浑噩噩。\\

哪怕原本在学习方面算是比较不错,想到周围人今后日积月累的学习时间,周可以说是没有一点宽裕。

选择打工的是自己,这份努力周并不打算落下,但学习量难免会减少,如何弥补这一点就是接下来面对的课题。\\

「你成绩还在稳步提升啊」

「你也是啊」\\

树嘴上虽然这么说,但成绩也有所提升,进入了张贴在布告栏上的排名,名次也比之前看到的更高。\\

「嗯。因为有个啰嗦的老爸,而且现在正是着眼于未来的时期」

「我可不想去考虑明年这时候的事情」\\

千岁抱着成绩单,垂头丧气地用没出息的声音接着说道。在她身后则是真昼面带微笑,有些困扰地垂着眉梢。

千岁似乎对成绩不满意,一脸无精打采的样子,但还是毫不犹豫地把成绩单递给树。树顺着成绩单上的文字看过去,露出尴尬的表情。\\

「小千……嗯,没有发生最糟糕的情况。完全是用擅长的学科填上了窟窿」

「唔,下次我会努力」

「不过分数比上次高,解题方法也掌握得差不多了,我认为有进步」

「昼儿~」\\

千岁似乎被真昼的解围感动,抱住真昼撒娇。真昼则是任由她抱着。

不,与其说是任由她抱着,更准确地讲,是真昼捕获了她,不让她逃走。\\

真昼对亲近的人非常照顾,她似乎也对千岁的成绩有些想法,脸上绽放出分外美丽的微笑。\\

「昼、昼儿?你的笑容好可怕哦?」

「没事的,还有一年的时间,只要努力就行了。剩下的就要看千岁的干劲和目标了。我今后也会陪你一起,就当是顺便复习了」

「……还要坚持?」

「当然。俗话说,罗马不是一天建成的。只要每天都临时抱佛脚,就能把知识刻在脑海里」

「那不叫临时抱佛脚,叫泡在书本里了……」

「这是所有考生都会走的路。朝着目标一起努力吧」

「噫!」\\

千岁这回发出了哀号,但真昼毫不在意,带着任谁看了都会觉得可爱的微笑,握着千岁的手。\\

千岁也明白再这样下去,她想考的大学就悬了,她没有甩开真昼的手,不过还是用眼神向这边求救,于是周和树不约而同挪开了目光。\\

并不是见死不救。绝对不是。\\

「……真昼对喜欢的人很斯巴达呢」

「她知道溺爱其实是在害人家吧」

「确实,例子可太多了」

「哦?那么宠人的椎名也会这样?」

「真昼其实很不饶人的。一开始的时候尤其冷淡」

「无法想象哎」\\

树哈哈笑着耸肩,他无从得知的是,真昼一开始对周的态度实在是非常冷淡。\\

从真昼的角度来看,不向不值得信任的男人展示软弱的一面,这样的警惕心也是可以理解的。等到关系稍微改善后,她有注意到了周自甘堕落的一面,变得不饶人也是理所当然。\\

不如说,能持之以恒地责备批评,这种亲切心和不急不躁反而是值得称赞的。\\

「她对外人不会表现出那种态度。我知道她是为我好才这么说,而且她说的也有道理,所以根本无法反驳」

「看来你以前相当自甘堕落」

「啰嗦。现在不是了」\\

这是周引以为傲的一点。\\

他和一年前的自己相比,简直可以说是天壤之别,这一点真昼应该也会认同吧。

地板上没有乱七八糟的杂物,打扫得干干净净,厨艺变得和一般人差不多,仪容比以前更加整洁,努力学习、锻炼身体,身材从豆芽菜变成了拿得出手的模样。一年前的自己要是听到这些,肯定会大吃一惊。\\

「意思是这个那个都是托了你老婆的福……那么,现在她对你那么好又是为什么?」

「因为我理解了她严格对待我的理由并加以改善,而且就算她不骂我,我也能维持好状态。不需要对已经能独当一面的人太过严厉,而且——」

「而且?」

「我过于自立,反倒让他想让我依赖他了。偶尔还拿头撞我,说自己的事情被抢了」\\

真昼似乎希望周能更加依赖她、向她撒娇,可是周哪能把家里的事情全丢给真昼做,那可太不好意思了。\\

本来因为打工,做饭的事情大都交给真昼了,所以休息日周也会下厨做菜。自己的家务事,他基本上会抽空去做掉。这些是理所当然的事情,不过真昼似乎心情有点复杂。\\

「啊,她希望你需要她啊」

「既然要一起生活,就应该分担彼此的负担。平常打工时我经常把事情交给她做,所以其他时间我有空的话,当然会帮忙做啊。真昼乖乖地放松休息就好了,结果还闹点小别扭」

「哦豁」

「怎么这时候开始坏笑?」

「没有没有。是说你们这对可爱的妻子和没想到还挺能干的丈夫」

「什么叫没想到啊?」

「关系你倒是不否认」

「啰嗦」

「别为了掩饰害羞就肘人」

「好啦,周君,不可以这么粗暴」

「就是就是」\\

真昼和千岁似乎谈完了,看到周摆出攻击姿态,她们便温和地责备起来。

周也不是真的想动手,但行为确实具有攻击性,所以被制止也是理所当然的,他只好乖乖退让。\\

「呜……抱歉」

「哎,这种时候大多都是阿树捉弄过头了。周不可能真的动手,所以应该是平时的打闹而已」

「小千,你到底有没有站在我这边?」

「我站在真昼儿这边哦~」

「好过分」

「赤泽也不可以太捉弄周君哦。因为周君在赤泽面前会变得有点孩子气」

「可是真昼也没站在我这边」

「我是中立的哦?」

「唔」\\

先不论心情上的因素,真昼的确经常站在中立的立场上说话,不会因为周是男朋友就偏袒他。尽管周对此能够理解,但话又说回来,他很想对「孩子气」这个评价提出异议。\\

周很想说,树才像小孩子一样捉弄人,可是真昼却戳了戳他的脸颊责备他,于是他差点说出口的不满就消散了,化作一声叹息从嘴唇的缝隙中滑落。\\

周也知道自己拿真昼没辙,所以为了吐出所剩的一点烦躁而叹了口气。而站在旁边的树则一脸笑嘻嘻的样子。

是不是给他一拳也没关系?周觉得自己冒出这种想法不是他的错。\\

「先不说这个了。真昼,恭喜你保持第一名」\\

看着树现在的表情,怕是自己的攻击性还会继续上升,于是他把树移出视野,然后转向真昼,发现真昼带着略带腼腆的表情。\\

「你还是老样子,该说努力程度惊人吗?我完全追不上你。真的很厉害」

「谢谢。不过因为时间上的问题,我比大家学得早……」

「昼儿被身边的人夸奖时,有时候不会坦率地接受呢」

「呜」

「如果是不熟的人,她就会只说一句谢谢,然后笑着带过。明明人家是真的觉得你很厉害」\\

正如千岁所说,真昼让周不要自卑,自己却往往不太能完全接受称赞。大概可以这么说:称赞的话她会听到耳朵里,但恐怕只有四成能进入她的内心。\\

对真昼来说,努力是理所当然的,虽然不至于完美主义,但她总是追求更高的目标。即使努力得到认可,出于谦虚,她有时也不会完全接受。她本人似乎有在留意,但偶尔还是会这样。\\

千岁能注意到这些细节,可见她很关注真昼,也很了解她,周不禁感到佩服。\\

「……我很高兴,谢谢」

「嗯嗯」

「小千,你把自己当成谁了?」

「当然是昼儿的挚友啊?」

「这么说倒也是」

「千岁,再多说一点」

「周君不要跟着起哄」

「因为真昼叫我不要自卑。真昼也得更认同自己才行」

「……真是的」\\

真昼坦率地接受了,所以周也没什么意见。如果这样还不接受的话,周打算之后要狠狠地夸她。

真昼似乎也看穿了周的眼神,身体打了个哆嗦。周决定把那当作是喜悦的颤抖。\\

「也恭喜周考了第四名~爱的力量真伟大」

「谢了。不过我还有努力的余地,不能只顾着打工,也要好好学习才行。学校里的排名只是过程,我得为明年提升自己才行」

「爱的力量被无视了呢」

「我都要吐槽吐到累了。先不说爱什么的,这确实是多亏有真昼的尽心尽力,我真的很感谢她,也觉得自己必须要努力」

「你真的比以前更有活力,或者说更坚强了」

「也多亏了身体变得更强壮了。我深刻体会到没有体力的话,什么事都做不了。剩下的就是干劲和毅力了」\\

原本算是比较无精打采的周能变得这么有活力,也一样是真昼的功劳,周对她满怀感激。

同时,在提升体力这方面,门胁和树给了他建议,还时不时陪他运动,功劳也有他们的一份。周难免心生感慨,自己真是交到了难得的朋友。

不过,周对调侃自己的树也不是没有意见,偶尔对他冷淡一点,也希望他见谅。\\

「年轻真好」

「我们同年啊……」

「可惜我是二月生的,年纪比你小」

「那树的体力比我更好,去加把劲努力吧」

「几乎!几乎是同年的啦!」

「嘻嘻」

「阿树的精力旺盛程度之后再确认,总之模拟考和定期考查都结束了,可以稍微歇息一小阵子了吧」

「虽然还是要正常上课」\\

可悲的是,考试结束并不代表课程也结束了。接下来是面向考试不停学习的阶段,放松的时间恐怕不会有多少。\\

发还答卷之后,还要订正错误并仔细听讲解,然后继续上课,轻松的时光已经过去了。\\

「想要更多休息」

「全人类都这么想」

「还有,再过差不多两周就是圣诞节了,两位有什么计划吗?」

「啊」\\

经千岁这么一说,周才想起圣诞节的存在。

没错,真昼的生日在十二月初,正好和考试期间重叠,所以周的注意力都放在了生日上。不过,十二月可是活动最密集的时期。

过不了多久,就将迎来世上的家人相庆、情侣彼此确认爱意的日子——尽管这一天原本并非庆祝这些的。

一般来说,这一特大活动是不会忘记的,但因为最近这一个月实在太忙,圣诞节的存在就完全抛到脑后去了。\\

「啊……我只顾着真昼的事情,都忘了」

「交往后的第一个圣诞节,哪有人会忘记的?」


「真、真抱歉,真昼。我完全没有计划」\\


周也觉得现在告诉她圣诞节还没有计划多少有些不妥,但说谎也不好,于是他没有隐瞒,老实地告诉了真昼。真昼本人则是毫不在意地摇了摇头。\\


「老实说,我也忘记这茬了……那个,因为之前发生过那种事,所以有点兴奋,或者说注意力都放在那上面了」\\

真昼本就对节日活动没有太多的热情,再加上生日就在附近,她也不记得这件事了。周也不知道他应该放心,还是去跟平常的情侣关系比较然后烦恼。

不过从表情里看得出,至少她没有感到不满,亡羊补牢,为时未晚。\\

「今年要怎么办?平安夜和圣诞节都悠闲地度过二人时光?」\\

去年,是树和千岁闯进周的家里,举办了圣诞派对,也因此暴露了周和真昼的关系,但这件事也成为了两人加深感情的契机,就结果而言,是个不错的圣诞节。


今年要怎么办?说到这儿,周和真昼面面相觑,烦恼地垂下视线。\\


「唔。老实说,我们平常就一直是两个人独处」


「没什么新鲜感呢」\\


说得直白一点,对周和真昼来说,两人独处并不是什么特别的事情。如果是在周的房间里独处,那自然是另当别论,但很不巧,周一直都极力避免把真昼带到自己的房间里。


两人独处也不会发生什么特别的事情,如果是在家里度过,可以预见圣诞节会和平常几乎没什么两样,顶多就是餐点会比较丰盛而已。\\


「已经是夫妻了呢」


「别打岔。真昼有什么想去的地方或想做的事情吗?」


「只要和周君在一起,去哪里都可以」


「……每次都是这么说」

「呵呵,也就是说,要夫妻之间亲密无间地过两天」

「我说啊……」

「千岁你们有计划要两个人一起出去吗?」

「啊,我们想先听听你们的安排。老实说,明年这时候我们可能就没空悠哉地过节了。不管是要两个人一起过,还是大家一起热热闹闹地过,都各有各的好」\\

她那带有一丝落寞的语气,或许有考虑到未来而感到忧郁的因素在里头,但更可能是感受到,大家一起聚会的时间正逐步减少。\\

明年这时候,除了学校推荐以外的学生都将进入最后冲刺阶段,因此,能像现在这样无忧无虑地聚在一起过圣诞节,恐怕这就是最后一次机会了。

真昼似乎想到了这一点,她环视了所有人一圈,缓缓地开口说:\\

「……如果大家方便的话,我想和大家一起过节。当然,如果你们两位有约会的计划,那你们的安排更优先」

「我也没问题,看你们俩」\\

周也认为在时间变得越来越紧张之前,能玩的时候还是应该尽量玩,和去年一样四个人一起过节就行了。\\

反正圣诞夜和圣诞节有两天,而且这两天都放假,只要其中一天能和真昼独处就行了。

真昼似乎也喜欢和千岁他们一起玩,周想尊重她的意愿。\\

「那就决定咯!大家一起去周家过节!」

「可以是可以,但每次都来我家啊」\\

虽然大家当场决定要一起玩,但场所似乎每次都是周的家里。

虽说基本上真昼都会在一起,但周也算是独居,而且空间宽敞,隔音措施完善,而且和所有人的家都还算近,往往是最方便的地点。\\

「这不是让树去昼儿家里多少有点问题嘛,而且昼儿也头疼。我家太小,而且哥哥和爸爸都很烦,绝对会来打扰的」

「很有可能」\\

虽然经常吵架,但千岁作为白河家的小女儿到底是备受疼爱,听说男朋友和一对朋友情侣要来,她的家人肯定会来看看情况。周跟树一块去过趟千岁的家里,那会儿也被这般那般地说了一顿,这是千岁所不喜的。

那么剩下的就是树家了——周和树对上视线,树脸上浮现了有些困扰的含蓄笑容。\\

「我家的话,该怎么说呢,我妈应该不会在意,我爸不在的话就完全没问题。不过他大概在家」

「因为我的存在会让他不爽」

「小千」

「毕竟事实就是如此」\\

树的父亲大地和千岁的关系实在是称不上好。周听当事人和树的说法,理解到他们之间存在没有恶意的矛盾,所以周也没什么可以多嘴的。\\

树像是责怪,又像是担心一样叫了千岁一声,然后他发现千岁的眼神非常平静,便闭上了嘴。\\

好像放弃了什么一样,千岁的眼神间尽管没有破罐破摔,却隐隐透出一抹想开了的模样。转瞬间,她跟周对上视线,又恢复了平时乐呵呵的表情。\\

「所以抱歉,去周家可以吗?」\\

既然千岁不打算坦露自己的心事,周这个外人也不能介入。\\

于是周也顺着千岁的意图,带着平常的表情摆出一副「真拿你没办法」的态度。\\

「我是无所谓啦,只要你们不在意」

「只要和阿树在一起,我去哪都行~」

「别在别人家卿卿我我」

「哎,都这会儿了还在乎这个?你们也可以卿卿我我啊?」

「谁要在别人面前卿卿我我啊?」

「不在别人面前,就会卿卿我我了是吧?」

「……啰嗦。我们是情侣,在自己家卿卿我我有什么问题」

「嚯嚯」

「嘿嘿」

「要不不让你来了」

「对不起,您大人有大量,请原谅小的」

「你下次注意点」

「遵命」\\

千岁装模做样地喊出了磕头时的那种声音,真昼也维持不住平时的表情,轻声笑了出来。\\

「嘻嘻,关系真好」

「昼儿也可以参与进来哦?比如来点抓衣带绕圈圈的桥段」

「绕圈圈……?」

「不来,更别让我去扮演反派代官。对真昼和对衣物我都不会有粗暴举动的」

「这方面倒还挺认真」

「谁搞这些无法无天的事啊。对物品也要爱护」

「在这种地方也会毫无保留地发挥善良本性,这就是周这个男人」

「这算是什么善良本性,就是普通的感性而已」

「周的普通可真好呢昼儿」

「如果不是平常如此,我大概也不会去多看他一眼了呢」

「我是不是被损了?」

「是在夸你、是在夸你」\\

被银铃般可爱又清脆的声音笑着这么说,周也不好抱怨什么,只能半眯着眼看向真昼。从她身上完全感觉不到恶意,周只好无可奈何地闭上嘴。\\

千岁见此情景,忍不住哈哈大笑起来。周则是抛去尖锐的目光让她闭嘴,然后夸张地叹了一口气。
