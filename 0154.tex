% \subsection{154 一瞬見えた色}
\subsection{154 一瞬見えた色}

% 自宅に帰って次の日、真っ先にした事は掃除だった。
回到家后第二天,首当其冲的事情便是打扫卫生。

% 流石に帰宅当日は疲れていたのでしなかったが、二週間も家を空けていれば部屋も埃が溜まっている。僅かなものではあるが、真昼も一緒に家で過ごすためなるべく清潔にしておきたいところだ。\\
到家当天实在是累,周便把这事放了放,但空了两周,家里便到处都积起了灰。虽然实际上灰也不多,但考虑到真昼也会呆在这个家里,还是把屋子好好打扫干净的为好。\\

% そんな訳で真昼仕込みの掃除術を駆使して、周は掃除をしていた。ちなみに真昼は真昼で自宅の掃除をしているらしいので、周一人である。\\
于是,周便使用着被真昼教会的技术,开始打扫起了屋子。顺带一提,真昼似乎是在清理着自己家里,因此周是孤军奋战。\\

% 真昼のお陰で掃除は得意ではないものの維持する事には問題がない。真昼いわく『ちゃんとこまめに掃除していれば大きな労力は要りません。後回しにするから不必要に労力と時間が奪われるのです』との事。
多亏了真昼的努力,周虽然称不上擅长打扫,但维持家中的清洁还是能做到的。正如真昼所言『打扫勤快点反而花不了太多功夫。一再拖延到时候花的时间精力』。

% 真昼の教えの通り定期的に軽い清掃をするだけで綺麗な状態を保てていた。\\
周遵从了真昼的叮嘱,只是定期做着简单的打扫,便让家里保持了整洁。\\

% 今回も、埃が多少家具に降っているだけなので、掃除に時間はかからなかった。\\
这次也只是家具上稍微积了点灰,打扫起来也不怎么花时间。\\

% さっと家具をほんのりと化粧する埃を拭いて掃除機をかけてついでに窓も拭き終えたところで、周は時計を見上げる。\\
周动作流畅地擦去了给家具打上淡妆的尘埃,拿吸尘器吸掉了灰,顺带擦了擦窗户,然后抬头看了下时间。\\

% 既に時刻は十五時過ぎ。
已经是下午三点多了。

% いつも通っているスーパーのセールは十六時から始まる事が多いので、そろそろ向かった方がいいだろう。\\
平常去的超市,打折活动常常在下午四点开始,现在差不多该出门了。\\

% (我ながら思うけど、所帯染みてきたなあ)\\
(自己这么一想,我也变得挺顾家了啊)\\

% スーパーへ行くのは帰省前に冷蔵庫を空にしたせいで、本日の夕食の材料がないのだ。朝昼はカップラーメンや冷食で済ませたが、夕飯はそうはいかない。
想到去超市,是因为回老家前清空了冰箱,因而家里没有做晚饭的材料了。虽然早饭和午饭都用杯面和不用下锅的东西对付了过去,但晚饭不能这么搞。

% 買い物担当は周であるが、材料費は折半だ。なるべく安く済ませようという考えはおかしくないのだが……高校生男子が食費を気にするのは些か所帯染みているだろう。\\
周负责买东西的事情,而费用则是各出一半。虽说尽量少花一点钱这一想法倒也挺正常……但男高中生考虑起伙食支出,有点给人一种成了家的感觉。\\

% 自分でも自分の変化にふっと笑って、とりあえず軽く汚れた服を着替えるべく自室に着替えを取りに行った。\\
面对这自己的变化,周自己也不禁微微一笑,然后走向自己房间,打算另取一套,换掉这身稍微有点脏的衣服。\\

% \\


% 「……ん?」\\
「……嗯?」\\

% スーパーに行く最中、考え事をしながら歩いていると、見覚えのある色素の薄い色の髪が見えた。
周正在去超市的路上,一边想着事情一边走着,突然眼里出现了一丛见惯了的淡色头发。

% つい振り返ってしまうが、当然後ろ姿しか見えない。真昼のような髪の長さでもなければ、そもそも性別からして違う。染めたような色ではなく天然のあの色の薄さは、珍しい。\\
周下意识地回过了头,但映在眼里的自然只有一个背影。这头发并没有真昼的那么长——倒不如说性别就不一样。不是染过的那种天然的淡色头发十分稀少。\\

% 珍しい事もあるもんだな、と思いつつ到着したスーパーに入って本日の夕食の材料をかごに放り込んでいると「あれ」と聞き覚えのある声が背後から聞こえた。\\
虽说稀少,倒也不是不可能啊,周一边想着一边进了超市,正在把今天晚饭的材料放进篮子里的时候,身后传来了一声耳熟的声音。\\

% 「こんな所で会うなんて珍しい」
「在这里遇上,还真是巧呢」

% 「九重か」\\
「九重啊」\\

% 門脇を通じて騎馬戦で親しくなった青年が、周と同じようにかごを腕に提げている。
靠门胁的关系在骑马战那时候互相熟悉了的青年,和周一样手上提着个篮子。

% ちなみにかごの中に入っているのはお菓子やジュースなので、彼の方が余程男子高校生らしい買い物をしていた。\\
顺带一提对面的篮子里装的都是些零食果汁一类,很像是男高中生会买的东西。\\

% 「藤宮って家こっちなの?」
「藤宫你家在这边吗?」

% 「おう。九重はこっちらへんじゃないと思ってたんだが……」
「啊。话说我记得九重你家不是住这边的啊……」

% 「僕はただ友達の家に泊まりだから買い出しにきただけ。藤宮は……ご飯?」
「我是现在跑去朋友家住,现在出来买东西啦。藤宫你是……食材?」

% 「ん。夕食の買い出しだよ」\\
「嗯。来买晚饭的材料」\\

% 見れば分かる通り、周の手にしたかごの中には生の鶏肉や大根、牛乳や豆腐といったおやつとはどう間違っても認識出来ないようなものが入っている。\\
一眼便知,周提着的这篮子里放着的都是些生鸡肉,萝卜,牛奶,豆腐这些,再怎么也不会看错成零食的东西。\\

% 「そういえば藤宮は一人暮らしなんだっけ。えらいね」
「说起来,藤宫你是一个人住来着。挺了不起的啊」

% 「まあ真昼がご飯作るんだけどな……」
「嘛做饭是靠真昼了啦……」

% 「……そういえば言ってたっけ……すごい生活してるよね」
「……好像你说过来着……这日子有点舒服啊」

% 「だな。真昼には感謝してるよ」\\
「是啊。真得感谢真昼呢」\\

% 彼女が居なければ周の食生活はズタボロだろう。掃除は多少出来るようになっても、未だに料理は不得意のままだ。
要是没有真昼,周的饮食恐怕是一片混乱。即便现在周多少能打扫干净屋子了,做饭的手艺依旧是惨不忍睹。

% 仮に居なくなってしまえば、周の今の生活は成り立たなくなる。\\
要是没了真昼,周现在的生活就无从谈起了。\\

% 小さく苦笑しながら「真昼様様だ」と呟けば、九重はそっとため息をつく。\\
周一边微微苦笑,一边小声感叹着「真是谢谢真昼了啊」,九重则叹了口气。\\

% 「なんというか、ほんと……あれだね、首ったけ?」
「真么说呢,你还真是……是叫,神魂颠倒?」

% 「そうだな。真昼もだけど」
「是啊。真昼他也差不多」

% 「自信満々に言えるんだね」
「你这可真是有自信啊」

% 「愛されてるって自覚は持ってるよ」\\
「我可是很明白自己正被爱着啊」\\

% 付き合う前は好意に自信が持てなかったが、今は違う。真昼に大切にされて好かれているのは自覚しているし、彼女が周の側に居る事を望んでいる事も分かっている。
在交往之前,周虽然有好意却没有自信,但如今却不同了。周已经有了正被真昼珍视、喜爱的自觉,也明白了呆在周的身边是她的希冀。

% 自意識過剰とかではなく純粋に事実だと認識していた。そう出来るようになったというのが、自信のついた証拠かもしれない。\\
周已经认清了,这并非自己的自我意识过剩,而是切切实实的事实。而这,大概也是对周自信的证明吧。\\

% あっさりと、淀みなく答えた周に、九重は先程まで苦笑していた周に代わって苦笑する。\\
听见了周流畅利落的答复,这回换成九重苦笑了起来。\\

% 「まあ、自信がついたならいい事だと思うよ。相思相愛なのにうじうじしてたあの時よりいいんじゃないの」
「嘛,我觉得有自信是好事啦。比起明明两情相悦却还是隔了层窗户纸那时候好多了」

% 「厳しいなあ」
「说的真狠啊」

% 「だってどう考えても好かれてるって見えてたのに。ま、僕には関係ないけど、君らが幸せならそれでいいんじゃないの」\\
「毕竟怎么看那都是喜欢对方的样子嘛。嘛,虽说跟我没什么关系,不过既然你们都很幸福,那不就皆大欢喜了」\\

% 肩を竦めた九重に彼なりの賛辞を感じて、頬を緩める。\\
九重耸了耸肩,周听着他那有自己风格的赞赏,弯起了嘴角。\\

% 「……ま、優太も納得してたし、僕はこれで丸く収まったと思ってるからね」
「……嘛,优太他也是接受了的样子,我也觉得这样也算是个圆满结束了呐」

% 「え?」
「诶?」

% 「ううん、なんでもない。じゃ、僕はレジに行くから」\\
「啊,没什么。那我先去结帐了」\\

% 何故そこで門脇、と思ったものの、追求をする前に九重はさっさとこちらに背を向けて去っていったので、周は困惑しつつもスマホにメモした夕食の材料をかごに放り込むべく彼に背を向けたのだった。
周正疑惑着为什么会跑出门胁的名字来,但在他追问之前,九重便迅速转过身子走了。周尽管心怀疑惑,也还是转过了身,继续对照着手机,挑选起了里面记着的晚饭材料。
