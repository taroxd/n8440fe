\subsection{}

「今年情人节,你想要什么样的巧克力?」\\

进入二月没多久的某一天。\\

今天不用打工,周和真昼一起悠闲地度过,为了减轻真昼的负担,周负责做晚饭的主要工作。在等待饭煮好的时候,真昼抛来了这个问题,周不禁眨了眨眼。\\

去年周根本没把情人节放在眼里,也没想过自己会和这种节日扯上关系,所以完全没放在心上。不过,今年的周再怎么也记得二月有情人节这回事了。

只是,他没想到真昼会当面问起。\\

「这次直接问了啊」\\

去年,两人还没交往的时候,真昼似乎是请千岁帮忙打听周的喜好,而今年因为两人正在交往,所以她就毫不客气地直接问了。

周还以为真昼会再稍微隐秘一点,没想到她问得这么光明正大,让身为收礼方的周有点吓到了。

看着周有些坐立不安的样子,真昼一边确认电锅上显示的煮饭时间,一边爽朗地笑了起来。\\

「都已经在交往了,要是还偷偷摸摸的,周君肯定也会注意到我在为情人节做准备的。让你装作没发现也挺不好意思的」

「世人都在讨论的话,确实很难忽略。你在这方面还挺讲究」

真昼在生日时就已经认真准备过了,不难想象情人节时她会去很用心地准备。

就算是迟钝的周,看到真昼偷偷准备什么东西的样子,联系到时期,也会想到是情人节。\\

既然如此,她便自然会想:干脆一开始就不准备惊喜,直接把周想要的东西送给他。虽然周觉得这样未免过于果断。\\

「反正瞒着你也会被发现,既然这样,难得有这个机会,当然要准备你想要的东西啊」

「我是能理解你的想法,可是我喜欢的东西你已经很清楚了吧?」\\

真昼去年已经通过千岁得知周不喜欢太甜的东西,而且从两人相遇到现在,她应该已经掌握了周的饮食喜好。

周心想,真昼应该能轻易做出他喜欢的东西……可是不知为何,真昼却一脸不服气的样子。\\

「我知道你不喜欢太甜的东西,可是之前做过的橙香四溢和巧克力蛋糕又太没意思了。我觉得喜好和想吃的东西是两回事,所以才问你现在想吃什么」

「可只要是真昼做的,我都喜欢」\\

这样说可能会被当成敷衍,但周实际上并没有特别的要求。不如说,这一年多共同生活的时间里,他切身感受到真昼做的任何东西都很好吃。\\

真昼总是能从某处拿出各种各样的食谱,熟练地做出料理,让周赞不绝口,周甚至觉得她是不是根本没有不会做的东西。在不知道的情况下,期待真昼会做什么料理也是一种乐趣。\\

所以,周真心认为什么都可以,但根据情况不同,这样也可能会触怒做菜的人,所以他才不太敢明说。\\

真昼听了,比起生气,更显得有些无奈。\\

「……我知道你说什么都可以是真心的,实际上也对任何东西都很满意,可是周君也要多加注意,别人问的时候如果回答什么都行,会让对方很伤脑筋的」

「抱歉。我的意思是希望你优先做你想做的东西。因为我喜欢你做的所有食物,所以很期待会拿到什么……做点心的时候,总有这些那些想做的东西吧,我希望你优先去做那些你想做的」\\

情人节这天,真昼会开开心心地做点心,周则是怀着感恩的心收下她的心意,对点心本身并没有什么执着。

既然知道会很好吃,那么就算不挑,等待着他的也一定是幸福,所以他希望能以真昼想做什么的心情为优先。\\

要是这么说,感觉又会惹她生气,于是周稍微委婉地表达自己的意思,而真昼似乎连这一点也看穿了,深深叹了口气。\\

「你这种地方特别没有欲望,真让人伤脑筋……真是的,不可以对女生说随便什么都行哦」

「我只会对你这么说……而且在其他女生面前我也没有这种给对面做选择题的机会吧」

「我知道」

「嗯,你清楚得很」\\

两人对彼此的想法心知肚明,于是相视而笑,相互表达意见。\\

「……真的没有想要的吗?」

「你不会觉得我对点心很了解吧?」

「其实你意外地清楚呢。和我去蛋糕店的时候,你不是经常看着橱窗和架子上的点心吗?」

「那不是在看我喜欢的,只是觉得你会喜欢而已」

「请你多考虑自己一点,我的喜好不重要」

「才不是。买蛋糕的时候,我想买最适合你的」

「真是的」\\

「你怎么老是这样」真昼撅起嘴唇表达不满,周回她「因为喜欢你啊」,结果换来比刚才更强烈的责备。\\

真昼的语气虽然强硬,但并不是真的生气,只是在掩饰害羞。周在至今的相处中已经很清楚这一点,所以只是轻笑带过,结果这次换真昼叹气了。\\

「……所以,你真的没有想要的吗?没有的话,我可以去向千岁请教秘密食谱哦?」

「拜托不要」\\

真昼立刻祭出最终手段,禁不住战栗的周连忙用仿佛不认识这个人的语气拒绝,但真昼依然笑容满面。\\

周很清楚,要是捉弄她捉弄得太过火,就会遭到惨痛的报复。尽管如此,他还是忍不住开口,因为只要向真昼展示好感,真昼都会感到高兴,即便她表现的方式有所不同。这次则是因为做得太过火了,才会变成这样。\\

周战战兢兢地回想起去年千岁精心制作的俄罗斯轮盘巧克力的威力。真昼大概是忍不住了,从嘴里呼出一大口气,愉快地笑了起来。\\

「嘻嘻,开玩笑的。周君不太能吃刺激性的东西呢」

「你也知道我比你还怕辣吧」

「所以我做菜的时候都会调整辣度。不过,如果要借用千岁的食谱,那应该要听从她的指示吧」

「对不起」

「我真的没有要付诸实行的意思哦?只是,如果你一直说随便什么都好,我会很伤脑筋……希望你能提出一些要求,比如类别或食材之类的」

「……甜度低一点,而且能保存一段时间的?一天吃完太浪费了。虽说既要减少糖分又要保存时间也很奇怪就是了」\\

不用防腐剂的话,点心能保存一段时间,基本上都是多亏糖的保湿作用。如果减少糖的量,保质期也就自然倾向于变短。\\

如果要减少甜度又想延长保质期,使用防腐剂会比较保险,但一般家庭不会在普通的礼物上使用防腐剂,真昼应该也不会这么做。而且市面上连有没有卖都很难说。\\

周心想自己提出了一个麻烦的要求,想要收回,但真昼却点头表示「我知道了」,比想象中更加干脆。\\

「那就不要做蛋糕类,做巧克力本身的加工点心比较好。如果尽可能选择水分较少的馅料,应该能保存一段时间,而且这样也能享受到各种风味」

「这部分就交给你决定,反正我什么都期待,只要你做想做的东西,我就很高兴了」\\

周不清楚详细情况,但真昼似乎已经想好要做什么了。他不知道该为此感到庆幸,还是该为让她鼓足干劲而感到抱歉。\\

周对似乎已经决定好要做什么的真昼说「别勉强自己哦」真昼却愣着不动。\\

「只要是你爱吃的东西,我什么都可以做哦?」

「最可怕的就是你真的什么都可以做,没有言过其实」

「我在这种时候是不会开玩笑的」

「我对平常的料理就很满足了啦」

「那我找机会把新学的食谱加进菜单里」

「好强的上进心……」\\

在小雪充满爱意的教育下,真昼的技能范围已经大得惊人,而且她似乎还以那股旺盛的学习能力和上进心学了新的食谱,身为受惠的一方,周在感到高兴的同时也担心了起来。\\

明明现在就已经很满足了,她还想进一步抓住周的胃袋是想做什么?\\

「敬请期待巧克力和料理哦」

「我会期待的,但真的别勉强自己哦?」

「我很分得清自己的能力」

「真羡慕你有这种自信」

「嘻嘻」\\

真昼自豪地露出有些淘气的笑容,周心想自己真是敌不过她,一边从抽屉里拿出饭勺,准备盛刚煮熟的米饭。
